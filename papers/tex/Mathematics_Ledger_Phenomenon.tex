\documentclass[11pt]{article}

% ── Page geometry ──
\usepackage[
  top=1.15in, bottom=1.15in,
  left=1.25in, right=1.25in
]{geometry}

% ── Fonts and typography ──
\usepackage[T1]{fontenc}
\usepackage{lmodern}
\usepackage{microtype}

% ── Mathematics ──
\usepackage{amsmath,amssymb,amsthm}
\usepackage{mathtools}
% ── Tables and lists ──
\usepackage{booktabs}
\usepackage{array}
\usepackage{enumitem}
\setlist{leftmargin=*, itemsep=2pt, parsep=0pt}

% ── Colour (restrained) ──
\usepackage[dvipsnames]{xcolor}
\definecolor{secblue}{RGB}{20,60,120}
\definecolor{linkblue}{RGB}{0,60,140}

% ── Hyperlinks ──
\usepackage[
  colorlinks=true,
  linkcolor=linkblue,
  citecolor=linkblue,
  urlcolor=linkblue
]{hyperref}
\usepackage[nameinlink,capitalize]{cleveref}

% ── Headers ──
\usepackage{fancyhdr}
\pagestyle{fancy}
\fancyhf{}
\renewcommand{\headrulewidth}{0.4pt}
\fancyhead[L]{\small\textsc{The Law of Mathematical Inevitability}}
\fancyhead[R]{\small\thepage}

% ── Theorem environments ──
\theoremstyle{plain}
\newtheorem{theorem}{Theorem}[section]
\newtheorem{proposition}[theorem]{Proposition}
\newtheorem{lemma}[theorem]{Lemma}
\newtheorem{corollary}[theorem]{Corollary}

\theoremstyle{definition}
\newtheorem{definition}[theorem]{Definition}
\newtheorem{example}[theorem]{Example}
\newtheorem{conjecture}[theorem]{Conjecture}

\theoremstyle{remark}
\newtheorem{remark}[theorem]{Remark}

% ── Notation ──
\newcommand{\phig}{\varphi}
\newcommand{\Jcost}{J}
\newcommand{\Jbit}{J_{\mathrm{bit}}}
\newcommand{\Rpos}{\mathbb{R}_{>0}}
\newcommand{\Rnn}{\mathbb{R}_{\ge 0}}
\DeclareMathOperator{\lcm}{lcm}

% ══════════════════════════════════════════════════════════════
\begin{document}
% ══════════════════════════════════════════════════════════════

\thispagestyle{empty}

\begin{center}
  {\LARGE\bfseries The Law of Mathematical Inevitability:\\[4pt]
   Numbers, Proofs, and Universal Reference\\[4pt]
   Forced by the d'Alembert Cost Equation}

  \bigskip

  {\large Jonathan Washburn}\\[4pt]
  {\small Recognition Science Research Institute, Austin, Texas}\\
  {\small\ttfamily washburn.jonathan@gmail.com}

  \bigskip

  {\small February 2026}
\end{center}

\medskip\hrule\medskip

\noindent\textbf{Abstract.}\;
The d'Alembert functional equation on $(\Rpos,\cdot)$, under
normalization and non-degeneracy, uniquely determines the cost
functional $\Jcost(x) = \tfrac{1}{2}(x+x^{-1})-1$~\cite{CostFunctional2026,Aczel1966}.
We derive three consequences for the foundations of inference:
\textbf{(1)}~the logarithm is the unique continuous additive
invariant vanishing on product-one sequences (via Cauchy's theorem);
\textbf{(2)}~the golden ratio $\phig$ is the unique base whose
geometric lattice admits a Fibonacci-type recursion;
\textbf{(3)}~a space is a universal referent for all positive-cost
objects if and only if its infimal cost is zero.
We compute explicit costs for a propositional-resolution example,
characterize the generators of the balanced-proof monoid, and pose
open problems connecting the framework to proof complexity.

\medskip\noindent
\textbf{MSC 2020:}\; 39B22 (functional equations), 03F20 (complexity
of proofs), 03A05 (philosophical and critical aspects of logic).

\medskip\noindent
\textbf{Keywords:}\; d'Alembert functional equation, Cauchy equation,
cost functional, golden ratio, proof complexity, universal reference,
Wigner effectiveness.

\medskip\hrule

\bigskip
\tableofcontents

\newpage

% ══════════════════════════════════════════════════════════════
\section{Introduction}\label{sec:intro}
% ══════════════════════════════════════════════════════════════

The \emph{d'Alembert functional equation}
\begin{equation}\label{eq:dalembert}
  g(s+t) + g(s-t) \;=\; 2\,g(s)\,g(t), \qquad s,t \in \mathbb{R},
\end{equation}
has been studied since d'Alembert~(1769) and Cauchy; its continuous
solutions are $g(t) = \cosh(\lambda t)$ for
$\lambda \in \mathbb{R}$~\cite{Aczel1966,Kannappan2009}.  The
\emph{multiplicative form}, via $x = e^s$, $y = e^t$,
$F(x) = g(\ln x) - 1$, yields
\begin{equation}\label{eq:rcl}
  F(xy) + F(x/y) \;=\; 2F(x)F(y) + 2F(x) + 2F(y),
  \qquad x,y > 0.
\end{equation}
Under normalization $F(1)=0$, non-degeneracy $F(x)>0$ for $x\ne 1$,
and calibration $F''_{\!\log}(0) = 1$, the unique solution
is~\cite{CostFunctional2026}
\begin{equation}\label{eq:jcost}
  \boxed{\;\Jcost(x) \;:=\; \tfrac{1}{2}\!\bigl(x + x^{-1}\bigr) - 1
  \;=\; \cosh(\ln x) - 1, \qquad x > 0.\;}
\end{equation}
This functional arises as the ``recognition cost''
in~\cite{RecognitionGeometry2025,Foundations2026,DimensionalRigidity2026}
and as the unique symmetric non-negative solution
to~\eqref{eq:rcl}~\cite{CostFunctional2026}.

In this paper we derive three consequences of $\Jcost$ for the
\emph{foundations of inference}: a uniqueness theorem for balance
conditions (\cref{sec:proofs}), a characterization of the self-similar
lattice (\cref{sec:numbers}), and an if-and-only-if characterization
of universal reference (\cref{sec:wigner}).  We state what is proved,
what is conjectured, and what remains open.

% ══════════════════════════════════════════════════════════════
\section{The Cost Functional}\label{sec:cost}
% ══════════════════════════════════════════════════════════════

\begin{definition}[Axioms]\label{def:axioms}
A \emph{d'Alembert cost functional} is a continuous function
$F : \Rpos \to \Rnn$ satisfying:
\begin{enumerate}[label=\textup{(A\arabic*)},nosep]
  \item\label{ax:norm} $F(1)=0$ and $F(x)>0$ for $x\ne 1$;
  \item\label{ax:recip} $F(x)=F(x^{-1})$ for all $x>0$;
  \item\label{ax:dalembert} $F(xy)+F(x/y)=2F(x)F(y)+2F(x)+2F(y)$
    for all $x,y>0$.
\end{enumerate}
\end{definition}

\begin{theorem}[Uniqueness; {\cite[Thm.~1]{CostFunctional2026},
cf.~\cite[Ch.~3]{Aczel1966}}]\label{thm:unique_cost}
The unique d'Alembert cost functional is
$\Jcost(x) = \tfrac{1}{2}(x+x^{-1})-1$.
\end{theorem}

\begin{proof}[Proof sketch]
Set $G(t):=F(e^t)+1$.  Then \ref{ax:dalembert} becomes the classical
$G(s+t)+G(s-t)=2G(s)G(t)$.  By~\cite[Thm.~3.1.3]{Aczel1966},
$G=\cosh(\lambda\,\cdot\,)$.
\ref{ax:norm} forces $\lambda\ne 0$; calibration fixes $\lambda=1$.
Hence $F=\cosh\circ\ln - 1 = \frac{1}{2}(x+x^{-1})-1$.
\end{proof}

\begin{remark}[Heritage]
The additive form~\eqref{eq:dalembert} was solved by d'Alembert~(1769);
Acz\'el~\cite{Aczel1966} and Acz\'el--Dhombres~\cite{AczelDhombres1989}
provide the modern theory.
Ref.~\cite{CostFunctional2026} adds the calibrated multiplicative form
with \ref{ax:norm}, selecting the cosh branch.
\end{remark}

\smallskip\noindent
\textbf{Key properties} (all \emph{derived}, not assumed):\;
$\Jcost\ge 0$ with $\Jcost=0 \Leftrightarrow x=1$\; (AM--GM);\;
$\Jcost(x)=\Jcost(x^{-1})$;\;
$\Jcost''(x)=x^{-3}>0$\; (strict convexity).

\begin{definition}\label{def:phi}
$\phig:=(1+\sqrt{5})/2$ (the golden ratio);
\quad $\Jbit:=\ln\phig\approx 0.4812$ (the ledger bit cost).
\end{definition}

% ══════════════════════════════════════════════════════════════
\section{The Self-Similar Lattice}\label{sec:numbers}
% ══════════════════════════════════════════════════════════════

\begin{definition}[$\phig$-Ladder]\label{def:ladder}
$L:\mathbb{Z}\to\Rpos$, $L(n)=\phig^n$.
\end{definition}

\begin{theorem}[Ladder]\label{thm:ladder}
$L$ is positive, strictly monotone, satisfies $L(0)=1$, and obeys the
Fibonacci recursion $L(n{+}2)=L(n{+}1)+L(n)$.
\end{theorem}

\begin{proof}
$\phig>1$ gives positivity and monotonicity.
$\phig^{n+2}=\phig^n\phig^2=\phig^n(\phig+1)=\phig^{n+1}+\phig^n$.
\end{proof}

\begin{theorem}[Uniqueness]\label{thm:self_similar}
If $\alpha>1$ satisfies $\alpha^2=\alpha+1$, then $\alpha=\phig$.
\end{theorem}

\begin{proof}
$\alpha=(1\pm\sqrt{5})/2$; the constraint $\alpha>1$ selects $\phig$.
\end{proof}

\begin{remark}
The condition $\alpha^2=\alpha+1$ arises precisely when
$\alpha^{n+2}=\alpha^{n+1}+\alpha^n$ (Fibonacci recursion on the
lattice).  No other base $\alpha>1$ satisfies this;
e.g., $2^{n+2}=4\cdot 2^n\ne 3\cdot 2^n$.
\end{remark}

\begin{definition}[Metric]\label{def:distance}
$d(m,n):=|m-n|\cdot\ln\phig$.
\end{definition}

\begin{proposition}\label{prop:metric}
$d$ is a metric on $\mathbb{Z}$.\; ($\ln\phig$ times the standard
metric.)
\end{proposition}

\begin{remark}
The cost $\Jcost(\phig^n)=\cosh(d(n,0))-1$ is a strictly monotone
convex function of $d$---not itself a metric, but an equivalent
measure of distance from unity.
\end{remark}

\begin{corollary}[Zeckendorf~{\cite{Zeckendorf1972}}]
Every positive integer has a unique sum of non-consecutive Fibonacci
numbers.
\end{corollary}

% ══════════════════════════════════════════════════════════════
\section{Balanced Ledger Sequences}\label{sec:proofs}
% ══════════════════════════════════════════════════════════════

\begin{definition}[Proof]\label{def:proof}
A \emph{recognition proof} $p=(r_1,\dots,r_N)$ with $r_k>0$ has
cost $C(p)=\sum\Jcost(r_k)$ and log-balance
$\beta(p)=\sum\ln r_k$.  It is \emph{balanced} when $\beta(p)=0$.
\end{definition}

\subsection*{Uniqueness of the balance condition}

\begin{theorem}[Cauchy on $(\Rpos,\cdot)$;
{\cite[Ch.~2]{Aczel1966}}]\label{thm:cauchy}
Every continuous homomorphism $\phi:(\Rpos,\cdot)\to(\mathbb{R},+)$
has the form $\phi=\lambda\ln$ for some $\lambda\in\mathbb{R}$.
\end{theorem}

\begin{proof}
Set $f(u):=\phi(e^u)$; then $f(u+v)=f(u)+f(v)$ continuously, so
$f=\lambda\,\mathrm{id}$~\cite[Thm.~2.1.1]{Aczel1966}.
\end{proof}

\begin{theorem}[Unique Balance]\label{thm:unique_balance}
If $\Phi$ is continuous, additive on sequences, and vanishes on every
product-one chain ($\prod r_k=1$), then
$\Phi=\lambda\sum\ln r_k$.
\end{theorem}

\begin{proof}
Additivity: $\Phi=\sum\phi(r_k)$.  Pair $(t,t^{-1})$ gives
$\phi(t)+\phi(t^{-1})=0$.  Triple $(r,s,(rs)^{-1})$ gives
$\phi(rs)=\phi(r)+\phi(s)$.
Apply \cref{thm:cauchy}.
\end{proof}

\begin{corollary}\label{cor:forced_balance}
Log-balance is the unique continuous additive closed-chain invariant
on $(\Rpos,\cdot)$.
\end{corollary}

\begin{theorem}[Monoid]\label{thm:compose}
Balanced proofs form a monoid; $C$ is a homomorphism to
$(\Rnn,+)$.
\end{theorem}

\begin{example}[Minimal balanced proof]\label{ex:phi_pair}
$p=(\phig,\phig^{-1})$:\;
$\beta=0$,\;
$C=2\Jcost(\phig)=\sqrt{5}-2\approx 0.236$.
\end{example}

\begin{remark}[Semantics]
The ratio $r$ of a proof step models an ``informational exchange
rate.''  \Cref{thm:unique_balance} holds \emph{regardless} of the
concrete assignment $r\mapsto$~inference-rule.  The mapping to
specific formal systems is an open problem
(\cref{sec:formal}).
\end{remark}

% ══════════════════════════════════════════════════════════════
\section{Further Consequences}\label{sec:further}
% ══════════════════════════════════════════════════════════════

\begin{proposition}[Canonical ordering]\label{prop:beauty}
$C$ induces a total preorder on balanced proofs, independent of the
choice of decreasing beauty functional.  A cost-minimizer
$p^*=\arg\min C$ exists for any provable theorem (cf.\
Erd\H{o}s's ``Proof from the Book''~\cite{Erdos1998}).
\end{proposition}

\begin{conjecture}[G\"odel saddle]\label{conj:saddle}
If a sentence requires unbounded self-reference depth, both proof and
refutation have unbounded $\Jcost$.
\end{conjecture}

The self-reference cost $S(n)\ge n\Jbit=n\ln\phig$ diverges
(Archimedean property).  The contribution is the interpretation:
$\Jbit$ is uniquely fixed by \ref{ax:norm}--\ref{ax:dalembert},
giving a machine-independent analogue of
Chaitin's~\cite{Chaitin1987} algorithmic incompleteness.

% ══════════════════════════════════════════════════════════════
\section{Toward Formal Proof Theory}\label{sec:formal}
% ══════════════════════════════════════════════════════════════

\begin{definition}[Complexity ratio]\label{def:ratio}
For a complexity measure $\kappa:\text{Formulas}\to\mathbb{N}$,
\begin{equation}\label{eq:ratio}
  r(A\to B) \;:=\; \frac{\kappa(B)+1}{\kappa(A)+1}.
\end{equation}
\end{definition}

\begin{proposition}\label{prop:tautology}
$A=B\implies r=1$, $\Jcost=0$.\;
A cyclic derivation $A\to\cdots\to A$ has $\prod r_k=1$ (balanced).
\end{proposition}

\subsection*{Worked example: propositional resolution}

\begin{example}\label{ex:resolution}
$(A\lor B)\land(\lnot A\lor B)\implies B$ by resolution on $A$,
with $\kappa=$\,clause width.

\smallskip
\begin{center}
\begin{tabular}{@{}lcccc@{}}
\toprule
Step & From & To & $r$ & $\Jcost(r)$ \\
\midrule
1 & $A\lor B\;(\kappa{=}2)$ & $B\;(\kappa{=}1)$ & $2/3$ & $1/12$ \\
2 & $\lnot A\lor B\;(\kappa{=}2)$ & $B\;(\kappa{=}1)$ & $2/3$ & $1/12$ \\
\midrule
\multicolumn{3}{@{}l}{Balance to cycle: add steps $(3/2,3/2)$}
  & & \\
\bottomrule
\end{tabular}
\end{center}

\smallskip\noindent
Derivation: $\beta=2\ln(2/3)\approx{-0.81}\ne 0$ (not balanced).\\
Balanced cycle $p'=(2/3,\,2/3,\,3/2,\,3/2)$:\;
$\beta=0$,\; $C=4/12=1/3$.
\end{example}

\subsection*{Proof monoid generators}

\begin{proposition}[Generators]\label{prop:generators}
The balanced-proof monoid over the $\phig$-lattice is generated by
the pairs $(\phig^k,\phig^{-k})$, $k\ge 1$.  The cheapest generator
is $(\phig,\phig^{-1})$ with $C=\sqrt{5}-2$.
\end{proposition}

\begin{proof}
Any balanced sequence $\sum n_k=0$ decomposes into matched
$\pm$\nobreakdash-pairs.  The cost
$2\Jcost(\phig^k)=\phig^k+\phig^{-k}-2$ is minimized at $k=1$.
\end{proof}

\subsection*{Open problems}

\begin{enumerate}[label=\textup{(Q\arabic*)},nosep]
  \item Does $C$ correlate with Frege proof length or resolution width?
  \item Is $C$-minimality equivalent to a known proof optimality?
  \item Does the generator structure correspond to an algebraic
    invariant of proof systems?
\end{enumerate}

% ══════════════════════════════════════════════════════════════
\section{Axiom of Choice: A Cost Interpretation}\label{sec:choice}
% ══════════════════════════════════════════════════════════════

\begin{theorem}\label{thm:finite}
$\Jcost(x)<\infty$ for all $x>0$.
\end{theorem}

\begin{theorem}\label{thm:empty}
$\Jcost(x)\to+\infty$ as $x\to 0^+$.
\end{theorem}

\begin{remark}
We do not derive AC from weaker axioms.  In any cost landscape where
$\Jcost<\infty$ on $\Rpos$ and $\Jcost(0^+)=\infty$, selection from
nonempty sets is always finitely accessible.  In compact settings,
$\arg\min_A\Jcost$ gives a constructive choice.
\end{remark}

% ══════════════════════════════════════════════════════════════
\section{Zero-Cost Universal Reference}\label{sec:wigner}
% ══════════════════════════════════════════════════════════════

\begin{theorem}[Backbone]\label{thm:backbone}
For any $(P,\Jcost_P)$ with $\Jcost_P(o)>0$, there exists $(M,0)$
with $\Jcost_M(s)=0<\Jcost_P(o)$.
\end{theorem}

\begin{theorem}[Necessity]\label{thm:only_zero}
If $M$ is a universal referent (i.e., for every $(P,\Jcost_P)$ and
$o$ with $\Jcost_P(o)>0$ there exists $s\in M$ with
$\Jcost_M(s)<\Jcost_P(o)$), then $\inf_M\Jcost_M=0$.
\end{theorem}

\begin{proof}
If $\inf=c>0$, take $P=\{o\}$ with $\Jcost_P(o)=c/2$: contradiction.
\end{proof}

\begin{theorem}[Sufficiency]\label{thm:converse}
If $\inf_M\Jcost_M=0$, then $M$ is a universal referent.
\end{theorem}

\begin{proof}
Given $\varepsilon=\Jcost_P(o)>0$, choose $s\in M$ with
$\Jcost_M(s)<\varepsilon$.
\end{proof}

\begin{corollary}[Characterization]\label{cor:iff}
$M$ is a universal referent $\;\Longleftrightarrow\;$ $\inf_M\Jcost_M=0$.
\end{corollary}

\begin{corollary}[Existence]\label{cor:exist}
$\ker(\Jcost)=\{1\}$ satisfies $\inf\Jcost=0$.
\end{corollary}

\begin{remark}[Wigner]
\Cref{cor:iff} is analogous to Shannon's source coding
theorem~\cite{Shannon1948}: zero intrinsic cost $\Leftrightarrow$
optimal compression.  Wigner's puzzle~\cite{Wigner1960} becomes a
theorem about cost asymmetry.
\end{remark}

% ══════════════════════════════════════════════════════════════
\section{Summary}\label{sec:summary}
% ══════════════════════════════════════════════════════════════

\begin{theorem}[Main]\label{thm:main}
Any d'Alembert cost functional (\cref{def:axioms}) determines:
\begin{enumerate}[nosep]
  \item $F=\Jcost$ uniquely \hfill (\cref{thm:unique_cost});
  \item $\phig$ as the unique self-similar base
    \hfill (\cref{thm:self_similar});
  \item the metric $d=|m-n|\ln\phig$
    \hfill (\cref{prop:metric});
  \item $\ln$ as the unique balance invariant
    \hfill (\cref{thm:unique_balance});
  \item the zero-cost subspace as the unique universal referent
    \hfill (\cref{cor:iff}).
\end{enumerate}
The $\phig$-lattice proof monoid is generated by
$\{(\phig^k,\phig^{-k}):k\ge 1\}$ (\cref{prop:generators}).
\end{theorem}

\begin{remark}[Scope]
These are necessary \emph{infrastructure} for mathematical reasoning,
not a derivation of ZFC or second-order arithmetic.  The contribution
is that this infrastructure is uniquely forced by a single functional
equation, and that concrete open problems
(\cref{sec:formal}) connect it to proof complexity.
\end{remark}

% ══════════════════════════════════════════════════════════════
\section{Discussion}\label{sec:discussion}
% ══════════════════════════════════════════════════════════════

\subsection*{Claims and non-claims}
Three structures (ordered lattice, balance condition, zero-cost
reference) are uniquely forced.  We do \emph{not} derive ZFC, refute
G\"odel, or derive AC from weaker axioms.

\subsection*{Prior work}

\begin{center}
\small
\renewcommand{\arraystretch}{1.15}
\begin{tabular}{@{}>{\bfseries}l p{9cm}@{}}
\toprule
Reference & Relation to this work \\
\midrule
Acz\'el~\cite{Aczel1966}
  & Classical d'Alembert theory; we apply it to cost and derive
    mathematical structures. \\
Tegmark~\cite{Tegmark2008}
  & Postulates math is fundamental; we derive \emph{specific}
    structures from one equation. \\
Wheeler~\cite{Wheeler1990}
  & ``It from Bit''; we quantify the bit: $\Jbit=\ln\phig$. \\
Chaitin~\cite{Chaitin1987}
  & Algorithmic incompleteness; $\Jcost$ replaces Kolmogorov
    complexity with a \emph{unique} measure. \\
Shannon~\cite{Shannon1948}
  & Source coding; zero-cost reference $\approx$ optimal
    compression (structural analogy). \\
Cauchy
  & $f(x{+}y)=f(x){+}f(y)$; \cref{thm:unique_balance} is a
    direct application to $(\Rpos,\cdot)$. \\
\bottomrule
\end{tabular}
\end{center}

\subsection*{Open questions}

\begin{enumerate}[label=\textup{(Q\arabic*)},nosep,start=4]
  \item Completion of $(\mathbb{Z},d)$ to a $\phig$-adic space?
  \item Topos structure on balanced proofs?
  \item Constructive selection via $\Jcost$-minimization in
    non-compact settings?
\end{enumerate}

% ══════════════════════════════════════════════════════════════
\section{Conclusion}\label{sec:conclusion}
% ══════════════════════════════════════════════════════════════

We have derived three consequences of the d'Alembert cost equation.

\emph{Balance uniqueness} (\cref{thm:unique_balance}): log-balance is
the unique continuous additive closed-chain invariant, by Cauchy's
classical theorem.  The worked resolution example
(\cref{ex:resolution}) and monoid generators
(\cref{prop:generators}) give a first algebraic picture.

\emph{Lattice uniqueness} (\cref{thm:self_similar}): $\phig$ is the
unique base admitting a Fibonacci recursion.

\emph{Universality} (\cref{cor:iff}): a space is a universal referent
iff its infimal cost is zero---both necessary and sufficient.

The most promising direction is the connection to proof complexity
(\cref{sec:formal}): does $\Jcost$-cost correlate with standard
proof-length measures?

\bigskip

% ══════════════════════════════════════════════════════════════
\begin{thebibliography}{99}
\small

\bibitem{Aczel1966}
J.~Acz\'el,
\emph{Lectures on Functional Equations and Their Applications},
Academic Press, 1966.

\bibitem{AczelDhombres1989}
J.~Acz\'el and J.~Dhombres,
\emph{Functional Equations in Several Variables},
Cambridge Univ.\ Press, 1989.

\bibitem{Kannappan2009}
P.~Kannappan,
\emph{Functional Equations and Inequalities with Applications},
Springer, 2009.

\bibitem{CostFunctional2026}
J.~Washburn and M.~Zlatanovi\'c,
Uniqueness of the canonical reciprocal cost,
arXiv:2602.05753 [math.CA], 2026.

\bibitem{RecognitionGeometry2025}
J.~Washburn,
The algebra of reality: a recognition science derivation of
physical law,
\emph{Axioms} \textbf{15}(2):90, 2025.

\bibitem{Foundations2026}
J.~Washburn,
Coherent comparison as information cost: a cost-first ledger
framework for discrete dynamics, 2026.

\bibitem{DimensionalRigidity2026}
J.~Washburn,
Dimensional rigidity: $D{=}3$ from linking and gap-45
synchronization, 2026.

\bibitem{Aboutness2026}
J.~Washburn,
The algebra of aboutness: reference as cost-minimizing
compression,
\emph{Entropy} (submitted), 2026.

\bibitem{Wigner1960}
E.\,P.~Wigner,
The unreasonable effectiveness of mathematics in the natural
sciences,
\emph{Comm.\ Pure Appl.\ Math.} \textbf{13}:1--14, 1960.

\bibitem{Tegmark2008}
M.~Tegmark,
The mathematical universe,
\emph{Found.\ Phys.} \textbf{38}:101--150, 2008.

\bibitem{Wheeler1990}
J.\,A.~Wheeler,
Information, physics, quantum: the search for links,
in \emph{Complexity, Entropy, and the Physics of Information},
1990.

\bibitem{Shannon1948}
C.\,E.~Shannon,
A mathematical theory of communication,
\emph{Bell Syst.\ Tech.\ J.} \textbf{27}:379--423, 1948.

\bibitem{Chaitin1987}
G.\,J.~Chaitin,
\emph{Algorithmic Information Theory},
Cambridge Univ.\ Press, 1987.

\bibitem{Erdos1998}
M.~Aigner and G.\,M.~Ziegler,
\emph{Proofs from THE BOOK},
Springer, 1998.

\bibitem{Zeckendorf1972}
E.~Zeckendorf,
Repr\'esentation des nombres naturels par une somme de nombres
de Fibonacci ou de nombres de Lucas,
\emph{Bull.\ Soc.\ Roy.\ Sci.\ Li\`ege}
\textbf{41}:179--182, 1972.

\end{thebibliography}

\end{document}
