\documentclass[11pt,a4paper]{article}

\usepackage[margin=1in]{geometry}
\usepackage[T1]{fontenc}
\usepackage{lmodern}
\usepackage{microtype}
\usepackage{amsmath,amssymb,amsthm}
\usepackage{mathtools}
\usepackage{booktabs}
\usepackage{enumitem}
\usepackage{xcolor}
\usepackage[hidelinks]{hyperref}
\usepackage{tikz}
\usetikzlibrary{arrows.meta,positioning}

% Theorem environments
\newtheorem{theorem}{Theorem}[section]
\newtheorem{proposition}[theorem]{Proposition}
\newtheorem{lemma}[theorem]{Lemma}
\newtheorem{corollary}[theorem]{Corollary}
\newtheorem{definition}[theorem]{Definition}
\newtheorem{remark}[theorem]{Remark}

% Notation
\newcommand{\phig}{\varphi}
\newcommand{\Jcost}{J}
\newcommand{\Ecoh}{E_{\mathrm{coh}}}
\newcommand{\alphaInv}{\alpha^{-1}}
\newcommand{\fgap}{f_{\mathrm{gap}}}
\newcommand{\dkappa}{\delta_\kappa}
\newcommand{\weig}{w_8}
\newcommand{\RS}{Recognition Science}
\newcommand{\SM}{Standard Model}

% Claim tags
\newcommand{\PROVED}{\textcolor{blue!70!black}{\footnotesize\textsf{[PROVED]}}}
\newcommand{\HYP}{\textcolor{orange!80!black}{\footnotesize\textsf{[HYP]}}}
\newcommand{\CERT}{\textcolor{teal}{\footnotesize\textsf{[CERT]}}}
\newcommand{\VAL}{\textcolor{purple!70!black}{\footnotesize\textsf{[VAL]}}}

\title{\textbf{The Fine-Structure Constant from Cube Geometry\\
and the Eight-Tick Projection}\\[0.5em]
\large Paper V of V: Derivation of $\alphaInv$ in Recognition Science}
\author{Jonathan Washburn\\
\small Recognition Science Research Institute, Austin, Texas\\
\small \texttt{washburn.jonathan@gmail.com}}
\date{\today}

\begin{document}
\maketitle

\begin{abstract}
The fine-structure constant $\alpha\approx 1/137$ is one of the most precisely
measured quantities in physics, yet its value has resisted theoretical
derivation from first principles for nearly a century.  This paper presents a
closed-form derivation of $\alphaInv$ within \RS{} (RS), using only three
ingredients traced to the discrete geometry of the 3-cube and the eight-tick
closure cycle:
\begin{enumerate}[nosep]
  \item A \textbf{geometric seed} $4\pi\cdot 11 = 44\pi\approx 138.230$,
        where $11=E_{\mathrm{passive}}$ is the passive edge count of the 3-cube
        and $4\pi$ is the solid angle of the full sphere.
  \item A \textbf{gap weight} $\fgap=\weig\cdot\ln\phig\approx 1.199$,
        where $\weig\approx 2.491$ is a parameter-free closed form derived from
        the DFT-8 projection of $\phig$-weighted modes on the eight-tick cycle.
  \item A \textbf{curvature correction}
        $\dkappa=-103/(102\pi^5)\approx -0.00331$,
        where $102=6\times 17$ (faces $\times$ wallpaper groups) and
        $103=102+1$ (Euler closure).
\end{enumerate}

The resulting prediction is: \HYP{}
\begin{equation*}
  \alphaInv = 4\pi\cdot 11 - \weig\cdot\ln\phig + \frac{103}{102\pi^5}
  \approx 137.0349.
\end{equation*}
This corresponds to the Thomson limit (IR baseline at $Q^2=0$) and differs
from the CODATA value $\alphaInv_{\mathrm{CODATA}}=137.035\,999\,206(11)$
by $\sim 8.1\,\mathrm{ppm}$---consistent with higher-order QED corrections
not included in the zero-loop structural prediction.

All integers in the formula ($11$, $102$, $103$) are derived from the same
counting layer used for particle masses (Papers~I--IV).
The gap weight $\weig$ has a parameter-free closed form whose equality with
the DFT-8 projection is fully proved in Lean~4 (the \texttt{w8\_projection\_equality}
theorem, with zero \texttt{sorry}).

No adjustable parameters appear at any stage.
\end{abstract}

\tableofcontents
\newpage

%=============================================================================
\section{Introduction}
%=============================================================================

\subsection{The fine-structure constant: status quo}

The fine-structure constant
\begin{equation}
  \alpha = \frac{e^2}{4\pi\epsilon_0\hbar c} \approx \frac{1}{137.036}
\end{equation}
governs the strength of electromagnetic interactions.  It is measured to
12 significant digits by combining the electron $g-2$ anomalous magnetic moment
with QED calculations at tenth order.  Yet no theoretical framework has
derived its value from first principles.  The SM treats $\alpha$ as an
input---a free parameter.

Attempts at deriving $\alpha$ from pure mathematics have a long and mostly
discredited history (Eddington's numerology, Wyler's formula, etc.).  What
distinguishes the present work is that $\alphaInv$ arises from the \emph{same}
discrete geometry (the 3-cube, the eight-tick closure, the counting layer)
that organizes the entire particle mass spectrum.  It is not an isolated
formula; it is part of a coherent framework with machine-verified foundations.

\subsection{What this paper derives}

We derive $\alphaInv$ as a sum of three terms, each with explicit combinatorial
origin in cube geometry.  The derivation yields the Thomson-limit (zero-momentum-transfer)
value.  The difference from the CODATA value ($\sim 8\,\mathrm{ppm}$) is
attributed to higher-order QED running effects between $Q^2=0$ and the
measurement scale, which are not part of the zero-parameter structural prediction.


%=============================================================================
\section{The Three Components}
\label{sec:components}
%=============================================================================

\subsection{Component 1: The geometric seed $4\pi\cdot E_{\mathrm{passive}}$}

The dominant term in $\alphaInv$ is: \HYP{}
\begin{equation}
  \alphaInv_{\mathrm{seed}} = 4\pi\cdot E_{\mathrm{passive}} = 4\pi\cdot 11 = 44\pi \approx 138.2300.
  \label{eq:seed}
\end{equation}

The factor $4\pi$ is the solid angle subtended by the full sphere---the
geometric weight of an isotropic recognition event in three dimensions.
The factor $E_{\mathrm{passive}}=11$ counts the passive edges of the 3-cube
(total edges $E=12$ minus one active edge $A=1$ per atomic tick).

\textbf{Interpretation}: the electromagnetic coupling strength at the Thomson
limit is set by the number of passive geometric channels available for photon
propagation on the cubic ledger, weighted by the isotropic solid angle.

\subsection{Component 2: The gap weight $\fgap=\weig\cdot\ln\phig$}

The gap weight subtracts a $\phig$-dependent correction: \HYP{}
\begin{equation}
  \fgap = \weig\cdot\ln\phig \approx 2.49057\times 0.48121 \approx 1.1985.
  \label{eq:gap}
\end{equation}

The weight $\weig$ is the central technical achievement of this paper.
It arises from the DFT-8 (discrete Fourier transform on the eight-tick cycle)
projection of $\phig$-weighted modes.

\subsubsection{Definition of $\weig$}

Consider the eight-tick cycle as a discrete signal processing domain.
The $\phig$-ladder generates a characteristic spectrum when sampled at
eight points.  The DFT-8 of this spectrum yields mode amplitudes, and the
projection onto the ledger-compatible subspace (modes satisfying the
eight-tick neutrality constraint) produces a scalar weight.

The parameter-free closed form is: \PROVED{}
\begin{equation}
  \weig = \frac{246 + 145\sqrt{2} - 102\sqrt{5} - 65\sqrt{10}}{7}
  \approx 2.49056927545.
  \label{eq:w8_closed}
\end{equation}

This is not a fitted number.  It is a specific algebraic expression in
$\sqrt{2}$, $\sqrt{5}$, and $\sqrt{10}$, derived from the DFT-8 projection
operator applied to $\phig$-weighted modes.

\subsubsection{The DFT-8 projection mechanism}

The eight-tick cycle has period 8, so its natural frequency basis is the
DFT-8.  For mode $k\in\{0,1,\ldots,7\}$, the basis function is
$\omega_k(t)=e^{-2\pi i k t/8}$ for $t\in\{0,1,\ldots,7\}$.

The $\phig$-ladder evaluates $\phig^{t}$ at each tick $t\in\{0,\ldots,7\}$.
The DFT-8 amplitude at mode $k$ is:
\begin{equation}
  \hat{f}(k) = \frac{1}{\sqrt{8}}\sum_{t=0}^{7}\phig^t\,e^{-2\pi i k t/8}.
\end{equation}

The eight-tick neutrality constraint ($\sum_{t=0}^{7}\delta_t=0$) removes the
$k=0$ mode.  The discrete-derivative spectrum (the ``stiffness'' of each mode)
is weighted by $\sin^2(\pi k/8)$.  The gap weight $\weig$ is the normalized
projection of $|\hat{f}(k)|^2$ onto this stiffness spectrum:
\begin{equation}
  \weig = \frac{64\sum_{k=1}^{7}|\hat{f}(k)|^2\,\sin^2(\pi k/8)}
  {\sum_{k=1}^{7}|\hat{f}(k)|^2}.
  \label{eq:w8_projection}
\end{equation}

\subsubsection{The projection equality theorem}

\begin{theorem}[Gap weight projection equality]
\label{thm:w8}
The DFT-8 projection formula~\eqref{eq:w8_projection} equals the
closed-form expression~\eqref{eq:w8_closed}:
\begin{equation}
  w_{8,\mathrm{projected}} = w_{8,\mathrm{closed}}.
\end{equation}
\end{theorem}

\noindent\textbf{Lean module}:
\texttt{IndisputableMonolith.Constants.GapWeight.ProjectionEquality.w8\_projection\_equality}
(zero \texttt{sorry}).

The proof proceeds via:
\begin{enumerate}[nosep]
  \item Parseval's theorem for DFT-8 (time-domain energy equals frequency-domain energy),
  \item Closed-form evaluation of $|\omega^k\cdot\phig-1|^2$ via the identity
        $\phig^2-2\phig\cos(k\pi/4)+1$,
  \item Geometric series summation for the $\phig$-DFT amplitudes,
  \item A polynomial identity verified by norm\_num.
\end{enumerate}

\subsection{Component 3: The curvature correction $\dkappa$}

The third term is a small curvature correction: \HYP{}
\begin{equation}
  \dkappa = -\frac{103}{102\pi^5} \approx -0.00331.
  \label{eq:delta_kappa}
\end{equation}

The integers have combinatorial origins:
\begin{itemize}[nosep]
  \item $102 = F\times W = 6\times 17$: the product of the cube face count
        and the wallpaper group count.  This represents the total number of
        distinct ``curvature channels'' on the ledger (each face paired with
        each planar symmetry group).
  \item $103 = 102+1$: an Euler closure correction (adding the identity
        element to the set of curvature channels).
  \item $\pi^5$: the volume factor of a 5-dimensional configuration space
        (the five independent components of the Ricci-flat condition in $D=3$
        spatial dimensions).
\end{itemize}


%=============================================================================
\section{Assembly: The Complete Formula}
\label{sec:assembly}
%=============================================================================

Combining the three components: \HYP{}
\begin{equation}
\boxed{
  \alphaInv = 4\pi\cdot 11 - \weig\cdot\ln\phig + \frac{103}{102\pi^5}.
}
  \label{eq:alpha_full}
\end{equation}

Evaluating numerically: \CERT{}
\begin{align}
  4\pi\cdot 11 &\approx 138.23007676 \\
  -\weig\cdot\ln\phig &\approx -1.19849 \\
  +103/(102\pi^5) &\approx +0.00331 \\
  \hline
  \alphaInv_{\mathrm{RS}} &\approx 137.03489.
\end{align}

For comparison: \VAL{}
\begin{equation}
  \alphaInv_{\mathrm{CODATA}} = 137.035\,999\,206(11).
\end{equation}

The difference:
\begin{equation}
  \frac{\alphaInv_{\mathrm{RS}}-\alphaInv_{\mathrm{CODATA}}}{\alphaInv_{\mathrm{CODATA}}}
  \approx -8.1\times 10^{-6} = -8.1\,\mathrm{ppm}.
  \label{eq:alpha_diff}
\end{equation}

\subsection{Interpretation of the discrepancy}

The RS derivation yields the Thomson-limit value---the fine-structure constant
at zero momentum transfer ($Q^2=0$), before QED vacuum polarization corrections.
The CODATA value is measured at finite $Q^2$ and incorporates higher-order QED
running effects (electron loops, muon loops, hadronic vacuum polarization, etc.).

The $8.1\,\mathrm{ppm}$ discrepancy is \emph{expected} as the difference between
the zero-loop structural prediction and the experimentally accessed regime.
Closing this gap would require computing the full QED vacuum polarization
corrections within the RS framework---a task for future work.


%=============================================================================
\section{Integer Provenance: Everything from the Counting Layer}
\label{sec:provenance}
%=============================================================================

Every integer in the $\alphaInv$ formula traces to the same counting layer
used for particle masses:

\begin{center}
\begin{tabular}{lrl}
\toprule
Integer & Value & Origin \\
\midrule
$E_{\mathrm{passive}}$ & 11 & $E_{\mathrm{total}}-A=12-1$ \\
$E_{\mathrm{total}}$ & 12 & Edges of 3-cube \\
$F$ & 6 & Faces of 3-cube \\
$W$ & 17 & 2D wallpaper groups \\
$102$ & $6\times 17$ & $F\times W$ \\
$103$ & $102+1$ & Euler closure \\
8 & $2^3$ & Vertices of 3-cube / eight-tick period \\
\bottomrule
\end{tabular}
\end{center}

The only non-integer inputs are:
\begin{itemize}[nosep]
  \item $\pi$ (geometric constant---the ratio of circumference to diameter),
  \item $\phig=(1+\sqrt{5})/2$ (forced by cost self-similarity, T6),
  \item $\sqrt{2},\sqrt{5},\sqrt{10}$ (appear in $\weig$ via the DFT-8
        projection; these are algebraic consequences of $\phig$ and the
        eight-tick geometry).
\end{itemize}

No measured quantity enters the derivation.  No parameter is tuned.


%=============================================================================
\section{Connection to the Mass Framework}
\label{sec:connection}
%=============================================================================

\subsection{$\alpha$ as a shared constant}

The fine-structure constant derived here is the \emph{same} $\alpha$ that
appears in the mass framework:
\begin{itemize}[nosep]
  \item In the lepton mass chain (Paper~II): the electron break $\delta_e$
        and generation steps $S_{e\to\mu}$, $S_{\mu\to\tau}$ contain
        $\alpha$-dependent corrections.
  \item In the CKM mixing predictions: $|V_{ub}|=\alpha/2$ and
        $|V_{us}|=\phig^{-3}-\tfrac{3}{2}\alpha$.
  \item In the PMNS predictions: $\sin^2\theta_{12}=\phig^{-2}-10\alpha$
        and $\sin^2\theta_{23}=\tfrac{1}{2}+6\alpha$.
\end{itemize}

The integer coefficients multiplying $\alpha$ in all these formulas are
cube-derived: $E_{\mathrm{passive}}=11$, $F=6$, $E-2=10$, $F/4=3/2$.
This unification---the same counting layer producing both $\alpha$ itself
and the coefficients of $\alpha$ in mass and mixing formulas---is a
non-trivial consistency check.

\subsection{The strong coupling $\alpha_s$}

For completeness, the RS framework also predicts the strong coupling
constant at the $Z$-mass scale: \HYP{}
\begin{equation}
  \alpha_s(M_Z) = \frac{2}{W} = \frac{2}{17} \approx 0.11765.
\end{equation}
PDG: $\alpha_s(M_Z)=0.1179\pm 0.0009$. \VAL{}

The contrast is revealing: $\alpha_{\mathrm{EM}}$ arises from
\emph{edge geometry} ($4\pi\cdot E_{\mathrm{passive}}$), while $\alpha_s$
arises from \emph{face symmetries} ($2/W$).  Both are cube-derived; the
different geometric channels (edges vs.\ faces) produce different coupling
strengths.


%=============================================================================
\section{Falsifiers}
\label{sec:falsifiers}
%=============================================================================

\begin{enumerate}
  \item \textbf{Formula structure}: if a simpler formula (fewer terms, different
        integers) reproduces $\alphaInv$ to comparable or better precision from
        the same inputs ($\phig$, cube counts), the present derivation is
        superseded.

  \item \textbf{$\weig$ value}: the closed-form $\weig$ is algebraically exact.
        Any numerical discrepancy between the Lean-certified value and an
        independent computation would indicate a proof error (not a physics
        error).

  \item \textbf{Thomson-limit interpretation}: if the RS prediction is shown to
        correspond to a regime other than $Q^2=0$ (e.g., a finite-$Q^2$ scale),
        the interpretation changes and the discrepancy analysis must be revised.

  \item \textbf{Integer provenance}: if the integers $11$, $102$, $103$ are
        shown to have alternative combinatorial origins that fit equally well,
        the uniqueness of the cube-geometry derivation is weakened.

  \item \textbf{Precision improvement}: future work could compute the full
        QED vacuum polarization corrections within RS.  If the resulting
        prediction moves \emph{away} from CODATA (rather than toward it),
        the structural derivation is in tension.
\end{enumerate}


%=============================================================================
\section{Conclusions}
\label{sec:conclusions}
%=============================================================================

This paper has derived the fine-structure constant $\alphaInv$ from first
principles within \RS{}, using three components---a geometric seed, a
DFT-8 gap weight, and a curvature correction---all traced to the same
counting layer ($E\!=\!12$, $E_p\!=\!11$, $F\!=\!6$, $W\!=\!17$) that
organizes particle masses.

The key results are:
\begin{enumerate}[nosep]
  \item $\alphaInv = 4\pi\cdot 11 - \weig\cdot\ln\phig + 103/(102\pi^5)
        \approx 137.035$, with no adjustable parameters.

  \item The gap weight $\weig=(246+145\sqrt{2}-102\sqrt{5}-65\sqrt{10})/7$
        is fully proved in Lean~4 via the DFT-8 projection equality theorem
        (zero \texttt{sorry}).

  \item The Thomson-limit prediction differs from CODATA by $8.1\,\mathrm{ppm}$,
        consistent with the expected effect of higher-order QED vacuum polarization.

  \item The same $\alpha$ appears in all mass and mixing predictions (Papers~II--III),
        with cube-derived integer coefficients, providing a non-trivial
        cross-consistency check.

  \item The strong coupling $\alpha_s(M_Z)=2/17\approx 0.1177$ is simultaneously
        predicted from face symmetries of the same 3-cube, in agreement with PDG.
\end{enumerate}

Together with Papers~I--IV, this completes the RS particle mass program: a
zero-parameter framework that derives the electromagnetic and strong coupling
constants, organizes all nine charged fermion masses, predicts CKM and PMNS
mixing, and extends to the neutrino sector---all from a single cost functional
and the discrete geometry of the 3-cube.

\begin{thebibliography}{99}
\bibitem{PDG2024} R.~L.~Workman \textit{et al.} [Particle Data Group],
  Prog.\ Theor.\ Exp.\ Phys.\ \textbf{2022}, 083C01 (2022) and 2024 update.
\bibitem{CODATA} E.~Tiesinga \textit{et al.},
  ``CODATA recommended values of the fundamental physical constants: 2018,''
  Rev.\ Mod.\ Phys.\ \textbf{93}, 025010 (2021) and 2022 update.
\bibitem{Schwinger1948} J.~Schwinger,
  ``On quantum-electrodynamics and the magnetic moment of the electron,''
  Phys.\ Rev.\ \textbf{73}, 416 (1948).
\bibitem{Washburn2025} J.~Washburn,
  ``The Algebra of Reality: A Recognition Science Derivation of Physical Law,''
  \textit{Axioms} \textbf{15}(2), 90 (2025).
\bibitem{PaperI} J.~Washburn, Paper~I of this series.
\bibitem{PaperII} J.~Washburn, Paper~II of this series.
\bibitem{PaperIII} J.~Washburn, Paper~III of this series.
\bibitem{PaperIV} J.~Washburn, Paper~IV of this series.
\end{thebibliography}

\end{document}
