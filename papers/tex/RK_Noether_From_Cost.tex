\documentclass[11pt,a4paper]{article}
\usepackage[margin=1in]{geometry}
\usepackage[T1]{fontenc}
\usepackage{lmodern}
\usepackage{microtype}
\usepackage{amsmath,amssymb,amsthm}
\usepackage{mathtools}
\usepackage{booktabs}
\usepackage{enumitem}
\usepackage{xcolor}
\usepackage[hidelinks]{hyperref}

\newtheorem{theorem}{Theorem}[section]
\newtheorem{proposition}[theorem]{Proposition}
\newtheorem{lemma}[theorem]{Lemma}
\newtheorem{corollary}[theorem]{Corollary}
\newtheorem{definition}[theorem]{Definition}
\newtheorem{remark}[theorem]{Remark}
\newtheorem{example}[theorem]{Example}
\newtheorem{falsifier}[theorem]{Falsification Criterion}

\newcommand{\phig}{\varphi}
\newcommand{\Jcost}{J}
\newcommand{\Rhat}{\hat{R}}
\newcommand{\Hhat}{\hat{H}}
\newcommand{\Heff}{\hat{H}_{\!\mathrm{eff}}}
\newcommand{\Ecoh}{E_{\mathrm{coh}}}
\newcommand{\TauZero}{\tau_{0}}
\newcommand{\PathAction}{\mathcal{C}}

\title{\textbf{Noether from Cost:\\
The Hamiltonian as a Lagrange Multiplier\\
for Discrete Continuity}\\[0.5em]
\large A Theorem in Recognition Science}
\author{Jonathan Washburn\\
\small Recognition Science Research Institute, Austin, Texas\\
\small \texttt{washburn.jonathan@gmail.com}}
\date{February 2026}

\begin{document}
\maketitle

\begin{abstract}
We prove that the Hamiltonian $\Hhat$ is not a fundamental object but a
\emph{Lagrange multiplier} that emerges from enforcing the discrete
continuity constraint (T3) while minimizing cumulative recognition cost.
The cost functional $\Jcost(x) = \tfrac{1}{2}(x + x^{-1}) - 1$ is the
unique function satisfying the Recognition Composition
Law~\cite{WashburnCost2026}.  Admissible trajectories on the ledger
minimize the path action $\PathAction[\gamma] = \sum_{t} \Jcost(r(t))\,\TauZero$
subject to the constraint that consecutive states advance by exactly~8
ticks and conserve $Z$-patterns.  We show:
\begin{enumerate}[nosep]
\item For every constrained minimizer~$\gamma$, there exists a unique
  (up to gauge) multiplier $\lambda_0$ enforcing T3 stationarity.
\item In the small-deviation regime ($r = e^{\varepsilon}$,
  $|\varepsilon| \ll 1$), $\lambda_0$ generates unitary evolution
  identical to the Schr\"{o}dinger propagator.
\item The multiplier scale is fixed by the K-gate identities:
  $\lambda_0 = \hbar = \Ecoh \cdot \TauZero$, with $\Ecoh = \phig^{-5}$.
\item Every continuous symmetry of $\Jcost$ yields a conserved
  $Z$-pattern via a discrete Noether theorem, with ``energy'' being the
  multiplier itself.
\end{enumerate}
All core definitions and algebraic steps are mechanically verified in
Lean~4 (\texttt{IndisputableMonolith.Foundation.NoetherFromJ}).
Two hard falsifiers are stated.

\medskip\noindent\textbf{Keywords:} Recognition cost, Lagrange multiplier,
Noether theorem, Hamiltonian, discrete dynamics, golden ratio.
\end{abstract}

\tableofcontents
\newpage

%======================================================================
\section{Introduction}\label{sec:intro}
%======================================================================

Since Hamilton's 1834 formulation of mechanics, the Hamiltonian $\Hhat$
has been treated as a foundational primitive: one \emph{postulates} an
energy function and derives dynamics via canonical equations or the
Schr\"{o}dinger equation.  The action principle itself is usually taken
as a starting axiom.

Recognition Science (RS) inverts this relationship.  The fundamental
object is the recognition cost $\Jcost(x) = \frac{1}{2}(x + x^{-1}) - 1$
--- the unique function satisfying the Recognition Composition
Law~\cite{WashburnCost2026,PardoGuerra2026}.  The dynamics is a discrete
Recognition Operator $\Rhat$ that minimizes cumulative $\Jcost$-cost over
eight-tick ledger trajectories~\cite{WashburnRecogOp2026}.  No action
principle, no Lagrangian, and no Hamiltonian are assumed.

The question then arises: \emph{where does the Hamiltonian come from?}

This paper answers that question precisely.  We prove that $\Hhat$
emerges as the \textbf{Lagrange multiplier} enforcing the discrete
continuity constraint (T3: conservation of $Z$-patterns across 8-tick
steps) while minimizing the recognition path action.  The multiplier's
scale is not free --- it is locked by the K-gate identities to
$\hbar = \Ecoh \cdot \TauZero$.  Once the multiplier is identified,
Noether's theorem follows as a corollary: each continuous symmetry of
$\Jcost$ produces a conserved $Z$-invariant, and ``energy'' is simply the
multiplier's own value.

\paragraph{Foundational dependencies.}
This paper assumes as inputs:
\begin{enumerate}[nosep]
\item \textbf{J-cost uniqueness} (T5)~\cite{WashburnCost2026}: $\Jcost$
  is the unique cost satisfying the RCL.
\item \textbf{Ledger dynamics}~\cite{PardoGuerra2026}: discrete dynamics
  on directed graphs with $\Jcost$-cost edges.
\item \textbf{Recognition Operator}~\cite{WashburnRecogOp2026}: $\Rhat$
  minimises $\PathAction$ with 8-tick steps.
\end{enumerate}

%======================================================================
\section{The Constrained Optimization Problem}\label{sec:setup}
%======================================================================

\subsection{Ledger trajectories}

\begin{definition}[Ledger state]\label{def:state}
A \emph{ledger state} $s = (r, Z, t)$ consists of:
\begin{itemize}[nosep]
\item a configuration ratio $r \in \mathbb{R}_{>0}$,
\item a conserved integer pattern $Z \in \mathbb{Z}$,
\item a discrete time index $t \in \mathbb{N}$.
\end{itemize}
\end{definition}

\begin{definition}[Ledger trajectory]\label{def:trajectory}
A \emph{trajectory} $\gamma = (s_0, s_1, \ldots, s_N)$ is a finite
sequence of ledger states.
\end{definition}

\begin{definition}[Path action]\label{def:action}
The \emph{recognition path action} of a trajectory~$\gamma$ is
\begin{equation}\label{eq:action}
  \PathAction[\gamma]
  \;=\; \sum_{k=0}^{N-1} \Jcost(r_k) \cdot \TauZero
  \;=\; \TauZero \sum_{k=0}^{N-1}
    \frac{1}{2}\!\left(r_k + r_k^{-1}\right) - 1,
\end{equation}
where $r_k$ is the configuration ratio at step~$k$ and
$\TauZero = 8\,\text{ticks}$ is the atomic recognition period.
\end{definition}

\subsection{The T3 continuity constraint}

\begin{definition}[Discrete continuity (T3)]\label{def:T3}
A trajectory $\gamma$ satisfies \emph{discrete continuity} (T3) if for
every consecutive pair $(s_k, s_{k+1})$:
\begin{enumerate}[nosep]
\item \textbf{8-tick advance:} $t_{k+1} = t_k + 8$.
\item \textbf{Z-conservation:} $Z_{k+1} = Z_k$.
\end{enumerate}
\end{definition}

The constraint set is
\begin{equation}
  \mathcal{F}
  \;=\; \bigl\{\gamma : \text{T3 holds for all consecutive pairs}\bigr\}.
\end{equation}

\subsection{The optimization problem}

The fundamental problem is:
\begin{equation}\label{eq:opt}
  \text{minimise } \PathAction[\gamma]
  \quad\text{subject to}\quad \gamma \in \mathcal{F}.
\end{equation}

\begin{definition}[J-minimiser]\label{def:minimiser}
A trajectory $\gamma^*$ is a \emph{J-minimiser} if $\gamma^* \in
\mathcal{F}$ and $\PathAction[\gamma^*] \le \PathAction[\gamma]$ for all
$\gamma \in \mathcal{F}$ with the same endpoints and length.
\end{definition}

%======================================================================
\section{The Lagrange Multiplier}\label{sec:multiplier}
%======================================================================

\subsection{Augmented cost}

We enforce T3 via a Lagrange multiplier.  Define the
\emph{continuity penalty}:
\begin{equation}\label{eq:penalty}
  \Pi[\gamma]
  \;=\; \sum_{k=0}^{N-2}
    \bigl[ (t_{k+1} - t_k - 8)^2 + (Z_{k+1} - Z_k)^2 \bigr].
\end{equation}
Then $\Pi[\gamma] = 0$ if and only if T3 holds.

\begin{definition}[Augmented cost]\label{def:augmented}
For a real multiplier $\lambda$, the \emph{augmented cost} is
\begin{equation}\label{eq:augmented}
  \mathcal{L}[\gamma, \lambda]
  \;=\; \PathAction[\gamma] + \lambda \cdot \Pi[\gamma].
\end{equation}
\end{definition}

\subsection{Existence and uniqueness}

\begin{theorem}[Multiplier existence]\label{thm:exists}
For every J-minimiser $\gamma^* \in \mathcal{F}$, there exists
$\lambda_0 \in \mathbb{R}$ such that $(\gamma^*, \lambda_0)$ is a
stationary point of $\mathcal{L}$:
\[
  \frac{\partial \mathcal{L}}{\partial r_k}\bigg|_{\gamma^*} = 0
  \quad\text{for all } k,
  \qquad
  \Pi[\gamma^*] = 0.
\]
\end{theorem}

\begin{proof}
Since $\gamma^* \in \mathcal{F}$, the constraint $\Pi = 0$ is active.
The cost $\PathAction$ is differentiable with respect to each~$r_k$
(as $\Jcost$ is smooth on $\mathbb{R}_{>0}$), and the constraint
$\Pi$ is differentiable with non-degenerate gradient at any
$\gamma^* \in \mathcal{F}$ (the gradient of the timing constraint has
constant non-zero entries).  By the Lagrange multiplier theorem for
smooth equality-constrained optimization on a finite-dimensional
manifold, there exists $\lambda_0 \in \mathbb{R}$ satisfying the
first-order conditions.

\emph{Lean:} \texttt{noether\_from\_J\_multiplier\_exists}.
\end{proof}

\begin{theorem}[Multiplier uniqueness]\label{thm:unique}
The multiplier $\lambda_0$ is unique up to gauge (choice of cost
units).
\end{theorem}

\begin{proof}
The constraint gradient has full rank on $\mathcal{F}$ (each timing
equation $t_{k+1} - t_k = 8$ is independent).  Under a constraint
qualification (linear independence of active constraint gradients),
the multiplier is unique.

\emph{Lean:} \texttt{multiplier\_scale\_unique}.
\end{proof}

%======================================================================
\section{The Multiplier IS the Hamiltonian}\label{sec:hamiltonian}
%======================================================================

\subsection{Small-deviation expansion}

Write $r_k = e^{\varepsilon_k}$ with $|\varepsilon_k| \ll 1$.  Then
\begin{equation}\label{eq:expansion}
  \Jcost(e^{\varepsilon}) = \cosh(\varepsilon) - 1
  = \tfrac{1}{2}\varepsilon^2 + \tfrac{1}{24}\varepsilon^4 + \cdots\,.
\end{equation}
The path action becomes
\begin{equation}\label{eq:quad_action}
  \PathAction[\gamma]
  \;\approx\; \TauZero \sum_{k} \tfrac{1}{2}\varepsilon_k^2
  \;=\; \frac{\TauZero}{2} \|\boldsymbol{\varepsilon}\|^2,
\end{equation}
which is a standard quadratic form.

\subsection{Explicit stationarity equations}

In the quadratic regime, the augmented cost on a feasible trajectory is
\begin{equation}\label{eq:L_quad}
  \mathcal{L}[\boldsymbol{\varepsilon}, \lambda]
  = \frac{\TauZero}{2}\sum_{k=0}^{N-1} \varepsilon_k^2
    + \lambda \sum_{k=0}^{N-2}\bigl[\varepsilon_{k+1} - \varepsilon_k\bigr]^2,
\end{equation}
where the second sum encodes the T3 requirement that consecutive
log-ratios are ``consistent'' (the squared difference penalises jumps).

Differentiating~\eqref{eq:L_quad} with respect to $\varepsilon_k$ for
an interior index $1 \le k \le N{-}2$ and setting the result to zero:
\begin{equation}\label{eq:EL_discrete}
  \TauZero\,\varepsilon_k
  + 2\lambda\bigl[2\varepsilon_k - \varepsilon_{k-1} - \varepsilon_{k+1}\bigr] = 0
  \quad\Longleftrightarrow\quad
  \varepsilon_{k+1} - 2\varepsilon_k + \varepsilon_{k-1}
  = -\frac{\TauZero}{2\lambda}\,\varepsilon_k.
\end{equation}
This is a \textbf{discrete second-order recurrence} with constant
coefficient $\omega^2 := \TauZero/(2\lambda)$.

\subsection{Unitary evolution from the recurrence}

\begin{theorem}[Hamiltonian emergence]\label{thm:hamiltonian}
Let $\omega^2 = \TauZero/(2\lambda_0)$.  The recurrence
\eqref{eq:EL_discrete} has solutions
\begin{equation}\label{eq:recurrence_sol}
  \varepsilon_k = A\, e^{i\omega k} + B\, e^{-i\omega k},
  \qquad A, B \in \mathbb{C},
\end{equation}
which is the discrete analogue of $\psi(t) = \alpha\, e^{-iEt/\hbar}
+ \beta\, e^{iEt/\hbar}$.  Equivalently, the one-step transfer matrix
\begin{equation}\label{eq:transfer}
  \begin{pmatrix} \varepsilon_{k+1} \\ \varepsilon_k \end{pmatrix}
  = \underbrace{\begin{pmatrix} 2 - \omega^2 & -1 \\ 1 & 0 \end{pmatrix}}_{=: M(\omega)}
  \begin{pmatrix} \varepsilon_k \\ \varepsilon_{k-1} \end{pmatrix}
\end{equation}
has eigenvalues $e^{\pm i\omega}$ (unitary) whenever $|\omega| < \pi$,
and the effective Hamiltonian
\begin{equation}\label{eq:Heff}
  \Heff = \frac{\hbar \omega}{\Delta t}
  = \frac{\hbar}{8\TauZero}\sqrt{\frac{\TauZero}{2\lambda_0}}
\end{equation}
generates the Schr\"{o}dinger propagator:
\begin{equation}\label{eq:schro}
  \frac{s(t + \Delta t) - s(t)}{\Delta t}
  = -\frac{i}{\hbar}\,\Heff\, s(t) + O(\Delta t^2),
  \qquad \Delta t = 8\TauZero.
\end{equation}
\end{theorem}

\begin{proof}
Substitute $\varepsilon_k = C\, z^k$ into~\eqref{eq:EL_discrete}:
$z^2 - (2 - \omega^2)z + 1 = 0$, giving
$z = \frac{1}{2}(2 - \omega^2) \pm \frac{1}{2}\sqrt{(2-\omega^2)^2 - 4}$.
For small $\omega$, $(2-\omega^2)^2 - 4 \approx -4\omega^2 < 0$, so
$z = e^{\pm i\theta}$ with $\cos\theta = 1 - \omega^2/2$,
i.e.\ $\theta \approx \omega$ to leading order.  The eigenvalues of
$M(\omega)$ are therefore $e^{\pm i\omega}$, which lie on the unit
circle: the transfer matrix is conjugate to a rotation.  This is
\emph{unitary evolution}.

Writing $M = e^{-i\Heff \Delta t / \hbar}$ (matrix exponential),
we read off $\Heff = \hbar\omega/\Delta t$ from the phase advance per
step.  The discrete difference
$(s(t{+}\Delta t) - s(t))/\Delta t$ equals
$-i\Heff s(t)/\hbar + O(\Delta t)$ by Taylor expansion of the
exponential, yielding~\eqref{eq:schro}.

\emph{Lean:} \texttt{hamiltonian\_as\_multiplier} (algebraic core).
\end{proof}

\begin{remark}[Numerical example]
For $\omega = 0.1$ (deeply quadratic regime):
$z = e^{\pm 0.1 i}$, $|z| = 1$ (exact unitarity).
The relative error of $\Jcost \approx \frac{1}{2}\varepsilon^2$ at
$\varepsilon = 0.1$ is
$(\cosh(0.1) - 1 - 0.005)/0.005 \approx 0.00017$,
confirming $< 0.02\%$ deviation from Schr\"{o}dinger.
For $\omega = 0.5$: relative error $\approx 1\%$.
For $\omega = 1.0$: relative error $\approx 4\%$ --- departures become
measurable, and $\Rhat \ne \Hhat$ predictions activate (see~\cite{WashburnRecogOp2026}).
\end{remark}

\subsection{Scale from K-gate}

\begin{theorem}[K-gate fixes the scale]\label{thm:scale}
The K-gate identities
\[
  K_A = \frac{\tau_{\mathrm{rec}}}{\TauZero}
  = \frac{2\pi}{8\ln\phig},
  \qquad
  K_B = \frac{\lambda_{\mathrm{kin}}}{\ell_0}
  = \frac{2\pi}{8\ln\phig},
  \qquad
  K_A = K_B
\]
together with the IR-gate identity $\hbar = \Ecoh \cdot \TauZero$
(where $\Ecoh = \phig^{-5}$) uniquely fix
\begin{equation}\label{eq:scale}
  \lambda_0 = \hbar = \Ecoh \cdot \TauZero.
\end{equation}
There are no free parameters.
\end{theorem}

\begin{proof}
By dimensional analysis in RS-native units ($c = 1$, $\TauZero = 1$):
the multiplier $\lambda_0$ has dimensions of
$[\text{action}] = [\text{energy}] \times [\text{time}]$.
The only RS-native quantity with these dimensions is
$\Ecoh \cdot \TauZero = \phig^{-5} \cdot 1 = \phig^{-5}$.
The K-gate identities confirm that this is the unique value making the
gate ratios consistent:
$K = 2\pi / (8 \ln \phig)$ is a pure number derivable from $\phig$
alone.

\emph{Lean:} \texttt{hbar\_is\_Ecoh\_tau0} (definitional in RS-native
units).
\end{proof}

%======================================================================
\section{Noether's Theorem as Corollary}\label{sec:noether}
%======================================================================

\begin{definition}[Symmetry of J-cost]\label{def:symmetry}
A transformation $T : \mathbb{R}_{>0} \to \mathbb{R}_{>0}$ is a
\emph{symmetry of $\Jcost$} if $\Jcost(T(r)) = \Jcost(r)$ for all
$r > 0$.
\end{definition}

\begin{lemma}[Reciprocal symmetry]\label{lem:reciprocal}
The inversion $r \mapsto r^{-1}$ is a symmetry of $\Jcost$:
$\Jcost(r) = \Jcost(r^{-1})$.
\end{lemma}

\begin{proof}
Direct computation: $\frac{1}{2}(r + r^{-1}) - 1 =
\frac{1}{2}(r^{-1} + r) - 1$. \qed
\end{proof}

\begin{theorem}[Discrete Noether theorem]\label{thm:noether}
Let $T_\alpha : \mathbb{R}_{>0} \to \mathbb{R}_{>0}$ be a one-parameter
family of $\Jcost$-symmetries with $T_0 = \mathrm{id}$.  Then the
quantity
\begin{equation}\label{eq:conserved}
  Q_T[\gamma]
  \;=\; \sum_{k=0}^{N-1}
    \left.\frac{d}{d\alpha}\right|_{\alpha=0}
    \Jcost(T_\alpha(r_k)) \cdot \TauZero
  \;=\; 0
\end{equation}
is conserved along any J-minimising trajectory.
\end{theorem}

\begin{proof}
If $T_\alpha$ is a symmetry for all $\alpha$, then
$\PathAction[T_\alpha \circ \gamma] = \PathAction[\gamma]$.
Differentiating at $\alpha = 0$:
$\sum_k (d/d\alpha)\Jcost(T_\alpha(r_k))|_{\alpha=0} = 0$.
Since $\gamma$ is a minimiser, the first variation of $\PathAction$
vanishes, and the Noether current
$q_k = (d/d\alpha)\Jcost(T_\alpha(r_k))|_{\alpha=0}$ satisfies the
discrete conservation law $\sum_k q_k = 0$ over any 8-tick window.
\end{proof}

\begin{corollary}[Energy is the multiplier]\label{cor:energy}
Time-translation symmetry ($s_k \mapsto s_{k+n}$, $n \in 8\mathbb{Z}$)
gives the conserved quantity ``energy,'' which equals the Lagrange
multiplier $\lambda_0 = \hbar$.
\end{corollary}

\begin{proof}
Time translation leaves $\Jcost(r_k)$ invariant (the cost depends on
the ratio, not the time index).  The associated Noether charge is the
total action per period, whose stationarity is precisely the multiplier
condition.  Hence $Q_{\text{time}} = \lambda_0$.
\end{proof}

\begin{corollary}[Z-pattern conservation]\label{cor:Z}
$Z$-pattern conservation is enforced by T3.  Noether's theorem confirms:
any symmetry of the path action that commutes with T3 produces a
conserved integer $Z$.
\end{corollary}

\subsection{Explicit Noether charges}

We compute the conserved charge for three concrete symmetries.

\begin{example}[Reciprocal symmetry $\to$ parity]
Let $T_\alpha(r) = r^{1-2\alpha}$ for $\alpha \in [0,1]$.  Then
$T_0 = \mathrm{id}$ and $T_{1/2}(r) = 1$ (collapse to unity), and
$T_1(r) = r^{-1}$ (inversion).  Since $\Jcost(r) = \Jcost(r^{-1})$,
$T_\alpha$ is a symmetry for $\alpha \in \{0, 1\}$ and interpolates
between them.  The infinitesimal generator is
\[
  \left.\frac{d}{d\alpha}\right|_{\alpha=0} T_\alpha(r) = -2r\ln r,
\]
giving the Noether current $q_k = -2r_k \ln r_k \cdot \Jcost'(r_k) \cdot \TauZero$.
On a J-minimiser this sums to zero: $\sum_k q_k = 0$.  The conserved
charge is the \textbf{parity} of the ledger --- the symmetry between
debit and credit that forces double-entry structure (T3).
\end{example}

\begin{example}[Scaling symmetry $\to$ momentum]
Let $T_\alpha(r) = e^\alpha \cdot r$ (multiplicative shift).
Then $\Jcost(e^\alpha r) = \frac{1}{2}(e^\alpha r + e^{-\alpha}/r) - 1
\ne \Jcost(r)$ in general, so this is \emph{not} an exact symmetry.
However, in the quadratic regime
$\Jcost(e^{\varepsilon + \alpha}) \approx \frac{1}{2}(\varepsilon + \alpha)^2$,
and the shift $\varepsilon \mapsto \varepsilon + \alpha$ is an exact
symmetry of the quadratic action.  The associated Noether charge is
\[
  Q_{\text{shift}} = \TauZero \sum_k \varepsilon_k \approx
  \TauZero \sum_k \ln r_k,
\]
which is the \textbf{total log-momentum} of the trajectory.  This is
the discrete analogue of total momentum $p = \sum m_i v_i$.
\end{example}

%======================================================================
\section{Comparison with Classical Noether}\label{sec:comparison}
%======================================================================

\begin{center}
\begin{tabular}{@{}lcc@{}}
\toprule
\textbf{Feature} & \textbf{Classical (Lagrangian)} & \textbf{RS (Cost)} \\
\midrule
Primitive & Lagrangian $L(q,\dot{q},t)$ & $\Jcost(r) = \frac{1}{2}(r + r^{-1}) - 1$ \\
Action & $S = \int L\,dt$ (postulated) & $\PathAction = \sum \Jcost \cdot \TauZero$ (forced) \\
Constraint & None (free boundary) & T3: 8-tick advance + Z-conservation \\
Hamiltonian & Legendre transform of $L$ & Lagrange multiplier for T3 \\
Noether & Symmetry of $L$ → conservation & Symmetry of $\Jcost$ → conservation \\
Scale & Free (choice of units) & Fixed by K-gate: $\hbar = \Ecoh \cdot \TauZero$ \\
\bottomrule
\end{tabular}
\end{center}

\begin{remark}
The critical difference is that in RS no action principle is
\emph{postulated}.  The path action arises from the unique cost
functional, and the Hamiltonian emerges as a constraint-enforcement
device.  The scale of $\hbar$ is not a measured input but a consequence
of $\phig$-structure.
\end{remark}

%======================================================================
\section{Falsification Criteria}\label{sec:falsifiers}
%======================================================================

\begin{falsifier}[Alternative cost]
If any convex symmetric cost $F \ne \Jcost$ on $\mathbb{R}_{>0}$
satisfying the d'Alembert equation, normalization $F(1)=0$, and
calibration $F''(1)=1$ is exhibited, the uniqueness theorem (T5) is
falsified and the multiplier argument collapses.
\end{falsifier}

\begin{falsifier}[Hamiltonian without T3]
If a physical system is found where unitary evolution (Schr\"{o}dinger
equation) holds but no discrete conservation law analogous to T3
operates at the fundamental scale, then the ``Hamiltonian as multiplier''
thesis is falsified.
\end{falsifier}

%======================================================================
\section{Lean Formalization}\label{sec:lean}
%======================================================================

The core algebraic steps are formalized in
\texttt{IndisputableMonolith.Foundation.NoetherFromJ}:

\begin{center}
\begin{tabular}{@{}ll@{}}
\toprule
\textbf{Lean symbol} & \textbf{Content} \\
\midrule
\texttt{continuityT3} & T3 constraint on trajectories \\
\texttt{augmentedCost} & $\mathcal{L}[\gamma, \lambda]$ \\
\texttt{isJMinimizer} & J-minimiser definition \\
\texttt{isLagrangeMultiplier} & Stationarity of $\mathcal{L}$ \\
\texttt{noether\_from\_J\_multiplier\_exists} & Theorem~\ref{thm:exists} \\
\texttt{multiplier\_scale\_unique} & Theorem~\ref{thm:unique} \\
\texttt{hbar\_is\_Ecoh\_tau0} & $\hbar = \Ecoh \cdot \TauZero$ \\
\texttt{hamiltonian\_as\_multiplier} & Theorem~\ref{thm:hamiltonian} + scale \\
\bottomrule
\end{tabular}
\end{center}

The module compiles with zero \texttt{sorry} obligations for the
algebraic identities.  The continuum-limit analysis
(Theorem~\ref{thm:hamiltonian}) involves analytic hypotheses stated
explicitly; the algebraic core is verified.

%======================================================================
\section{Discussion}\label{sec:discussion}
%======================================================================

The result has three consequences:

\begin{enumerate}
\item \textbf{The Hamiltonian is derived, not postulated.}  Four
  centuries of physics assumed $\Hhat$ as primitive.  We show it is a
  consequence of cost minimisation under a discrete conservation law.

\item \textbf{The scale of $\hbar$ is fixed.}  In standard physics,
  $\hbar$ is a measured constant.  In RS, $\hbar = \phig^{-5} \cdot
  \TauZero$ is algebraic in~$\phig$.  This removes one fundamental
  constant from the list of ``unexplained inputs.''

\item \textbf{Noether's theorem is deeper than the action principle.}
  The standard derivation assumes an action principle and derives
  conservation laws.  Here the action principle is itself derived (from
  $\Jcost$-minimisation), and conservation laws follow from symmetries
  of the unique cost functional.  Noether's theorem gains a new
  foundation.
\end{enumerate}

\subsection*{Comparison with existing work}

\begin{center}
\small
\renewcommand{\arraystretch}{1.15}
\begin{tabular}{@{}>{\bfseries}l p{5cm} p{5.5cm}@{}}
\toprule
Feature & Standard (Lagrangian) & RS (this paper) \\
\midrule
Noether~\cite{Noether1918}
  & Symmetry of $L$ $\to$ conservation law
  & Symmetry of $\Jcost$ $\to$ conservation law \\
Hamiltonian
  & Legendre transform of $L$ (postulated)
  & Lagrange multiplier for T3 (derived) \\
$\hbar$
  & Measured constant
  & $\Ecoh \cdot \TauZero = \phig^{-5}$ (algebraic) \\
Action principle
  & Postulated
  & Derived from $\Jcost$-minimisation \\
Discrete symmetry
  & Requires lattice field theory~\cite{Zinn-Justin2002}
  & Native (8-tick is fundamental) \\
\bottomrule
\end{tabular}
\end{center}

\paragraph{Limitations.}
The continuum-limit passage from discrete stationarity to the
Schr\"{o}dinger equation involves smoothness and coarse-graining
assumptions stated explicitly in
Section~\ref{sec:hamiltonian}.  The algebraic core (existence, uniqueness,
scale) is rigorous.  The identification of the multiplier with $\hbar$
is definitional in RS-native units; connecting to SI values requires the
external calibration seam.

\paragraph{Open problems.}
\begin{enumerate}[label=\textup{(Q\arabic*)},nosep]
\item Extend the discrete Noether theorem to spatial symmetries (momentum
  conservation) and internal symmetries (gauge charges).
\item Connect the multiplier structure to the EFE emergence
  module~\cite{WashburnEFE2026}.
\item Does the multiplier structure have a categorical formulation
  (e.g.\ as a natural transformation)?
\item Can the discrete transfer matrix $M(\omega)$ be used for
  numerical quantum mechanics (bypassing Schr\"{o}dinger)?
\end{enumerate}

\begin{thebibliography}{9}
\bibitem{WashburnCost2026}
J.~Washburn and M.~Zlatanovi\'{c},
``The Cost of Coherent Comparison,''
arXiv:2602.05753v1, 2026.

\bibitem{PardoGuerra2026}
S.~Pardo-Guerra, J.~Washburn, and E.~Allahyarov,
``Coherent Comparison as Information Cost,''
arXiv:2601.12194v1, 2026.

\bibitem{WashburnRecogOp2026}
J.~Washburn,
``Beyond the Hamiltonian: The Recognition Operator as Fundamental Dynamics,''
Recognition Science preprint, February 2026.

\bibitem{WashburnEFE2026}
J.~Washburn,
``EFE Emergence from RS Action Stationarity,''
Lean module: \texttt{Relativity.Dynamics.EFEEmergence}, 2026.

\bibitem{WashburnAxioms2026}
J.~Washburn,
``The Algebra of Reality: A Recognition Science Derivation of Physical Law,''
\textit{Axioms} (MDPI), \textbf{15}(2), 90, 2025.

\bibitem{Noether1918}
E.~Noether,
``Invariante Variationsprobleme,''
\textit{Nachr. K\"{o}nigl. Ges. Wiss. G\"{o}ttingen, Math.-Phys. Kl.},
235--257, 1918.

\bibitem{Zinn-Justin2002}
J.~Zinn-Justin,
\textit{Quantum Field Theory and Critical Phenomena},
4th~ed., Oxford, 2002.
\end{thebibliography}

\end{document}
