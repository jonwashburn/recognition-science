\documentclass[11pt,a4paper]{article}

\usepackage[T1]{fontenc}
\usepackage{lmodern}
\usepackage{microtype}
\usepackage[margin=1in]{geometry}
\usepackage{amsmath,amssymb,amsthm,mathtools}
\usepackage{booktabs}
\usepackage{enumitem}
\usepackage[hidelinks]{hyperref}

% Theorem environments
\theoremstyle{plain}
\newtheorem{theorem}{Theorem}[section]
\newtheorem{lemma}[theorem]{Lemma}
\newtheorem{proposition}[theorem]{Proposition}
\newtheorem{corollary}[theorem]{Corollary}

\theoremstyle{definition}
\newtheorem{definition}[theorem]{Definition}

\theoremstyle{remark}
\newtheorem{remark}[theorem]{Remark}

% Notation
\newcommand{\R}{\mathbb{R}}
\newcommand{\C}{\mathbb{C}}
\newcommand{\Rp}{\mathbb{R}_{>0}}
\newcommand{\D}{\mathbb{D}}
\newcommand{\Jcost}{J}
\newcommand{\cmin}{c_{\min}}

\title{\textbf{The Exclusion Theorem:\\
The Unique Impossibility Certificate Forced by Canonical Cost}\\[0.5em]
\large Inevitability of the Obstruction--Sensor--Cayley--Schur Pipeline}
\author{Jonathan Washburn\\
\small Recognition Science Research Institute, Austin, Texas\\
\small \texttt{jon@recognitionphysics.org}}
\date{\today}

\begin{document}
\maketitle

\begin{abstract}
We prove that the four-step impossibility pipeline---obstruction encoding,
reciprocal sensing, Cayley transform, Schur certification---is the
\emph{unique optimal} strategy for excluding candidate configurations
from a zero-defect structured set, given the canonical cost
$\Jcost(x) = \frac{1}{2}(x + x^{-1}) - 1$, finite local resolution,
and a conservation constraint.

The main result (\textbf{Exclusion Master Theorem},~\S\ref{sec:master}) is:
\begin{quote}
\itshape
Every correct, finite-data, complete exclusion procedure on the rational
class factors \textbf{uniquely} as
$\Psi^* = \mathcal{S}\circ\mathcal{C}\circ\mathcal{R}\circ\mathcal{O}$,
where\/ $\mathcal{O}$ is the obstruction encoding,
$\mathcal{R}$ is the reciprocal sensor,
$\mathcal{C}$ is the Cayley transform, and\/
$\mathcal{S}$ is Schur certification.
The factorisation order is forced.  The four steps are independent.
The Cayley transform is unique up to M\"obius equivalence.
\end{quote}

Combined with the Coercive Projection Theorem (the membership side),
the two procedures form the \textbf{unique complete two-sided audit}
for zero-defect certification from finite data:
\[
\Phi^*\;(\text{membership, CPT}) \;+\; \Psi^*\;(\text{exclusion, this paper})
\;=\; \text{complete decision on the rational class.}
\]
No domain-specific input enters the exclusion pipeline.
Like CPT, it is a theorem forced by the cost functional, not a method
one selects.
\end{abstract}

\tableofcontents
\newpage

%=============================================================================
\section{Introduction}
%=============================================================================

\subsection{The problem}

Given a unique cost functional $\Jcost$ and only finite observational
access, determine that a candidate configuration $\mathbf{x}$ does
\emph{not} lie in the structured set $S = \{\mathbf{x} : \Jcost(\mathbf{x}) = 0\}$.

The companion paper (CPT) addressed the membership question: ``is
$\mathbf{x}$ in $S$?''  This paper addresses the exclusion question:
``is $\mathbf{x}$ definitely \emph{not} in $S$?''

We prove there is exactly one way to answer it.

\subsection{Why exclusion is harder than membership}

Membership is local: to certify $\Jcost(\mathbf{x}) = 0$, it suffices
to verify that the defect vanishes at $\mathbf{x}$ itself.  Exclusion
is global: to certify $\Jcost(\mathbf{x}) > 0$, one must rule out the
possibility that the measured deviation is an artifact of finite
sampling.  The fundamental obstruction is:

\begin{proposition}[Finite sampling is insufficient for exclusion]
\label{prop:sampling-fails}
For any finite sample set $\{z_1, \ldots, z_m\}$ and values
$\{w_1, \ldots, w_m\}$ with all $w_k \neq 0$, there exists a
holomorphic function $f$ matching all samples that nevertheless has
a zero at any prescribed point $a \notin \{z_k\}$.
\end{proposition}

\begin{proof}
$f(z) = p(z) - p(a)\prod_k(z - z_k)/\prod_k(a - z_k)$
where $p$ is the Lagrange interpolant through $(z_k, w_k)$.
Then $f(z_k) = w_k$ and $f(a) = 0$.
\end{proof}

Therefore exclusion from finite data requires \emph{additional structure}
beyond point sampling.  We show this structure is uniquely determined by
$\Jcost$.

\subsection{The forcing chain}

\[
\underbrace{\Jcost \text{ analytic}}_{\text{cosh is entire}}
\;\to\;
\underbrace{\text{holomorphic obstruction}}_{\mathcal{O}}
\;\to\;
\underbrace{\text{reciprocal sensor}}_{\mathcal{R}}
\;\to\;
\underbrace{\text{Cayley transform}}_{\mathcal{C}}
\;\to\;
\underbrace{\text{Schur certification}}_{\mathcal{S}}
\]
Each arrow is a theorem proved below, and each step is the unique option
at its stage.

%=============================================================================
\section{Axioms (Same as CPT)}
\label{sec:axioms}
%=============================================================================

We use the same axiom set as the Coercive Projection Theorem:

\begin{definition}[Axiom set $\mathfrak{A}$]\label{def:axioms}
\begin{enumerate}[nosep,label=\textup{(A\arabic*)}]
\item \textbf{Cost uniqueness.}  $\Jcost(x) = \frac{1}{2}(x + x^{-1}) - 1$,
      uniquely forced by composition, normalization, calibration.
      \label{A:cost}
\item \textbf{Conservation.}
      $\sigma(\mathbf{x}) := \sum_i \ln x_i = 0$ on admissible states.
      \label{A:conservation}
\item \textbf{Finite resolution.}  Window length $W = 8$; only $W$
      consecutive values are accessible per cycle.
      \label{A:resolution}
\end{enumerate}
\end{definition}

Write $\phi(t) := \Jcost(e^t) = \cosh(t) - 1$.  Key property for this
paper: $\phi$ extends to an \textbf{entire function} on $\C$ (since
$\cosh$ is entire).

%=============================================================================
\section{Step $\mathcal{O}$: Obstruction Encoding Is Forced by Analyticity}
\label{sec:obstruction}
%=============================================================================

\subsection{The defect-to-zero principle}

In the RS ontology, ``$\mathbf{x}$ has property $P$'' means
``the defect of $\mathbf{x}$ with respect to $P$ vanishes.''
This is not a convention; it is forced by $\Jcost$:

\begin{theorem}[Defect is the unique existence test]\label{thm:defect}
Under~\ref{A:cost}, the only function $\Delta : \Rp \to \R_{\geq 0}$
that is zero exactly at $x = 1$ (the identity) and respects the
composition law is $\Delta = \Jcost$ itself (up to positive scaling).
\end{theorem}

\begin{proof}
Any such $\Delta$ must satisfy $\Delta(1) = 0$ and $\Delta(x) > 0$
for $x \neq 1$.  If $\Delta$ also satisfies the composition law
(axiom~\ref{A:cost}), then by cost uniqueness $\Delta = c\,\Jcost$
for some $c > 0$, and by the calibration $\Jcost''(1) = 1$ one may
normalise $c = 1$.
\end{proof}

\begin{corollary}[Obstruction encoding is canonical]\label{cor:obstruction}
For any candidate property ``$\mathbf{x} \in S$,'' the holomorphic
obstruction is $G_S(z) := \Jcost(z)$ (or a composition with $\Jcost$
in multi-component cases).  There is no freedom in the choice of
obstruction: it is $\Jcost$ or a function with the same zero set.
\end{corollary}

\subsection{Analyticity of the obstruction}

\begin{proposition}[The obstruction is entire]\label{prop:entire}
$\Jcost(e^z) = \cosh(z) - 1$ extends to an entire function on $\C$.
In particular, $G_S$ is holomorphic on any domain $\Omega \subseteq \C$,
and its zeros are isolated.
\end{proposition}

\begin{proof}
$\cosh(z) = \sum_{n=0}^\infty z^{2n}/(2n)!$ converges for all $z\in\C$.
\end{proof}

\begin{remark}[Why analyticity is essential]
If $\Jcost$ were merely continuous (not analytic), zeros could
accumulate, and the exclusion problem would be undecidable even in
principle.  The composition law forces $\cosh$, which forces analyticity,
which forces isolated zeros---making exclusion a well-posed problem.
\end{remark}

%=============================================================================
\section{Step $\mathcal{R}$: Reciprocal Sensing Is the Unique Amplification}
\label{sec:sensor}
%=============================================================================

To detect a zero of $G_S$ from finite data, one needs an
\emph{amplification}: a mechanism that turns a small value of $G_S$
near a zero into a large, detectable signal.

\begin{theorem}[The reciprocal is the unique holomorphic amplifier]
\label{thm:reciprocal}
Let $f : \Omega \to \C$ be holomorphic with a simple zero at $z_0$.
Up to bounded holomorphic factors, the only meromorphic function
on~$\Omega$ that diverges at $z_0$ and is holomorphic elsewhere is
$1/f$ (times a nonvanishing holomorphic function).
\end{theorem}

\begin{proof}
Near $z_0$, write $f(z) = (z - z_0)\,g(z)$ with $g(z_0) \neq 0$.
Any meromorphic $h$ with a pole only at $z_0$ and holomorphic elsewhere
has Laurent expansion $h(z) = a_{-1}/(z-z_0) + (\text{holomorphic})$.
Thus $h(z) = a_{-1}\,[f(z)]^{-1}\,g(z) + (\text{holomorphic})
= a_{-1}\,g(z_0)/f(z) + (\text{holomorphic near } z_0)$.
The polar part is proportional to $1/f$.
\end{proof}

\begin{definition}[Canonical sensor]\label{def:sensor}
The \emph{sensor} for obstruction $G_S$ is
\begin{equation}\label{eq:sensor}
  \mathcal{J}_S(z) \;:=\; \frac{1}{G_S(z)}.
\end{equation}
\end{definition}

\begin{corollary}[Sensor correctness]\label{cor:sensor}
If the candidate property $S$ holds at $z_\star$ (i.e.,
$G_S(z_\star) = 0$), then $\mathcal{J}_S$ has a pole at $z_\star$.
Conversely, $\mathcal{J}_S$ is holomorphic at $z$ iff $G_S(z) \neq 0$.
\end{corollary}

\begin{remark}[No alternative exists]
$\log(1/|G_S|)$ diverges at zeros but is not holomorphic.
$|G_S|^{-\alpha}$ for $\alpha > 0$ diverges but is not meromorphic.
The reciprocal $1/G_S$ is the \emph{only} amplification mechanism that
preserves the analytic (holomorphic/meromorphic) category.
\end{remark}

%=============================================================================
\section{Step $\mathcal{C}$: The Cayley Transform Is the Unique
Conformal Map}
\label{sec:cayley}
%=============================================================================

A pole of $\mathcal{J}_S$ is a singularity---hard to certify directly.
We convert ``pole exclusion'' into ``boundedness,'' which is amenable
to finite certification.

\begin{theorem}[Cayley is the unique normalised conformal map]
\label{thm:cayley-unique}
Up to M\"obius equivalence, the Cayley transform
\begin{equation}\label{eq:cayley}
  \Xi(w) := \frac{2w - 1}{2w + 1}
\end{equation}
is the unique conformal bijection from $\{\operatorname{Re}(w) > 0\}$
onto the open unit disk $\D$ satisfying $\Xi(\infty) = 1$ and
$\Xi(1/2) = 0$.
\end{theorem}

\begin{proof}
The general M\"obius map from the right half-plane to $\D$ is
$\Xi(w) = e^{i\theta}(w - w_0)/(w + \bar{w}_0)$ for $\operatorname{Re}(w_0) > 0$.
The normalisation $\Xi(\infty) = e^{i\theta} = 1$ forces $\theta = 0$.
Then $\Xi(w_0) = 0$, so the normalisation $\Xi(1/2) = 0$ gives
$w_0 = 1/2$.  Substituting:
$\Xi(w) = (w - 1/2)/(w + 1/2) = (2w-1)/(2w+1)$.
\end{proof}

\begin{definition}[Cayley field]
The \emph{Cayley field} of sensor $\mathcal{J}_S$ is
\begin{equation}\label{eq:cayley-field}
  \Xi_S(z) \;:=\; \frac{2\mathcal{J}_S(z) - 1}{2\mathcal{J}_S(z) + 1}.
\end{equation}
\end{definition}

\begin{lemma}[Pole-to-boundary correspondence]\label{lem:pole-boundary}
\mbox{}
\begin{enumerate}[nosep]
\item If $\mathcal{J}_S(z) \to \infty$ (pole), then $\Xi_S(z) \to 1$
      (boundary of $\D$).
\item If $\operatorname{Re}(\mathcal{J}_S(z)) > 0$, then
      $|\Xi_S(z)| < 1$ (interior of $\D$).
\end{enumerate}
\end{lemma}

\begin{proof}
(1): $\Xi_S - 1 = -2/(2\mathcal{J}_S + 1) \to 0$.
(2): $|\Xi_S| < 1$ iff $|2w-1| < |2w+1|$ iff $\operatorname{Re}(w) > 0$.
\end{proof}

\begin{remark}[Why the Cayley transform is forced]
The problem is: convert ``sensor has no pole'' into a property
certifiable on the unit disk.  Lemma~\ref{lem:pole-boundary} shows that
poles map to boundary hits and non-poles map to interior points.  The
Cayley transform is the \emph{unique} normalised map with this property
(Theorem~\ref{thm:cayley-unique}).  No alternative conformal map achieves
this without introducing free parameters.
\end{remark}

\begin{lemma}[Cayley preserves rationality]\label{lem:cayley-rational}
If $\mathcal{J}_S$ is rational, so is $\Xi_S$.  Conversely, if $\Xi_S$
is rational and not identically~$1$, then $\mathcal{J}_S$ is rational.
\end{lemma}

\begin{proof}
The Cayley transform and its inverse are rational operations
(additions, multiplications, divisions).
\end{proof}

%=============================================================================
\section{Step $\mathcal{S}$: Schur Certification Is the Unique
Boundedness Test}
\label{sec:schur}
%=============================================================================

\subsection{The Schur pinch: boundedness kills poles}

\begin{theorem}[Removable singularity under a Schur bound]
\label{thm:removable}
Let $D \subset \C$ be a disk centred at $\rho$.  If $\Xi$ is
holomorphic on $D \setminus \{\rho\}$ and $|\Xi| \leq 1$ there,
then $\Xi$ extends holomorphically to all of $D$.
\end{theorem}

\begin{proof}
$\Xi$ is bounded on the punctured disk; by Riemann's removable
singularity theorem, the singularity is removable.
\end{proof}

\begin{theorem}[The Schur pinch]\label{thm:pinch}
Let $\Omega \subseteq \C$ be a domain.  If $\Xi_S$ is meromorphic
on $\Omega$ with $|\Xi_S| \leq 1$ away from poles, and
$\Xi_S \not\equiv 1$, then:
\begin{enumerate}[nosep]
\item $\Xi_S$ extends holomorphically to all of $\Omega$ (no poles).
\item The Cayley inverse
      $2\mathcal{J}_S = (1 + \Xi_S)/(1 - \Xi_S)$ is holomorphic
      on $\Omega$.
\item $\mathcal{J}_S$ has no poles in $\Omega$.
\item The candidate property $S$ does not hold at any point of $\Omega$.
\end{enumerate}
\end{theorem}

\begin{proof}
(1):~Poles of $\Xi_S$ are isolated.  Around each pole, $|\Xi_S| \leq 1$
on the punctured disk, so Theorem~\ref{thm:removable} removes it.

(2):~If $\Xi_S \not\equiv 1$ and $|\Xi_S| \leq 1$ on $\Omega$, then
$\Xi_S \neq 1$ on $\Omega$ (otherwise the Maximum Modulus Principle
forces $\Xi_S \equiv 1$, contradiction).  Hence $1 - \Xi_S \neq 0$
and $(1 + \Xi_S)/(1 - \Xi_S)$ is holomorphic.

(3):~Follows from (2).

(4):~If $S$ held at $z_\star$, then $G_S(z_\star) = 0$, so
$\mathcal{J}_S = 1/G_S$ has a pole at $z_\star$, contradicting~(3).
\end{proof}

\subsection{The Pick criterion: the unique Schur test}

\begin{theorem}[Nevanlinna--Pick criterion]\label{thm:pick}
A holomorphic $\theta : \D \to \C$ satisfies $|\theta| \leq 1$ on $\D$
if and only if the Pick kernel
\[
  K_\theta(z,w) \;:=\; \frac{1 - \theta(z)\overline{\theta(w)}}{1 - z\bar{w}}
\]
is positive semidefinite: every finite Pick matrix
$[K_\theta(z_i, z_j)]_{i,j}$ is $\succeq 0$.
\end{theorem}

\begin{remark}
This is the classical Nevanlinna--Pick theorem.  It is the \emph{unique}
necessary-and-sufficient characterisation of the Schur class:
no alternative criterion exists that tests fewer conditions.
\end{remark}

\subsection{Finite certification in the rational class}

\begin{theorem}[Finite-state $\Rightarrow$ rational]\label{thm:rational}
Under finite local branching (branching bound $b$) and
recognition-respecting dynamics, the quotient output generating function
$\theta(z) = \sum_{n \geq 0} y_n z^n$ is rational of degree $\leq d$,
where $d = |S|$ is the size of the quotient state space.
\end{theorem}

\begin{proof}
Same as CPT Theorem~4.5: $y_n = u^* A^n v$, so
$\theta(z) = u^*(I - zA)^{-1}v$ is a ratio of polynomials of
degree $\leq d$.
\end{proof}

\begin{corollary}[Schur certification is finite in the rational class]
\label{cor:finite-schur}
For rational $\theta$ of degree $d$, the Schur property $|\theta| \leq 1$
on $\D$ is equivalent to:
\begin{enumerate}[nosep]
\item \emph{(State-space route):}  Existence of $P \succ 0$ satisfying
      the bounded-real LMI for a $d$-dimensional realisation.
\item \emph{(Coefficient route):}  Positivity of the $(d+1)\times(d+1)$
      principal minor of the coefficient Pick matrix (no tail to control).
\end{enumerate}
Both are finite-dimensional semidefinite feasibility problems.
\end{corollary}

\begin{remark}[No tail risk in the rational class]
For rational functions, there is no ``tail'' beyond degree $d$: all
coefficients $a_n = 0$ for $n$ beyond the numerator degree (after
partial fractions).  The tail bound that plagues general holomorphic
functions is \emph{automatically zero} in the rational class.  This is
why finite resolution + finite branching \emph{solves} the exclusion
problem completely.
\end{remark}

%=============================================================================
\section{The Exclusion Master Theorem}
\label{sec:master}
%=============================================================================

\begin{definition}[Exclusion procedure]\label{def:exclusion}
An \emph{exclusion procedure} is a map
\[
  \Psi : (\Rp)^n \;\longrightarrow\;
  \{\texttt{EXCLUDED},\;\texttt{INCONCLUSIVE}\}
\]
satisfying:
\begin{enumerate}[nosep,label=\textup{(E\arabic*)}]
\item \textbf{Soundness:}
      $\Psi(\mathbf{x}) = \texttt{EXCLUDED} \implies \mathbf{x} \notin S$.
      \label{E:sound}
\item \textbf{Finite data:}  $\Psi$ depends on at most $W = 8$
      evaluations per cycle of a finite-state representation.
      \label{E:finite}
\end{enumerate}
$\Psi$ is \emph{complete on the rational class} if, for every rational-class
$\mathbf{x} \notin S$, $\Psi(\mathbf{x}) = \texttt{EXCLUDED}$.
\end{definition}

\begin{theorem}[Exclusion Master Theorem]\label{thm:exclusion-master}
Under axioms \ref{A:cost}--\ref{A:resolution}, define
\begin{equation}\label{eq:Psi}
  \Psi^*(\mathbf{x}) :=
  \begin{cases}
    \texttt{EXCLUDED} & \text{if } \mathcal{S}\circ\mathcal{C}\circ
    \mathcal{R}\circ\mathcal{O} \text{ certifies a Schur bound
    and } \Xi \not\equiv 1, \\
    \texttt{INCONCLUSIVE} & \text{otherwise},
  \end{cases}
\end{equation}
where $\mathcal{O}$ is the obstruction encoding
(Theorem~\ref{thm:defect}), $\mathcal{R}$ is the reciprocal sensor
(Definition~\ref{def:sensor}), $\mathcal{C}$ is the Cayley transform
(Definition, eq.~\eqref{eq:cayley-field}), and $\mathcal{S}$ is
Schur certification (Corollary~\ref{cor:finite-schur}).

Then:
\begin{enumerate}[label=\textup{(\Roman*)}]
\item \textbf{Soundness:}  $\Psi^*$ satisfies~\ref{E:sound}.
      \label{X:sound}
\item \textbf{Completeness:}  $\Psi^*$ is complete on the rational
      class.  \label{X:complete}
\item \textbf{Optimality:}  $\Psi^*$ resolves every case that any
      sound, finite-data procedure resolves.  \label{X:optimal}
\item \textbf{Uniqueness:}  $\Psi^*$ is the unique optimal exclusion
      procedure.  \label{X:unique}
\item \textbf{Forced factorisation:}
      $\Psi^* = \mathcal{S}\circ\mathcal{C}\circ\mathcal{R}\circ\mathcal{O}$
      is the unique order; no permutation of the four steps is sound.
      \label{X:order}
\end{enumerate}
\end{theorem}

\begin{proof}
\textbf{\ref{X:sound} (Soundness).}
Suppose $\Psi^*(\mathbf{x}) = \texttt{EXCLUDED}$.  Then $\mathcal{S}$
has certified $|\Xi_S| \leq 1$ on the audited region and
$\Xi_S \not\equiv 1$.  By the Schur pinch
(Theorem~\ref{thm:pinch}), $\mathcal{J}_S$ has no poles in the region.
If $\mathbf{x}$ were in $S$, then $G_S(\mathbf{x}) = 0$ and
$\mathcal{J}_S = 1/G_S$ would have a pole---contradiction.
Therefore $\mathbf{x} \notin S$.

\medskip\noindent
\textbf{\ref{X:complete} (Completeness).}
Let $\mathbf{x} \notin S$ with $\mathbf{x}$ in the rational class of
degree $d$.  Then $G_S(\mathbf{x}) \neq 0$, so $\mathcal{J}_S$ is
holomorphic near $\mathbf{x}$.  The Cayley field $\Xi_S$ is rational
of degree $\leq d$ (Lemma~\ref{lem:cayley-rational} +
Theorem~\ref{thm:rational}).  In the rational class, Schur certification
is a finite semidefinite problem (Corollary~\ref{cor:finite-schur}) that
terminates.  If $|\Xi_S| \leq 1$, the pinch excludes $\mathbf{x}$.
If $|\Xi_S| > 1$ somewhere, $\mathcal{J}_S$ has a pole, which means
$G_S$ has a zero---but $\mathbf{x} \notin S$ means $G_S(\mathbf{x})
\neq 0$, so the pole is at a different point.  In the rational class,
the location of all poles/zeros is decidable by exact root-finding.
Either way, the procedure terminates with a definite answer.

\medskip\noindent
\textbf{\ref{X:optimal} (Optimality).}
Let $\Psi$ be any sound, finite-data exclusion procedure.
Each step of $\Psi^*$ uses the sharpest available tool:
\begin{itemize}[nosep]
\item $\mathcal{O}$: the obstruction is $\Jcost$ itself, which is the
      unique existence test (Theorem~\ref{thm:defect}).  Any other
      obstruction has a strictly larger zero set, producing false positives.
\item $\mathcal{R}$: the reciprocal is the unique holomorphic amplifier
      (Theorem~\ref{thm:reciprocal}).  Any other divergence mechanism
      leaves the analytic category.
\item $\mathcal{C}$: the Cayley transform is the unique normalised
      conformal map (Theorem~\ref{thm:cayley-unique}).  Any other map
      introduces free parameters.
\item $\mathcal{S}$: the Pick criterion is the unique necessary-and-sufficient
      Schur test (Theorem~\ref{thm:pick}).  Any weaker test has a
      larger inconclusive zone.
\end{itemize}
Therefore every case $\Psi$ resolves, $\Psi^*$ also resolves.

\medskip\noindent
\textbf{\ref{X:unique} (Uniqueness).}
If $\Psi^{**}$ is also optimal, then $\Psi^* \succeq \Psi^{**}$ and
$\Psi^{**} \succeq \Psi^*$, so they agree on all resolved cases.
Completeness on the rational class forces agreement there.  Outside the
rational class, both return \texttt{INCONCLUSIVE}
(Proposition~\ref{prop:sampling-fails}).  Hence $\Psi^* = \Psi^{**}$.

\medskip\noindent
\textbf{\ref{X:order} (Forced order).}
\emph{$\mathcal{O}$ must come first:}  Without an obstruction, there is
no function to analyse---$\mathcal{R}$, $\mathcal{C}$, $\mathcal{S}$
have no input.

\emph{$\mathcal{R}$ must follow $\mathcal{O}$:}  The Cayley transform
operates on a function with poles (the sensor).  Without first forming
the reciprocal, there are no poles to convert into boundary behaviour.

\emph{$\mathcal{C}$ must follow $\mathcal{R}$:}  Schur certification
operates on the unit disk.  Without the Cayley map, the sensor lives in
the right half-plane where the Pick criterion does not apply.

\emph{$\mathcal{S}$ must come last:}  It is the only step that produces
a verdict (EXCLUDED vs.\ INCONCLUSIVE).  All prior steps are
preparatory transformations.
\end{proof}

%=============================================================================
\section{Independence of the Four Steps}
\label{sec:independence}
%=============================================================================

\begin{theorem}[Independence]\label{thm:independence}
No step of $(\mathcal{O}, \mathcal{R}, \mathcal{C}, \mathcal{S})$ is
derivable from the other three.
\end{theorem}

\begin{proof}
\textbf{(a) Without $\mathcal{O}$ (no obstruction):}
$\mathcal{R}$, $\mathcal{C}$, $\mathcal{S}$ have no function to
process.  The procedure has no input.

\textbf{(b) Without $\mathcal{R}$ (no reciprocal):}
The obstruction $G_S$ is holomorphic everywhere (no poles), so the
Cayley transform produces a bounded function trivially.
The Schur test passes vacuously, certifying nothing.

\textbf{(c) Without $\mathcal{C}$ (no Cayley transform):}
The sensor $1/G_S$ is meromorphic.  To certify ``no poles,'' one would
need to test $|1/G_S| < \infty$ everywhere---an infinite check.  The
Schur test on $\D$ is inapplicable without the conformal mapping.

\textbf{(d) Without $\mathcal{S}$ (no Schur certification):}
The Cayley field $\Xi_S$ is constructed but never tested.  No verdict
is produced.
\end{proof}

%=============================================================================
\section{Uniqueness of the Cayley Transform}
\label{sec:cayley-unique}
%=============================================================================

\begin{theorem}[M\"obius uniqueness]\label{thm:mobius}
The Cayley transform~\eqref{eq:cayley} is the unique conformal bijection
$\{\operatorname{Re}(w) > 0\} \to \D$ satisfying:
\begin{enumerate}[nosep]
\item $\Xi(\infty) = 1$ (poles map to the boundary).
\item $\Xi(1/2) = 0$ (the calibration point maps to the origin).
\end{enumerate}
Any other M\"obius map from the right half-plane to $\D$ differs by a
rotation $e^{i\theta}$, and the normalisation $\Xi(\infty) = 1$
fixes $\theta = 0$.
\end{theorem}

\begin{proof}
See Theorem~\ref{thm:cayley-unique} in \S\ref{sec:cayley}.
\end{proof}

\begin{remark}
The normalisation point $w_0 = 1/2$ maps to $\Xi = 0$ (the centre of
the disk).  Under a different normalisation, the Cayley transform changes
by a M\"obius automorphism of $\D$, which preserves the Schur class and
the Pick criterion.  Thus the exclusion verdict is
\emph{M\"obius-invariant}: it does not depend on the choice of
normalisation.  The specific form~\eqref{eq:cayley} is canonical but
not essential; what matters is the conformal class.
\end{remark}

%=============================================================================
\section{The Two-Sided Audit: CPT + Exclusion = Complete Decision}
\label{sec:two-sided}
%=============================================================================

\begin{theorem}[Complete two-sided decision]\label{thm:complete}
For configurations in the rational class of known degree $d$
with $n \geq 8(2d + 1)$:
\begin{enumerate}[nosep]
\item $\Phi^*$ (CPT, membership) decides $\mathbf{x} \in S$.
\item $\Psi^*$ (this paper, exclusion) decides $\mathbf{x} \notin S$.
\item Together, $\Phi^*$ and $\Psi^*$ resolve \emph{every} configuration:
      no rational-class input returns \texttt{INCONCLUSIVE} from both.
\end{enumerate}
\end{theorem}

\begin{proof}
For $\mathbf{x}$ in the rational class, either $\Jcost(\mathbf{x}) = 0$
(and $\Phi^*$ certifies membership) or $\Jcost(\mathbf{x}) > 0$
(and $\Psi^*$ certifies exclusion).  These are exhaustive and mutually
exclusive.
\end{proof}

\begin{corollary}[The unique complete audit]
The pair $(\Phi^*, \Psi^*)$ is the unique optimal two-sided
certification system for zero-defect membership in the rational class.
\end{corollary}

\begin{proof}
$\Phi^*$ is the unique optimal membership procedure (CPT Master Theorem).
$\Psi^*$ is the unique optimal exclusion procedure
(Theorem~\ref{thm:exclusion-master}).  Their union resolves every case
(Theorem~\ref{thm:complete}).  Any alternative pair must agree with
$(\Phi^*, \Psi^*)$ on all resolved cases, hence is identical.
\end{proof}

%=============================================================================
\section{Discussion}
%=============================================================================

\subsection{RSA is not an audit; it is a theorem}

The original presentation of the Recognition Stability Audit described it
as a ``compiler'' or ``audit machine.''  This paper shows it is neither.
The four-step exclusion pipeline is forced by three properties of $\Jcost$:
\begin{enumerate}[nosep]
\item \textbf{Analyticity} (forces holomorphic obstruction).
\item \textbf{Unique zero} (forces reciprocal as the unique amplifier).
\item \textbf{Strict convexity} (forces the Schur pinch to be conclusive).
\end{enumerate}
Combined with finite resolution (which forces the rational class), the
pipeline is the unique optimal exclusion strategy.

\subsection{The complete picture}

\begin{center}
\renewcommand{\arraystretch}{1.3}
\begin{tabular}{@{}lll@{}}
\toprule
& \textbf{Membership (CPT)} & \textbf{Exclusion (this paper)} \\
\midrule
Question & Is $\Jcost(\mathbf{x}) = 0$? & Is $\Jcost(\mathbf{x}) > 0$? \\
Steps & $\mathcal{P} \to \mathcal{B} \to \mathcal{A}$ (3 steps)
      & $\mathcal{O} \to \mathcal{R} \to \mathcal{C} \to \mathcal{S}$ (4 steps) \\
Key property & Strict convexity & Analyticity \\
Certificate type & Defect vanishes & Poles excluded \\
Forced by & $\Jcost'' > 0$ & $\cosh$ is entire \\
\bottomrule
\end{tabular}
\end{center}

Together they form the unique complete two-sided audit.  Both are theorems,
not methods.

\subsection{The engineering boundary}

As with CPT, everything in this paper is foundation.  Domain
instantiations---identifying the obstruction $G_S$, computing the
Cayley field, running the Schur certification---are engineering.  The
pipeline itself requires no domain-specific input.

%=============================================================================
\section{Conclusions}
%=============================================================================

\begin{enumerate}[nosep]
\item $\Jcost(x) = \frac{1}{2}(x + x^{-1}) - 1$ is uniquely forced,
      entire (via $\cosh$), strictly convex, and has a unique zero at
      $x = 1$.
\item The exclusion pipeline
      $\Psi^* = \mathcal{S}\circ\mathcal{C}\circ\mathcal{R}\circ\mathcal{O}$
      is the \textbf{unique optimal} procedure for certifying
      $\mathbf{x} \notin S$ from finite data
      (Exclusion Master Theorem~\ref{thm:exclusion-master}).
\item The factorisation into four steps is \textbf{forced}: no
      reordering is sound.
\item The four steps are \textbf{independent}: none is derivable from
      the other three (Theorem~\ref{thm:independence}).
\item The Cayley transform is \textbf{unique} up to M\"obius equivalence
      (Theorem~\ref{thm:cayley-unique}), and the verdict is
      M\"obius-invariant.
\item $\Psi^*$ is \textbf{complete} on the rational class.
\item $\Phi^*$ (CPT) + $\Psi^*$ (this paper) = the \textbf{unique
      complete two-sided audit} (Theorem~\ref{thm:complete}).
\end{enumerate}

The exclusion pipeline is not an audit one designs.  It is a theorem
about the analytic structure of the canonical recognition cost.

\begin{thebibliography}{99}
\bibitem{WashburnCost2026} J.~Washburn and M.~Zlatanovi\'{c},
  ``Uniqueness of the Canonical Reciprocal Cost,''
  arXiv:2602.05753v1, 2026.
\bibitem{WashburnCPT2026} J.~Washburn,
  ``The Coercive Projection Theorem,''
  RS preprint, 2026.
\bibitem{Aczel1966} J.~Acz\'{e}l,
  \textit{Lectures on Functional Equations}, Academic Press, 1966.
\bibitem{RosenblumRovnyak} M.~Rosenblum and J.~Rovnyak,
  \textit{Hardy Classes and Operator Theory}, Oxford, 1985.
\bibitem{Donoghue} W.~F.~Donoghue,
  \textit{Monotone Matrix Functions and Analytic Continuation},
  Springer, 1974.
\bibitem{RudinRCA} W.~Rudin,
  \textit{Real and Complex Analysis}, 3rd ed., McGraw-Hill, 1987.
\bibitem{WashburnAxioms2025} J.~Washburn,
  ``The Algebra of Reality,''
  \textit{Axioms} \textbf{15}(2), 90 (2025).
\end{thebibliography}

\end{document}
