\documentclass[11pt,a4paper]{article}

\usepackage[T1]{fontenc}
\usepackage{lmodern}
\usepackage{microtype}
\usepackage[margin=1in]{geometry}
\usepackage{amsmath,amssymb,amsthm,mathtools}
\usepackage{booktabs}
\usepackage{enumitem}
\usepackage[hidelinks,breaklinks]{hyperref}

\theoremstyle{plain}
\newtheorem{theorem}{Theorem}[section]
\newtheorem{lemma}[theorem]{Lemma}
\newtheorem{proposition}[theorem]{Proposition}
\newtheorem{corollary}[theorem]{Corollary}

\theoremstyle{definition}
\newtheorem{definition}[theorem]{Definition}
\newtheorem{axiom}[theorem]{Axiom}
\newtheorem{postulate}[theorem]{Postulate}

\theoremstyle{remark}
\newtheorem{remark}[theorem]{Remark}

\newcommand{\R}{\mathbb{R}}
\newcommand{\Z}{\mathbb{Z}}
\newcommand{\N}{\mathbb{N}}
\newcommand{\Rp}{\mathbb{R}_{>0}}
\newcommand{\Jcost}{J}
\newcommand{\phig}{\varphi}

\title{\textbf{Three Attributes Force a Unique Mathematical Framework}\\[0.4em]
\large Core Inevitability, Reverse Implication, and Extended Closure}
\author{Jonathan Washburn\\
\small Recognition Science Research Institute, Austin, Texas\\
\small \texttt{jon@recognitionphysics.org}}
\date{February 15, 2026}

\begin{document}
\maketitle

\begin{abstract}
We formalise three classical attributes as axioms on a comparison-cost
system: omniscience (full discriminability), omnipotence (unique
cost-minimising admissible action), and omnipresence (complete and
globally accessible state geometry). We prove a core inevitability
result: these axioms force a unique cost law
\[
\Jcost(x)=\frac{1}{2}(x+x^{-1})-1
\]
and force the unique ambient spatial dimension $D=3$ under the
simultaneous requirements of topological memory and non-segregation.

We then prove the reverse implication for the core framework, yielding
a biconditional at the level of core axioms. Extended consequences
that are standard in Recognition Science, namely the golden-ratio rung,
8-tick periodicity, and the $W=17$ bridge, are stated separately with
their explicit additional postulates.

The paper is self-contained and separates strictly proved claims from
postulate-dependent extensions.

\medskip\noindent
\textbf{MSC 2020:} 39B52, 51K05, 57M25, 26A51.

\medskip\noindent
\textbf{Keywords:} functional equations, d'Alembert equation,
dimensional selectivity, metric completeness, linking invariants.
\end{abstract}

\tableofcontents
\newpage

%=============================================================================
\section{Introduction}
\label{sec:intro}
%=============================================================================

If a system can discriminate every state, execute every consistent
transformation, and be present everywhere, what mathematical structure
is forced?

This paper gives a precise answer in two layers.
\begin{enumerate}[nosep]
\item A core theorem that is rigorous under explicit axioms.
\item An extended closure layer whose extra assumptions are
clearly declared.
\end{enumerate}

The core theorem is the part that should be read as strict
mathematics. The closure layer records how the broader Recognition
Science chain is obtained from additional structural choices.

\subsection{Prior results used}

The d'Alembert classification and convexity regularity follow classical
functional-equation analysis \cite{Aczel1966,Kuczma2009}. The
canonical reciprocal cost uniqueness result is established in
\cite{CostUnique} and in the peer-reviewed
\cite{WashburnRahnamai2026}. Recognition-geometry context appears in
\cite{RecogGeom}.

%=============================================================================
\section{Formal Setup}
\label{sec:setup}
%=============================================================================

\begin{definition}[Costed comparison system]
A costed comparison system is a tuple
\[
\mathcal{C}=(\mathcal{S},\iota,C)
\]
with:
\begin{enumerate}[nosep,label=\textup{(\alph*)}]
\item a nonempty state set $\mathcal{S}$,
\item an injective scale map $\iota:\mathcal{S}\to\Rp$,
\item a cost map $C:\Rp\to\R_{\ge 0}$.
\end{enumerate}
For $a,b\in\mathcal{S}$ define the ratio
\[
x_{ab}:=\frac{\iota(a)}{\iota(b)}.
\]
\end{definition}

\begin{definition}[Log coordinate and metric]
Define $y:=\log\circ\iota:\mathcal{S}\to\R$ and
\[
d_{\log}(a,b):=|y(a)-y(b)|.
\]
When we refer to metric completeness in this paper, we mean
completeness of $(\mathcal{S},d_{\log})$.
\end{definition}

\begin{remark}
Using $d_{\log}$ avoids ambiguity about metric structure. The cost
$C$ is not assumed to be itself a metric.
\end{remark}

%=============================================================================
\section{Three Axioms}
\label{sec:axioms}
%=============================================================================

\begin{axiom}[Omniscience]
\label{ax:omni}
There exists a costed comparison system $(\mathcal{S},\iota,C)$ such
that:
\begin{enumerate}[nosep,label=\textup{(\alph*)}]
\item $\iota$ is injective (perfect discriminability),
\item $C(1)=0$,
\item for all $x,y>0$,
\[
C(xy)+C(x/y)=P(C(x),C(y))
\]
for a symmetric polynomial $P$,
\item calibration at identity:
\[
\lim_{t\to 0}\frac{2C(e^t)}{t^2}=1.
\]
\end{enumerate}
\end{axiom}

\begin{axiom}[Omnipotence]
\label{ax:omnip}
Admissible transformations are exactly finite-cost ones, and every
admissible constrained optimisation has a unique minimiser. In
particular:
\begin{enumerate}[nosep,label=\textup{(\alph*)}]
\item $C''(x)>0$ for all $x>0$,
\item the conservation quantity
\[
\sigma(\mathbf{x})=\sum_i\log x_i
\]
is invariant under admissible dynamics.
\end{enumerate}
\end{axiom}

\begin{axiom}[Omnipresence]
\label{ax:omnip2}
\begin{enumerate}[nosep,label=\textup{(\alph*)}]
\item $(\mathcal{S},d_{\log})$ is complete,
\item the carrier is a full lattice $\Z^D$ for some $D\in\N$,
\item any two lattice states are connected by a finite admissible path,
\item interaction histories admit a topological-memory model:
closed trajectories are represented by embedded loops in an ambient
continuum $\R^D$.
\end{enumerate}
\end{axiom}

%=============================================================================
\section{Core Forward Theorems}
\label{sec:forward}
%=============================================================================

\subsection{The forced cost law}

\begin{theorem}[Core inevitability of the canonical cost]
\label{thm:cost}
Under Axioms \ref{ax:omni} and \ref{ax:omnip}, the unique admissible
cost is
\[
\Jcost(x)=\frac{1}{2}(x+x^{-1})-1.
\]
\end{theorem}

\begin{proof}
Define $f(t):=1+C(e^t)$. The composition law becomes a continuous
d'Alembert equation
\[
f(t+s)+f(t-s)=2f(t)f(s),\qquad f(0)=1.
\]
By classical classification \cite{Aczel1966,Kuczma2009}, continuous
solutions are $f\equiv 1$, $f(t)=\cos(at)$, or $f(t)=\cosh(at)$.

Strict convexity excludes $f\equiv 1$. Nonnegativity of $C$ gives
$f\ge 1$, excluding cosine except the trivial case already removed.
Hence $f(t)=\cosh(at)$. Calibration yields
\[
1=\lim_{t\to 0}\frac{2C(e^t)}{t^2}
=\lim_{t\to 0}\frac{2(\cosh(at)-1)}{t^2}=a^2.
\]
So $a=1$ and
\[
C(e^t)=\cosh t-1.
\]
Substitute $t=\log x$.
\end{proof}

\begin{corollary}[Derived properties]
\label{cor:properties}
For all $x>0$:
\begin{enumerate}[nosep,label=\textup{(\roman*)}]
\item $\Jcost(x)=\Jcost(x^{-1})$,
\item $\Jcost(x)=\frac{(x-1)^2}{2x}\ge 0$ with equality iff $x=1$,
\item $\Jcost''(x)=x^{-3}>0$,
\item $\Jcost(x)\ge \frac{1}{2}(\log x)^2$.
\end{enumerate}
\end{corollary}

\begin{proof}
(i) and (ii) are direct algebra. (iii) is direct differentiation.
(iv) follows from $\cosh u-1\ge u^2/2$ with $u=\log x$.
\end{proof}

\subsection{Dimension selectivity}

\begin{theorem}[Core dimension theorem]
\label{thm:D3}
Under Axiom \ref{ax:omnip2}(d), the unique ambient dimension satisfying
both:
\begin{enumerate}[nosep,label=\textup{(\alph*)}]
\item nontrivial topological memory of loop interaction, and
\item no topological segregation of reachable states,
\end{enumerate}
is $D=3$.
\end{theorem}

\begin{proof}
Interpret loop interaction in ambient $\R^D$.

\textbf{$D=2$ fails (segregation).}
Jordan curve theorem: any simple closed curve separates $\R^2$ into
inside and outside, violating global accessibility.

\textbf{$D\ge 4$ fails (no memory).}
For an embedded loop $\gamma\subset\R^D$, $D\ge 4$ implies
codimension $\ge 3$. Alexander duality yields trivial first homology
for the complement in the relevant linking degree, so loop-linking
memory vanishes.

\textbf{$D=3$ works.}
Loops can link with nonzero linking number, giving stable interaction
memory, while a single embedded loop does not separate $\R^3$.

Hence only $D=3$ satisfies both requirements.
\end{proof}

\begin{remark}
The theorem explicitly uses ambient continuum topology for interaction
invariants and lattice discreteness for state accessibility. This is
the continuum-lattice bridge required by Axiom \ref{ax:omnip2}(d).
\end{remark}

\subsection{Core master theorem}

\begin{theorem}[Core structural inevitability]
\label{thm:coremaster}
Axioms \ref{ax:omni}--\ref{ax:omnip2} force a unique core framework:
\begin{enumerate}[nosep,label=\textup{(\roman*)}]
\item canonical cost law $\Jcost$,
\item strict convex unique dynamics under conservation,
\item ambient dimension $D=3$ from memory plus non-segregation.
\end{enumerate}
\end{theorem}

%=============================================================================
\section{Reverse Direction for the Core Framework}
\label{sec:reverse}
%=============================================================================

\begin{definition}[Core framework object]
\label{def:coreframework}
A core framework object is data
\[
\mathfrak{F}=(\mathcal{S},\iota,\Jcost,d_{\log},D)
\]
with:
\begin{enumerate}[nosep,label=\textup{(\alph*)}]
\item $\iota:\mathcal{S}\to\Rp$ injective,
\item $\Jcost(x)=\frac12(x+x^{-1})-1$,
\item $d_{\log}(a,b)=|\log\iota(a)-\log\iota(b)|$ complete,
\item admissible dynamics defined as unique minimisers of strictly
convex $\Jcost$-objectives under conservation constraints,
\item ambient topological interaction model in $\R^D$ with $D=3$.
\end{enumerate}
\end{definition}

\begin{theorem}[Reverse implication for the core]
\label{thm:reversecore}
Every core framework object (Definition \ref{def:coreframework})
satisfies Axioms \ref{ax:omni}--\ref{ax:omnip2}.
\end{theorem}

\begin{proof}
\textbf{Omniscience:} injective $\iota$ gives discriminability; $\Jcost$
satisfies d'Alembert form and calibration at identity.

\textbf{Omnipotence:} strict convexity of $\Jcost$ gives unique
minimisers; conservation is enforced by the admissible constraint class.

\textbf{Omnipresence:} $d_{\log}$ is complete by assumption, lattice
carrier is given, finite-path accessibility is part of the admissible
dynamics, and topological interaction model is included.
\end{proof}

\begin{corollary}[Core biconditional]
\label{cor:corebiconditional}
At core level:
\[
(\text{Omniscience}\wedge\text{Omnipotence}\wedge\text{Omnipresence})
\Longleftrightarrow
(\text{Core framework object}).
\]
\end{corollary}

%=============================================================================
\section{Extended Closure Layer (Explicitly Postulated)}
\label{sec:extended}
%=============================================================================

The following consequences are standard in Recognition Science, but
they require assumptions not contained in the core axioms.

\begin{postulate}[Self-similar reciprocal closure]
\label{post:phi}
The refinement map on positive ratios is the minimal reciprocal
self-similarity law
\[
x\mapsto 1+\frac{1}{x},
\]
equivalently fixed points satisfy $x^2=x+1$.
\end{postulate}

\begin{postulate}[Local update graph]
\label{post:8}
One update epoch is represented by a Hamiltonian traversal of the
hypercube $Q_D$.
\end{postulate}

\begin{postulate}[Symmetry bridge]
\label{post:17}
At $D=3$, the effective face-symmetry catalogue is identified with the
17 wallpaper-group classes.
\end{postulate}

\begin{theorem}[Extended consequences]
\label{thm:extended}
Assume Postulates \ref{post:phi}--\ref{post:17}. Then:
\begin{enumerate}[nosep,label=\textup{(\roman*)}]
\item $\phig=(1+\sqrt5)/2$ is the unique positive fixed point.
\item For $D=3$, the minimal epoch length is $2^3=8$.
\item Gap-45 synchronisation:
$45=T(9)=1+\cdots+9$, and
$\mathrm{lcm}(8,45)=360$.
\item The bridge count gives $W=17$ and
$E_p+F=11+6=17$ at $D=3$.
\end{enumerate}
\end{theorem}

\begin{proof}
(i) solve $x^2-x-1=0$. (ii) $Q_D$ has $2^D$ vertices and a Hamiltonian
cycle visits each once. (iii) arithmetic. (iv) direct count with the
stated bridge identification.
\end{proof}

\begin{remark}[Status discipline]
Statements in Theorems \ref{thm:cost}, \ref{thm:D3},
\ref{thm:coremaster}, and \ref{thm:reversecore} are core theorems.
Statements in Theorem \ref{thm:extended} are conditional on explicit
postulates.
\end{remark}

%=============================================================================
\section{Tautological Interpretation}
\label{sec:tautology}
%=============================================================================

The three core axioms align with classical logical laws:
\begin{center}
\renewcommand{\arraystretch}{1.25}
\begin{tabular}{@{}lll@{}}
\toprule
\textbf{Axiom} & \textbf{Logical law} & \textbf{Core mathematical witness} \\
\midrule
Omniscience & Identity & $\Jcost(1)=0$, injective scale map \\
Omnipotence & Non-contradiction & strict convex unique minimiser \\
Omnipresence & Excluded middle & complete metric coverage \\
\bottomrule
\end{tabular}
\end{center}

Under this reading, the core framework is the geometric realisation of
``$a=a$'' after adding coherent composition and completeness
constraints.

%=============================================================================
\section{Discussion}
\label{sec:discussion}
%=============================================================================

\subsection{Why this split matters}

A common failure mode in foundational papers is mixing proved claims
with architecture-level assumptions. The split in this paper is
intended to avoid that:
\begin{itemize}[nosep]
\item Core layer: mathematically forced.
\item Extended layer: mathematically transparent conditional bridge.
\end{itemize}

\subsection{Philosophical context}

The question ``why something rather than nothing'' (Leibniz
\cite{Leibniz1714}) is represented here as boundary-exclusion in a
complete cost geometry. The broader ``mathematics as ontology'' line
has modern formulations in Tegmark \cite{Tegmark2008} and Wheeler's
information-first perspective \cite{Wheeler1990}. The present paper
is narrower: it specifies one explicit core structure and proves its
uniqueness from stated axioms.

%=============================================================================
\section{Conclusion}
\label{sec:conclusion}
%=============================================================================

We proved a core inevitability and a reverse implication:
\begin{enumerate}[nosep]
\item The three axioms force a unique core framework.
\item The core framework satisfies the three axioms.
\end{enumerate}
Hence a core biconditional holds.

Extended RS closure claims are recorded with explicit postulates, so
the logical status of each claim is visible and auditable.

%=============================================================================
% REFERENCES
%=============================================================================
\begin{thebibliography}{99}

\bibitem{Aczel1966}
J.~Acz\'el,
\textit{Lectures on Functional Equations and Their Applications},
Academic Press, 1966.

\bibitem{Kuczma2009}
M.~Kuczma,
\textit{An Introduction to the Theory of Functional Equations and
Inequalities},
2nd ed., Birkh\"auser, 2009.

\bibitem{CostUnique}
J.~Washburn and M.~Zlatanovi\'c,
``Uniqueness of the Canonical Reciprocal Cost,''
arXiv:2602.05753, 2026.

\bibitem{WashburnRahnamai2026}
J.~Washburn and A.~Rahnamai Barghi,
``Reciprocal Convex Costs for Ratio Matching: Axiomatic
Characterization,''
\textit{Axioms}, 2026.
doi:10.3390/axioms1010000.

\bibitem{RecogGeom}
J.~Washburn, M.~Zlatanovi\'c, and E.~Allahyarov,
``Recognition Geometry,''
\textit{Axioms} \textbf{15}(2), 90, 2025.

\bibitem{Leibniz1714}
G.~W.~Leibniz,
``Principles of Nature and Grace, Based on Reason,'' 1714.

\bibitem{Tegmark2008}
M.~Tegmark,
``The Mathematical Universe,''
\textit{Foundations of Physics} \textbf{38}, 101--150, 2008.

\bibitem{Wheeler1990}
J.~A.~Wheeler,
``Information, Physics, Quantum: The Search for Links,''
in \textit{Complexity, Entropy and the Physics of Information},
Addison-Wesley, 1990.

\end{thebibliography}

\end{document}
