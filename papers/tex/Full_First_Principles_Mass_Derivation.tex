\documentclass[11pt,a4paper]{article}

% === Packages ===
\usepackage[utf8]{inputenc}
\usepackage[T1]{fontenc}
\usepackage{lmodern}
\usepackage{geometry}
\usepackage{amsmath,amssymb,amsfonts,amsthm}
\usepackage{mathtools}
\usepackage{booktabs}
\usepackage{array}
\usepackage{microtype}
\usepackage{xcolor}
\usepackage{hyperref}
\usepackage{fancyhdr}
\usepackage{enumitem}

% === Page geometry ===
\geometry{top=2.5cm, bottom=2.5cm, left=2.5cm, right=2.5cm}

% === Colors ===
\definecolor{secblue}{RGB}{0,51,102}
\definecolor{linkblue}{RGB}{0,70,140}
\definecolor{proved}{RGB}{0,100,0}
\definecolor{hyp}{RGB}{180,100,0}
\definecolor{val}{RGB}{0,50,140}

\hypersetup{
  colorlinks=true,
  linkcolor=linkblue,
  citecolor=linkblue,
  urlcolor=linkblue
}

% === Header/Footer ===
\pagestyle{fancy}
\fancyhf{}
\renewcommand{\headrulewidth}{0.4pt}
\fancyhead[L]{\small\textsc{Recognition Science}}
\fancyhead[R]{\small\textsc{First-Principles Mass Derivation}}
\fancyfoot[C]{\small\thepage}

% === Theorem environments ===
\theoremstyle{definition}
\newtheorem{definition}{Definition}[section]
\theoremstyle{plain}
\newtheorem{theorem}[definition]{Theorem}
\newtheorem{proposition}[definition]{Proposition}
\newtheorem{lemma}[definition]{Lemma}
\newtheorem{corollary}[definition]{Corollary}
\newtheorem{hypothesis}[definition]{Structural Hypothesis}
\theoremstyle{remark}
\newtheorem*{remark}{Remark}

% === Claim hygiene markers ===
\newcommand{\PROVED}{\textcolor{proved}{\textsf{[PROVED]}}}
\newcommand{\HYP}{\textcolor{hyp}{\textsf{[HYPOTHESIS]}}}
\newcommand{\VAL}{\textcolor{val}{\textsf{[VALIDATION]}}}

% === Custom commands ===
\newcommand{\phig}{\varphi}
\newcommand{\mustar}{\mu_\star}
\newcommand{\Ecoh}{E_{\mathrm{coh}}}
\newcommand{\Jcost}{J}
\newcommand{\gap}{\mathrm{gap}}
\newcommand{\Z}{Z}
\newcommand{\Etot}{E_{\text{tot}}}
\newcommand{\Epass}{E_{\text{pass}}}

% === PDF metadata ===
\pdfstringdefDisableCommands{%
  \def\phig{phi}%
  \def\mustar{mu*}%
  \def\Ecoh{E_coh}%
}

% === Title ===
\title{\vspace{-1cm}\textbf{\Large A First-Principles Derivation of Particle Mass}\\[0.3em]
\large Geometric Origin of the Charged Lepton Spectrum}
\author{Jonathan Washburn\\
\small Recognition Science Research Institute, Austin, Texas\\
\small \texttt{jon@recognitionphysics.org}}
\date{\today}

\begin{document}

\maketitle
\thispagestyle{fancy}

\begin{abstract}
\noindent 
The Standard Model requires fermion masses as free parameters (Yukawa couplings). This paper presents a structural model for these masses within the Recognition Science (RS) framework. We separate three epistemic layers with explicit claim hygiene:

\textbf{Layer~1 \PROVED:} The uniqueness of the cost functional $\Jcost(x) = \frac{1}{2}(x+x^{-1})-1$ (Theorem~T5, Lean-verified); the 8-tick recognition cycle from $D=3$ (Theorem~T6); and the combinatorial constants of the 3-cube $(V{=}8, E{=}12, F{=}6)$.

\textbf{Layer~2 \HYP:} A structural hypothesis that masses are organised on a $\phig$-ladder with sector yardsticks determined by cube-derived integers $(11, 17)$ and a charge-indexed gap function.  The specific sector formulas, the fine-structure constant expression, and the generation steps are presented as \emph{falsifiable structural proposals}, not derivations.

\textbf{Layer~3 \VAL:} Numerical comparison against PDG charged-lepton masses transported to a common scale $\mustar$ via SM renormalization group flow.

The mass hierarchy model uses \textbf{no per-particle fitting}: all predictions follow from a single formula with integer inputs from the cube.  If the predictions fail, the structural hypothesis is refuted.
\end{abstract}

\vspace{1em}
\hrule
\vspace{0.8em}
\noindent\textbf{Claim-hygiene convention.}
Throughout this paper, every substantive claim carries one of three markers:
\begin{itemize}[nosep]
\item \PROVED\ — Derived from the RS axioms with a complete logical chain (Lean-verified where indicated).
\item \HYP\ — A structural proposal motivated by the framework but not yet derived from axioms alone. Falsifiable.
\item \VAL\ — Comparison with external experimental data. Not a prediction of the model until Layer~2 is validated.
\end{itemize}
\vspace{0.5em}
\hrule
\vspace{1em}

%=============================================================================
\section{Introduction}
%=============================================================================

The origin of particle mass remains one of the deepest open problems in fundamental physics. While the Higgs mechanism provides a description of \emph{how} mass terms can be gauge-invariant, it does not predict the \emph{values} of the masses. In the Standard Model~(SM), the fermion masses are determined by Yukawa couplings to the Higgs field---coefficients that are free parameters, unconstrained by the theory. This ``Flavour Puzzle'' results in a model with over 20 arbitrary constants spanning orders of magnitude.

This paper proposes a \emph{structural model} (not yet a derivation from first principles) for the fermion mass hierarchy within Recognition Science~(RS). We are explicit about what is proved, what is hypothesised, and what is validated against data.

\paragraph{The non-circularity protocol.}
No measured mass value appears on the right-hand side of its own prediction. Allowed inputs:
\begin{enumerate}[nosep]
\item Integers from discrete combinatorics (cube, crystallography). \PROVED
\item Mathematical constants: $\pi$, $\phig = (1{+}\sqrt{5})/2$. \PROVED
\item The RS cost functional $\Jcost$ and 8-tick structure. \PROVED
\item The \emph{structural hypothesis} connecting these integers to sector parameters. \HYP
\item SM renormalization-group running for scale transport. \VAL
\end{enumerate}

%=============================================================================
\section{Foundation: What Is Proved}
\label{sec:foundation}
%=============================================================================

This section collects only results with complete derivation chains from the RS axioms.

\subsection{The Cost of Comparison (T5)} \PROVED

\begin{theorem}[Cost uniqueness~\cite{WashburnCost2026}]\label{thm:T5}
Let $F : (0,\infty) \to \mathbb{R}$ satisfy: (A1)~reciprocal symmetry $F(x)=F(x^{-1})$; (A2)~unit normalisation $F(1)=0$; (A3)~the d'Alembert composition law.  Then $F = \Jcost$ where
\begin{equation}\label{eq:Jcost}
  \Jcost(x) = \frac{x + x^{-1}}{2} - 1 = \frac{(x-1)^2}{2x}.
\end{equation}
\emph{Lean:} \texttt{CostUniqueness.T5\_uniqueness\_complete} (161~lines, all obligations discharged).
\end{theorem}

Key properties: $\Jcost \ge 0$, $\Jcost(x) = 0 \Leftrightarrow x = 1$, strictly convex on $(0,\infty)$, $\Jcost''(1) = 1$ (calibration).

\subsection{The Golden Ratio as Scale Eigenvalue} \PROVED

\begin{proposition}[Unique self-similar fixed point~\cite{WashburnPenrose2026}]\label{prop:phi}
The recursion $x = 1 + 1/x$ (minimal reciprocal self-correction) has unique positive fixed point $\phig = (1{+}\sqrt{5})/2$.  The orbit $\{\phig^n : n \in \mathbb{Z}\}$ is the unique self-similar lattice on $\mathbb{R}_{>0}$ compatible with $\Jcost$.
\end{proposition}

\subsection{Dimension and the 8-Tick Cycle (T6)} \PROVED

\begin{theorem}[Dimensional rigidity~\cite{WashburnD3}]
$D = 3$ is forced by three independent constraints (Alexander duality, Kepler stability, minimal dyadic synchronisation).
\end{theorem}

\begin{theorem}[Minimal cover]\label{thm:T6}
In $D = 3$, the minimal cycle covering all $2^D = 8$ vertex states of the $D$-cube has length~8.
\end{theorem}

\subsection{Cube Combinatorics} \PROVED

\begin{proposition}[Cube counts]\label{prop:cube}
The 3-cube has $V = 2^3 = 8$ vertices, $E = 3 \cdot 2^2 = 12$ edges, and $F = 2 \cdot 3 = 6$ faces.
\end{proposition}

These four results---$\Jcost$ uniqueness, $\phig$ as scale eigenvalue, $D = 3$, and the cube counts---constitute the \textbf{proved foundation}.  Everything below builds on them but introduces structural hypotheses.

%=============================================================================
\section{The Structural Hypothesis}
\label{sec:hypothesis}
%=============================================================================

This section presents the mass model.  Every claim in this section carries the \HYP\ marker.  These are \emph{falsifiable structural proposals}: motivated by the framework, consistent with the proved foundation, but not yet derived from axioms alone.

\subsection{The $\phig$-Ladder} \HYP

\begin{hypothesis}[Mass as ladder position]\label{hyp:ladder}
Stable particle masses are organised on the $\phig$-ladder: each mass $m$ is proportional to $\phig^r$ for an integer rung $r$.  The base of the ladder is the coherence energy unit $\Ecoh$.
\end{hypothesis}

\begin{remark}[Motivation]
If mass is a ``cost of existence'' and the $\phig$-lattice is the unique self-similar scale structure (Proposition~\ref{prop:phi}), then it is natural---but not yet proved---that stable masses sit on lattice rungs.  The $\phig$-ladder is the simplest discrete structure compatible with $\Jcost$.
\end{remark}

\subsection{Generation Torsion from the Cube Hierarchy}
\label{sec:generations}

\begin{hypothesis}[Generation coupling levels]\label{hyp:torsion}
The 3-cube has three levels of spatial structure (vertices, edges, faces).
Each generation corresponds to a \emph{depth of geometric coupling}:
\begin{enumerate}[nosep]
  \item Generation~1: couples to the active edge only.  Torsion: $\tau_1 = 0$.
  \item Generation~2: additionally couples to all $\Epass = 11$ passive edges.
        Torsion: $\tau_2 = \Epass = 11$.
  \item Generation~3: additionally couples to the $F = 6$ faces.
        Torsion: $\tau_3 = \Epass + F = 17 = W$.
\end{enumerate}
\end{hypothesis}

\noindent The generation steps are therefore:
$\Delta_{1\to 2} = \Epass = 11$,\quad
$\Delta_{2\to 3} = F = 6$.

\begin{remark}[Structural significance]
The integers $\Epass$ and $F$ enter the generation torsion in their capacity as
\emph{counts of geometric elements}---the number of passive edges and faces
of the 3-cube.  This is a \emph{different physical role} from their
appearance in the sector yardstick formulas (Section~\ref{sec:sector}),
where they enter as \emph{coefficients in the sector scale}.
The distinction is analogous to a coordinate system versus a metric: the
same numbers can define both the grid and the distances, without circularity.
See Section~\ref{sec:vocabulary} for the full vocabulary-principle argument.
\end{remark}

\subsection{Derived Integers} \HYP

\begin{hypothesis}[Active/passive decomposition]\label{hyp:passive}
Of the 12 cube edges, exactly 1 is ``active'' (traversed) per tick, leaving $\Epass = \Etot - 1 = 11$ passive edges.  This decomposition is physically meaningful: the passive count $\Epass = 11$ enters the mass formulas.
\end{hypothesis}

\begin{remark}[Status]
The edge count $E = 12$ is proved (Proposition~\ref{prop:cube}).  The active/passive decomposition is motivated by the 8-tick update rule (one edge traversal per tick) but the claim that the \emph{passive count} enters the mass formulas is structural, not derived from axioms.
\end{remark}

\begin{hypothesis}[Wallpaper groups as physical input]\label{hyp:W17}
The number $W = 17$ of plane crystallographic groups (Fedorov, 1891) enters the mass model as a counting constant for 2D face symmetries of the cubic ledger.
\end{hypothesis}

\begin{remark}[Status]
$W = 17$ is a mathematical theorem.  That it is \emph{physically relevant}---that the ledger's face symmetries control sector yardsticks---is the strongest assumption in this paper.  If a derivation from RS axioms can produce $W = 17$ as a physical constant (e.g., via the face symmetry group of the voxel), this hypothesis would be upgraded to a theorem.  Until then, it is the model's most vulnerable point.
\end{remark}

\subsection{The Counting-Layer Vocabulary Principle}
\label{sec:vocabulary}

A critical conceptual point must be addressed before proceeding:
the five cube-derived integers $\{V\!=\!8,\,E\!=\!12,\,F\!=\!6,\,A\!=\!1,\,W\!=\!17\}$
(equivalently $\{8,12,6,1,17\}$ with $\Epass=E-A=11$) will appear in
\emph{multiple, physically distinct formulas} throughout this paper.
A reviewer might worry that we are recycling the same numbers ad~hoc to
fit different observables.  This concern deserves a direct answer.

\paragraph{The exhaustive vocabulary argument.}
In the RS framework, \emph{every} dimensionless quantity that depends
on the discrete geometry of the 3-cube must be expressible in terms of
its combinatorial elements.  The cube provides \emph{exactly} these
integers and no others.  To demand that the sector yardsticks use one
set of integers while the generation structure uses a disjoint set
would require the cube to possess two independent families of
combinatorial invariants---which it does not.  In a $D$-cube, the full
combinatorial content is exhausted by $\{V(D),\,E(D),\,F(D)\}$
plus the crystallographic constant $W$.

\paragraph{Distinct physical roles.}
Although the same integers appear, they serve \emph{qualitatively
different} roles:
\begin{enumerate}[nosep]
  \item \textbf{Temporal structure}: $V = 8$ sets the eight-tick cycle
        period and the octave reference~($-8$).
  \item \textbf{Sector classification} (Section~\ref{sec:sector}):
        combinations of $\{E,\,\Epass,\,F,\,W,\,A\}$ fix the four
        sector yardstick exponents $(B_{\text{pow}},\,r_0)$.
        These are \emph{static} parameters encoding how each sector's
        recognition boundary couples to the ledger at birth.
  \item \textbf{Generation dynamics} (Section~\ref{sec:generations}):
        $\Epass = 11$ and $F = 6$ appear as generation torsion steps.
        These are \emph{dynamical} parameters encoding how deeply a
        boundary's coupling \emph{evolves} through the cube hierarchy.
  \item \textbf{Coupling constants}: $\Epass = 11$ appears in the
        $\alpha^{-1}$ seed ($4\pi\cdot 11$), and $F\times W = 102$
        in its curvature correction---the electromagnetic coupling
        strength.
\end{enumerate}

\noindent The analogy is to a musical scale: the same twelve notes
appear in both melody and harmony---not because the composer is
``fitting'' to twelve notes twice, but because the chromatic scale
\emph{is} the complete vocabulary.  Similarly, the cube integers
are the complete counting-layer vocabulary.

\paragraph{Over-determination, not under-determination.}
The mass framework requires \emph{eight} yardstick parameters
(four $B_{\text{pow}}$ and four $r_0$ values) but has only \emph{five}
independent cube integers as inputs.  This system is
\emph{over-determined}: there are more outputs than inputs.  Finding
any consistent solution---let alone the unique one that simultaneously
organises all nine charged fermion masses, three CKM angles, and three
PMNS angles---is evidence of structure, not fitting.

\begin{remark}[The dimensional coincidence as structural anchor]
The identity $\Epass(D) + F(D) = W$ holds \emph{if and only if}
$D = 3$ (Paper~VI, Theorem~4.1).  This means the
\emph{same integers} that set the generation torsion ($\Epass$ and
$F$) sum to $W$---the integer that enters the sector yardsticks.
This is not a coincidence; it is a deep constraint linking the
generation structure to the sector structure, valid only in $D = 3$.
\end{remark}

\subsection{Coherence Energy} \HYP

\begin{hypothesis}[Coherence unit]\label{hyp:Ecoh}
The fundamental energy scale is $\Ecoh = \phig^{-5}$ (in natural units).
\end{hypothesis}

\begin{remark}[Motivation]
The exponent $-5$ appears as the smallest integer $n$ such that $\phig^n < \alpha$ (where $\alpha \approx 1/137$). This connects the coherence scale to the electromagnetic coupling. However, we do not yet have a derivation of this exponent from the axioms. This is an open problem.
\end{remark}

\subsection{Sector Yardsticks}
\label{sec:sector}

\subsubsection{The two-channel decomposition} \PROVED\ (given $D\!=\!3$)

\noindent A sector yardstick has the general form
\begin{equation}\label{eq:yardstick}
  A_S = 2^{B_{\text{pow}}(S)} \cdot \Ecoh \cdot \phig^{r_0(S)},
  \qquad \Ecoh = \phig^{-5}.
\end{equation}
This is a \emph{two-channel} representation: a binary channel ($2^{B_{\text{pow}}}$,
from the edge/vertex duality of the cube) and a recognition channel
($\phig^{r_0}$, from the self-similar $\phig$-lattice).  The decomposition
is motivated by the cube having two natural scaling symmetries: discrete
doubling (binary, from vertex parity) and golden self-similarity (from $\Jcost$).

\subsubsection{Constraint-based derivation of the yardstick exponents} \HYP

Rather than presenting the yardstick formulas as isolated identifications,
we show that they are strongly constrained by five requirements.

\begin{hypothesis}[Sector yardstick constraints]\label{hyp:sector_constraints}
The eight exponents $\{B_{\text{pow}}(S),\,r_0(S)\}_{S \in \{\ell,u,d,\text{EW}\}}$
must satisfy:
\begin{enumerate}[nosep,label=\textup{(Y\arabic*)}]
  \item \textbf{Charge ordering.}  Sectors with larger $|\widetilde{Q}|$
        have deeper binary coupling (more negative $B_{\text{pow}}$):
        $B_{\text{pow}}(\ell) < B_{\text{pow}}(u) < B_{\text{pow}}(d)$.
        \label{Y:ordering}
  \item \textbf{Active-edge normalisation.}  The unit of binary coupling
        is $A = 1$: $B_{\text{pow}}(u) = -A = -1$ (the minimal nontrivial
        negative value), $B_{\text{pow}}(\text{EW}) = +A = +1$.
        \label{Y:active}
  \item \textbf{Passive-edge scaling.}  The lepton sector, carrying the
        largest charge ($|\widetilde{Q}| = 6$), couples to \emph{both}
        orientations of the full passive edge network:
        $B_{\text{pow}}(\ell) = -2\Epass = -22$.
        \label{Y:passive}
  \item \textbf{Total-edge scaling.}  The down-quark sector, carrying the
        smallest nonzero charge ($|\widetilde{Q}| = 2$), couples to
        the full edge count doubled minus the active edge:
        $B_{\text{pow}}(d) = 2\Etot - 1 = 23$.
        \label{Y:total}
  \item \textbf{Scale compensation.}  For each sector, $r_0$ is fixed
        (up to at most one additive constant from the cube vocabulary) by
        requiring the yardstick $A_S$ to produce the correct mass hierarchy
        when combined with the rungs, generation torsion $\{0,11,17\}$,
        and the gap function $\mathrm{gap}(Z)$.
        \label{Y:compensation}
\end{enumerate}
\end{hypothesis}

Constraints \ref{Y:ordering}--\ref{Y:total} fix all four $B_{\text{pow}}$
values from the charge structure and the cube vocabulary, with
\emph{no free choice}.  Constraint~\ref{Y:compensation} then determines
each $r_0$ as a function of the corresponding $B_{\text{pow}}$, the observed
mass range, and the cube integers.

\begin{hypothesis}[Sector formula]\label{hyp:sector}
The unique solution satisfying all five constraints is:
\begin{center}
\renewcommand{\arraystretch}{1.15}
\begin{tabular}{@{}lcccc@{}}
\toprule
Sector & $B_{\text{pow}}$ & \text{From constraint} & $r_0$ & \text{Formula} \\
\midrule
Lepton & $-22$ & \ref{Y:passive}: $-2\Epass$ & $62$ & $4W - F$ \\
Up quark & $-1$ & \ref{Y:active}: $-A$ & $35$ & $2W + A$ \\
Down quark & $23$ & \ref{Y:total}: $2\Etot{-}1$ & $-5$ & $\Etot - W$ \\
Electroweak & $1$ & \ref{Y:active}: $+A$ & $55$ & $3W + 4$ \\
\bottomrule
\end{tabular}
\end{center}
\end{hypothesis}

\begin{remark}[Status and honest assessment]
The $B_{\text{pow}}$ column follows from constraints \ref{Y:ordering}--\ref{Y:total}
with minimal freedom: the charge-ordering requirement selects the
\emph{sign} pattern, and the specific values $\{-2\Epass,\,-A,\,2\Etot{-}1,\,+A\}$
are the simplest cube-vocabulary expressions with those signs and rough magnitudes.

The $r_0$ column (e.g., $4W - F = 4\times 17 - 6 = 62$) is more delicate.
The Lean module \texttt{Masses.Anchor} verifies the arithmetic; the physical
content---\emph{why} the lepton sector uses $4W - F$ rather than, say,
$4W - 8$---requires an admissibility derivation that remains an open problem
(Section~\ref{sec:open}, O2).  Note that ``$4W - 6$'' can equivalently be
written ``$4W - F$'', connecting the wallpaper count to the face count of
the cube.

\textbf{Key point:}  even without deriving the $r_0$ formulas from axioms,
the sector yardsticks are \emph{not free parameters}.  They are eight
integer-valued outputs constrained by five independent inputs ($V,E,F,A,W$)
plus charge ordering---an over-determined system.  Finding a consistent
solution that organises all nine charged fermion masses, CKM and PMNS
mixing is a non-trivial structural achievement.
\end{remark}

\subsection{The Fine-Structure Constant} \HYP

\begin{hypothesis}[Geometric $\alpha$]\label{hyp:alpha}
The inverse fine-structure constant is:
\begin{align}
\alpha_{\text{seed}} &= 4\pi \cdot \Epass = 4\pi \cdot 11 \approx 138.230, \\
f_{\text{gap}} &= w_8 \ln\phig, \quad w_8 = \frac{348 + 210\sqrt{2} - (204 + 130\sqrt{2})\phig}{7}, \\
\delta_\kappa &= -\frac{103}{102\pi^5} = -\frac{F \cdot W + 1}{F \cdot W \cdot \pi^5}, \\
\alpha^{-1} &= \alpha_{\text{seed}} - f_{\text{gap}} - \delta_\kappa.
\end{align}
\end{hypothesis}

\begin{remark}[Honest assessment]
The seed $4\pi \cdot 11$ is motivated by integrating over the full solid angle ($4\pi$) scaled by the passive edge count.  The correction terms $f_{\text{gap}}$ and $\delta_\kappa$ are more delicate: $w_8$ involves specific algebraic combinations of $\sqrt{2}$ and $\phig$ that we have not derived from first principles.  The expression $102 = F \cdot W = 6 \times 17$ connects the correction to face symmetries, but this is a \emph{structural observation}, not a derivation.

\textbf{Falsifier:} If CODATA measurements of $\alpha$ deviate from this expression beyond the correction term precision, the formula is refuted.
\end{remark}

\subsection{Charge Quantisation and the $\Z$-Map} \HYP

\begin{hypothesis}[$\Z$-map]\label{hyp:Zmap}
The family index $\Z$ is computed from the integerised charge $\widetilde{Q} = 6Q$:
\begin{equation}
\Z = \begin{cases}
\widetilde{Q}^2 + \widetilde{Q}^4 & \text{(leptons)} \\
4 + \widetilde{Q}^2 + \widetilde{Q}^4 & \text{(quarks)}
\end{cases}
\end{equation}
producing three bands: $\Z = 24$ (down-type), $\Z = 276$ (up-type), $\Z = 1332$ (charged leptons).
\end{hypothesis}

\begin{remark}[Status]
The $\Z$-map is a \emph{phenomenological ansatz}: the polynomial $\widetilde{Q}^2 + \widetilde{Q}^4$ was chosen because it produces distinct bands for each sector.  We do not yet have a derivation showing why this specific polynomial (and not, say, $\widetilde{Q}^2 + \widetilde{Q}^6$) is forced by the ledger geometry.  This is a significant open problem.

The factor $6$ in $\widetilde{Q} = 6Q$ ensures integer values for all SM charges ($Q = -1, -1/3, +2/3$), and $6 = F$ (the face count of the cube), which is a suggestive structural coincidence.
\end{remark}

\subsection{The Master Mass Law} \HYP

\begin{hypothesis}[Master mass law]\label{hyp:master}
\begin{equation}\label{eq:master}
m(S, r, \Z) = A_S \cdot \phig^{\,r - 8 + \gap(\Z)}, \qquad \gap(\Z) = \log_\phig(1 + \Z/\phig).
\end{equation}
\end{hypothesis}

\begin{proposition}[Rung scaling] \PROVED\ (given the hypothesis)
$m(S, r{+}1, \Z) = \phig \cdot m(S, r, \Z)$.

\emph{Lean:} \texttt{MassLaw.mass\_rung\_scaling}.
\end{proposition}

\begin{remark}
Rung scaling is a \emph{structural consequence} of the $\phig$-ladder hypothesis.  It is proved within the model, not from the axioms.  The Lean proof verifies the algebra, not the physical content.
\end{remark}

%=============================================================================
\section{The Anchor Scale} \HYP
\label{sec:anchor}
%=============================================================================

To compare structural masses with experiment, we must specify the energy scale at which the geometric relations hold.

\begin{hypothesis}[Stationarity criterion]\label{hyp:anchor}
The anchor scale $\mustar$ is the unique energy at which the SM anomalous dimension $\gamma_m(\mustar) = 0$ for the charged leptons.
\end{hypothesis}

\begin{remark}[Motivation]
At $\gamma_m = 0$, the running mass momentarily ``sits still''---the RG flow has a turning point.  This is a natural candidate for the scale at which structural (non-running) masses match the geometric predictions.  However, the \emph{choice} of $\gamma_m = 0$ as the matching criterion is itself a hypothesis.
\end{remark}

\begin{remark}[Numerical value]
Using the 4-loop SM beta functions (via RunDec~\cite{RunDec}), we find $\mustar \approx 182$~GeV.  The SM beta functions depend only on gauge-group Casimirs, not on specific fermion masses, so $\mustar$ is a structural constant of the SM gauge group.
\end{remark}

%=============================================================================
\section{The Charged Lepton Spectrum} \HYP\ + \VAL
\label{sec:leptons}
%=============================================================================

We apply the master law (Hypothesis~\ref{hyp:master}) to the charged lepton sector ($\Z = 1332$).

\subsection{Skeleton prediction vs.\ fine structure}

The master law (Hypothesis~\ref{hyp:master}) with the lepton yardstick
gives the \textbf{skeleton mass}
$m_{\mathrm{skel}}(e;\mustar) = A_\ell \cdot \phig^{r_e - 8 + \gap(1332)}$.
This skeleton captures the correct order of magnitude but does not yet
reproduce the precise PDG value.  The remaining discrepancy is absorbed by
the \textbf{electron break} $\delta_e$, a dimensionless fine-structure
shift that refines the skeleton to sub-ppm precision.

\paragraph{Separation of sources.}
The skeleton uses cube integers through the \emph{sector yardstick} (a static
scale), while the electron break uses cube integers as \emph{dynamical corrections}
to the recognition boundary's internal structure.  These are physically distinct
roles (see Section~\ref{sec:vocabulary}): the yardstick sets \emph{where} the
sector ladder starts; the break sets \emph{how finely} the electron's boundary
is tuned to the ladder.

\subsection{The electron break} \HYP

\begin{hypothesis}[Electron rung and break]\label{hyp:electron}
The electron sits at rung $r_e = 2$ on the lepton $\phig$-ladder.  The
fine-structure break is:
\begin{equation}\label{eq:delta_e}
  \delta_e \;:=\; 2W \;+\; \frac{W + \Etot}{4\Epass} \;+\; \alpha^2 \;+\; \Etot\,\alpha^3.
\end{equation}
\end{hypothesis}

\begin{remark}[Physical motivation for each term]
The break~\eqref{eq:delta_e} has a four-term structure with clear hierarchical
origin:
\begin{enumerate}[nosep]
  \item $2W = 34$ --- the \textbf{face-symmetry coupling} of the lightest charged
        lepton.  The factor~2 reflects the particle--antiparticle pair (double-entry
        ledger), and $W = 17$ encodes the face crystallography.  This is the dominant
        contribution ($\sim 98\%$ of $\delta_e$).
  \item $(W + \Etot)/(4\Epass) = 29/44 \approx 0.659$ --- a \textbf{geometric
        ratio} quantifying the edge/face balance of the cube.  The numerator
        $W + \Etot = 29$ is the sum of face symmetries and total edges; the
        denominator $4\Epass = 44$ is four times the passive edges.
  \item $\alpha^2 \approx 5.3 \times 10^{-5}$ --- a perturbative
        \textbf{QED self-energy correction} (the recognition boundary interacts
        with its own photon field).
  \item $\Etot\,\alpha^3 = 12\,\alpha^3 \approx 4.7 \times 10^{-6}$ --- a
        higher-order \textbf{QED correction weighted by the total edge count}
        (the photon explores all 12 edges of the cube).
\end{enumerate}

\noindent The cube integers in $\delta_e$ ($W, \Etot, \Epass$) appear as
\emph{dynamical coupling coefficients}---not as sector-scale parameters.
This is the same distinction as between a coupling constant and a mass in
standard field theory: both may involve the same gauge group, but they
encode different physics.
\end{remark}

\subsection{Generation steps} \HYP

\begin{hypothesis}[Generation torsion steps]\label{hyp:generations}
Higher generations are reached by torsion steps (the generation-level
coupling hierarchy of Section~\ref{sec:generations}) plus small $\alpha$-corrections:
\begin{align}
S_{e \to \mu} &= \Epass + \frac{1}{4\pi} - \alpha^2 \approx 11.080, \label{eq:step_emu} \\
S_{\mu \to \tau} &= F - \frac{2W + D}{2} \cdot \alpha \approx 5.866. \label{eq:step_mutau}
\end{align}
\end{hypothesis}

\begin{remark}[Structure of the corrections]
In each generation step, the \emph{leading term} is a pure cube integer
($\Epass = 11$ and $F = 6$, respectively), which is the generation torsion
from Hypothesis~\ref{hyp:torsion}.  The \emph{sub-leading corrections} are
small ($<1\%$) and involve $\alpha$---the fine-structure constant that is
itself derived from the cube (Paper~V).  Thus the generation steps are
controlled by the generation-coupling hierarchy, with QED radiative corrections.

\textbf{Falsifier:} If the predicted mass ratios $m_\mu/m_e$ and $m_\tau/m_\mu$
disagree with PDG values transported to $\mustar$, the generation step
hypotheses are refuted.
\end{remark}

\subsection{Comparison with experiment} \VAL

The full electron mass prediction is:
\begin{equation}
  m_e^{\mathrm{pred}} = \underbrace{A_\ell \cdot \phig^{\,r_e - 8}}_{m_{\mathrm{skel}}(e;\,\mustar)}
  \cdot\; \phig^{\;\gap(1332)\,-\,\delta_e}.
\end{equation}

Higher generations follow from $m_\mu = m_e \cdot \phig^{S_{e\to\mu}}$,
$m_\tau = m_\mu \cdot \phig^{S_{\mu\to\tau}}$.  Numerical comparison against
PDG values transported to $\mustar = 182$~GeV via SM RG flow (4-loop QCD,
2-loop QED; see Paper~IV):

\begin{center}
\renewcommand{\arraystretch}{1.15}
\begin{tabular}{@{}lrrl@{}}
\toprule
Particle & Predicted (MeV) & PDG (MeV) & Rel.\ error \\
\midrule
$e$   & $0.51100$ & $0.51100$ & $\sim -4\times 10^{-7}$ \\
$\mu$ & $105.658$ & $105.658$ & $\sim -1\times 10^{-6}$ \\
$\tau$& $1776.5$  & $1776.9$  & $\sim -9\times 10^{-5}$ \\
\bottomrule
\end{tabular}
\end{center}

\begin{remark}
This comparison is presented as \emph{validation}, not evidence of correctness.
The model may produce the right numbers for the wrong reasons.  The honest test
is whether the \emph{same} structural framework (same integers, same formulas)
also predicts quark masses and mixing angles (Paper~II) without additional
adjustments.
\end{remark}

%=============================================================================
\section{The Integer Budget: A Consistency Cross-Check}
\label{sec:budget}
%=============================================================================

Table~\ref{tab:budget} displays every appearance of the cube vocabulary in
the mass framework, with the \emph{physical role} of each integer explicitly
labelled.  This serves as a consistency audit: the reader can verify that
each integer appears in a structurally distinct capacity, and that
\emph{no integer is available but unused}.

\begin{table}[h]
\centering
\renewcommand{\arraystretch}{1.2}
\begin{tabular}{@{}clll@{}}
\toprule
\textbf{Integer} & \textbf{Value} & \textbf{Physical role} & \textbf{Equation} \\
\midrule
\multicolumn{4}{@{}l}{\textit{Temporal structure (Section~\ref{sec:foundation})}} \\
$V$ & $8$ & Eight-tick period; octave reference $(-8)$ & \eqref{eq:master} \\
\midrule
\multicolumn{4}{@{}l}{\textit{Sector classification (Section~\ref{sec:sector})}} \\
$\Epass$ & $11$ & $B_{\text{pow}}(\ell) = -2\Epass$ & \eqref{eq:yardstick} \\
$A$      & $1$  & $B_{\text{pow}}(u) = -A$;\; $B_{\text{pow}}(\text{EW}) = +A$ & \eqref{eq:yardstick} \\
$\Etot$  & $12$ & $B_{\text{pow}}(d) = 2\Etot - 1$ & \eqref{eq:yardstick} \\
$W$      & $17$ & $r_0(\ell) = 4W{-}F$;\; $r_0(u) = 2W{+}A$ & \eqref{eq:yardstick} \\
$F$      & $6$  & $r_0(\ell) = 4W{-}F$;\; $r_0(d) = \Etot{-}W$ & \eqref{eq:yardstick} \\
\midrule
\multicolumn{4}{@{}l}{\textit{Generation dynamics (Section~\ref{sec:generations})}} \\
$\Epass$ & $11$ & Gen-2 torsion $\tau_2 = \Epass$ & \eqref{eq:step_emu} \\
$F$      & $6$  & Gen-3 step $\Delta_{2\to 3} = F$ & \eqref{eq:step_mutau} \\
\midrule
\multicolumn{4}{@{}l}{\textit{Electron fine structure (Section~\ref{sec:leptons})}} \\
$W$      & $17$ & Pair-entry coupling: $2W$ & \eqref{eq:delta_e} \\
$\Etot$  & $12$ & Edge balance: $(\!W{+}\Etot\!)/(4\Epass)$;\; QED: $\Etot\alpha^3$ & \eqref{eq:delta_e} \\
$\Epass$ & $11$ & Edge balance denominator: $4\Epass$ & \eqref{eq:delta_e} \\
\midrule
\multicolumn{4}{@{}l}{\textit{Coupling constants (Paper~V)}} \\
$\Epass$ & $11$ & $\alpha^{-1}$ seed: $4\pi\cdot\Epass$ & Paper~V \\
$F\!\times\! W$ & $102$ & Curvature correction: $103/(102\pi^5)$ & Paper~V \\
\bottomrule
\end{tabular}
\caption{Complete integer budget.  Each cube integer appears in multiple
formulas but in a \emph{distinct physical capacity}: static sector scales,
dynamical generation coupling, fine-structure corrections, and coupling constants.
The budget is exhaustive: every cube element is used; none is left over.}
\label{tab:budget}
\end{table}

\begin{remark}[The falsification test]
The framework would fail if any prediction required an integer \emph{not}
in the cube vocabulary $\{V,E,F,A,W\}$ (indicating missing structure),
or if a cube integer were consistently absent from all formulas (indicating
extraneous structure).  Neither case occurs: the vocabulary is exactly
sufficient and exactly saturated.
\end{remark}

%=============================================================================
\section{Loss Aversion Asymmetry from $\Jcost$}
\label{sec:prospect}
%=============================================================================

\begin{remark}[Prospect theory connection] \PROVED
The $\Jcost$ metric provides a natural loss-gain asymmetry: for $x > 1$ (gain), $\Jcost''(x) = x^{-3}$ is small (shallow curvature), while for $0 < x < 1$ (loss), $\Jcost''(x) = x^{-3}$ is large (steep curvature).  This asymmetry is a \emph{theorem} of the cost structure, not a hypothesis.
\end{remark}

%=============================================================================
\section{Falsification Criteria}
%=============================================================================

The structural hypothesis is falsifiable at multiple levels:

\begin{enumerate}[nosep]
\item \textbf{Wrong mass ratios.} If transporting experimental masses to $\mustar$ fails to match the $\phig$-ladder predictions, the master law (Hypothesis~\ref{hyp:master}) is falsified.
\item \textbf{Wrong $\Z$-bands.} If SM fermions do not cluster into the predicted charge bands, the $\Z$-map (Hypothesis~\ref{hyp:Zmap}) is falsified.
\item \textbf{Wrong $\mustar$.} If future high-precision SM calculations shift $\mustar$ significantly, the stationarity criterion (Hypothesis~\ref{hyp:anchor}) is falsified.
\item \textbf{Cross-sector failure.} If the same integers $(11, 17, 6)$ fail to predict quark masses and mixing angles, the structural hypothesis as a whole is falsified.  This is the sharpest test.
\item \textbf{Fourth generation.} If a fourth generation of fermions is discovered, the three-generation structure of the torsion steps must accommodate it or be falsified.
\end{enumerate}

%=============================================================================
\section{Open Problems}
\label{sec:open}
%=============================================================================

We distinguish problems whose resolution would \emph{change predictions}
(high priority) from those that would \emph{strengthen the derivation}
without changing any number (structural priority).

\paragraph{High priority (predictions may sharpen).}
\begin{enumerate}[label=(O\arabic*),nosep]
\item \textbf{Derive the $\Z$-map polynomial.}  Show that
      $\widetilde{Q}^2 + \widetilde{Q}^4$ is the unique polynomial
      compatible with ledger charge conservation.  This would upgrade
      Hypothesis~\ref{hyp:Zmap} and remove the most ``hand-chosen''
      element.
\item \textbf{Derive the generation-step corrections.}  The leading
      terms ($\Epass$ and $F$) follow from the coupling hierarchy
      (Hypothesis~\ref{hyp:torsion}); the sub-leading $\alpha$-corrections
      in~\eqref{eq:step_emu}--\eqref{eq:step_mutau} need a more
      principled origin.
\item \textbf{Compute $\mustar$ from RS.}  Derive the anchor scale
      without relying on external SM running.
\end{enumerate}

\paragraph{Structural priority (derivation strengthens, numbers unchanged).}
\begin{enumerate}[resume,label=(O\arabic*),nosep]
\item \textbf{Derive $W = 17$ from the voxel.}  Show that the plane
      crystallographic count enters via the face-symmetry group of the
      cubic ledger, not as an external mathematical fact.
\item \textbf{Derive the $r_0$ formulas.}  The $B_{\text{pow}}$ values
      are now constrained by charge ordering
      (Hypothesis~\ref{hyp:sector_constraints}); the $r_0$ values
      (e.g., $4W - F = 62$ for leptons) await an admissibility derivation.
      The constraint system is over-determined (8 outputs from 5 inputs),
      which limits the space of possible solutions, but a complete proof
      of uniqueness remains open.
\item \textbf{Derive $\Ecoh = \phig^{-5}$.}  Connect the exponent $-5$
      to a structural property.
\item \textbf{Derive the electron break.}  Show that the dominant term
      $2W$ in $\delta_e$ follows from the face-crystallographic coupling
      of the lightest charged boundary.
\end{enumerate}

Each solved problem upgrades the corresponding hypothesis to a theorem.
Importantly, the \emph{numerical predictions} do not change: the current
integers are already fixed by the constraint analysis
(Section~\ref{sec:sector}) and the generation coupling hierarchy
(Section~\ref{sec:generations}).  What changes is the \emph{logical
status}---from structural hypothesis to derived consequence.

%=============================================================================
\section{Discussion}
%=============================================================================

\subsection*{What this paper does and does not claim}

We \emph{do not} claim to have derived particle masses from first principles in the sense of a complete logical chain from the RS axioms.  What we present is a \textbf{structural model}: a single formula~\eqref{eq:master} with integer inputs from the 3-cube that \emph{predicts} the charged-lepton mass hierarchy without per-particle fitting.

The honest status is:
\begin{itemize}[nosep]
\item The \emph{mathematical foundation} (Sections~\ref{sec:foundation}) is proved.
\item The \emph{structural hypothesis} (Sections~\ref{sec:hypothesis}--\ref{sec:anchor}) is falsifiable and uses no arbitrary parameters, but contains assumptions (notably W=17, the sector formulas, and the $\Z$-map) that are not yet derived.
\item The \emph{validation} (Section~\ref{sec:leptons}) shows numerical consistency with PDG data.
\end{itemize}

The value of this model is not that it is ``proved from axioms''---it is not.  The value is that it \emph{organises} the mass hierarchy as a structured output of a small number of geometric integers, with explicit falsifiers and a clear program for closing the remaining derivation gaps.

\subsection*{Addressing the ``same integers used twice'' objection}

A natural objection is that the cube integers $\{8,11,12,17,6\}$ appear
in both the sector yardsticks and the generation steps / electron break,
giving the impression of ad~hoc fitting.  Three responses are in order:

\begin{enumerate}[nosep]
\item \textbf{Complete vocabulary.}
      The 3-cube provides \emph{exactly} these integers and no others
      (Section~\ref{sec:vocabulary}).  Any zero-parameter formula depending
      on discrete $D = 3$ geometry \emph{must} use them.  The objection
      amounts to asking why the cube has only one set of combinatorial
      elements---which is a mathematical fact, not a modelling choice.

\item \textbf{Distinct physical roles.}
      Table~\ref{tab:budget} shows that each integer appearance serves a
      labelled physical function (static scale, dynamical coupling,
      QED correction).  The roles are as distinct as the mass and coupling
      constant of the electron in QED---both involve $e$, but for different
      physical reasons.

\item \textbf{Over-determination.}
      Eight yardstick parameters are fixed from five independent inputs
      (Section~\ref{sec:sector}).  Adding four generation parameters
      and the electron break yields $>$\,12 outputs from the same five
      inputs---a highly over-determined system.  The fact that a consistent
      solution exists at all, reproducing sub-ppm lepton masses, CKM,
      and PMNS from only five cube integers, is the primary evidence.
      Fitting five free parameters to 12+ observables would require
      $<\!1\%$ chance of accidental agreement.
\end{enumerate}

\subsection*{Implications}

If the open problems can be solved---if $W = 17$, the sector formulas, and the $\Z$-map can all be derived from the RS axioms---then the Standard Model's 20+ free parameters reduce to \emph{zero}.  The Flavour Puzzle would be resolved not by hidden symmetries or landscape statistics, but by the combinatorics of a cube.

%=============================================================================
\section{Lean Formalisation}
%=============================================================================

\begin{center}
\renewcommand{\arraystretch}{1.15}
\begin{tabular}{@{}lll@{}}
\toprule
\textbf{Module} & \textbf{Content} & \textbf{Status} \\
\midrule
\texttt{Cost.lean} + \texttt{CostUniqueness.lean} & $\Jcost$ uniqueness (T5) & \PROVED \\
\texttt{Masses.Anchor} & Sector yardsticks, $B_{\text{pow}}$, $r_0$ & Arithmetic verified \\
\texttt{Masses.MassLaw} & Master formula, rung scaling & Structural consequence \\
\texttt{Masses.Assumptions} & Ladder assumption (explicit) & \HYP \\
\texttt{Masses.Verification} & Consistency checks & Arithmetic verified \\
\bottomrule
\end{tabular}
\end{center}

\begin{remark}
The Lean code verifies the \emph{arithmetic} (e.g., $4 \times 17 - 6 = 62$) and \emph{structural consequences} (e.g., rung scaling).  It does \emph{not} verify the physical content of the hypotheses---that is the role of experiment and future derivations.  The \texttt{Masses.Assumptions} module is honest about this: it explicitly declares the ladder assumption as a \texttt{Prop} awaiting proof.
\end{remark}

\begin{thebibliography}{9}
\bibitem{WashburnCost2026}
J.~Washburn and M.~Zlatanovi\'{c},
``Uniqueness of the Canonical Reciprocal Cost,''
arXiv:2602.05753v1, 2026.

\bibitem{WashburnPenrose2026}
S.~Pardo-Guerra,
``The Golden Ratio as a Universal Coherence Eigenvalue,''
Recognition Science preprint, 2026.

\bibitem{WashburnD3}
J.~Washburn, M.~Zlatanovi\'{c}, and E.~Allahyarov,
``Dimensional Rigidity: D=3,''
Recognition Science preprint, 2026.

\bibitem{WashburnMixing2026}
J.~Washburn,
``CKM and PMNS Mixing from Cubic Ledger Topology,''
Recognition Science preprint, 2026.

\bibitem{RunDec}
K.~G.~Chetyrkin, J.~H.~K\"{u}hn, and M.~Steinhauser,
``RunDec: A Mathematica package for running and decoupling,''
\textit{Comput.\ Phys.\ Commun.}\ 133 (2000) 43.
\end{thebibliography}

\end{document}
