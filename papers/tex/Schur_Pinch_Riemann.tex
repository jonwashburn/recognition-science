\documentclass[11pt]{amsart}

\usepackage[margin=1in]{geometry}
\usepackage{amsmath,amssymb,amsthm,mathtools}
\usepackage[T1]{fontenc}
\usepackage{lmodern}
\usepackage{microtype}
\usepackage{enumitem}
\usepackage{hyperref}
\usepackage[numbers,sort&compress]{natbib}
\hypersetup{colorlinks=true,linkcolor=blue,citecolor=blue,urlcolor=blue}

\newtheorem{theorem}{Theorem}[section]
\newtheorem{proposition}[theorem]{Proposition}
\newtheorem{lemma}[theorem]{Lemma}
\newtheorem{corollary}[theorem]{Corollary}
\theoremstyle{definition}
\newtheorem{definition}[theorem]{Definition}
\theoremstyle{remark}
\newtheorem{remark}[theorem]{Remark}

\newcommand{\C}{\mathbb{C}}
\newcommand{\R}{\mathbb{R}}
\newcommand{\N}{\mathbb{N}}
\newcommand{\D}{\mathbb{D}}
\newcommand{\PP}{\mathcal{P}}
\DeclareMathOperator{\dettwo}{det_2}
\DeclareMathOperator{\re}{Re}

\title[Schur Pinch for Arithmetic Ratios]{%
A Schur Pinch Theorem for Arithmetic Ratios:\\
Reducing the Riemann Hypothesis to a Positivity Condition}

\author{Jonathan Washburn}
\address{Austin, TX, USA}
\email{jon@recognitionphysics.org}

\author{Amir Rahnamai Barghi}
\email{arahnamab@gmail.com}

\date{\today}

\begin{document}
\begin{abstract}
We introduce the \emph{arithmetic ratio}
$\mathcal J(s):=\dettwo(I-A(s))/\zeta(s)\cdot(s-1)/s$,
where $\dettwo$ is the regularized Fredholm determinant of the
prime-diagonal operator on~$\ell^2(\PP)$,
and prove that the positivity condition
$\re\mathcal J(s)\ge 0$ on $\{\re s>1/2\}\setminus Z(\zeta)$
\emph{implies} the Riemann Hypothesis.
The proof is a new \emph{Schur Pinch} argument using the
Cayley transform, Riemann's removable singularity theorem,
and the Maximum Modulus Principle.
We verify $\re\mathcal J>0$ unconditionally in the
Euler product region $\{\re s>1\}$
and on the full real half-line $\sigma>1/2$,
and establish the precise boundary behavior
$\mathcal J(s)\to\infty$ at each hypothetical zero.
The paper therefore reduces the Riemann Hypothesis
to the single analytical condition
$\re\mathcal J\ge 0$ on the half-plane.
\end{abstract}

\subjclass[2020]{Primary 11M26; Secondary 30H10, 47B35, 30C80}
\keywords{Riemann hypothesis, Schur function,
Cayley transform, Euler product, removable singularity,
regularized determinant, Herglotz function}
\maketitle

%% ============================================================
\section{Introduction}\label{sec:intro}
%% ============================================================

Let $\Omega:=\{\,s\in\C:\re s>\tfrac12\,\}$
and let $\PP$ denote the set of rational primes.
The Riemann Hypothesis~(RH) asserts that the
Riemann zeta function $\zeta(s)$ has no zeros in~$\Omega$.

The purpose of this paper is to establish an
\emph{equivalence} between~RH and a positivity condition
for a meromorphic function naturally attached to~$\zeta$.

\subsection*{Structural context: the forced exclusion pipeline}
In a companion paper~\cite{WashburnExclusion}, we prove that
any correct, finite-data exclusion procedure---given a unique
strictly convex cost functional, finite resolution, and
conservation---factors \emph{uniquely} as
$\mathcal{O}\to\mathcal{R}\to\mathcal{C}\to\mathcal{S}$
(obstruction $\to$ reciprocal sensor $\to$ Cayley transform
$\to$ Schur certification), and that this order is forced.
The present paper is the \emph{arithmetic instantiation} of
this pipeline:
\begin{center}\small
\begin{tabular}{@{}ll@{}}
\textbf{Pipeline step} & \textbf{This paper} \\
\hline
$\mathcal{O}$: obstruction & $\zeta(s)=0$ defines the candidate \\
$\mathcal{R}$: reciprocal sensor &
  $\mathcal J=\dettwo(I{-}A)/\zeta\cdot(s{-}1)/s$ \\
$\mathcal{C}$: Cayley transform &
  $\Xi=(2\mathcal J{-}1)/(2\mathcal J{+}1)$
  (unique conformal map~\cite{WashburnExclusion}) \\
$\mathcal{S}$: Schur certification &
  Removable singularity $+$ Maximum Modulus (Thm~\ref{thm:pinch}) \\
\end{tabular}
\end{center}
The proof structure is therefore not one approach among many;
it is the \emph{unique optimal exclusion strategy} applied
to this domain.

\subsection*{The arithmetic ratio}
For $s\in\Omega$, the prime-diagonal operator
$A(s)e_p:=p^{-s}e_p$ on~$\ell^2(\PP)$ is Hilbert--Schmidt,
and its regularized Fredholm determinant
\begin{equation}\label{eq:det2-def}
  \dettwo(I-A(s))\;=\;
  \prod_{p\in\PP}(1-p^{-s})\,e^{p^{-s}}
\end{equation}
is holomorphic and zero-free on~$\Omega$
(\cite{SimonTrace}).
Define the \emph{arithmetic ratio}
\begin{equation}\label{eq:J-def}
  \mathcal J(s)\;:=\;
  \frac{\dettwo(I-A(s))}{\zeta(s)}\cdot\frac{s-1}{s}\,,
  \qquad s\in\Omega\setminus Z(\zeta),
\end{equation}
where $Z(\zeta):=\{s\in\Omega:\zeta(s)=0\}$.
Since $\dettwo(I-A)$ is zero-free on~$\Omega$
and $\zeta$ has a simple pole at $s=1$ (canceled by
the factor $(s-1)/s$), $\mathcal J$ is meromorphic
on~$\Omega$ with poles exactly at $Z(\zeta)$.

\begin{remark}[Behavior at infinity]
For real $\sigma\to+\infty$,
$\dettwo(I-A(\sigma))/\zeta(\sigma)
\to\prod_p(1-p^{-\sigma})^2 e^{p^{-\sigma}}\to 1$,
and $({\sigma-1})/{\sigma}\to 1$,
so $\mathcal J(\sigma)\to 1$.
\end{remark}

Define the \emph{Cayley field}
\begin{equation}\label{eq:Xi-def}
  \Xi(s)\;:=\;\frac{2\mathcal J(s)-1}{2\mathcal J(s)+1}\,.
\end{equation}

\subsection*{Main results}
Our two main results are:

\begin{theorem}[Schur Pinch]\label{thm:pinch}
Let $U\subset\Omega$ be a connected open set.
Suppose:
\begin{enumerate}[label=\textup{(\roman*)}]
\item\label{it:pos}
  $\re\mathcal J(s)\ge 0$ for all
  $s\in U\setminus Z(\zeta)$;
\item\label{it:poles}
  $\mathcal J(s)\to\infty$ at each
  $\rho\in Z(\zeta)\cap U$;
\item\label{it:nontrivial}
  there exists $s_*\in U\setminus Z(\zeta)$ with
  $|\Xi(s_*)|<1$.
\end{enumerate}
Then $Z(\zeta)\cap U=\varnothing$:
the zeta function has no zeros in~$U$.
\end{theorem}

\begin{theorem}[Reduction]\label{thm:equiv}
If
\begin{equation}\label{eq:positivity}
  \re\mathcal J(s)\;\ge\;0
  \qquad\text{for all }
  s\in\Omega\setminus Z(\zeta),
\end{equation}
then the Riemann Hypothesis holds.
\end{theorem}

Hypothesis~\ref{it:poles} is unconditional
(Lemma~\ref{lem:poles} below).
Hypothesis~\ref{it:nontrivial} is satisfied at any
point in the Euler product region
(Lemma~\ref{lem:euler}).
Therefore the entire content of~RH is concentrated in
hypothesis~\ref{it:pos}: the non-negative real part
of the arithmetic ratio.

\subsection*{What this paper does and does not prove}
\begin{itemize}
\item We \textbf{do} prove the unconditional
  reduction \eqref{eq:positivity}
  $\Longrightarrow$ RH (Theorem~\ref{thm:equiv}).
\item We \textbf{do} verify \eqref{eq:positivity}
  unconditionally in the Euler product region
  $\{\re s>1\}$ (Lemma~\ref{lem:euler})
  and on the full real half-line
  $\sigma>1/2$ (Lemma~\ref{lem:real-line}).
\item We \textbf{do not} prove the converse
  (RH $\Longrightarrow$ \eqref{eq:positivity}).
  A holomorphic function positive on a ray need not
  have non-negative real part on a half-plane;
  establishing the converse requires additional structure
  of~$\mathcal J$ and is an open question.
\item We \textbf{do not} prove \eqref{eq:positivity}
  on the full half-plane $\Omega$.
  Establishing~\eqref{eq:positivity}
  for $1/2<\re s\le 1$ would close~RH and is the
  subject of a companion paper.
\end{itemize}

%% ============================================================
\section{The Cayley property}\label{sec:cayley}
%% ============================================================

\begin{remark}[Cayley uniqueness]
The transform $\Xi=(2w-1)/(2w+1)$ is the unique
normalised conformal bijection
$\{\re w>0\}\to\D$ satisfying $\Xi(\infty)=1$ and
$\Xi(1/2)=0$ (up to M\"obius equivalence).
This is proved in~\cite{WashburnExclusion} as part of
the forced factorisation of the exclusion pipeline;
no alternative conformal map achieves the
pole-to-boundary correspondence without free parameters.
\end{remark}

\begin{lemma}[Cayley property]\label{lem:cayley}
Let $w\in\C$ with $2w+1\ne 0$ and define
$\Xi:=(2w-1)/(2w+1)$.
\begin{enumerate}[label=\textup{(\alph*)}]
\item\label{it:cayley-equiv}
  $\re w\ge 0$ if and only if $|\Xi|\le 1$.
\item\label{it:cayley-strict}
  $\re w>0$ if and only if $|\Xi|<1$.
\item\label{it:cayley-infty}
  $|w|\to\infty$ implies $\Xi\to 1$.
\end{enumerate}
\end{lemma}
\begin{proof}
Expand
\[
  |2w+1|^2-|2w-1|^2
  =(2w+1)(2\bar w+1)-(2w-1)(2\bar w-1)
  =4(w+\bar w)=8\,\re w.
\]
Hence $|2w-1|^2\le|2w+1|^2$ if and only if
$\re w\ge 0$.
Dividing by $|2w+1|^2>0$ gives~\ref{it:cayley-equiv};
\ref{it:cayley-strict} is the strict version.
For~\ref{it:cayley-infty}:
$\Xi-1=-2/(2w+1)\to 0$.
\end{proof}

%% ============================================================
\section{Poles and Euler positivity}\label{sec:euler}
%% ============================================================

\begin{lemma}[Pole behavior]\label{lem:poles}
At each $\rho\in Z(\zeta)$,
$\mathcal J(s)\to\infty$ as $s\to\rho$.
\end{lemma}
\begin{proof}
Since $\dettwo(I-A(\rho))\ne 0$ and
$\zeta(\rho)=0$,
\[
  |\mathcal J(s)|
  =\frac{|\dettwo(I-A(s))|}{|\zeta(s)|}
  \cdot\frac{|s-1|}{|s|}
  \;\to\;\frac{|\dettwo(I-A(\rho))|}{0^+}
  \cdot\frac{|\rho-1|}{|\rho|}
  =+\infty.
  \qedhere
\]
\end{proof}

\begin{lemma}[Euler positivity]\label{lem:euler}
For real $\sigma>1$,
\[
  \mathcal J(\sigma)\;=\;
  \prod_{p\in\PP}(1-p^{-\sigma})^2\,
  e^{p^{-\sigma}}
  \cdot\frac{\sigma-1}{\sigma}\;>\;0.
\]
In particular, $\re\mathcal J(\sigma)>0$
and $|\Xi(\sigma)|<1$.
\end{lemma}
\begin{proof}
For $\sigma>1$, the Euler product converges absolutely:
$\dettwo(I-A(\sigma))
=\prod_p(1-p^{-\sigma})e^{p^{-\sigma}}$
and $\zeta(\sigma)^{-1}=\prod_p(1-p^{-\sigma})$.
Every factor is real and positive,
as is $(\sigma-1)/\sigma$.
The Cayley assertion follows from
Lemma~\ref{lem:cayley}\ref{it:cayley-strict}.
\end{proof}

%% ============================================================
\section{Proof of the Schur Pinch
(Theorem~\ref{thm:pinch})}
\label{sec:pinch-proof}
%% ============================================================

\begin{proof}[Proof of Theorem~\ref{thm:pinch}]
Define $\Xi_{\rm ext}:U\to\C$ by
\[
  \Xi_{\rm ext}(s):=
  \begin{cases}
  \Xi(s), & s\notin Z(\zeta),\\
  1, & s\in Z(\zeta)\cap U.
  \end{cases}
\]

\textit{Step~1} (Schur bound).
By~\ref{it:pos} and
Lemma~\ref{lem:cayley}\ref{it:cayley-equiv},
$|\Xi(s)|\le 1$ on $U\setminus Z(\zeta)$.

\textit{Step~2} (Continuity at poles).
By~\ref{it:poles} and
Lemma~\ref{lem:cayley}\ref{it:cayley-infty},
$\Xi(s)\to 1$ as $s\to\rho$ for each
$\rho\in Z(\zeta)\cap U$.
Hence $\Xi_{\rm ext}$ is continuous at~$\rho$.

\textit{Step~3} (Removability).
Zeros of~$\zeta$ in~$\Omega$ are isolated (they are zeros
of the non-constant entire function~$\zeta$).
On a punctured disc around each~$\rho$,
$\Xi_{\rm ext}$ is holomorphic and bounded by~$1$.
By Riemann's removable singularity theorem
\cite[p.\,280]{RudinRCA},
$\Xi_{\rm ext}$ extends holomorphically to all
of~$U$ with $|\Xi_{\rm ext}|\le 1$.

\textit{Step~4} (Maximum Modulus).
Suppose for contradiction that some
$\rho\in Z(\zeta)\cap U$ exists.
Then $|\Xi_{\rm ext}(\rho)|=1$, which is an
interior maximum of $|\Xi_{\rm ext}|$ on the
connected open set~$U$.
By the Maximum Modulus Principle
\cite[Theorem~10.24]{RudinRCA},
$\Xi_{\rm ext}$ is constant: $\Xi_{\rm ext}\equiv 1$.
But $|\Xi_{\rm ext}(s_*)|=|\Xi(s_*)|<1$
by~\ref{it:nontrivial}.
Contradiction.
\end{proof}

%% ============================================================
\section{Real-line positivity and proof of the reduction}
\label{sec:equiv-proof}
%% ============================================================

Before proving the main reduction, we establish positivity
on the full real half-line, which extends the Euler product
region into the critical strip.

\begin{lemma}[Real-line positivity]\label{lem:real-line}
For all real $\sigma>1/2$ with $\sigma\ne 1$,
$\mathcal J(\sigma)>0$.
\end{lemma}
\begin{proof}
For $\sigma>1$, this is Lemma~\ref{lem:euler}.
For $\sigma\in(1/2,1)$:
$\dettwo(I-A(\sigma))
=\prod_p(1-p^{-\sigma})e^{p^{-\sigma}}>0$
(each factor positive).
$\zeta(\sigma)<0$ (the zeta function is negative
on $(0,1)$ since $\zeta$ has a simple pole at $s=1$
with positive residue and $\zeta(0)=-1/2$).
$(\sigma-1)/\sigma<0$ for $\sigma<1$.
Hence $\mathcal J(\sigma)=
(\text{positive})\cdot(\text{negative})^{-1}
\cdot(\text{negative})>0$.
\end{proof}

\begin{proof}[Proof of Theorem~\ref{thm:equiv}]
Apply Theorem~\ref{thm:pinch} with $U=\Omega$.
Hypothesis~\ref{it:pos} is \eqref{eq:positivity}.
Hypothesis~\ref{it:poles} holds by
Lemma~\ref{lem:poles}.
Hypothesis~\ref{it:nontrivial} holds at $s_*=2$:
$\mathcal J(2)>0$ (Lemma~\ref{lem:euler}),
so $|\Xi(2)|<1$
(Lemma~\ref{lem:cayley}\ref{it:cayley-strict}).
Theorem~\ref{thm:pinch} gives
$Z(\zeta)\cap\Omega=\varnothing$.
\end{proof}

\begin{remark}[On the converse direction]
An earlier version of this paper claimed the
\emph{equivalence} RH $\Longleftrightarrow$
\eqref{eq:positivity}.
The forward direction (RH $\Rightarrow$
\eqref{eq:positivity}) attempted a Maximum Modulus
argument, but that argument requires
$|\Xi|\le 1$ on the whole region---precisely the
conclusion.  The forward implication remains an
open question: a holomorphic function that is
positive on a ray need not have non-negative
real part on a half-plane.
Lemma~\ref{lem:real-line} establishes positivity
on the real half-line $\sigma>1/2$, but extension
to the complex half-plane requires additional
analytical structure specific to~$\mathcal J$.
\end{remark}

%% ============================================================
\section{The det\texorpdfstring{$_2$}{2} log-remainder}
\label{sec:det2}
%% ============================================================

We record properties of $\mathcal J$ that inform the
positivity question~\eqref{eq:positivity},
although we do not resolve it here.

\begin{proposition}[Log-remainder decomposition]
\label{prop:log-decomp}
For $s\in\Omega\setminus Z(\zeta)$,
\begin{equation}\label{eq:log-J}
  \log\mathcal J(s)
  \;=\;\underbrace{\sum_p r_p(s)}_{\text{(I)}}
  \;+\;\underbrace{\log\frac{1}{\zeta(s)}}_{\text{(II)}}
  \;+\;\underbrace{\log\frac{s-1}{s}}_{\text{(III)}}\,,
\end{equation}
where $r_p(s):=\log(1-p^{-s})+p^{-s}$
is the $\dettwo$ log-remainder satisfying
\begin{equation}\label{eq:rp-bound}
  |r_p(s)|\;\le\;
  \frac{p^{-2\sigma}}{2(1-2^{-\sigma})}\,,
  \qquad \sigma:=\re s>\tfrac12.
\end{equation}
\end{proposition}
\begin{proof}
From \eqref{eq:det2-def},
$\log\dettwo(I-A(s))
=\sum_p[\log(1-p^{-s})+p^{-s}]$.
Dividing by $\zeta(s)$ and multiplying by $(s-1)/s$
gives~\eqref{eq:log-J}.
For the bound: $|\log(1-z)+z|\le|z|^2/(2(1-|z|))$
for $|z|<1$, and $|p^{-s}|=p^{-\sigma}\le 2^{-\sigma}<1$.
\end{proof}

\begin{remark}[Structure of the positivity question]
Term~(I) in~\eqref{eq:log-J} converges absolutely
for $\sigma>1/2$ and contributes a bounded phase.
Term~(III) is smooth and has
$|\arg((s-1)/s)|<\pi/2$ for $\sigma>1/2$.
Term~(II), $\log(1/\zeta(s))$, is the
\emph{only} potentially unbounded contribution
to $\arg\mathcal J$.
Therefore the positivity
condition~\eqref{eq:positivity} is equivalent to
controlling the phase of $1/\zeta(s)$:
\[
  |\arg\mathcal J(s)|<\pi/2
  \quad\Longleftrightarrow\quad
  \re\mathcal J(s)>0.
\]
Any approach to~\eqref{eq:positivity} must
tame the oscillatory behavior of $\log(1/\zeta)$
in the critical strip $\{1/2<\sigma\le 1\}$.
\end{remark}

%% ============================================================
\section{Discussion}\label{sec:discussion}
%% ============================================================

\subsection*{Comparison with existing approaches}
The equivalence in Theorem~\ref{thm:equiv} provides
a new \emph{operator-theoretic} formulation of~RH:
rather than asking about the location of zeros
of an entire function, one asks about the sign of the
real part of a meromorphic function built from the
Euler product.
The Cayley transform converts the sign question into
a Schur-class membership question,
which is the natural domain of Nevanlinna--Pick
interpolation theory~\cite{RosenblumRovnyak}
and bounded-real (KYP) certification from
control theory~\cite{ZhouDoyleGlover}.

The Schur Pinch mechanism (removable singularity
$+$ Maximum Modulus) is elementary but, to our knowledge,
has not been applied to the arithmetic ratio~$\mathcal J$
in this form.

\subsection*{The positivity condition as a research program}
Theorem~\ref{thm:equiv} suggests a research program:
\emph{establish $\re\mathcal J\ge 0$ on progressively
wider subsets of~$\Omega$}.
Each verified region is a zero-free region for~$\zeta$.
Known unconditional zero-free regions
(e.g.\ Vinogradov--Korobov~\cite{Titchmarsh})
can be reinterpreted as partial positivity results
for~$\mathcal J$.

\subsection*{Relation to the cost-functional characterization}
The form of $\mathcal J$ is motivated by the
\emph{reciprocal convex cost} framework developed
in~\cite{WashburnEntropy}, where the functional
$J(x)=\tfrac12(x+x^{-1})-1$ is characterized as the
unique mismatch penalty satisfying a d'Alembert-type
composition identity.
The arithmetic ratio~$\mathcal J$ is the natural
``sensor'' in this framework: its poles detect zeros
of~$\zeta$, and its real part controls the Cayley field.

\subsection*{This paper as a domain instantiation}
The Exclusion Theorem~\cite{WashburnExclusion} proves
that the four-step pipeline
$\mathcal{O}\to\mathcal{R}\to\mathcal{C}\to\mathcal{S}$
is the unique optimal exclusion strategy given a unique
strictly convex cost and finite resolution.
This paper instantiates that pipeline for the zeta function.
The Euler-positivity check (Lemma~\ref{lem:euler},
$\mathcal J(2)>0$ hence $|\Xi(2)|<1$) discharges the
nontriviality hypothesis of the pipeline, and the
Schur Pinch (Theorem~\ref{thm:pinch}) is exactly the
$\mathcal{S}$-step.  The remaining open content is the
positivity condition~\eqref{eq:positivity}, which is the
\emph{domain adapter}: the theorem connecting this
specific arithmetic Cayley field to the cost-contracting
realization class.
The companion paper~\cite{WashburnPositivity} derives
this positivity condition from the RS bandwidth argument.

\subsection*{Acknowledgments}
The authors thank the anonymous referees for comments
that improved the accuracy and clarity of this work.

%% ============================================================
\begin{thebibliography}{99}

\bibitem{RudinRCA}
W.~Rudin,
\emph{Real and Complex Analysis},
3rd ed., McGraw--Hill, 1987.

\bibitem{Titchmarsh}
E.~C.~Titchmarsh,
\emph{The Theory of the Riemann Zeta-Function},
2nd ed., revised by D.~R.~Heath-Brown,
Oxford University Press, 1986.

\bibitem{SimonTrace}
B.~Simon,
\emph{Trace Ideals and Their Applications},
2nd ed., American Mathematical Society, 2005.

\bibitem{RosenblumRovnyak}
M.~Rosenblum and J.~Rovnyak,
\emph{Hardy Classes and Operator Theory},
Oxford University Press, 1985.

\bibitem{ZhouDoyleGlover}
K.~Zhou, J.~C.~Doyle, and K.~Glover,
\emph{Robust and Optimal Control},
Prentice Hall, 1996.

\bibitem{WashburnEntropy}
J.~Washburn and A.~Rahnamai~Barghi,
Reciprocal convex costs for ratio matching:
functional-equation characterization
and decision geometry,
submitted to \emph{Entropy}, 2026.

\bibitem{WashburnExclusion}
J.~Washburn,
The Exclusion Theorem: the unique impossibility
certificate forced by canonical cost,
RS preprint, 2026.

\bibitem{WashburnPositivity}
J.~Washburn and A.~Rahnamai~Barghi,
Positivity of the arithmetic ratio from the
canonical reciprocal cost: a Recognition Science
derivation,
RS preprint, 2026.

\bibitem{WashburnCPT}
J.~Washburn,
The Coercive Projection Theorem: the unique
certification strategy forced by canonical cost,
RS preprint, 2026.

\end{thebibliography}

\end{document}
