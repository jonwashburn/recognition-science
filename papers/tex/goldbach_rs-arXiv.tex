\documentclass[11pt]{article}
\usepackage[utf8]{inputenc}
\usepackage[T1]{fontenc}
\usepackage{lmodern}
\usepackage{amsmath,amssymb,amsthm}
\usepackage{geometry}
\usepackage{booktabs}
\usepackage{enumitem}
\usepackage{xcolor}
\usepackage[hidelinks]{hyperref}
\geometry{margin=1in}

\title{Goldbach via a Mod-8 Kernel:\\
Density-One Positivity, Short-Interval Bounds,\\
and a Conditional Dispersion Improvement}
\author{Jonathan Washburn\\Recognition Physics Institute}
\date{February 2026}

% Theorem environments
\newtheorem{theorem}{Theorem}[section]
\newtheorem{lemma}[theorem]{Lemma}
\newtheorem{proposition}[theorem]{Proposition}
\newtheorem{corollary}[theorem]{Corollary}
\newtheorem{conjecture}[theorem]{Conjecture}
\theoremstyle{definition}
\newtheorem{definition}[theorem]{Definition}
\theoremstyle{remark}
\newtheorem{remark}[theorem]{Remark}

% Claim hygiene markers
\newcommand{\PROVED}{\textcolor{blue!70!black}{\textsf{[PROVED]}}}
\newcommand{\COND}{\textcolor{orange!80!black}{\textsf{[CONDITIONAL]}}}
\newcommand{\CONJ}{\textcolor{red!70!black}{\textsf{[CONJECTURE]}}}

\begin{document}
\maketitle

\begin{abstract}
We develop a classical circle-method framework for Goldbach's conjecture
using a mod-8 periodic kernel $K_8$ that preserves the natural residue
structure and isolates the $2$-adic local factor.
On major arcs we obtain a positive main term
$(c_8(2m)+o(1))\,\mathfrak{S}(2m)\,N/\log^2 N$
with $c_8(2m)\in\{1,\frac{1}{2}\}$.
On minor arcs we prove \emph{unconditional} density-one positivity
via mean-square bounds and convert fourth-moment control into
bounded gaps between exceptional even integers $\ll(\log N)^8$.
We formulate a precise \emph{conjecture} on medium-arc dispersion
(a log-power saving $\delta_{\mathrm{med}}>0$ on a three-tier arc
decomposition) and show that it would lower the short-interval
exponent to $(\log N)^{8-\delta_{\mathrm{med}}}$ and yield
uniform positivity beyond an explicit threshold $N_0$.
An unconditional Chen/Selberg variant (prime $+$ almost-prime)
and a GRH conditional template are recorded for comparison.

\medskip\noindent
\textbf{Claim hygiene.}
Every result is marked \PROVED{} (unconditional),
\COND{} (follows from a stated conjecture), or
\CONJ{} (the conjecture itself).
\end{abstract}

\tableofcontents

%% ============================================================
\section{Introduction}\label{sec:intro}
%% ============================================================

Goldbach's conjecture asserts that every even integer $2m>2$
is a sum of two primes.
We develop a circle-method framework using a mod-8 periodic kernel
$K_8$ that preserves the natural residue structure and isolates
the $2$-adic local factor.

\paragraph{Contributions.}
\begin{enumerate}[label=(\roman*)]
\item \PROVED{} A mod-8 kernel $K_8$ with a positive major-arc
  main term involving a $2$-adic gate $c_8(2m)\in\{1,\frac{1}{2}\}$.
\item \PROVED{} Unconditional density-one positivity and
  bounded gaps $\ll(\log N)^8$ between exceptions.
\item \CONJ{} A medium-arc dispersion conjecture (log-power
  saving $\delta_{\mathrm{med}}>0$).
\item \COND{} Improved exponent $(\log N)^{8-\delta_{\mathrm{med}}}$
  and uniform positivity beyond explicit $N_0$,
  conditional on the conjecture.
\item \PROVED{} An unconditional Chen/Selberg variant
  (prime $+$ almost-prime) with computable threshold.
\end{enumerate}

%% ============================================================
\section{The mod-8 kernel and circle-method setup}\label{sec:kernel}
%% ============================================================

\subsection{Mod-8 kernel}
Let $\chi_8$ be the primitive real Dirichlet character modulo $8$:
\[
\chi_8(n)=\begin{cases}
0,& n\equiv 0,2,4,6\pmod{8},\\
{+}1,& n\equiv 1,7\pmod{8},\\
{-}1,& n\equiv 3,5\pmod{8},
\end{cases}
\]
and define for even $2m$ the switch
\[
\varepsilon(2m)=\begin{cases}
{+}1,& 2m\equiv 0,2\pmod{8},\\
{-}1,& 2m\equiv 4,6\pmod{8}.
\end{cases}
\]
Set the aligned kernel
\begin{equation}\label{eq:K8}
K_8(n,m)\;:=\;\tfrac12\,\mathbf{1}_{n\text{ odd}}\,
\mathbf{1}_{2m-n\text{ odd}}\,
\bigl(1+\varepsilon(2m)\,\chi_8(n)\,\chi_8(2m-n)\bigr),
\end{equation}
which is periodic modulo $8$ in both arguments and, for each
residue class $2m\bmod 8$, keeps a positive proportion of
odd--odd pairs.

\subsection{Smoothed bilinear form}
Write $\Lambda$ for the von Mangoldt function.
Fix $N\asymp m$ and a smooth cutoff
$\eta\in C_c^\infty((0,2))$ with $\eta\equiv 1$ on $[\frac14,\frac74]$.
Define
\[
R_8(2m;N)\;:=\;\sum_{n\ge 1}\Lambda(n)\,\Lambda(2m{-}n)\,
K_8(n,m)\,\eta\!\Bigl(\tfrac{n}{N}\Bigr)\,
\eta\!\Bigl(\tfrac{2m-n}{N}\Bigr).
\]
Let
\[
S(\alpha)=\sum_{n\ge 1}\Lambda(n)\,e(\alpha n)\,
\eta\!\Bigl(\tfrac{n}{N}\Bigr),\qquad
S_{\chi_8}(\alpha)=\sum_{n\ge 1}\Lambda(n)\,\chi_8(n)\,
e(\alpha n)\,\eta\!\Bigl(\tfrac{n}{N}\Bigr).
\]
Expanding \eqref{eq:K8} gives
\begin{equation}\label{eq:circle}
R_8(2m;N)=\tfrac12\int_0^1 S(\alpha)^2 e(-2m\alpha)\,d\alpha
\;{+}\;\tfrac12\,\varepsilon(2m)\int_0^1 S_{\chi_8}(\alpha)^2
e(-2m\alpha)\,d\alpha,
\end{equation}
up to negligible even--even contributions.

%% ============================================================
\section{Unconditional results}\label{sec:unconditional}
%% ============================================================

\subsection{Major arcs and the 2-adic gate}\PROVED

Let $\mathfrak{M}$ be the standard major arcs.
Classical analysis (Vaughan~\cite{Vaughan1997}, Chs.~3--4) yields:

\begin{proposition}[Major-arc main term]\label{prop:major}
Uniformly for even $2m\le 2N$,
\[
\int_{\mathfrak{M}}\Bigl(\tfrac12 S^2
+\tfrac12\varepsilon S_{\chi_8}^2\Bigr)
e(-2m\alpha)\,d\alpha
\;=\;\bigl(c_8(2m)+o(1)\bigr)\,\mathfrak{S}(2m)\,
\frac{N}{\log^2 N},
\]
where
\[
c_8(2m)=\begin{cases}
1,&2m\equiv 0,4\pmod{8},\\
\tfrac12,&2m\equiv 2,6\pmod{8},
\end{cases}
\]
and the singular series satisfies the uniform lower bound
\[
\mathfrak{S}(2m)\;\ge\;c_0\;:=\;2C_2
\;=\;2\prod_{p>2}\frac{p(p-2)}{(p-1)^2}\;\approx\;1.32032.
\]
\end{proposition}

\begin{proof}
Standard Hardy--Littlewood singular series analysis.
The factor $c_8$ is the $2$-adic weight from the odd--odd
gating in \eqref{eq:K8}.
The lower bound on $\mathfrak{S}(2m)$ follows from its
Euler product since each factor $(p-1)/(p-2)\ge 1$
for $p\mid m$, $p>2$.
See \cite[Ch.~4]{Vaughan1997}.
\end{proof}

\subsection{Density-one positivity}\PROVED

\begin{theorem}[Density-one positivity]\label{thm:density-one}
For almost all even $2m\le 2N$,
\[
R_8(2m;N)=\bigl(c_8(2m)+o(1)\bigr)\,\mathfrak{S}(2m)\,
\frac{N}{\log^2 N}\;>\;0.
\]
The set of even $2m\le 2N$ with $R_8(2m;N)=0$ has
asymptotic density~$0$.
\end{theorem}

\begin{proof}
On the minor arcs $\mathfrak{m}$, standard mean-square bounds
(Vaughan's identity, large sieve, zero-density estimates;
\cite[Ch.~3]{Vaughan1997}, \cite[Ch.~13]{MontgomeryVaughan2007})
give, for any fixed $A>0$,
\[
\int_{\mathfrak{m}}|S(\alpha)|^2\,d\alpha\ll
\frac{N}{(\log N)^A},\qquad
\int_{\mathfrak{m}}|S_{\chi_8}(\alpha)|^2\,d\alpha\ll
\frac{N}{(\log N)^A}.
\]
By Cauchy--Schwarz the minor-arc contribution is
$\ll N/(\log N)^A$.
Choosing $A>2$ and averaging over $m$ gives the density-one
conclusion.
\end{proof}

\subsection{Short-interval positivity at exponent~8}\PROVED

\begin{theorem}[Bounded gaps between exceptions]\label{thm:short-interval}
There exists an absolute constant $C>0$ such that,
for all sufficiently large $N$,
every interval of $H\ge C(\log N)^8$ consecutive
values of $m$ contains some even $2m$ with
$R_8(2m;N)>0$.
\end{theorem}

\begin{proof}
Write the minor-arc remainder as
\[
F(2m;N):=\tfrac12\int_{\mathfrak{m}}S^2 e(-2m\alpha)\,d\alpha
+\tfrac12\varepsilon\int_{\mathfrak{m}}S_{\chi_8}^2
e(-2m\alpha)\,d\alpha.
\]
By Parseval and the classical fourth-moment bound
(\cite[Ch.~13]{MontgomeryVaughan2007}),
\[
\sum_{2m\in\mathcal{M}}|F(2m;N)|^2\;\le\;
I_{\mathrm{minor}}(N)\;\ll\;N^2(\log N)^4.
\]
Set $T(N):=\frac14 c_0\,N/\log^2 N$ (half the worst-case
major-arc lower bound, using $\min c_8=\frac12$).
By Markov's inequality, the number of $m$ in any window of
length $H$ with $|F(2m;N)|\ge T(N)$ is at most
$I_{\mathrm{minor}}(N)/T(N)^2\ll(\log N)^8$.
If $H$ exceeds this, at least one $m$ has
$|F|<T$, hence $R_8>0$.
\end{proof}

\begin{remark}
The $K_8$-combined fourth moment
$I_{\mathrm{minor}}^{K_8}:=
\frac12\int|S|^4+\frac12\int|S_{\chi_8}|^4$
satisfies
$\sum|F|^2\le I_{\mathrm{minor}}^{K_8}
\le\frac12 I_{\mathrm{minor}}$,
giving a factor-of-two constant improvement (same exponent~8).
\end{remark}

\subsection{Chen/Selberg variant}\PROVED

\begin{proposition}[Prime $+$ almost-prime]\label{prop:chen}
There exists a computable $M_0$ such that for all even
$2m\ge M_0$,
\[
2m\;=\;p\;+\;P_2,
\]
with $p$ prime and $P_2$ a product of at most two primes.
\end{proposition}

\begin{proof}[Proof sketch]
Replace $\Lambda$ by a Selberg lower-bound sieve weight $W$
adapted to primes and $P_2$'s.
The $K_8$ gate adjusts only the local factor at $2$
(the $c_8$ switch).
By Chen's method
(\cite{Chen1973}, adapted to finitely many congruence
conditions as in \cite{Vaughan1997}),
the modified bilinear form $R_8^{(2)}(2m;N)>0$ for all
$2m\ge M_0$.
The threshold $M_0$ depends on sieve constants,
Bombieri--Vinogradov level, zero-density estimates,
and the circle-method constants from Proposition~\ref{prop:major}.
\end{proof}

%% ============================================================
\section{The medium-arc dispersion conjecture}\label{sec:conjecture}
%% ============================================================

We now isolate the single unproved ingredient that, if
established, would improve the exponent from~$8$ and yield
uniform positivity.

\subsection{Three-tier arc decomposition}
Fix parameters
\[
Q=\frac{N^{1/2}}{(\log N)^4},\qquad
Q'=\frac{N^{2/3}}{(\log N)^6}.
\]
Define
\begin{align*}
\mathfrak{M}&=\bigcup_{1\le q\le Q}\bigcup_{(a,q)=1}
\Bigl\{\alpha:\Bigl|\alpha-\tfrac{a}{q}\Bigr|
\le\tfrac{Q}{qN}\Bigr\},\\
\mathfrak{M}_{\mathrm{med}}&=
\bigcup_{Q<q\le Q'}\bigcup_{(a,q)=1}
\Bigl\{\alpha:\Bigl|\alpha-\tfrac{a}{q}\Bigr|
\le\tfrac{Q'}{qN}\Bigr\}\setminus\mathfrak{M},\\
\mathfrak{m}_{\mathrm{deep}}&=
[0,1)\setminus(\mathfrak{M}\cup\mathfrak{M}_{\mathrm{med}}).
\end{align*}

\begin{lemma}[Local $L^4$ on short arcs]\label{lem:local-L4}\PROVED{}
For any finitely supported sequence $(c_x)$ and $B\in(0,1]$,
\[
\int_{|\beta|\le B}\Bigl|\sum_x c_x\,e(\beta x)\Bigr|^4
d\beta\;\le\;2B\,\Bigl(\sum_x|c_x|^2\Bigr)^2.
\]
\end{lemma}
\begin{proof}
Expand the fourth power and integrate termwise:
$\int_{-B}^B e(\beta(u-v))\,d\beta\le 2B$.
Cauchy--Schwarz on the shift-convolution gives the claim.
\end{proof}

\subsection{The conjecture}

\begin{conjecture}[Medium-arc $L^4$ saving]\label{conj:dispersion}
\CONJ{}
With the three-tier decomposition above and
Vaughan partition $U=V=N^{1/3}$, there exist absolute
constants $C_{\mathrm{disp}}>0$ and $\delta_{\mathrm{med}}>0$
such that, for all sufficiently large $N$,
\begin{equation}\label{eq:dispersion}
\int_{\mathfrak{M}_{\mathrm{med}}}
\bigl(|S(\alpha)|^4+|S_{\chi_8}(\alpha)|^4\bigr)\,d\alpha
\;\le\;C_{\mathrm{disp}}\,N^2\,(\log N)^{4-\delta_{\mathrm{med}}}.
\end{equation}
\end{conjecture}

\begin{remark}[Why this is not yet proved]\label{rem:gap}
The natural attack is:
apply Lemma~\ref{lem:local-L4} to each bilinear piece
$\mathcal{B}(\alpha)$ on arcs near $a/q$,
then sum over reduced $a\bmod q$ using completion modulo $q$
and the additive large sieve.
The large sieve controls the \emph{$L^2$ average} over residues:
\[
\sum_{\substack{a\bmod q\\(a,q)=1}}
\Bigl|\sum_{m,n}A_m B_n e\bigl(\tfrac{a}{q}mn\bigr)\Bigr|^2
\;\le\;(q+M+N/M)\,MN\,(\log N)^C.
\]
However, the $L^4$ bound requires controlling the
\emph{square} of the quadratic form:
$\sum_a\bigl(\sum_x|\cdots|^2\bigr)^2$.
Passing from the $L^2$ average to the $L^4$ norm requires
either a pointwise (maximum) bound over residues $a$,
or a direct fourth-moment large sieve---neither of which
follows from the standard additive large sieve alone.

Techniques from Deshouillers--Iwaniec~\cite{DeshouillersIwaniec}
(Kloosterman-sum dispersion),
Duke--Friedlander--Iwaniec~\cite{DukeFriedlanderIwaniec}
(bilinear forms with Kloosterman sums),
and their exposition in
Iwaniec--Kowalski~\cite{IwaniecKowalski} can potentially
close this gap, but the adaptation to the present $L^4$-on-medium-arcs
setting requires substantial additional work
that we do not complete here.
\end{remark}

\begin{remark}[Plausibility]
The saving is plausible because on medium arcs ($Q<q\le Q'$),
the modular structure modulo $q$ introduces cancellation in
bilinear sums beyond what the trivial $L^2$-to-$L^4$ passage
provides.
The factor $\log(Q'/Q)/\log N$ is positive and
provides the natural candidate for $\delta_{\mathrm{med}}$.
\end{remark}

%% ============================================================
\section{Conditional consequences of the conjecture}\label{sec:conditional}
%% ============================================================

Throughout this section, all results assume
Conjecture~\ref{conj:dispersion} with some
$\delta_{\mathrm{med}}>0$.

\subsection{Coercivity inequality}\COND

\begin{lemma}[Medium-arc coercivity]\label{lem:coercivity}
Assume Conjecture~\ref{conj:dispersion}.
Set $C_{\mathrm{meas}}:=\mathrm{meas}(\mathfrak{M}_{\mathrm{med}})
\le\bigl(\frac{12}{\pi^2}\log\frac{Q'}{Q}+2\bigr)\frac{Q'}{N}$.
Then uniformly for $2m\le 2N$,
\[
R_8(2m;N)\;\ge\;\int_{\mathfrak{M}}\cdots
\;-\;\frac{1}{\sqrt{2}}\,C_{\mathrm{meas}}^{1/2}\,
\mathcal{D}_{\mathrm{med}}^{1/2}
\;-\;\epsilon_{\mathrm{deep}}(N),
\]
where $\mathcal{D}_{\mathrm{med}}=
\int_{\mathfrak{M}_{\mathrm{med}}}(|S|^4+|S_{\chi_8}|^4)\,d\alpha$
and $\epsilon_{\mathrm{deep}}\ll N/(\log N)^A$ for any fixed $A\ge 6$.
\end{lemma}

\begin{proof}
Split $[0,1)=\mathfrak{M}\cup\mathfrak{M}_{\mathrm{med}}
\cup\mathfrak{m}_{\mathrm{deep}}$.
On $\mathfrak{M}_{\mathrm{med}}$, Cauchy--Schwarz gives
$|\int S^2 e|\le\mathrm{meas}^{1/2}\cdot(\int|S|^4)^{1/2}$.
On $\mathfrak{m}_{\mathrm{deep}}$, use $|\int f^2 e|\le\int|f|^2$
and the classical mean-square bound.
\end{proof}

\subsection{Improved short-interval exponent}\COND

\begin{theorem}[Short-interval improvement]\label{thm:improved-interval}
Assume Conjecture~\ref{conj:dispersion} with
$\delta_{\mathrm{med}}>0$.
Then there exists $C>0$ such that, for all large $N$,
every interval of $H\ge C(\log N)^{8-\delta_{\mathrm{med}}}$
consecutive values of $m$ contains some even $2m$ with
$R_8(2m;N)>0$.
\end{theorem}

\begin{proof}
Replace $I_{\mathrm{minor}}(N)\ll N^2(\log N)^4$ in the proof
of Theorem~\ref{thm:short-interval} by the improved bound
$I_{\mathrm{minor}}^{K_8}(N)\le
\frac12 C_{\mathrm{disp}}N^2(\log N)^{4-\delta_{\mathrm{med}}}
+\frac12 C_{\mathrm{deep}}N^2(\log N)^4$.
The medium-arc term dominates, giving
$H_0\ll(\log N)^{8-\delta_{\mathrm{med}}}$.
\end{proof}

\subsection{Uniform pointwise positivity}\COND

\begin{theorem}[Uniform positivity beyond $N_0$]\label{thm:uniform}
Assume Conjecture~\ref{conj:dispersion} with
$\delta_{\mathrm{med}}\ge 10^{-3}$ and
$C_{\mathrm{disp}}\le 10^3$.
Then there exists an explicit $N_0$ (depending on
$C_{\mathrm{disp}}$, $\delta_{\mathrm{med}}$, and the
deep-minor constant) such that for every $N\ge N_0$
and all even $2m\le 2N$,
\[
R_8(2m;N)\;>\;0.
\]
Conservative examples: $\log N_0\approx 66$ for
$C_{\mathrm{disp}}=10^3$.
\end{theorem}

\begin{proof}
By Lemma~\ref{lem:coercivity} and Proposition~\ref{prop:major},
it suffices to have
\[
\sqrt{C_{\mathrm{meas}}\cdot C_{\mathrm{disp}}}\,
(\log N)^{2-\delta_{\mathrm{med}}/2}
+C_{\mathrm{deep}}/(\log N)^4
\;\le\;\tfrac12\,c_0/\log^2 N.
\]
Since $C_{\mathrm{meas}}\ll N^{-1/3}(\log N)^{-5}$,
the left side decays as $N^{-1/6}$ times log-powers,
while the right is $O(1/\log^2 N)$.
The inequality holds for $N\ge N_0$ with
$N_0$ computable from the constants.
\end{proof}

%% ============================================================
\section{GRH conditional theorem}\label{sec:grh}
%% ============================================================

\begin{theorem}[Goldbach under GRH]\label{thm:grh}\PROVED{}
(relative to GRH)
Assume GRH for Dirichlet $L$-functions and standard
zero-free/zero-density estimates.
Then there exists $N_0$ such that for all $N\ge N_0$
and all even $2m\le 2N$, $R(2m;N)>0$.
A finite verification below $2N_0$ completes the proof.
\end{theorem}

\begin{proof}[Sketch]
Under GRH, pointwise minor-arc estimates of size
$\ll N/(\log N)^A$ hold for each $2m$, giving
$R_8(2m;N)>0$ for all large $2m$.
\end{proof}

%% ============================================================
\section{Discussion: what is proved and what remains}
\label{sec:discussion}
%% ============================================================

\subsection{Summary table}

\begin{center}
\small
\begin{tabular}{@{}lll@{}}
\toprule
\textbf{Result} & \textbf{Status} & \textbf{Reference} \\
\midrule
Major-arc positivity ($c_8\cdot\mathfrak{S}\cdot N/\log^2 N$)
  & \PROVED & Prop.~\ref{prop:major} \\
Density-one positivity
  & \PROVED & Thm.~\ref{thm:density-one} \\
Bounded gaps $\ll(\log N)^8$
  & \PROVED & Thm.~\ref{thm:short-interval} \\
Chen variant (prime $+$ $P_2$)
  & \PROVED & Prop.~\ref{prop:chen} \\
Medium-arc $L^4$ saving $\delta_{\mathrm{med}}>0$
  & \CONJ & Conj.~\ref{conj:dispersion} \\
Improved exponent $8-\delta_{\mathrm{med}}$
  & \COND & Thm.~\ref{thm:improved-interval} \\
Uniform positivity beyond $N_0$
  & \COND & Thm.~\ref{thm:uniform} \\
Goldbach under GRH
  & \PROVED{} (rel.~GRH) & Thm.~\ref{thm:grh} \\
\bottomrule
\end{tabular}
\end{center}

\subsection{The bottleneck}

The entire conditional tower rests on a single unproved
ingredient: an $L^4$ saving on medium arcs
(Conjecture~\ref{conj:dispersion}).
As explained in Remark~\ref{rem:gap}, the obstacle is
the passage from $L^2$ large-sieve control (which is standard)
to $L^4$ control (which requires either a pointwise bound
over residues or a direct fourth-moment large sieve).

\paragraph{Approaches that might close the gap.}
\begin{itemize}
\item The Deshouillers--Iwaniec dispersion method
  \cite{DeshouillersIwaniec}, if adapted to bound
  the fourth moment of bilinear sums on medium arcs directly.
\item Heath-Brown's fourth-moment large sieve, applied to
  the Vaughan-decomposed pieces with medium-arc localization.
\item A type-II estimate using Kloosterman refinements
  (Duke--Friedlander--Iwaniec~\cite{DukeFriedlanderIwaniec})
  to control individual residue contributions pointwise.
\end{itemize}
Any of these, if carried out to yield $\delta_{\mathrm{med}}>0$,
would immediately activate all conditional results in
Section~\ref{sec:conditional}.

\subsection{The role of the mod-8 kernel}

The $K_8$ kernel serves two purposes:
\begin{enumerate}
\item It isolates the $2$-adic structure cleanly,
  producing an explicit gate $c_8\in\{1,\frac{1}{2}\}$
  rather than a generic local factor.
\item The $K_8$-combined fourth moment
  $I^{K_8}=\frac12\int|S|^4+\frac12\int|S_{\chi_8}|^4$
  gives a constant-factor improvement over the plain bound
  (same exponent, better prefactor).
\end{enumerate}
In the context of Recognition Science, the mod-8 structure
resonates with the 8-tick periodicity of the recognition
operator $\hat{R}$, providing a natural number-theoretic
instantiation of the 8-phase ledger structure.

\subsection{Computational verification}

For any explicit $N_0$ (conditional or under GRH),
the residual range $4\le 2m\le 2N_0$ can be closed by
deterministic computation:
segmented sieve $\to$ odd-only bitset $\to$ mod-8 gate
$+$ wheel-840 $\to$ first-hit search per even $n$.
This is essentially linear time in $N_0$ up to log factors.

%% ============================================================
\begin{thebibliography}{9}

\bibitem{HardyLittlewood}
G.~H.~Hardy and J.~E.~Littlewood,
``Some problems of `Partitio Numerorum'; III,''
\emph{Acta Math.} \textbf{44} (1923).

\bibitem{Vaughan1997}
R.~C.~Vaughan,
\emph{The Hardy--Littlewood Method}, 2nd ed.,
Cambridge University Press, 1997.

\bibitem{MontgomeryVaughan2007}
H.~L.~Montgomery and R.~C.~Vaughan,
\emph{Multiplicative Number Theory~I},
Cambridge University Press, 2007.

\bibitem{Chen1973}
J.~R.~Chen,
``On the representation of a large even integer as the sum
of a prime and the product of at most two primes,''
\emph{Sci.\ Sinica} \textbf{16} (1973), 157--176.

\bibitem{DeshouillersIwaniec}
J.-M.~Deshouillers and H.~Iwaniec,
``Kloosterman sums and Fourier coefficients of cusp forms,''
\emph{Invent.\ Math.} \textbf{70} (1982/83), 219--288.

\bibitem{DukeFriedlanderIwaniec}
W.~Duke, J.~Friedlander, and H.~Iwaniec,
``Bilinear forms with Kloosterman sums,''
\emph{Invent.\ Math.} \textbf{128} (1997), 23--43.

\bibitem{IwaniecKowalski}
H.~Iwaniec and E.~Kowalski,
\emph{Analytic Number Theory},
AMS Colloq.\ Publ., Vol.~53, 2004.

\end{thebibliography}

\end{document}
