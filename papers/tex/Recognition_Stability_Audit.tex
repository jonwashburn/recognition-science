\documentclass[11pt]{article}

\usepackage{amsmath,amssymb,amsthm,mathtools}
\usepackage{geometry}
\usepackage{booktabs}
\usepackage[hidelinks]{hyperref}
\usepackage{microtype}

\geometry{letterpaper, margin=1in}

% Theorem environments
\newtheorem{theorem}{Theorem}
\newtheorem{lemma}{Lemma}
\newtheorem{proposition}{Proposition}
\newtheorem{corollary}{Corollary}
\newtheorem{definition}{Definition}
\newtheorem{remark}{Remark}
\newtheorem{algorithm}{Algorithm}

% Short macros
\newcommand{\R}{\mathbb{R}}
\newcommand{\C}{\mathbb{C}}
\newcommand{\Rp}{\mathbb{R}_{>0}}
\newcommand{\D}{\mathbb{D}}
\newcommand{\F}{\mathcal{F}}
\newcommand{\ledger}{\mathcal{L}}
\newcommand{\Rhat}{\widehat{R}}
\newcommand{\mhol}{m_{\mathrm{hol}}}

\title{The Recognition Stability Audit:\\
IA compiler for impossibility certificates:\\
Realizable Cayley Fields and Finite Schur Certificates from Canonical Recognition Cost}
\author{Jonathan Washburn\\Recognition Science --- Recognition Physics Institute\\Austin, Texas, USA\\\texttt{jon@recognitionphysics.org}}
\date{\today}

\begin{document}
\maketitle

\begin{abstract}
We formalize the Recognition Stability Audit (RSA), a reusable \emph{impossibility audit} for candidate states whose existence would force an \emph{infinite recognition cost}, modeled as blow-up of a canonical reciprocal sensor.
Conceptually, RSA is a \emph{compiler}: it attempts to convert an existence claim into a pole mechanism for a bounded analytic field, and then eliminates that pole by a finite certificate.
The audit has three layers: a canonical reciprocal cost functional $J(x)$ (forced by normalization and composition axioms \cite{Washburn2025Cost}); an RS defect/obstruction encoding that turns an existence claim into a holomorphic obstruction $G$ and canonical sensor $\mathcal J=1/G$; and a Cayley transform producing a bounded field $\Xi$ whose Schur control excludes sensor poles by a Schur/Herglotz pinch.
To make the certification step fully intrinsic in the full-derivation route, we fix a formal realizability model in which the audited Cayley field is \emph{realizable}: it is generated by an explicit 8-tick recognition process with finite local branching (hence finite-state/rational) and by a cost-driven contractive tick dynamics (hence stable finite-dimensional realizations).
In this realizable class we record two fully finite Schur certification routes: a state-space bounded-real (KYP/LMI) certificate \cite{ZhouDoyleGlover1996} and a Pick-gap-plus-tail certificate from Taylor data (Nevanlinna--Pick/Schur theory \cite{RosenblumRovnyak,Donoghue}) with tail bounds derivable from stable realization data.
In the bounded-degree rational regime, RSA upgrades from a sound semi-decision for impossibility to a full decision procedure by exact root-location tests.
We illustrate the audit with a certified far-field zeta instantiation \cite{WashburnRiemannDec31} and outline a complementary existence-side architecture for Hodge-type problems (companion manuscripts). We define a \emph{Universal Recognition Class} of problems admissible to this audit and conjecture that many well-posed existence problems in physical geometry and arithmetic admit the Step~0 realizability model and therefore lie within it.
\end{abstract}

\section{Motivation and scope}

Recognition Science treats recognition as a constrained operation: it consumes a nonzero, quantifiable resource.
The operational consequence is:

\medskip
\noindent\textbf{Principle (Finite recognition).}
\emph{A physically realizable state cannot require infinite recognition cost.}
\medskip

RSA is a general method for turning this principle into a checkable certificate.
It is built as a \emph{template} with explicit proof obligations, but it is best read as a \emph{compiler/audit machine}: given a candidate existence claim, RSA attempts to compile it into a finite certificate of impossibility.

\subsection*{How to read this manuscript: RSA as a compiler}
RSA has a front-end, a middle-end, and a back-end.

\medskip
\noindent\textbf{Front-end (existence $\Rightarrow$ obstruction zero).}
Encode the candidate claim as an obstruction $G$ whose zeros are forced by the candidate mechanism; then define the sensor $\mathcal J:=1/G$ so ``candidate occurs'' becomes ``sensor blows up'' (Section~\ref{subsec:rs-defect}).

\medskip
\noindent\textbf{Middle-end (sensor $\Rightarrow$ bounded Cayley field).}
Apply the Cayley transform to obtain a bounded field $\Xi$ so that a pole of $\mathcal J$ becomes a boundary hit for $\Xi$; then classical Schur/Herglotz pinching turns global Schur control of $\Xi$ into pole-freeness of $\mathcal J$ (Section~\ref{sec:pinch} and Corollary~\ref{cor:no-poles}).

\medskip
\noindent\textbf{Back-end (global boundedness $\Rightarrow$ finite certificate).}
Global Schur control is an analytic statement, so RSA supplies finite certification regimes (bounded-real/KYP or Pick-gap-plus-tail).
The crucial bridge is that the audited Cayley field is not arbitrary: in the full-derivation route it is assumed \emph{realizable} in the Step~0 sense (8-tick finite-window plus cost-driven contractive dynamics), which places it in a finite-complexity class where finite certificates are complete or can be made quantitative (Theorem~\ref{thm:realizability-finite-complexity} and Corollary~\ref{cor:intrinsic-cost-certification}).

\medskip
\noindent\textbf{The remaining domain module (what makes a flagship instantiation a closed theorem).}
For any specific target (zeta, Hodge, PDE blow-up, \dots), the one domain-specific obligation is to prove a theorem identifying the audited Cayley field built from that domain obstruction with a Step~0 realizable Cayley field model (including explicit degree/stability constants, or an explicit tail bound).
Once that module is supplied, the rest of RSA is a closed pipeline: one computes a finite certificate and concludes impossibility on the audited region.

\subsection*{What RSA does and does not claim}
\begin{itemize}
\item \textbf{What RSA can certify:} conditional \texttt{IMPOSSIBLE\_STATE} for candidates whose existence would force sensor blow-up in the audited region.
\item \textbf{What RSA does not do in isolation:} prove existence \emph{in general}. Passing an audit is therefore \texttt{INCONCLUSIVE} unless paired with a complementary existence-side mechanism (Definition~\ref{def:urc-dec}); in the finite-dimensional rational regime such an existence-side mechanism is automatic by exact root testing (Theorem~\ref{thm:decision-rational}).
\end{itemize}

\subsection*{Claim taxonomy (for referee-facing scope hygiene)}
\begin{center}
\small
\begin{tabular}{@{}ll@{}}
\toprule
\textbf{Type} & \textbf{Status in this manuscript} \\
\midrule
Cost uniqueness (Theorem~\ref{thm:J-unique}) & Proved here (functional equation $\Rightarrow$ ODE $\Rightarrow$ $\cosh$) \\
Pick/Schur background (Theorem~\ref{thm:pick-criterion}) & Classical; see \cite{RosenblumRovnyak,Donoghue} \\
Realizability $\Rightarrow$ finite complexity (Theorem~\ref{thm:realizability-finite-complexity}) & Proved here (under the explicit 8-tick model) \\
Finite certificate $\Rightarrow$ global Schur (Theorem~\ref{thm:finite-certificate-global}) & Proved here (packaging of standard criteria) \\
Finite-sampling obstruction (Proposition~\ref{prop:no-finite-sampling}) & Proved here \\
Schur pinch $\Rightarrow$ audited impossibility (Theorem~\ref{thm:correctness}) & Proved here (given stated hypotheses) \\
Decision in the rational regime (Theorem~\ref{thm:decision-rational}) & Stated and proved (finite algebraic root-location) \\
URC coverage conjecture & Conjecture (Section~\ref{sec:urc}) \\
\bottomrule
\end{tabular}
\end{center}

\subsection*{Conceptual dictionary (Rosetta Stone)}
To orient experts from different fields, we map RSA terms to their standard analogs:
\begin{center}
\small
\begin{tabular}{@{}ll@{}}
\toprule
\textbf{RSA Term} & \textbf{Standard Analog} \\
\midrule
Sensor Cost Blow-up & Control Theory: Instability / Positive Feedback Loop \\
Schur Certification & Operator Theory: Nevanlinna--Pick Interpolation \\
Reciprocal Cost & Physics: Action / Entropy \\
\bottomrule
\end{tabular}
\end{center}

\section{Canonical recognition cost}

\begin{definition}[Deviant]
A \emph{deviant} is a multiplicative deviation $x\in\Rp$ from the identity state $x=1$.
\end{definition}

\begin{definition}[Canonical reciprocal cost]
Define $J:\Rp\to\R$ by
\begin{equation}\label{eq:Jdef}
J(x) \;=\; \frac12\!\left(x+x^{-1}\right)-1.
\end{equation}
\end{definition}

\begin{proposition}[Basic properties]
For all $x\in\Rp$:
\begin{enumerate}
\item (Reciprocity) $J(x)=J(x^{-1})$.
\item (Normalization) $J(1)=0$.
\item (Nonnegativity) $J(x)=\dfrac{(x-1)^2}{2x}\ge 0$, with equality iff $x=1$.
\item (Divergence) $J(x)\to\infty$ as $x\to 0^+$ or $x\to\infty$.
\end{enumerate}
\end{proposition}

\begin{definition}[Log coordinates]
Write $x=e^t$ with $t\in\R$. Then
\begin{equation}\label{eq:cosh}
J(e^t)=\cosh(t)-1.
\end{equation}
\end{definition}

\subsection{Why this \texorpdfstring{$J$}{J} is not a dial}
RSA relies on the idea that the cost functional is \emph{forced}, not chosen.
The following uniqueness theorem isolates a minimal hypothesis set that pins down $J$.

\begin{theorem}[Uniqueness under composition and calibration]\label{thm:J-unique}
Let $F:\Rp\to\R$ satisfy:
\begin{enumerate}
\item $F(1)=0$,
\item for all $x,y>0$,
\begin{equation}\label{eq:composition}
F(xy)+F(x/y)=2F(x)F(y)+2F(x)+2F(y),
\end{equation}
\item (Unit log-curvature at identity)
\begin{equation}\label{eq:curv}
\lim_{t\to 0}\frac{2F(e^t)}{t^2}=1.
\end{equation}
\end{enumerate}
Then $F(x)=J(x)$ for all $x>0$.
\end{theorem}

\begin{proof}
Define $H:\R\to\R$ by $H(t):=F(e^t)+1$.
Then $H(0)=F(1)+1=1$.
Substituting $x=e^t$ and $y=e^u$ into \eqref{eq:composition} gives
\[
F(e^{t+u})+F(e^{t-u})
=2F(e^t)F(e^u)+2F(e^t)+2F(e^u).
\]
Adding $2$ to both sides and regrouping yields d'Alembert's equation
\begin{equation}\label{eq:dalembert}
H(t+u)+H(t-u)=2H(t)H(u)\qquad (t,u\in\R).
\end{equation}
Setting $t=0$ in \eqref{eq:dalembert} gives $H(u)+H(-u)=2H(0)H(u)=2H(u)$, hence $H$ is even.

The curvature condition \eqref{eq:curv} is exactly the second-order calibration
\[
\lim_{t\to 0}\frac{H(t)-1}{t^2}=\lim_{t\to 0}\frac{F(e^t)}{t^2}=\frac12.
\]
In particular, $H$ has second derivative at $0$ with $H''(0)=1$.

Now fix $t\in\R$ and rewrite \eqref{eq:dalembert} as a symmetric second-difference identity:
\[
H(t+u)-2H(t)+H(t-u)=2H(t)\bigl(H(u)-1\bigr)\qquad (u\neq 0).
\]
Dividing by $u^2$ and letting $u\to 0$ gives
\[
\lim_{u\to 0}\frac{H(t+u)-2H(t)+H(t-u)}{u^2}
=2H(t)\cdot \lim_{u\to 0}\frac{H(u)-1}{u^2}
=2H(t)\cdot \frac12
=H(t).
\]
Thus $H$ is twice differentiable and satisfies the ODE $H''(t)=H(t)$ for all $t$.
The general $C^2$ solutions of $H''=H$ are $H(t)=a e^t+b e^{-t}$.
Since $H$ is even we have $a=b$, and since $H(0)=1$ we get $a=b=\tfrac12$.
Therefore $H(t)=\cosh(t)$ and hence, for $x>0$ with $t=\log x$,
\[
F(x)=H(\log x)-1=\cosh(\log x)-1=\frac12(x+x^{-1})-1=J(x).
\]
\end{proof}

\begin{remark}
The functional equation \eqref{eq:composition} is a multiplicative form of d'Alembert's equation (compare \eqref{eq:dalembert}), and the calibration \eqref{eq:curv} selects the hyperbolic branch and fixes its scale.
\end{remark}

\subsection{A cost-minimizing neutrality projection (derived from \texorpdfstring{$J$}{J})}
The full-derivation route requires a concrete mechanism that connects Recognition Cost to dynamics.
Here we record a canonical ``projection-to-neutrality'' update rule that is \emph{forced} once $J$ is fixed.
The point is not to import additional structure, but to show that a standard RS move---enforcing a conservation/neutrality constraint by a minimal-cost correction---has a unique solution because $J$ is strictly convex in log-coordinates.

\begin{definition}[Skew and neutrality manifold]\label{def:skew-neutrality}
Fix $n\in\mathbb N$ and consider a multiplicative ledger vector $x=(x_1,\dots,x_n)\in(\Rp)^n$.
Define its \emph{log-skew} by
\[
\sigma(x)\;:=\;\sum_{i=1}^n \log(x_i).
\]
Define the \emph{neutrality manifold} (constraint surface)
\[
\mathcal M\;:=\;\{\,x\in(\Rp)^n:\ \sigma(x)=0\,\}.
\]
\end{definition}

\begin{definition}[Neutralizing corrections and their $J$-cost]\label{def:neutralizing-corrections}
Given $x\in(\Rp)^n$, a \emph{neutralizing correction} is a vector $r=(r_1,\dots,r_n)\in(\Rp)^n$ such that the corrected state
\[
x' \;=\; x\odot r \;:=\; (x_1r_1,\dots,x_nr_n)
\]
lies in $\mathcal M$.
Equivalently, in log-coordinates $t_i:=\log(r_i)$, this constraint is
\[
\sum_{i=1}^n t_i \;=\; -\sigma(x).
\]
Define the total correction cost by
\[
\mathrm{Cost}(r)\;:=\;\sum_{i=1}^n J(r_i).
\]
\end{definition}

\begin{theorem}[Canonical $J$-projection to neutrality]\label{thm:J-projection}
For every $x\in(\Rp)^n$, the minimization problem
\[
\min\{\ \mathrm{Cost}(r):\ r\in(\Rp)^n,\ x\odot r\in \mathcal M\ \}
\]
has a unique minimizer, namely
\[
r_1=\cdots=r_n=\exp\!\Big(-\frac{\sigma(x)}{n}\Big).
\]
Equivalently, the unique minimal-cost neutralization is the uniform rescaling
\[
x'_i \;=\; x_i\cdot \exp\!\Big(-\frac{\sigma(x)}{n}\Big)\qquad (i=1,\dots,n).
\]
\end{theorem}
\begin{proof}
Write $t_i=\log(r_i)$ so that $r_i=e^{t_i}$ and the neutrality constraint is $\sum_i t_i=-\sigma(x)$.
Using \eqref{eq:cosh}, we have $J(e^{t_i})=\cosh(t_i)-1$.
Define $\phi(t):=\cosh(t)-1$, which is strictly convex on $\R$ (since $\phi''(t)=\cosh(t)>0$).
Then
\[
\mathrm{Cost}(r)=\sum_{i=1}^n J(e^{t_i})=\sum_{i=1}^n \phi(t_i).
\]
By Jensen's inequality,
\[
\frac1n\sum_{i=1}^n \phi(t_i)\ \ge\ \phi\!\Big(\frac1n\sum_{i=1}^n t_i\Big)
\;=\;\phi\!\Big(-\frac{\sigma(x)}{n}\Big),
\]
with equality if and only if $t_1=\cdots=t_n$ (strict convexity).
Therefore the unique minimizer is $t_i=-\sigma(x)/n$ for all $i$, i.e.\ $r_i=\exp(-\sigma(x)/n)$.
Substituting into $x'=x\odot r$ gives the stated formula.
\end{proof}

\begin{corollary}[Log-space orthogonal projection and nonexpansiveness]\label{cor:logspace-projection}
Let $x\in(\Rp)^n$ and write $y_i=\log(x_i)$.
Define $x'\in\mathcal M$ by the canonical $J$-projection (Theorem~\ref{thm:J-projection}) and write $y'_i=\log(x'_i)$.
Then
\[
y'_i \;=\; y_i - \frac{1}{n}\sum_{j=1}^n y_j\qquad (i=1,\dots,n),
\]
so $y'\in \R^n$ is the Euclidean orthogonal projection of $y$ onto the hyperplane $H=\{y\in\R^n:\sum_i y_i=0\}$.
In particular, the map $y\mapsto y'$ is $1$-Lipschitz and satisfies $\|y'\|\le \|y\|$.
\end{corollary}
\begin{proof}
From Theorem~\ref{thm:J-projection} we have $x'_i=x_i\exp(-\sigma(x)/n)$.
Taking logarithms gives
\[
y'_i=\log(x'_i)=\log(x_i)-\frac{\sigma(x)}{n}=y_i-\frac{1}{n}\sum_{j=1}^n y_j.
\]
This is exactly the formula for orthogonal projection onto $H$ (subtract the mean).
Orthogonal projections in Euclidean space are $1$-Lipschitz and satisfy $\|y'\|\le \|y\|$.
\end{proof}

\begin{remark}[Why this matters for RSA]
Theorem~\ref{thm:J-projection} is a second ``program to certificate-grade theorem'' upgrade.
It shows that once a neutrality constraint is imposed, the cost-minimizing correction step is \emph{unique} and \emph{explicit}.
This is the simplest rigorous sense in which Recognition Cost generates dynamics: neutralization is not an arbitrary repair, but the forced minimizer of a convex cost.
Corollary~\ref{cor:logspace-projection} adds an important structural fact: in log-coordinates, the correction is an \emph{orthogonal projection}, hence nonexpansive.
This is the beginning of the stability/contractivity story needed to justify finite-dimensional bounded-real certificates from cost dynamics.
\end{remark}

\subsection{A contractive tick model from Recognition Cost (proximal form)}\label{subsec:prox-contract}
The bounded-real (KYP) certificate used by RSA is a \emph{contractivity} statement: it asserts the existence of a storage function that decreases along trajectories up to an input--output energy balance (Remark~\ref{rem:kyp-dissipation}).
To make this intrinsic to Recognition Cost, one needs a mathematically explicit ``one-tick'' dynamics rule whose only primitive is the cost functional.

A standard way to turn a convex cost into a well-posed discrete-time dynamics is a \emph{proximal step}: ``move as little as possible while paying down cost.''
We record the resulting contraction mechanism in the canonical $J$-geometry.

\begin{definition}[Cost-regularized neutrality step]\label{def:prox-step}
Fix $n\ge 1$ and let $H=\{y\in\R^n:\sum_i y_i=0\}$.
Let $\phi(t)=\cosh(t)-1$ so that $J(e^t)=\phi(t)$ by \eqref{eq:cosh}.
For $\lambda>0$, define the map $\Pi_\lambda:\R^n\to H$ by
\[
\Pi_\lambda(y)\ :=\ \arg\min_{y'\in H}\ \Big(\frac12\|y'-y\|^2\;+\;\lambda\sum_{i=1}^n \phi(y'_i)\Big).
\]
\end{definition}

\begin{lemma}[Strong convexity and contraction]\label{lem:prox-contraction}
For each $\lambda>0$, the minimizer in Definition~\ref{def:prox-step} exists and is unique.
Moreover $\Pi_\lambda$ is a strict contraction:
\[
\|\Pi_\lambda(y)-\Pi_\lambda(\tilde y)\|\ \le\ \frac{1}{1+\lambda}\,\|y-\tilde y\|
\qquad (y,\tilde y\in\R^n).
\]
\end{lemma}
\begin{proof}
The function $\phi$ is $1$-strongly convex on $\R$ because $\phi''(t)=\cosh(t)\ge 1$.
Therefore $y'\mapsto \sum_i \phi(y'_i)$ is $1$-strongly convex on $\R^n$ (hence on the affine subspace $H$), and adding the strictly convex quadratic term $\frac12\|y'-y\|^2$ makes the objective in Definition~\ref{def:prox-step} strictly convex on $H$.
Since the objective is coercive on $H$, a unique minimizer exists.

For the contraction estimate, set
\[
f(y'):=\sum_{i=1}^n \phi(y'_i) + \iota_H(y'),
\]
where $\iota_H$ is $0$ on $H$ and $+\infty$ off $H$.
Then $f$ is $1$-strongly convex, so its subdifferential $\partial f$ is $1$-strongly monotone:
for $u\in\partial f(x)$ and $v\in\partial f(\tilde x)$,
\[
\langle x-\tilde x,\,u-v\rangle\ \ge\ \|x-\tilde x\|^2.
\]
Let $x=\Pi_\lambda(y)$ and $\tilde x=\Pi_\lambda(\tilde y)$.
The first-order optimality conditions are
\[
0\in x-y+\lambda\,\partial f(x),\qquad 0\in \tilde x-\tilde y+\lambda\,\partial f(\tilde x),
\]
so $(y-x)/\lambda\in \partial f(x)$ and $(\tilde y-\tilde x)/\lambda\in\partial f(\tilde x)$.
Applying strong monotonicity gives
\[
\Big\langle x-\tilde x,\ \frac{y-x}{\lambda}-\frac{\tilde y-\tilde x}{\lambda}\Big\rangle
\ \ge\ \|x-\tilde x\|^2.
\]
Multiplying by $\lambda$ and rearranging yields
\[
\langle x-\tilde x,\ y-\tilde y\rangle\ \ge\ (1+\lambda)\,\|x-\tilde x\|^2.
\]
By Cauchy--Schwarz, $\|x-\tilde x\|\,\|y-\tilde y\|\ge (1+\lambda)\|x-\tilde x\|^2$, hence $\|x-\tilde x\|\le \frac{1}{1+\lambda}\|y-\tilde y\|$.
\end{proof}

\begin{remark}[How this feeds bounded-real certificates]
Lemma~\ref{lem:prox-contraction} exhibits an explicit, purely cost-driven source of strict contraction in log-coordinates.
In a finite-resolution quotient (Section~\ref{sec:rg4}), such contractive tick maps yield stable finite-dimensional realizations, which is the structural input needed to make bounded-real (KYP/LMI) certification intrinsic rather than assumed.
For a domain instantiation, the corresponding adapter theorem is: identify the audited Cayley field with the transfer of a realization governed by such a cost-regularized neutrality step (or equivalently, exhibit a dissipation inequality of the form in Remark~\ref{rem:kyp-dissipation}).
When this identification is available, the stability hypotheses needed for state-space certification are discharged \emph{from the RS dynamics itself} rather than imported as an external modeling assumption.
\end{remark}

\begin{lemma}[Contraction implies stable linearization]\label{lem:contraction-linearization}
Let $F:\R^d\to\R^d$ be differentiable and assume it is globally Lipschitz with constant $L<1$, i.e.\ $\|F(x)-F(y)\|\le L\|x-y\|$ for all $x,y$.
Then for every $x\in\R^d$, the Jacobian matrix $DF(x)$ satisfies $\|DF(x)\|\le L$, hence $\rho(DF(x))\le L<1$.
\end{lemma}
\begin{proof}
Fix $x\in\R^d$ and a unit vector $v$.
For small $t\neq 0$, the mean-value estimate gives
\[
\frac{\|F(x+tv)-F(x)\|}{|t|}\ \le\ L\,\frac{\|tv\|}{|t|}=L.
\]
Taking $t\to 0$ and using differentiability yields $\|DF(x)v\|\le L$.
Taking the supremum over unit vectors $v$ gives $\|DF(x)\|\le L$.
Finally, for any linear operator $A$, the spectral radius satisfies $\rho(A)\le \|A\|$, so $\rho(DF(x))<1$.
\end{proof}

\begin{remark}[Stable $A$ from a cost-driven tick]
Lemma~\ref{lem:prox-contraction} shows that the RS-prox tick $\Pi_\lambda$ is Lipschitz with constant $L=1/(1+\lambda)<1$.
Whenever one passes to a differentiable finite-dimensional coordinate model (e.g.\ by working in log-coordinates on a finite-resolution quotient and linearizing around a reference trajectory), Lemma~\ref{lem:contraction-linearization} implies that the resulting one-tick linear update matrix $A$ satisfies $\rho(A)<1$.
This is exactly the stability hypothesis required in the bounded-real certificate (Theorem~\ref{thm:bounded-real}) and the tail bound from a realization (Lemma~\ref{lem:tail-from-realization}).
\end{remark}

\section{Recognition Geometry and finite local resolution (RG4)}\label{sec:rg4}

Recognition Geometry (RG) formalizes a measurement-first viewpoint: recognizers are primitive, and observable space is a quotient by indistinguishability.
RSA uses one axiom from this framework as a \emph{universally reusable} restriction: finite local resolution.

\begin{definition}[Finite local resolution (RG4)]
Let $\mathcal C$ be a configuration space, $\mathcal E$ an event space, $\mathcal N(c)$ a neighborhood system, and $R:\mathcal C\to\mathcal E$ a recognizer.
RG4 asserts:
for every $c\in\mathcal C$ there exists $U\in\mathcal N(c)$ such that $R(U)$ is finite, i.e.\ $|R(U)|<\infty$.
\end{definition}
\begin{remark}
In the full Recognition Geometry axiom list \cite{WashburnRecognitionGeometry2025}, \emph{indistinguishability} is RG3 and \emph{finite local resolution} is RG4. We use only the finite-resolution content here.
\end{remark}

\begin{remark}[Why RG4 matters for RSA]
Finite certificates can only imply global control if the audited objects belong to a restricted class.
RG4 provides the correct kind of restriction in an operational setting: it excludes idealized ``infinite precision'' recognizers and motivates finite-complexity hypotheses for the fields that RSA audits.
\end{remark}

\subsection{Deriving RG4 from an 8-tick physical realizability model}
The full-derivation route requires that ``finite resolution'' and ``finite complexity'' are not merely asserted as desiderata.
They must follow from a precise model of what it means for a recognition procedure to be physically realizable in one finite window.
We now record a minimal (and deliberately explicit) discrete-time model that captures the 8-tick constraint and implies RG4.

\begin{definition}[One-tick reachability and 8-tick reachability]\label{def:reachability}
Let $\ledger$ be a set of internal ledger states.
A \emph{one-tick reachability operator} is a map $\Rhat:\ledger\to \mathcal P(\ledger)$, where $\Rhat(\ell)$ is the set of states reachable from $\ell$ in one recognition tick.
Define the $n$-tick reachable set recursively by
\[
\Rhat^{(0)}(\ell)=\{\ell\},\qquad
\Rhat^{(n+1)}(\ell)=\bigcup_{\ell'\in \Rhat^{(n)}(\ell)} \Rhat(\ell').
\]
The \emph{8-tick reachable set} is $\Rhat^{(8)}(\ell)$.
\end{definition}

\begin{definition}[Neighborhoods induced by reachability]\label{def:reachability-neighborhoods}
Given $\Rhat$, define a neighborhood system $\mathcal N_{\Rhat}$ on $\ledger$ by declaring $U\subseteq \ledger$ to be a neighborhood of $\ell$ if it contains the entire 8-tick reachable set:
\[
U\in \mathcal N_{\Rhat}(\ell)\quad \Longleftrightarrow\quad \Rhat^{(8)}(\ell)\subseteq U.
\]
In particular, $\Rhat^{(8)}(\ell)\in \mathcal N_{\Rhat}(\ell)$ for every $\ell$.
\end{definition}

\begin{definition}[Finite local branching (per tick)]\label{def:finite-branching}
We say $\Rhat$ has \emph{finite local branching} if for every $\ell\in\ledger$ the set $\Rhat(\ell)$ is finite.
If there exists a uniform bound $b\in\mathbb N$ such that $|\Rhat(\ell)|\le b$ for all $\ell$, we say $\Rhat$ has \emph{branching bound $b$}.
\end{definition}

\begin{lemma}[Finite branching implies finite 8-tick reachability]\label{lem:finite-8tick}
If $\Rhat$ has branching bound $b$, then for every $\ell\in\ledger$, the 8-tick reachable set $\Rhat^{(8)}(\ell)$ is finite and
\[
|\Rhat^{(8)}(\ell)|\ \le\ 1+b+b^2+\cdots+b^8\ \le\ \frac{b^9-1}{b-1}\qquad (b\ge 2),
\]
with the obvious bound $|\Rhat^{(8)}(\ell)|\le 9$ when $b=1$.
\end{lemma}
\begin{proof}
By induction on $n$.
For $n=0$ the set $\Rhat^{(0)}(\ell)=\{\ell\}$ is finite.
Assume $\Rhat^{(n)}(\ell)$ is finite and $|\Rhat(\ell')|\le b$ for all $\ell'$.
Then $\Rhat^{(n+1)}(\ell)=\bigcup_{\ell'\in \Rhat^{(n)}(\ell)} \Rhat(\ell')$ is a finite union of finite sets, hence finite, and
\[
|\Rhat^{(n+1)}(\ell)|\le \sum_{\ell'\in \Rhat^{(n)}(\ell)} |\Rhat(\ell')|\le b\,|\Rhat^{(n)}(\ell)|.
\]
Iterating gives $|\Rhat^{(n)}(\ell)|\le 1+b+\cdots+b^n$, and in particular the stated 8-tick bound.
\end{proof}

\begin{definition}[8-tick realizable recognizer]\label{def:8tick-realizable}
Let $M:\ledger\to\mathcal E$ be a measurement/recognizer output map to an event set $\mathcal E$.
We say $M$ is \emph{8-tick realizable at $\ell$} (with respect to $\Rhat$) if the set of reachable outcomes
\[
M(\Rhat^{(8)}(\ell))\ :=\ \{\,M(\ell'):\ \ell'\in \Rhat^{(8)}(\ell)\,\}
\]
is finite.
\end{definition}

\begin{proposition}[8-tick realizability implies RG4 (finite local resolution)]\label{prop:8tick-implies-rg4}
Assume $\Rhat$ has finite local branching (Definition~\ref{def:finite-branching}) and $M:\ledger\to\mathcal E$ is a recognizer.
Equip $\ledger$ with the reachability-induced neighborhood system $\mathcal N_{\Rhat}$ (Definition~\ref{def:reachability-neighborhoods}).
Then for every $\ell\in\ledger$ there exists a neighborhood $U\in \mathcal N_{\Rhat}(\ell)$ such that $M(U)$ is finite; in other words, RG4 holds for the configuration space $\ledger$, event space $\mathcal E$, neighborhood system $\mathcal N_{\Rhat}$, and recognizer $M$.
\end{proposition}
\begin{proof}
Fix $\ell\in\ledger$ and take $U:=\Rhat^{(8)}(\ell)$.
By Definition~\ref{def:reachability-neighborhoods} we have $U\in \mathcal N_{\Rhat}(\ell)$.
By Lemma~\ref{lem:finite-8tick}, the set $U$ is finite, hence its image $M(U)$ is finite.
\end{proof}

\begin{remark}[What has (and has not) been assumed]
Proposition~\ref{prop:8tick-implies-rg4} makes explicit the exact content needed to obtain RG4 from an 8-tick model:
finite local branching of the one-tick reachability operator.
This is where physical realizability enters: if in one tick the system could branch into infinitely many distinct next-states, then no finite-window audit can be meaningful.
In this manuscript, the role of Recognition Cost is to supply an explicit, non-branching and contractive tick mechanism under finite cost budgets (Section~\ref{subsec:prox-contract}).
In particular, the canonical $J$-projection step (Theorem~\ref{thm:J-projection}) is single-valued and nonexpansive in log-coordinates (Corollary~\ref{cor:logspace-projection}), so it introduces no branching and supplies a natural Lyapunov structure.
\end{remark}

\subsection{From finite resolution to a finite-state model (the first missing bridge)}
RSA needs a \emph{finite-to-global} bridge: a reason that a finite computation (``eight ticks,'' or any finite certificate) can control the whole interior of an audited region.
In the full-derivation route, this bridge is not assumed. It is extracted from Recognition Geometry by passing to a \emph{local recognition quotient} and then representing its induced dynamics as a finite-state system.

\begin{definition}[Indistinguishability and local recognition quotient]\label{def:local-quotient}
Fix a recognizer $R:\mathcal C\to\mathcal E$ and a subset $U\subseteq \mathcal C$.
Define an equivalence relation $\sim_R$ on $U$ by
\[
c_1\sim_R c_2 \quad\Longleftrightarrow\quad R(c_1)=R(c_2).
\]
Write $U/{\sim_R}$ for the set of equivalence classes (the local recognition quotient).
\end{definition}

\begin{lemma}[RG4 implies a finite local quotient]\label{lem:rg4-finite-quotient}
If $R(U)$ is finite, then $U/{\sim_R}$ is finite and $|U/{\sim_R}|\le |R(U)|$.
\end{lemma}
\begin{proof}
Each equivalence class in $U/{\sim_R}$ is mapped by $R$ to a single value in $R(U)$, and different classes map to different values.
Thus the induced map $U/{\sim_R}\to R(U)$ is injective, so $|U/{\sim_R}|\le |R(U)|<\infty$.
\end{proof}

\begin{definition}[Recognition-respecting update and induced quotient dynamics]\label{def:quotient-dynamics}
Let $T:U\to U$ be a one-step update map (one ``tick'') on configurations.
We say $T$ is \emph{recognition-respecting on $U$} if
\[
c_1\sim_R c_2 \;\Longrightarrow\; T(c_1)\sim_R T(c_2)\qquad (c_1,c_2\in U).
\]
In this case the quotient map $\bar T:U/{\sim_R}\to U/{\sim_R}$ defined by $\bar T([c])=[T(c)]$ is well-defined.
\end{definition}

\begin{definition}[Finite-state realization (local, discrete-time)]\label{def:finite-state-realization}
Assume $U/{\sim_R}$ is finite and $T$ is recognition-respecting on $U$.
Let $S:=U/{\sim_R}$, choose an initial class $s_0\in S$, and fix an \emph{output map} $O:S\to\C$.
The resulting discrete-time process is the finite-state system
\[
s_{n+1}=\bar T(s_n),\qquad y_n=O(s_n)\qquad (n\ge 0),
\]
with state space $S$ and output sequence $(y_n)_{n\ge 0}$.
\end{definition}

\begin{theorem}[Finite-state $\Rightarrow$ rational generating function]\label{thm:finite-state-rational}
In the setup of Definition~\ref{def:finite-state-realization}, define the power series
\[
F(z)\;:=\;\sum_{n\ge 0} y_n\,z^n \qquad (|z|<1).
\]
Then $F$ is a \emph{rational} function of $z$.
More precisely, if $d=|S|$ then there exists a $d\times d$ matrix $A$ and vectors $u,v\in\C^d$ such that
\[
F(z)=u^\ast (I-zA)^{-1}v,
\]
so the denominator of $F$ divides $\det(I-zA)$ and has degree at most $d$.
\end{theorem}
\begin{proof}
Enumerate $S=\{1,\dots,d\}$ and let $e_j$ be the standard basis of $\C^d$.
Define $A$ by $A e_j := e_{\bar T(j)}$, so $A^n e_{s_0}=e_{s_n}$.
Let $v:=e_{s_0}$ and let $u\in\C^d$ be the vector with entries $u_j := O(j)$, so that $y_n = u^\ast A^n v$.
Then for $|z|<1$,
\[
F(z)=\sum_{n\ge 0} u^\ast A^n v\, z^n
=u^\ast\Big(\sum_{n\ge 0}(zA)^n\Big)v
=u^\ast(I-zA)^{-1}v,
\]
where the Neumann series converges for $|z|<1$ because $S$ is finite and $A$ is a bounded linear operator on $\C^d$.
The right-hand side is a ratio of polynomials because $(I-zA)^{-1}=\mathrm{adj}(I-zA)/\det(I-zA)$.
\end{proof}

\begin{corollary}[Finite-state $\Rightarrow$ state-space transfer form]\label{cor:finite-state-state-space}
In the setup of Theorem~\ref{thm:finite-state-rational}, the same function $F$ admits a (finite-dimensional) state-space representation of the standard discrete-time transfer form
\[
F(z)=D+z\,C\,(I-zA)^{-1}B,
\]
as in Definition~\ref{def:state-space}.
Concretely, one may take $B=v$, $C=u^\ast A$, and $D=u^\ast v$.
\end{corollary}
\begin{proof}
Using $(I-zA)^{-1}=I+zA(I-zA)^{-1}$, we compute
\[
F(z)=u^\ast(I-zA)^{-1}v
=u^\ast v + z\,u^\ast A (I-zA)^{-1}v,
\]
which has the required form with $D=u^\ast v$, $C=u^\ast A$, and $B=v$.
\end{proof}

\begin{corollary}[Finite resolution $\Rightarrow$ rational coefficients]\label{cor:finite-state-rational-coeffs}
In the setup of Theorem~\ref{thm:finite-state-rational}, suppose the output map $O:S\to\C$ takes values in a subfield $K\subseteq\C$.
Then the resulting generating function $F$ lies in $K(z)$.
In particular, if $O$ takes values in $\mathbb{Q}$ (finite-resolution outputs), then $F\in\mathbb{Q}(z)$.
\end{corollary}
\begin{proof}
In the proof of Theorem~\ref{thm:finite-state-rational}, the matrix $A$ has entries in $\{0,1\}\subseteq\mathbb{Q}$ by construction, and the vectors $v$ and $u$ have entries in $\mathbb{Q}$ and $K$ respectively.
Therefore $(I-zA)^{-1}=\mathrm{adj}(I-zA)/\det(I-zA)$ has entries in $\mathbb{Q}(z)\subseteq K(z)$, and multiplying on the left by $u^\ast$ and on the right by $v$ shows $F\in K(z)$.
\end{proof}

\begin{theorem}[Realizability $\Rightarrow$ finite complexity (rational audited class)]\label{thm:realizability-finite-complexity}
Assume an 8-tick reachability model $(\ledger,\Rhat)$ with finite local branching (Definition~\ref{def:finite-branching}) and a recognizer $R:\ledger\to\mathcal E$.
Fix $\ell_0\in\ledger$ and set $U:=\Rhat^{(8)}(\ell_0)$.
Assume a recognition-respecting one-tick update $T:U\to U$ (Definition~\ref{def:quotient-dynamics}) and an output map $O:U/{\sim_R}\to K$ into a subfield $K\subseteq\C$.
Let $(s_n)_{n\ge 0}$ be the induced quotient dynamics on $S:=U/{\sim_R}$ with output $y_n:=O(s_n)$ (Definition~\ref{def:finite-state-realization}), and define
\[
\theta(z)\;:=\;\sum_{n\ge 0} y_n\,z^n\qquad (|z|<1).
\]
Then $\theta\in K(z)$ is a rational function whose denominator has degree at most $|S|$.
\end{theorem}
\begin{proof}
By Lemma~\ref{lem:finite-8tick}, the 8-tick reachable set $U=\Rhat^{(8)}(\ell_0)$ is finite.
Hence $R(U)$ is finite and Lemma~\ref{lem:rg4-finite-quotient} implies $S=U/{\sim_R}$ is finite.
Since $T$ is recognition-respecting, the induced quotient update $\bar T:S\to S$ is well-defined (Definition~\ref{def:quotient-dynamics}), so Definition~\ref{def:finite-state-realization} applies.
Theorem~\ref{thm:finite-state-rational} gives rationality of $\theta$, and Corollary~\ref{cor:finite-state-rational-coeffs} places $\theta$ in $K(z)$.
\end{proof}

\begin{remark}[Connection to the Cayley audit]
Once an audited Cayley field $\theta$ belongs to a rational class (as in Theorem~\ref{thm:realizability-finite-complexity}), the associated sensor and obstruction inherit rationality through the Cayley transform and its inverse (Lemma~\ref{lem:cayley-rational}).
This is the precise sense in which finite-window realizability turns analytic auditing into a finite-dimensional algebraic problem.
\end{remark}

\begin{remark}[What this buys RSA]
Theorem~\ref{thm:finite-state-rational} is the first hard upgrade from ``program'' to ``certificate-grade theorem'':
under RG4 and recognition-respecting dynamics, the audited object is not an arbitrary holomorphic function.
It lies in a \emph{finite-dimensional rational family} whose global behavior is determined by finitely many parameters.
We use this in Lemma~\ref{lem:tail-from-realization} and Theorem~\ref{thm:finite-certificate-global} to replace ``assume a tail bound'' with a derived, finite-complexity certificate.
\end{remark}

\section{Sensors as obstruction reciprocals}

RSA needs a disciplined way to define ``correct sensors'' across domains.
The clean universal pattern is: define a sensor as the reciprocal of an obstruction that vanishes exactly when the candidate mechanism occurs.

\begin{definition}[Candidate and obstruction]\label{def:candidate-obstruction}
Fix a domain $\Omega$ (typically a region in $\C$ or a parameter space) and a candidate statement $S$.
An \emph{obstruction} is a function $G_S:\Omega\to\C$ such that, for some $z_\star\in\Omega$,
\[
S \text{ occurs at } z_\star \;\Longrightarrow\; G_S(z_\star)=0.
\]
If one has the biconditional $S$ at $z_\star \Longleftrightarrow G_S(z_\star)=0$, we call $G_S$ \emph{exact}.
\end{definition}

\begin{definition}[Sensor as reciprocal]\label{def:sensor-reciprocal}
Given an obstruction $G_S$ that is nonzero on admissible states, define the sensor
\[
\mathcal{J}_S(z)\;:=\;\frac{1}{G_S(z)}.
\]
\end{definition}

\begin{lemma}[Sensor correctness is a reduction]\label{lem:sensor-reduction}
If $S$ at $z_\star$ implies $G_S(z_\star)=0$ and $G_S$ is holomorphic near $z_\star$ with a simple zero, then $\mathcal{J}_S$ has a pole at $z_\star$.
\end{lemma}

\begin{remark}
This is the only truly domain-agnostic way to make ``sensor correctness'' rigorous: the domain work lives in constructing $G_S$ and proving the implication $S\Rightarrow G_S=0$.
\end{remark}

\subsection{RS defect semantics makes obstruction-zeros canonical}\label{subsec:rs-defect}
In Recognition Science, ``existence'' is not a primitive logical predicate; it is defined operationally in terms of a defect/cost that must converge to zero at a stable configuration.
This viewpoint supplies a canonical source of obstructions:
\emph{a candidate holds exactly when its defect vanishes}.
We record this as an explicit schema.

\begin{definition}[Defect functional (RS encoding of an existence claim)]\label{def:defect-functional}
Fix a domain $\Omega$ and a candidate statement $S$.
A \emph{defect functional} for $S$ is a map
\[
\Delta_S:\Omega\to \R_{\ge 0}
\]
with the intended semantics:
\[
S \text{ holds at } z \quad\Longleftrightarrow\quad \Delta_S(z)=0.
\]
\end{definition}

\begin{proposition}[Existence $\Rightarrow$ obstruction zero (schema)]\label{prop:existence-obstruction-zero}
Let $\Delta_S$ be a defect functional in the sense of Definition~\ref{def:defect-functional}.
Define the obstruction $G_S:=\Delta_S$.
Then for any $z_\star\in\Omega$,
\[
S \text{ holds at } z_\star \;\Longrightarrow\; G_S(z_\star)=0.
\]
If moreover $\Delta_S$ is \emph{exact} (i.e.\ the biconditional holds), then $G_S$ is an exact obstruction in the sense of Definition~\ref{def:candidate-obstruction}.
\end{proposition}
\begin{proof}
Immediate from the defining semantics of $\Delta_S$.
\end{proof}

\begin{remark}[Analytic RS-admissibility]
RSA is an analytic audit, so one typically works in a setting where the defect admits an analytic representative:
either $\Delta_S$ itself extends to a holomorphic function on $\Omega$, or one constructs a holomorphic $G_S$ with the same zero set as $\Delta_S$.
Under this analytic admissibility, the reciprocal sensor $\mathcal J_S=1/G_S$ is a canonical blow-up detector for the candidate.
\end{remark}

\section{Bounded Cayley fields and the Schur/Herglotz pinch}\label{sec:pinch}

\subsection{The Cayley transform}

\begin{definition}[Bounded transform]
Define the bounded field $\Xi_S$ by
\begin{equation}\label{eq:Cayley}
\Xi_S(z)\;=\;\frac{2\mathcal{J}_S(z)-1}{2\mathcal{J}_S(z)+1}.
\end{equation}
\end{definition}

\begin{lemma}[Cayley transform preserves rationality]\label{lem:cayley-rational}
If $\mathcal J_S$ is a rational function of $z$, then $\Xi_S$ defined by \eqref{eq:Cayley} is rational.
Conversely, if $\Xi_S$ is rational and not identically $1$, then the Cayley inverse
\begin{equation}\label{eq:cayley-inverse}
2\mathcal J_S(z)\;=\;\frac{1+\Xi_S(z)}{1-\Xi_S(z)}
\end{equation}
is rational on its domain of definition, hence $\mathcal J_S$ is rational.
\end{lemma}
\begin{proof}
This is immediate from algebra: the Cayley transform \eqref{eq:Cayley} and its inverse \eqref{eq:cayley-inverse} are obtained by a finite number of additions, multiplications, and divisions.
These operations preserve rationality.
The nontriviality condition $\Xi_S\not\equiv 1$ ensures the inverse expression is not identically undefined.
\end{proof}

\begin{lemma}[Pole-to-boundary behavior]
If $\mathcal{J}_S(z)\to\infty$ along an approach to $z=z_\star$, then $\Xi_S(z)\to 1$ along that approach.
\end{lemma}
\begin{proof}
Write
\[
\Xi_S(z)-1 \;=\; \frac{2\mathcal J_S(z)-1}{2\mathcal J_S(z)+1}-1
\;=\; \frac{-2}{2\mathcal J_S(z)+1}.
\]
If $\mathcal J_S(z)\to\infty$ along an approach, then the denominator tends to $\infty$ and hence $\Xi_S(z)-1\to 0$.
\end{proof}

\begin{lemma}[Right half-plane to unit disk]
If $\Re(\mathcal{J}_S(z))>0$ on a region, then $|\Xi_S(z)|<1$ on that region.
\end{lemma}
\begin{proof}
Let $w=\mathcal J_S(z)$ and $\Xi=\frac{2w-1}{2w+1}$.
Then $|\Xi|<1$ is equivalent to $|2w-1|<|2w+1|$.
Squaring both sides gives
\[
|2w-1|^2 < |2w+1|^2
\;\Longleftrightarrow\;
(2w-1)(2\overline w-1) < (2w+1)(2\overline w+1)
\;\Longleftrightarrow\;
-2(w+\overline w) < 2(w+\overline w),
\]
which is equivalent to $\Re(w)>0$.
\end{proof}

\subsection{Why a Schur bound prevents poles (removability pinch)}

\begin{lemma}[Removable singularity under a Schur bound]\label{lem:removable-schur}
Let $D\subset\C$ be a disc centered at $\rho$.
If $\Xi$ is holomorphic on $D\setminus\{\rho\}$ and satisfies $|\Xi|<1$ there, then $\Xi$ extends holomorphically to $D$.
\end{lemma}
\begin{proof}
Since $\Xi$ is bounded on the punctured disc, Riemann's removable singularity theorem applies; see, e.g., \cite{RudinRCA}.
\end{proof}

\begin{corollary}[Schur bound prevents poles of the Cayley inverse]\label{cor:no-poles}
Let $U\subset\Omega$ be a domain.
Assume $\Xi$ is meromorphic on $U$ and satisfies $|\Xi|\le 1$ on $U$ away from its poles.
Then $\Xi$ extends holomorphically to $U$, and the Cayley inverse
\[
2\mathcal{J}\;=\;\frac{1+\Xi}{1-\Xi}
\]
extends holomorphically to $U$; in particular $\mathcal{J}$ has no poles in $U$.
\end{corollary}
\begin{proof}
The poles of $\Xi$ form a discrete subset of $U$.
On a punctured disc around any pole, $|\Xi|\le 1$ implies $\Xi$ is bounded, hence removable by Lemma~\ref{lem:removable-schur}.
Thus $\Xi$ extends holomorphically across all its poles.
If $\Xi$ is not identically $1$ on $U$, then $\Xi\neq 1$ on $U$ (otherwise $|\Xi|$ would achieve its maximum $1$ at an interior point and the Maximum Modulus Principle would force $\Xi\equiv 1$).
Therefore the Cayley inverse $(1+\Xi)/(1-\Xi)$ is holomorphic on $U$ and equals $2\mathcal J$ wherever originally defined, so $\mathcal J$ has no poles in $U$.
\end{proof}

\begin{remark}[Degenerate case $\Xi\equiv 1$]
If $\Xi(s_0)=1$ at an interior point and $|\Xi|\le 1$ in a connected region, then $\Xi\equiv 1$ by the Maximum Modulus Principle.
In RSA applications one excludes this by a normalization (e.g.\ a limit at ``infinity'') or by an explicit nontriviality check.
\end{remark}

\section{Why finite sampling needs a complexity bound}

The following elementary fact is the key ``referee objection'' RSA must address.

\begin{proposition}[No finite sampling can rule out poles without extra structure]\label{prop:no-finite-sampling}
Fix distinct sample points $z_1,\dots,z_n\in\C$ and target values $w_1,\dots,w_n\in\C$.
For any point $a\in\C\setminus\{z_1,\dots,z_n\}$ there exists a meromorphic function $f$ on $\C$ such that $f(z_k)=w_k$ for all $k$ and $f$ has a pole at $a$.
\end{proposition}

\begin{proof}
Let $p$ be the Lagrange interpolating polynomial with $p(z_k)=w_k$.
Define
\[
f(z)\;:=\;p(z)\;+\;\frac{(z-z_1)\cdots(z-z_n)}{z-a}.
\]
Then $f(z_k)=p(z_k)=w_k$ for all $k$, and $f$ has a pole at $z=a$.
\end{proof}

\begin{remark}[Tail risk and the Pick-gap-plus-tail guardrail]
Therefore RSA cannot honestly claim that ``8 samples'' alone give global pole-freeness.
Any finite sampling regime is vulnerable to \emph{tail risk}: without additional structure, the true audited field can ``wiggle'' between sample points (or develop an interior singularity) while still matching the sampled values.
RSA therefore never treats finite samples as a global certificate by themselves.

To make a finite computation global one needs a \emph{finite-complexity control statement}:
either a finite-dimensional restriction (finite state/degree, as derived from 8-tick realizability in Theorem~\ref{thm:realizability-finite-complexity}),
or a proved tail bound in a norm controlling the interior.
The coefficient \emph{Pick-gap-plus-tail} mechanism (Proposition~\ref{prop:finite-gap-tail}) is the explicit guardrail: it converts a finite spectral gap together with a quantitative tail bound into positivity of the infinite Pick operator, hence a global Schur conclusion.
In the stable realization regime, such tail bounds can be derived from stability data (Lemma~\ref{lem:tail-from-realization}).
\end{remark}

\section{Eight-tick discretization (operational sampling form)}

Recognition Science motivates an eight-phase cadence (an ``eight-tick window'').
In the full-derivation route, this cadence is not merely philosophical: it is the operational reason that \emph{finite-window certificates} are the only admissible kind of audit.
Section~\ref{sec:rg4} made this explicit by modeling one-tick reachability $\Rhat$ and proving that finite local branching implies finite 8-tick outcome sets (Proposition~\ref{prop:8tick-implies-rg4}), i.e.\ a concrete route to finite local resolution.

Once an audited domain $\Omega$ is normalized to the unit disk $\D$ by a chart $\psi:\Omega\to\D$, one natural way to represent an eight-tick audit is to allocate one sample per tick on a symmetric sampling pattern on $\D$.
We record the standard Nevanlinna--Pick matrix for \emph{values} at these eight points, and we emphasize exactly what such a finite value certificate does (and does not) certify without additional finite-complexity control.

\begin{definition}[Eight-tick sampling set]\label{def:eight-tick-sampling}
Fix $r\in(0,1)$ and define the eight roots of unity $\omega_k=e^{2\pi i k/8}$, $k=0,\dots,7$.
The eight-tick sampling points are
\[
z_k \;=\; r\,\omega_k \in \D.
\]
\end{definition}

\begin{definition}[Value-sample Pick matrix]\label{def:pick-values}
Let $\theta:\D\to\C$ be holomorphic and set $\xi_k=\theta(z_k)$ at the points in Definition~\ref{def:eight-tick-sampling}.
Define the $8\times 8$ \emph{value-sample Pick matrix}
\[
P^{\mathrm{val}}
\;=\;
\left[\frac{1-\xi_i\overline{\xi_j}}{1-z_i\overline{z_j}}\right]_{i,j=0}^{7}.
\]
\end{definition}

\begin{proposition}[Schur $\Rightarrow$ value-sample Pick positivity]\label{prop:schur-implies-pick}
If $\theta$ is Schur on $\D$, then $P^{\mathrm{val}}\succeq 0$.
\end{proposition}

\begin{remark}[What finite Pick positivity actually means]
If $P^{\mathrm{val}}\succeq 0$, the Pick criterion implies that there exists \emph{some} Schur function $\widetilde\theta$ with $\widetilde\theta(z_k)=\theta(z_k)$ for $k=0,\dots,7$.
This does \emph{not} by itself imply that the specific function $\theta$ is Schur (Proposition~\ref{prop:no-finite-sampling}).
To use value-sample positivity as a global certificate for this specific function $\theta$, one must also control uniqueness:
either $\theta$ is known to lie in a finite-dimensional class where finitely many data determine $\theta$ (e.g.\ the finite-state/rational regime of Theorem~\ref{thm:realizability-finite-complexity}),
or one proves a quantitative deviation bound (a tail bound) controlling $\theta-\widetilde\theta$ in the interior.
\end{remark}

\section{Finite certificates for global Schur control}

RSA reduces impossibility claims to the absence of poles of a sensor, via a Schur bound for the associated Cayley field.
To make this \emph{algorithmic}, we need a finite certificate that implies a global Schur bound on $\D$.
We record two complementary routes:
(i) a \emph{state-space bounded-real} certificate that is exact when a finite-dimensional realization is available,
and (ii) a \emph{Pick-gap-plus-tail} certificate that is robust when working from Taylor data with rigorous error bars.

\subsection{State-space bounded-real certificate (exact in the finite-dimensional class)}\label{subsec:bounded-real}

\begin{definition}[State-space realization]\label{def:state-space}
A (discrete-time) state-space realization of a scalar holomorphic function $\theta:\D\to\C$ is a quadruple $(A,B,C,D)$ with $A\in\C^{d\times d}$, $B\in\C^{d\times 1}$, $C\in\C^{1\times d}$, $D\in\C$ such that
\begin{equation}\label{eq:state-space-transfer}
\theta(z)\;=\;D\;+\;z\,C\,(I-zA)^{-1}B.
\end{equation}
\end{definition}

\begin{theorem}[Discrete-time bounded-real lemma (scalar form)]\label{thm:bounded-real}
Assume $\theta$ admits a realization $(A,B,C,D)$ with $\rho(A)<1$ (spectral radius strictly less than $1$), so that \eqref{eq:state-space-transfer} is holomorphic on $\D$.
Then $\theta$ is Schur on $\D$ (i.e.\ $|\theta(z)|\le 1$ for all $z\in\D$) if and only if there exists a Hermitian matrix $P\succ 0$ such that
\begin{equation}\label{eq:brl-lmi}
\begin{pmatrix}
P - A^\ast P A - C^\ast C & -A^\ast P B - C^\ast D \\
-B^\ast P A - D^\ast C & 1 - B^\ast P B - D^\ast D
\end{pmatrix}\ \succeq\ 0.
\end{equation}
This is the standard discrete-time bounded-real (KYP) lemma; see, e.g., \cite{ZhouDoyleGlover1996}.
\end{theorem}

\begin{remark}[Why this matters for the full-derivation route]
Theorem~\ref{thm:bounded-real} turns Schur certification into a finite-dimensional semidefinite feasibility problem.
In the full-derivation route, RG4 yields finite-dimensionality (Section~\ref{sec:rg4}, Theorem~\ref{thm:finite-state-rational}).
Recognition Cost supplies a canonical source of contraction in log-coordinates via the proximal tick model (Section~\ref{subsec:prox-contract}, Lemma~\ref{lem:prox-contraction});
and strict contraction implies a stable linearization (Lemma~\ref{lem:contraction-linearization}), which is the structural origin of the stability hypothesis $\rho(A)<1$ in state-space certification.
This route is \emph{exact} on the finite-dimensional class: there is no tail bound to assume.
\end{remark}

\begin{remark}[KYP/LMI as a cost (dissipation) inequality]\label{rem:kyp-dissipation}
Control-theoretically, the matrix $P$ is a \emph{storage} (Lyapunov) certificate for the state $x_k$ of the realization:
one interprets $V(x)=x^\ast P x$ as an ``internal cost'' and the LMI \eqref{eq:brl-lmi} as the existence of a quadratic dissipation inequality of the form
\[
V(x_{k+1})-V(x_k)\ \le\ \|u_k\|^2-\|y_k\|^2
\]
along all trajectories of the state-space system.
This is precisely the formal meaning of ``contractive/passive dynamics'' for a finite-dimensional recognition process.
Thus, in the full-derivation route, the key RS-facing obligation is not ``assume bounded-real'' but rather:
exhibit a cost-driven mechanism that yields such a storage inequality (perhaps after passing to log-coordinates and a finite-resolution quotient), thereby making the bounded-real certificate intrinsic.
Corollary~\ref{cor:intrinsic-cost-certification} records one explicit route in a cost-contracting regime.
\end{remark}

\begin{corollary}[Intrinsic certification in the cost-contracting regime]\label{cor:intrinsic-cost-certification}
Suppose the pulled-back Cayley field $\theta$ admits a finite-dimensional realization \eqref{eq:state-space-transfer} whose one-tick state update $x\mapsto Ax$ arises as the linearization of a differentiable strict contraction map with Lipschitz constant $L<1$ (for example, a differentiable coordinate model of the RS proximal tick in Section~\ref{subsec:prox-contract}).
Then $\rho(A)<1$ (Lemma~\ref{lem:contraction-linearization}), so $\theta$ is holomorphic on $\D$ and both certification regimes become fully finite without additional stability or tail assumptions:
\begin{itemize}
\item the bounded-real (KYP/LMI) test applies in state space (Theorem~\ref{thm:bounded-real});
\item the coefficient Pick-gap certificate can use a derived tail bound from realization data (Lemma~\ref{lem:tail-from-realization}), hence implies a global Schur bound by Theorem~\ref{thm:finite-certificate-global}.
\end{itemize}
\end{corollary}

\subsection{Schur class and Pick kernels on the disk}

\begin{definition}[Schur kernel]
Let $\theta:\D\to\C$ be holomorphic.
Define its Pick kernel
\[
K_\theta(z,w)\;:=\;\frac{1-\theta(z)\overline{\theta(w)}}{1-z\overline w}.
\]
\end{definition}

\begin{theorem}[Pick criterion, functional form]\label{thm:pick-criterion}
A holomorphic $\theta:\D\to\C$ is Schur (i.e.\ $|\theta|\le 1$ on $\D$) if and only if $K_\theta$ is a positive semidefinite kernel on $\D$ (equivalently: every finite Pick matrix $[K_\theta(z_i,z_j)]$ is positive semidefinite).
\end{theorem}
\begin{remark}
This is the classical Nevanlinna--Pick/Schur criterion; see \cite[Ch.~2]{RosenblumRovnyak} or \cite{Donoghue}.
\end{remark}

\subsection{Coefficient Pick matrix and a stability lemma}

\begin{definition}[Coefficient Pick matrix]
Write $\theta(z)=\sum_{n\ge 0} a_n z^n$ and expand
\[
K_\theta(z,w)\;=\;\sum_{i,j\ge 0} P_{ij}\, z^i \overline w^{\,j}.
\]
The infinite Hermitian matrix $P=[P_{ij}]_{i,j\ge 0}$ is the \emph{coefficient Pick matrix} of $\theta$.
\end{definition}

\begin{lemma}[Coefficient formula]\label{lem:pick-coeff-formula}
With notation above, for all $i,j\ge 0$,
\[
P_{ij}\;=\;\delta_{ij}\;-\;\sum_{k=0}^{\min\{i,j\}} a_{i-k}\,\overline{a_{j-k}}.
\]
Equivalently $P=I-AA^*$ where $A$ is the lower-triangular Toeplitz matrix $A_{ij}=a_{i-j}$ for $i\ge j$ and $0$ for $i<j$.
\end{lemma}

\begin{definition}[Weighted tail]
For $N\ge 1$, define the Dirichlet-type tail size
\[
\varepsilon_N^2\;:=\;\sum_{n\ge N}(n+1)\,|a_n|^2.
\]
\end{definition}

\begin{lemma}[Tail bound from a contractive realization]\label{lem:tail-from-realization}
Assume $\theta$ admits a state-space realization \eqref{eq:state-space-transfer} and write its Taylor coefficients as $\theta(z)=\sum_{n\ge 0} a_n z^n$.
If $\|A\|\le \rho<1$ (operator norm on $\C^d$), then for every $N\ge 1$,
\[
\varepsilon_N^2
\;=\;\sum_{n\ge N}(n+1)\,|a_n|^2
\;\le\;
\|C\|^2\,\|B\|^2\left(\frac{(N+1)\rho^{2(N-1)}}{1-\rho^2}+\frac{\rho^{2N}}{(1-\rho^2)^2}\right).
\]
\end{lemma}
\begin{proof}
Expanding \eqref{eq:state-space-transfer} as a power series gives $a_0=D$ and, for $n\ge 1$,
\[
a_n \;=\; C\,A^{n-1}B.
\]
Therefore $|a_n|\le \|C\|\,\|A\|^{n-1}\,\|B\|\le \|C\|\,\rho^{n-1}\,\|B\|$ for all $n\ge 1$.
Hence, for any $N\ge 1$,
\[
\varepsilon_N^2
\;=\;\sum_{n\ge N}(n+1)\,|a_n|^2
\;\le\;\|C\|^2\,\|B\|^2\sum_{n\ge N}(n+1)\rho^{2(n-1)}.
\]
Let $r=\rho^2\in(0,1)$. A standard geometric-sum computation yields
\[
\sum_{n\ge N}(n+1)r^{n-1}
\;=\;\frac{(N+1)r^{N-1}}{1-r}+\frac{r^N}{(1-r)^2}.
\]
Substituting $r=\rho^2$ gives the stated bound.
\end{proof}

\begin{lemma}[Tail-to-infinite stability]\label{lem:pick-tail-stability}
There is an absolute constant $C\le 2$ such that, for each $N\ge 1$,
the operator difference $P(\theta)-P(\theta^{(\le N-1)})$ has norm at most $C\,\varepsilon_N$, where $\theta^{(\le N-1)}(z)=\sum_{n=0}^{N-1} a_n z^n$.
\end{lemma}

\begin{proposition}[Finite gap $+$ tail bound $\Rightarrow$ Schur]\label{prop:finite-gap-tail}
Let $\theta$ be holomorphic on $\D$ with coefficient Pick matrix $P(\theta)$.
Fix $N\ge 1$ and assume:
\begin{enumerate}
\item (finite gap) the $N\times N$ principal minor satisfies $P_N(\theta)\succeq \delta I_N$ for some $\delta>0$;
\item (tail bound) $C\,\varepsilon_N<\delta$ with $C$ as in Lemma~\ref{lem:pick-tail-stability}.
\end{enumerate}
Then $P(\theta)\succeq 0$ as an operator on $\ell^2(\mathbb N_0)$, hence $\theta$ is Schur on $\D$.
\end{proposition}

\begin{theorem}[Finite certificate $\Rightarrow$ global Schur bound]\label{thm:finite-certificate-global}
Let $\theta:\D\to\C$ be holomorphic. Suppose one of the following holds:
\begin{enumerate}
\item \emph{(State-space bounded-real certificate)} $\theta$ admits a stable realization $(A,B,C,D)$ with $\rho(A)<1$ and there exists $P\succ 0$ satisfying the bounded-real LMI \eqref{eq:brl-lmi} (Theorem~\ref{thm:bounded-real});
\item \emph{(Coefficient Pick gap $+$ tail certificate)} there exist $N\ge 1$ and $\delta>0$ such that $P_N(\theta)\succeq \delta I_N$ and $C\varepsilon_N<\delta$ (Proposition~\ref{prop:finite-gap-tail}).
\end{enumerate}
Then $\theta$ is Schur on $\D$.
\end{theorem}
\begin{proof}
In case (1), Theorem~\ref{thm:bounded-real} implies that $\theta$ is Schur.
In case (2), Proposition~\ref{prop:finite-gap-tail} implies that $\theta$ is Schur.
\end{proof}

\begin{remark}[Algorithmic certificate]
In applications, one computes $a_0,\dots,a_{N-1}$ with rigorous error bars, computes a certified spectral gap lower bound $\delta$, and proves an explicit tail bound $\varepsilon_N$.
The strict inequality $C\varepsilon_N<\delta$ converts the finite computation into a global Schur bound.
\emph{In the finite-dimensional realization regime,} a tail bound can be obtained directly from contractivity/stability data (Lemma~\ref{lem:tail-from-realization}), so the only remaining numerical work is to certify the finite gap.
\end{remark}

\section{The Recognition Stability Audit (RSA)}

\begin{algorithm}[Recognition Stability Audit (RSA), certificate form]\label{alg:RSA}
\textbf{Input:} A candidate state $S$ and an audited region $\Omega$.

\textbf{Output:} \texttt{IMPOSSIBLE\_STATE} or \texttt{INCONCLUSIVE}.

\medskip
\noindent\textbf{Step 0 (Structures to fix).}
Fix:
\begin{itemize}
\item an RS-style defect/obstruction encoding of the candidate on $\Omega$:
either a defect functional $\Delta_S$ with semantics $S$ at $z \Leftrightarrow \Delta_S(z)=0$ (Definition~\ref{def:defect-functional}),
or directly a holomorphic obstruction $G_S$ whose zero set encodes $S$ (Definition~\ref{def:candidate-obstruction}).
In either case, define the sensor by $\mathcal J_S:=1/G_S$ (Definition~\ref{def:sensor-reciprocal} and Section~\ref{subsec:rs-defect});
\item (analytic admissibility) $G_S$ is holomorphic on the chart domain and has a simple zero at any candidate point (so that $S\Rightarrow$ pole of $\mathcal J_S$ is automatic by Lemma~\ref{lem:sensor-reduction});
\item a domain normalization $\psi:\Omega\to\D$ on which $\mathcal{J}_S$ is holomorphic away from the candidate poles;
\item a \emph{realizable audited Cayley field model} for the pulled-back Cayley field $\theta(z):=\Xi_S(\psi^{-1}(z))$ in the full-derivation route:
\begin{itemize}
\item \emph{finite-window (8-tick) realizability:} an explicit 8-tick reachability model $(\ledger,\Rhat)$ with finite local branching and a recognition-respecting induced quotient dynamics whose output generating function is $\theta$; in particular $\theta$ is finite-state/rational (Theorem~\ref{thm:realizability-finite-complexity});
\item \emph{cost-driven realizability:} a cost-governed one-tick dynamics (e.g.\ the RS proximal neutrality step in log-coordinates, Section~\ref{subsec:prox-contract}) together with a differentiable coordinate model so that the induced tick is a strict contraction and its linearization yields a stable state update matrix $A$ (Lemma~\ref{lem:contraction-linearization}); consequently $\theta$ admits a stable finite-dimensional realization and lies in the intrinsic finite certification regime (Corollary~\ref{cor:intrinsic-cost-certification}).
\end{itemize}
\end{itemize}

\noindent\textbf{Step 1 (Bounded field).}
Form $\Xi_S$ by \eqref{eq:Cayley} and pull back $\theta(z)=\Xi_S(\psi^{-1}(z))$.

\noindent\textbf{Step 2 (Finite Schur certification).}
Choose a certification regime:
\begin{itemize}
\item \emph{(State-space regime)} when a finite-dimensional realization of $\theta$ is available, certify Schur contractivity by the bounded-real LMI (Theorem~\ref{thm:bounded-real}).
\item \emph{(Coefficient regime)} compute Taylor data for $\theta$ and certify a finite Pick gap plus tail bound (Proposition~\ref{prop:finite-gap-tail}).
\item \emph{(Point-sample regime)} sample $\theta$ at finitely many points and certify positivity together with the derived finite-complexity control (e.g.\ a rational/finite-state realization as in Theorem~\ref{thm:finite-state-rational}) that makes finite interpolation global.
\end{itemize}
Any successful certificate implies that $\theta$ is Schur on $\D$ (Theorem~\ref{thm:finite-certificate-global}).

\noindent\textbf{Step 3 (Pinch).}
If the Schur certificate succeeds on $\D$ (hence on $\Omega$), conclude that $\mathcal{J}_S$ has no poles in $\Omega$ (Corollary~\ref{cor:no-poles}).

\noindent\textbf{Step 4 (Decision).}
If the candidate mechanism requires such a pole in $\Omega$, return \texttt{IMPOSSIBLE\_STATE}. Otherwise return \texttt{INCONCLUSIVE}.
In the finite-dimensional rational regime, this can be upgraded to a full decision procedure by an existence-side root test (Algorithm~\ref{alg:rational-decision}).
\end{algorithm}

\subsection*{Implementation checklist (what an application must prove)}
To instantiate RSA in a new domain, the following items are the usual proof obligations:
\begin{itemize}
\item \textbf{(S1) RS defect/obstruction encoding:} provide a defect functional $\Delta_S$ with semantics $S\Leftrightarrow \Delta_S=0$ (Definition~\ref{def:defect-functional}) and an analytic representative $G_S$ with the same zeros (Section~\ref{subsec:rs-defect}); then $\mathcal J_S:=1/G_S$ is the canonical sensor (Definition~\ref{def:sensor-reciprocal}) and $S\Rightarrow$ blow-up is a reduction lemma (Lemma~\ref{lem:sensor-reduction}).
\item \textbf{(S2) Analytic class:} the pulled-back Cayley field belongs to a class where global control is meaningful (e.g.\ holomorphic on the chart domain).
\item \textbf{(S3) Finite-to-global bridge:} a derived finite-complexity restriction (finite-state/rational realization) sufficient to globalize a finite certificate (Section~3, Theorem~\ref{thm:finite-state-rational}); in practice one may discharge this via an explicit realization, or by a certified tail bound when working in coefficient form.
\item \textbf{(S4) Nontriviality:} exclude the degenerate case $\Xi\equiv 1$ on the audited region.
\item \textbf{(S5) Audit artifacts:} record the finite certificate (matrix gap, tail bound) in a reproducible form.
\end{itemize}

\section{Correctness theorem (what RSA actually proves)}

\begin{theorem}[Schur pinch $\Rightarrow$ impossibility in the audited region]\label{thm:correctness}
Assume:
\begin{enumerate}
\item (Correct sensor reduction) the candidate $S$ is encoded by an RS defect/obstruction pair on $\Omega$ (Section~\ref{subsec:rs-defect}) with a holomorphic representative $G_S$ having a simple zero at any candidate point, and $\mathcal J_S:=1/G_S$;
\item (Schur bound) the Cayley field $\Xi_S$ is Schur on $\Omega$ (i.e.\ $|\Xi_S|\le 1$ on $\Omega$);
\item (Nontriviality) $\Xi_S\not\equiv 1$ on $\Omega$.
\end{enumerate}
Then the candidate $S$ does not hold at any point of $\Omega$.
\end{theorem}

\begin{proof}
Suppose for contradiction that $S$ holds at some point $z_\star\in\Omega$.
By hypothesis (Schur bound) and (Nontriviality), Corollary~\ref{cor:no-poles} applies and implies that the Cayley inverse
\[
2\mathcal J_S=\frac{1+\Xi_S}{1-\Xi_S}
\]
extends holomorphically across $\Omega$; in particular, $\mathcal J_S$ has \emph{no poles} in $\Omega$.

On the other hand, by hypothesis (Correct sensor reduction), $S$ at $z_\star$ forces $G_S(z_\star)=0$ and $\mathcal J_S:=1/G_S$.
Since $G_S$ is holomorphic near $z_\star$ with a simple zero, Lemma~\ref{lem:sensor-reduction} implies that $\mathcal J_S=1/G_S$ has a pole at $z_\star$.
This contradicts the pole-freeness of $\mathcal J_S$ on $\Omega$ established above.
\end{proof}

\begin{corollary}[RSA soundness in certificate form]\label{cor:rsa-soundness}
In the setting of Algorithm~\ref{alg:RSA}, if RSA succeeds in Step~2 in certifying a Schur bound for $\Xi_S$ on $\Omega$ and $\Xi_S\not\equiv 1$, then RSA's output \texttt{IMPOSSIBLE\_STATE} is correct: the candidate does not occur in $\Omega$.
\end{corollary}
\begin{proof}
Under the stated hypotheses, Theorem~\ref{thm:correctness} applies directly.
\end{proof}

\section{Flagship Instantiations: Arithmetic and Geometry}

We demonstrate the universality of the RSA template by summarizing two implementations: one in analytic number theory (Riemann Hypothesis) and one in complex geometry (Hodge Conjecture). These case studies show that the abstract ``sensor'' and ``finite check'' concepts map to concrete, standard mathematical objects.

\subsection{Case I (Arithmetic): The Riemann Zeta Zero-Free Region}
\begin{remark}[Admissibility tier for Case I]
This instantiation is presented as a \emph{recognition-admissible impossibility certificate} (Definition~\ref{def:urc-imp}): a correct sensor reduction and a finite Schur certificate yield a terminating \texttt{IMPOSSIBLE\_STATE} conclusion on the audited far-field region.
It is not used here as a full decision procedure in the sense of Theorem~\ref{thm:decision-rational}; the rational-regime decision mechanism applies when the pulled-back obstruction is a bounded-degree rational function.
\end{remark}
\textbf{Target:} Rule out zeros of $\zeta(s)$ in the far-field half-plane $\Re s \ge 0.6$.

\begin{itemize}
\item \textbf{Monster:} A zero $\rho$ of $\zeta(s)$ with $\Re \rho \ge 0.6$.
\item \textbf{Sensor:} An arithmetic ratio $\mathcal{J}(s)$ whose poles in $\Re s>1/2$ encode zeros of $\zeta$ (e.g.\ $\mathcal{J}(s)=\frac{\det_2(I-A(s))}{\zeta(s)}\cdot\frac{s}{s-1}$ in a raw gauge; see \cite{WashburnRiemannDec31}).
\item \textbf{Mechanism:} If $\zeta(\rho)=0$, the denominator vanishes, forcing $\mathcal{J}(\rho)$ to have a pole.
\item \textbf{Bounded Field:} $\Theta(s) = \frac{2\mathcal{J}(s)-1}{2\mathcal{J}(s)+1}$. Pole $\Rightarrow \Theta(\rho)\to 1$.
\item \textbf{Finite Check (Pick gap + tail).} In the audited far-field implementation \cite{WashburnRiemannDec31}, a $16\times 16$ coefficient Pick minor is certified SPD at $\sigma_0=0.7$ with spectral gap $\delta_{\rm cert}=0.6273368612$, and the coefficient tail is certified as $\sum_{n\ge 16}(n+1)|a_n(0.7)|^2\le 0.0127$, hence $\varepsilon_{16}\le 0.113$. With a perturbation constant $C\le 2$, one checks $C\varepsilon_{16}\le 0.226<0.627=\delta_{\rm cert}$, yielding a global Schur bound on $\{\Re s>0.7\}$.
\item \textbf{Hybrid cover to $\Re s\ge 0.6$.} A certified rectangle bound on $[0.6,0.7]\times[0,20]$ plus an explicit far-$|t|$ bound extends the Schur property to $\{\Re s>0.6\}$ \cite{WashburnRiemannDec31}.
\item \textbf{Result:} The Schur/Herglotz pinch removes poles of $\mathcal{J}$ in the audited region, hence excludes zeros of $\zeta$ there: \texttt{IMPOSSIBLE\_STATE} for a far-field counterexample.
\end{itemize}
\noindent (See \cite{WashburnRiemannDec31} for a fully auditable implementation with certified artifacts.)

\subsection{Case II (Geometry): The Hodge Conjecture}
\begin{remark}[Role of Case II]
Case~II is explicitly an \emph{existence-side} architecture sketch intended to complement RSA in geometric settings (Definition~\ref{def:urc-dec}).
It is labeled as an architecture sketch because the full domain adapter theorems (that identify the audited Cayley field with a cost-contracting realizable class, and that provide the needed defect/obstruction encoding with analytic admissibility) are substantial geometric proofs and are recorded in companion manuscripts.
\end{remark}
\textbf{Target (architecture sketch with quantitative modules):} Record how a Hodge-type existence question can be organized into finite-resolution modules with explicit tail controls, suitable as an \emph{existence-side} complement to RSA (Definition~\ref{def:urc-dec}).
\textbf{Reframed as Impossibility:} Rule out a persistent ``non-algebraic gap'' (a strictly positive lower bound on a calibration/cohomological defect) by constructing a refinement sequence whose defect is forced to vanish.

\begin{itemize}
\item \textbf{Monster:} A rational Hodge class $\gamma$ that cannot be realized by an algebraic cycle.
\item \textbf{Finite-resolution direction labels (dictionary module).} A stable dictionary fit produces per-cell weights and robust discrete labels (winner-take-all away from ties) for strongly positive $(p,p)$ data; see \cite{WashburnStableDirectionDictionaries}.
\item \textbf{Holomorphic manufacturing (analytic module).} On a projective K\"ahler manifold, large tensor powers admit holomorphic complete intersections whose local sheets match prescribed tangent templates with uniform $C^1$ control on the Bergman scale $\mhol^{-1/2}$; see \cite{WashburnBergmanScaleHolomorphic}.
\item \textbf{Deterministic interface geometry (corner-exit module).} Corner-exit slivers force deterministic face incidence under small $C^1$ perturbations and yield per-face boundary mass control $\simeq v^{(k-1)/k}$; see \cite[Prop.~(G1)--(G2)]{WashburnCornerExitSlivers}.
\item \textbf{Combinatorial coherence (prefix module).} A prefix-template selection rule confines mismatch across a face to a terminal tail, and under slow variation the unmatched tail has $O(h)$ relative size; see \cite{WashburnPrefixTemplate}.
\item \textbf{Tail control via flat norm (gluing module).} A global weighted estimate bounds the residual boundary in flat norm,
\[
\F(\partial T^{\mathrm{raw}})\ \lesssim\ \varrho\,h^2\sum m_{Q,a}^{(k-1)/k},
\]
and implies a vanishing-mass correction when $\F(\partial T^{\mathrm{raw}})\to 0$; see \cite{WashburnWeightedFlat}.
\item \textbf{Exact homology via period locking (quantization module).} Fixed-dimension discrepancy rounding controls all periods to $<1/4$, and a tiny boundary correction then forces exact integrality by lattice locking; see \cite{WashburnCohomologyQuantization}.
\item \textbf{Validation on toy models.} As a sanity check, this calibrated-current machinery recovers the standard $(p,p)$-cohomology generators on projective space $\mathbb{CP}^n$ by strictly minimizing the comass norm, reproducing known results via the variational pathway.
\end{itemize}
\noindent Together these modules supply a plausible \emph{existence-side certificate} architecture for Hodge-type problems: finite-resolution recognition produces coherent local pieces, a quantified tail bound forces vanishing boundary defect under refinement, and integrality locks the global class.

\section{The Universal Recognition Class}\label{sec:urc}

These examples suggest a generalization. RSA is not merely a heuristic: on a well-defined admissible class it gives a reproducible \emph{impossibility certificate}. Upgrading to a full \emph{decision} procedure is automatic in the finite-dimensional rational regime (Theorem~\ref{thm:decision-rational}); outside that regime it requires additional completeness hypotheses (i.e.\ a complementary existence witness mechanism in the sense of Definition~\ref{def:urc-dec}).

\begin{definition}[Recognition-Admissible (impossibility form)]\label{def:urc-imp}
A mathematical existence problem $P$ is \emph{recognition-admissible for impossibility} if it can be transformed into data $(\Omega, \Delta_P, G_P, \mathcal{C}_{\text{fin}})$ where:
\begin{enumerate}
\item $\Omega$ is a domain with a valid complex-analytic or geometric structure.
\item $\Delta_P:\Omega\to\R_{\ge 0}$ is a defect functional with semantics $P$ at $z \Leftrightarrow \Delta_P(z)=0$ (Definition~\ref{def:defect-functional}).
\item $G_P$ is a holomorphic representative on $\Omega$ with the same zero set as $\Delta_P$ (analytic admissibility; Section~\ref{subsec:rs-defect}); define the canonical sensor $\mathcal J_P:=1/G_P$ (Definition~\ref{def:sensor-reciprocal}).
Assume that at any candidate point the zero of $G_P$ is simple, so that $P\Rightarrow$ pole of $\mathcal J_P$ is automatic (Lemma~\ref{lem:sensor-reduction}).
\item $\mathcal{C}_{\text{fin}}$ is a finite-complexity control regime (finite-state/rational/tail-bounded) sufficient to convert a finite certificate into global Schur control on $\Omega$.
\end{enumerate}
\end{definition}

\begin{definition}[Recognition-Admissible (decidable form)]\label{def:urc-dec}
A problem $P$ is \emph{recognition-admissible for decision} if, in addition to Definition~\ref{def:urc-imp}, there exists a complementary \emph{existence-side} mechanism that yields a terminating certificate of $P$ when $P$ holds (e.g.\ a constructive witness, or an exact analytic reconstruction of $G_P$ from finite data).
\end{definition}

\begin{theorem}[A proved URC subclass: realizable audited Cayley fields]\label{thm:urc-realizable}
Let $P$ be a mathematical existence problem on a domain $\Omega$.
Assume $P$ admits the Step~0 structures of Algorithm~\ref{alg:RSA}, including:
(i) an RS defect/obstruction encoding with analytic admissibility and simple zeros, and
(ii) the \emph{realizable audited Cayley field model} (finite-window 8-tick realizability and cost-driven realizability) for the pulled-back Cayley field $\theta$.
Then $P$ is recognition-admissible for impossibility (Definition~\ref{def:urc-imp}).

If, in addition, after normalization $\psi:\Omega\to\D$ the pulled-back obstruction $\widetilde G_P(z)=G_P(\psi^{-1}(z))$ is a bounded-degree rational function over a computable number field (as in Theorem~\ref{thm:decision-rational}), then $P$ is recognition-admissible for decision (Definition~\ref{def:urc-dec}).
\end{theorem}
\begin{proof}
By Step~0, $P$ comes with a defect/obstruction encoding, a holomorphic representative $G_P$ with simple zeros at candidate points, and a normalization to the disk, so the analytic items of Definition~\ref{def:urc-imp} are satisfied.
The Step~0 realizability model supplies an explicit finite-complexity control regime $\mathcal C_{\mathrm{fin}}$:
finite-window realizability yields a finite-state/rational audited class for the Cayley field (Theorem~\ref{thm:realizability-finite-complexity}), and cost-driven realizability places $\theta$ in the intrinsic finite certification regime (Corollary~\ref{cor:intrinsic-cost-certification}).
Thus $P$ is recognition-admissible for impossibility.

If the additional rational/degree/computability hypothesis holds, Theorem~\ref{thm:decision-rational} furnishes a terminating existence-side mechanism (root witness vs.\ certified no-root), hence Definition~\ref{def:urc-dec} holds.
\end{proof}

\begin{theorem}[Decision in the finite-dimensional rational class]\label{thm:decision-rational}
Assume $P$ is recognition-admissible for impossibility (Definition~\ref{def:urc-imp}) and that, after a domain normalization $\psi:\Omega\to\D$, the pulled-back obstruction
\[
\widetilde G_P(z)\;:=\;G_P(\psi^{-1}(z))
\]
is a \emph{rational} function of $z$ with coefficients in a computable number field and with a known degree bound.
Then there is a terminating decision procedure that either:
\begin{itemize}
\item produces a witness point $z_\star\in\D$ with $\widetilde G_P(z_\star)=0$ (hence $P$ holds at $\psi^{-1}(z_\star)$), or
\item certifies that $\widetilde G_P$ has no zeros in $\D$ (hence $P$ does not hold anywhere in $\Omega$).
\end{itemize}
\end{theorem}
\begin{proof}
Write $\widetilde G_P(z)=p(z)/q(z)$ with coprime polynomials $p,q$ over the coefficient field; then $\widetilde G_P(z_\star)=0$ if and only if $p(z_\star)=0$.
Since $p$ has bounded degree and computable coefficients, there are standard terminating exact procedures to decide whether $p$ has a root in the open unit disk (e.g.\ Schur--Cohn/Jury-type root-location tests, or algebraic root isolation over number fields).
If such a root exists, the procedure returns one (as an algebraic number, hence an explicit witness); otherwise it certifies that no such root exists.
\end{proof}

\begin{remark}[Why the rational decision hypothesis is natural in the full-derivation route]
In the full-derivation route, finite local resolution plus recognition-respecting dynamics yields a finite-state/rational class for the audited Cayley field (Theorem~\ref{thm:finite-state-rational}).
Rationality is preserved under the Cayley transform and its inverse (Lemma~\ref{lem:cayley-rational}), so whenever the Cayley field is rational, the associated sensor $\mathcal J_P$ and obstruction $G_P=1/\mathcal J_P$ are rational as well (away from the degenerate case $\Xi\equiv 1$).
Thus Theorem~\ref{thm:decision-rational} captures the most literal ``finite description $\Rightarrow$ finite decision'' regime.
\end{remark}

\begin{corollary}[Rational regime implies decidable admissibility]\label{cor:rational-implies-urc-dec}
Under the hypotheses of Theorem~\ref{thm:decision-rational}, the problem $P$ is recognition-admissible for decision (Definition~\ref{def:urc-dec}).
\end{corollary}
\begin{proof}
Theorem~\ref{thm:decision-rational} provides a terminating existence-side mechanism (a witness root when $P$ holds, or a certified no-root conclusion when $P$ fails).
\end{proof}

\begin{algorithm}[Decision procedure in the finite-dimensional rational class]\label{alg:rational-decision}
\textbf{Input:} A problem $P$ in the admissibility class of Definition~\ref{def:urc-imp}, together with a normalization $\psi:\Omega\to\D$ and a rational representation (with degree bound) of the pulled-back obstruction $\widetilde G_P(z)=G_P(\psi^{-1}(z))$.

\textbf{Output:} \texttt{WITNESS} (with $z_\star$) or \texttt{IMPOSSIBLE\_STATE}.

\medskip
\noindent\textbf{Step 1 (Optional nonexistence certificate).}
Attempt a finite Schur certificate for the pulled-back Cayley field $\theta$ (the field certified in Step~2 of Algorithm~\ref{alg:RSA}).
If a Schur bound is certified on $\D$ and $\theta\not\equiv 1$, return \texttt{IMPOSSIBLE\_STATE} by Corollary~\ref{cor:rsa-soundness}.

\noindent\textbf{Step 2 (Exact root test).}
Write $\widetilde G_P(z)=p(z)/q(z)$ with coprime polynomials over the coefficient field.
Use an exact terminating root-location procedure to decide whether $p$ has a root in $\D$ (Theorem~\ref{thm:decision-rational}).

\noindent\textbf{Step 3 (Decision).}
If a root $z_\star\in\D$ exists, output \texttt{WITNESS} with $z_\star$ (so $P$ holds at $\psi^{-1}(z_\star)$).
Otherwise output \texttt{IMPOSSIBLE\_STATE}.
\end{algorithm}

\begin{proposition}[RSA is a semi-decision procedure for impossibility]\label{prop:rsa-semi-decision}
For any problem in the admissibility class of Definition~\ref{def:urc-imp}, RSA is a terminating certificate procedure for \texttt{IMPOSSIBLE\_STATE} whenever the Schur certificate succeeds on the audited region.
\end{proposition}
\begin{proof}
Fix $P$ in the admissibility class of Definition~\ref{def:urc-imp}.
By construction there is a holomorphic obstruction $G_P$ whose zero set encodes $P$, and the canonical sensor is $\mathcal J_P:=1/G_P$.
By the simple-zero hypothesis, if $P$ held at some $z_\star\in\Omega$ then $\mathcal J_P$ would have a pole at $z_\star$ (Lemma~\ref{lem:sensor-reduction}).

RSA forms the Cayley field and runs a finite Schur certification procedure in a finite-complexity regime.
When this Schur certificate succeeds on the audited region, the Schur pinch (Corollary~\ref{cor:no-poles}) implies that $\mathcal J_P$ has no poles in that region.
Therefore $P$ cannot hold there, and RSA returns \texttt{IMPOSSIBLE\_STATE}.
\end{proof}

\begin{remark}[Decision requires an existence-side certificate]
RSA by itself is not an existence prover; therefore Definition~\ref{def:urc-dec} makes explicit that a full decision procedure requires an additional mechanism beyond the Schur/pole-exclusion audit.
\end{remark}
\begin{remark}[RSA + existence modules (how URC\_dec is intended to be used)]
In practice, a URC\_dec claim is typically assembled from two complementary pieces:
(i) an RSA-style \emph{impossibility} certificate that rules out candidates by excluding sensor blow-up, and
(ii) a domain-specific \emph{existence} architecture that constructs (or certifies) admissible witnesses when the object exists.
Case~II outlines one such existence architecture (dictionary $\to$ manufacturing $\to$ deterministic interfaces $\to$ flat-norm tail control $\to$ period locking), intended as the complement to RSA in geometric settings.
\end{remark}

\section{Gödel/undecidability objections and scope}\label{sec:godel}
Claims of ``universal algorithms'' in mathematics are immediately met with Gödelian and Turing-style objections.
This manuscript avoids contradiction by an explicit scope restriction: RSA is an \emph{audit} for stability of \emph{configurations} in a cost-theoretic ontology.
It does not claim a decision procedure for arbitrary formal sentences, and it does not claim to settle Gödel-style questions about provability in arithmetic.

\subsection*{Gödel's setup vs.\ RSA's setup}
Gödel's incompleteness theorems concern effectively axiomatized formal proof systems capable of encoding arithmetic and an internal provability predicate; truth is evaluated externally (e.g.\ in the standard model) \cite{Godel1931,Tarski1936}.
Recognition Science instead treats ``truth'' as stabilization/cost-minimization and ``existence'' as convergence to zero defect; this is a different target \cite{WashburnGodelDissolution,WashburnRecognitionGeometry2025}.

\subsection*{Self-reference and ``non-configurations''}
A key mechanism behind incompleteness is self-reference.
In the RS ontology, the direct analog is a \emph{self-referential stabilization query} (a configuration asserting its own non-stabilization). The Gödel dissolution manuscript argues that such objects have no fixed point under coercive dynamics and are therefore \emph{outside the ontology} (``non-configurations'') \cite{WashburnGodelDissolution}.
For the purposes of this manuscript, the key point is the scope consequence: self-referential ``this does not stabilize'' objects are excluded from the recognition-admissible input class, so Gödel-style diagonal statements do not arise as configurations that RSA is meant to audit.

\subsection*{Example: RSA failure on the Halting Problem}
To demonstrate that RSA respects undecidability (and to make the boundary concrete), consider applying it to the Halting Problem.
If one attempts to construct a sensor $\mathcal{J}_H$ that blows up exactly when a Turing machine $M$ halts, the attempt fails at Step~0:
either the sensor is not computable (evaluation requires an unbounded simulation), or the Step~0 realizability model needed to globalize a finite Schur certificate cannot be discharged without already deciding whether $M$ halts.
In either case the audit cannot produce a terminating certificate, and it correctly returns \texttt{INCONCLUSIVE} (or the instantiation fails to meet the admissibility requirements), reflecting that RSA cannot decide undecidable problems.

\subsection*{What this does and does not buy}
This resolves the \emph{logical} objection ``does RSA contradict incompleteness?'' by clarifying that RSA does not target unrestricted arithmetic truth.
It does \emph{not} by itself prove that all mathematical problems admit the Step~0 realizability model (and hence lie in the admissible class of Section~\ref{sec:urc}); that remains a substantive coverage conjecture requiring domain-by-domain reductions and finite-complexity control.

\subsection*{Conjecture: Coverage of the realizable Cayley model}
We conjecture that \emph{all well-posed existence problems in physical geometry and arithmetic} can be reduced into the Step~0 framework of Algorithm~\ref{alg:RSA} in its \emph{full-derivation} form:
namely, each such problem admits an analytically admissible RS defect/obstruction encoding and a realizable audited Cayley field model (finite-window 8-tick realizability plus cost-driven contractive dynamics) so that the proved mechanism of Theorem~\ref{thm:urc-realizable} applies.
Concretely:
\begin{itemize}
\item \textbf{Number theory:} zeros of $L$-functions are encoded as obstruction zeros (hence sensor poles) and the audited Cayley field arises from a finite-window, certified recognition ratio (as in Case~I).
\item \textbf{Geometry:} existence of special cycles/metrics is encoded by coercive defect functionals with holomorphic representatives, and the audited Cayley field is governed by cost-driven coercive dynamics (as suggested by the Case~II architecture).
\item \textbf{Physics:} mass gaps and regularity questions are encoded by finite-cost recognition constraints whose induced tick dynamics is contractive under the canonical cost.
\end{itemize}
Under this conjecture, the open content is not the audit mechanism (proved in this manuscript) but the \emph{coverage}: showing that the domains of interest admit the Step~0 realizability model.

\begin{remark}[Relationship to ``solve all math'']
Any claim approaching ``all mathematical problems are solvable by a set algorithm'' must be interpreted as a claim about the scope of the admissible class (Section~\ref{sec:urc}) and/or about an RS-internal semantics of truth/existence. Without such a scope change, classical undecidability results apply.
\end{remark}

\section{Limitations and open problems}\label{sec:limits}
\begin{itemize}
\item \textbf{Existence-side certificates outside the rational class.} RSA is designed for impossibility. In the finite-dimensional rational regime, an existence-side decision mechanism is available by exact root testing (Theorem~\ref{thm:decision-rational}). Beyond that regime (general holomorphic obstructions without a degree bound), existence-side certificates are genuinely domain-dependent.
\item \textbf{Finite-complexity and stability justification.} RSA needs a finite-to-global bridge. Section~\ref{sec:rg4} shows one explicit route: an 8-tick reachability model with finite local branching yields RG4 and a finite-state/rational class (Theorem~\ref{thm:finite-state-rational}). On the stability side, Recognition Cost supplies a canonical strict contraction mechanism in log-coordinates (Section~\ref{subsec:prox-contract}, Lemma~\ref{lem:prox-contraction}), and strict contraction implies stable linearization (Lemma~\ref{lem:contraction-linearization}); in stable realization regimes this also yields an explicit Taylor tail bound (Lemma~\ref{lem:tail-from-realization}). What remains open in broad generality is to connect the \emph{domain-specific} audited Cayley field to a realization governed by such cost-driven dynamics, so that the stability hypotheses are discharged intrinsically rather than by external modeling.
\item \textbf{URC coverage conjecture.} The claim that a very broad class of important problems admits the Step~0 realizability model (and hence falls in URC) is conjectural and should be tested domain-by-domain.
\end{itemize}

\section{Conclusion}

RSA packages a standard analytic mechanism (Cayley transform + Schur/Herglotz pinch) together with an explicit finite certification step that globalizes control on the unit disk.
Two complementary certification regimes are recorded: a state-space bounded-real (KYP/LMI) certificate (exact in the finite-dimensional realization class) and a Pick-gap-plus-tail certificate from Taylor data (with tail bounds derivable from stable realization data; Lemma~\ref{lem:tail-from-realization}).

Its universality is the universality of a pipeline:
\[
\text{candidate}\Rightarrow \text{obstruction zero}\Rightarrow \text{sensor pole}\Rightarrow \Xi\to 1.
\]
When a strict Schur certificate succeeds (and the degenerate case $\Xi\equiv 1$ is excluded), the pinch removes poles of the sensor on the audited region; Theorem~\ref{thm:correctness} then implies that the candidate does not occur there.
In the finite-dimensional rational regime, an existence-side mechanism is also available by exact root-location tests (Theorem~\ref{thm:decision-rational}), yielding a full decision procedure (Algorithm~\ref{alg:rational-decision}).

\begin{thebibliography}{99}

\bibitem{Godel1931}
K.~Gödel, ``Über formal unentscheidbare Sätze der Principia Mathematica und verwandter Systeme I,''
\emph{Monatshefte für Mathematik und Physik}, 38 (1931), 173--198.

\bibitem{Tarski1936}
A.~Tarski, ``Der Wahrheitsbegriff in den formalisierten Sprachen,''
\emph{Studia Philosophica}, 1 (1936), 261--405.

\bibitem{Aczel1966}
J.~Aczél, \emph{Lectures on Functional Equations and Their Applications},
Academic Press, 1966.

\bibitem{Kuczma2009}
M.~Kuczma, \emph{An Introduction to the Theory of Functional Equations and Inequalities}, 2nd ed.,
Birkhäuser, 2009.

\bibitem{RosenblumRovnyak}
M.~Rosenblum and J.~Rovnyak, \emph{Hardy Classes and Operator Theory},
Oxford University Press, 1985.

\bibitem{Donoghue}
W.~F.~Donoghue, \emph{Monotone Matrix Functions and Analytic Continuation},
Springer, 1974.

\bibitem{ZhouDoyleGlover1996}
K.~Zhou, J.~C.~Doyle, and K.~Glover, \emph{Robust and Optimal Control},
Prentice Hall, 1996.

\bibitem{RudinRCA}
W.~Rudin, \emph{Real and Complex Analysis}, 3rd ed., McGraw--Hill, 1987.

\bibitem{Washburn2025Cost}
J.~Washburn, ``Cost Is Not a Dial: A Self-Contained Uniqueness Theorem for the Canonical Reciprocal Cost on $\mathbb{R}_{>0}$,''
manuscript (2025). \;(\path{papers/tex/canonical_cost_uniqueness.tex})

\bibitem{WashburnRecognitionGeometry2025}
J.~Washburn, ``Recognition Geometry: A Complete Mathematical Framework,''
manuscript (Dec.\ 2025). \;(\path{papers/tex/recognition-geometry-dec-23.tex})

\bibitem{WashburnGodelDissolution}
J.~Washburn, ``Gödel's Theorem Does Not Obstruct Physical Closure: A Cost-Theoretic Resolution via Recognition Science,''
manuscript (Dec.\ 2025). \;(\path{papers/tex/godel_dissolution.tex})

\bibitem{WashburnRiemannDec31}
J.~Washburn, ``Riemann (Dec 31): Far-field Schur certification via Pick gaps + certified artifacts,''
manuscript (2025/2026). \;(\path{papers/tex/Riemann-Dec-31.tex})

\bibitem{WashburnStableDirectionDictionaries}
J.~Washburn, ``Stable Direction Dictionaries for Strongly Positive $(p,p)$-Forms via Regularized Simplex Fits,''
manuscript. \;(\path{papers/tex/paper-1-stable-direction-dictionaries.tex})

\bibitem{WashburnBergmanScaleHolomorphic}
J.~Washburn, ``Bergman-Scale Holomorphic Manufacturing of Prescribed Tangent Templates in Projective K\"ahler Manifolds,''
manuscript.
\newblock (\path{Paper-2-Bergman-Scale-Holomorphic.tex})

\bibitem{WashburnCornerExitSlivers}
J.~Washburn, ``Corner-Exit Slivers for Calibrated Sheet Constructions: Deterministic Face Incidence and Uniform Boundary Control,''
manuscript.
\newblock (\path{Paper-3-Corner-Exit-Slivers.tex})

\bibitem{WashburnPrefixTemplate}
J.~Washburn, ``Prefix-Template Bookkeeping: Deterministic Coherence up to $O(h)$ Face Edits,''
manuscript. \;(\path{papers/tex/Paper-4-Prefix-Template.tex})

\bibitem{WashburnWeightedFlat}
J.~Washburn, ``Weighted Flat-Norm Gluing for Sliver Microstructures and Vanishing-Mass Boundary Correction,''
manuscript. \;(\path{papers/tex/Paper-5-Weighted-Flat.tex})

\bibitem{WashburnCohomologyQuantization}
J.~Washburn, ``Cohomology Quantization for Microstructured Calibrated Currents via Discrepancy Rounding,''
manuscript. \;(\path{papers/tex/Paper-6-Cohomology.tex})

\end{thebibliography}

\end{document}

