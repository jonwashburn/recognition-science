\documentclass[11pt,a4paper]{article}
\usepackage[margin=1in]{geometry}
\usepackage[T1]{fontenc}
\usepackage{lmodern}
\usepackage{microtype}
\usepackage{amsmath,amssymb,amsthm}
\usepackage{mathtools}
\usepackage{booktabs}
\usepackage{array}
\usepackage{enumitem}
\usepackage{xcolor}
\usepackage[hidelinks]{hyperref}

\newtheorem{theorem}{Theorem}[section]
\newtheorem{proposition}[theorem]{Proposition}
\newtheorem{lemma}[theorem]{Lemma}
\newtheorem{corollary}[theorem]{Corollary}
\newtheorem{definition}[theorem]{Definition}
\newtheorem{remark}[theorem]{Remark}
\newtheorem{example}[theorem]{Example}
\newtheorem{falsifier}[theorem]{Falsification Criterion}

\newcommand{\phig}{\varphi}
\newcommand{\Jcost}{J}
\newcommand{\Rhat}{\hat{R}}

\title{\textbf{Simplicial Ledger Topology:\\
Coordinate-Free Cost Geometry\\
on 3-Complexes}\\[0.5em]
\large A Mathematical Foundation for Recognition Science}
\author{Jonathan Washburn\\
\small Recognition Science Research Institute, Austin, Texas\\
\small \texttt{washburn.jonathan@gmail.com}}
\date{February 2026}

\begin{document}
\maketitle

\begin{abstract}
The Recognition Science ledger has been presented on the cubic lattice $\mathbb{Z}^3$. We show the construction is \emph{coordinate-free}: any simplicial 3-complex admits a J-cost sheaf whose stationarity conditions are independent of the triangulation. Specifically, we define a cost-weighted simplicial complex $(\mathcal{K}, \Jcost)$ where each 3-simplex carries a recognition potential $\psi$, and prove:
\begin{enumerate}[nosep]
\item The local cost $\Jcost_{\mathrm{loc}}(\sigma, \psi) = \Jcost(\psi) \cdot \mathrm{vol}(\sigma)$ defines a non-negative measure on the complex.
\item The global cost $\mathcal{C}[\psi] = \sum_\sigma \Jcost_{\mathrm{loc}}(\sigma, \psi(\sigma))$ has well-defined variational equations.
\item The Euler--Lagrange equations of $\mathcal{C}$ on the simplicial complex are invariant under stellar subdivision (refinement independence).
\item In the continuum limit (mesh $\to 0$), the stationarity conditions converge to a second-order PDE whose leading term recovers the Jacobi determinant formula $\delta\sqrt{|g|}/\delta g^{\mu\nu} = -\frac{1}{2}\sqrt{|g|}\,g_{\mu\nu}$, connecting to the Einstein field equations.
\end{enumerate}
The construction provides a rigorous mathematical foundation for the claim that ``gravity is J-cost stationarity on a 3-manifold.''  Core definitions are formalized in Lean~4 (\texttt{IndisputableMonolith.Foundation.SimplicialLedger}).

\medskip\noindent\textbf{Keywords:} simplicial complex, J-cost, sheaf, Euler--Lagrange, refinement invariance, coordinate-free, recognition geometry.
\end{abstract}

\tableofcontents
\newpage

%======================================================================
\section{Introduction}\label{sec:intro}
%======================================================================

Recognition Science derives dynamics from a unique cost functional $\Jcost(x) = \frac{1}{2}(x + x^{-1}) - 1$ on $\mathbb{R}_{>0}$~\cite{WashburnCost2026}.  The physical substrate is a discrete ledger evolving in 8-tick steps~\cite{PardoGuerra2026}.  Previous presentations use the cubic lattice $\mathbb{Z}^3$, which raises a natural question: \emph{is the physics coordinate-dependent?}

This paper proves it is not.  We construct the J-cost variational problem on an arbitrary simplicial 3-complex and show that the stationarity equations depend only on the topology and the cost functional, not on the particular triangulation.  This is the discrete analogue of diffeomorphism invariance in general relativity.

The key insight is that $\Jcost$ is \emph{intrinsic}: it depends on the configuration ratio $\psi$ (a dimensionless positive real), not on coordinates.  The volume element $\mathrm{vol}(\sigma)$ of each 3-simplex provides the natural measure, and the product $\Jcost(\psi) \cdot \mathrm{vol}(\sigma)$ is a coordinate-free local cost.

\paragraph{Foundational dependencies.}
\begin{enumerate}[nosep]
\item J-cost uniqueness (T5)~\cite{WashburnCost2026}.
\item Ledger dynamics~\cite{PardoGuerra2026}.
\item Dimensional rigidity $D = 3$~\cite{WashburnD3_2026}.
\end{enumerate}

%======================================================================
\section{Simplicial 3-Complexes}\label{sec:complex}
%======================================================================

\begin{definition}[3-Simplex]\label{def:simplex}
A \emph{3-simplex} $\sigma$ is an ordered quadruple of vertices $(v_0, v_1, v_2, v_3)$ in $\mathbb{R}^3$ with strictly positive oriented volume:
\[
  \mathrm{vol}(\sigma) = \frac{1}{6}\left|\det\begin{pmatrix} v_1 - v_0 \\ v_2 - v_0 \\ v_3 - v_0 \end{pmatrix}\right| > 0.
\]
\end{definition}

\begin{definition}[Simplicial 3-complex]\label{def:complex}
A \emph{simplicial 3-complex} $\mathcal{K}$ is a finite collection of 3-simplices such that:
\begin{enumerate}[nosep]
\item Every face of a simplex in $\mathcal{K}$ is in $\mathcal{K}$ (closure).
\item The intersection of any two simplices is either empty or a common face (compatibility).
\item $\mathcal{K}$ is non-empty.
\end{enumerate}
We write $|\mathcal{K}|$ for the underlying topological space (union of all simplices).
\end{definition}

\begin{definition}[Stellar subdivision]\label{def:stellar}
A \emph{stellar subdivision} of $\mathcal{K}$ at a simplex $\sigma$ introduces a new vertex $v^*$ at the barycenter of $\sigma$ and replaces $\sigma$ with the cone from $v^*$ over $\partial\sigma$.  The result $\mathcal{K}'$ satisfies $|\mathcal{K}'| = |\mathcal{K}|$ (same underlying space).
\end{definition}

%======================================================================
\section{The J-Cost Sheaf}\label{sec:sheaf}
%======================================================================

\begin{definition}[Recognition potential]\label{def:potential}
A \emph{recognition potential} on $\mathcal{K}$ is a function
$\psi : \mathcal{K}_3 \to \mathbb{R}_{>0}$
assigning a positive real to each 3-simplex.
\end{definition}

\begin{definition}[J-cost sheaf]\label{def:sheaf}
The \emph{J-cost sheaf} $(\mathcal{K}, \Jcost, \psi)$ consists of:
\begin{itemize}[nosep]
\item The simplicial complex $\mathcal{K}$.
\item The cost functional $\Jcost(x) = \frac{1}{2}(x + x^{-1}) - 1$.
\item A recognition potential $\psi$.
\end{itemize}
\end{definition}

\begin{definition}[Local cost]\label{def:local}
The \emph{local cost} on a 3-simplex $\sigma$ is
\begin{equation}\label{eq:local}
  \Jcost_{\mathrm{loc}}(\sigma, \psi) = \Jcost(\psi(\sigma)) \cdot \mathrm{vol}(\sigma).
\end{equation}
\end{definition}

\begin{lemma}[Local cost non-negativity]\label{lem:local_nonneg}
$\Jcost_{\mathrm{loc}}(\sigma, \psi) \geq 0$ for all $\sigma$ and $\psi$.
\end{lemma}

\begin{proof}
$\Jcost(x) \geq 0$ for all $x > 0$ (by AM-GM: $\frac{1}{2}(x + x^{-1}) \geq 1$), and $\mathrm{vol}(\sigma) > 0$ by definition. \qed
\end{proof}

\begin{lemma}[Local cost zero iff identity]\label{lem:local_zero}
$\Jcost_{\mathrm{loc}}(\sigma, \psi) = 0$ if and only if $\psi(\sigma) = 1$.
\end{lemma}

\begin{proof}
$\mathrm{vol}(\sigma) > 0$, so $\Jcost_{\mathrm{loc}} = 0 \Leftrightarrow \Jcost(\psi(\sigma)) = 0 \Leftrightarrow \psi(\sigma) = 1$. \qed
\end{proof}

%======================================================================
\section{Global Cost and Variational Equations}\label{sec:global}
%======================================================================

\begin{definition}[Global cost]\label{def:global}
The \emph{global cost} of a potential $\psi$ on $\mathcal{K}$ is
\begin{equation}\label{eq:global}
  \mathcal{C}[\psi] = \sum_{\sigma \in \mathcal{K}_3} \Jcost_{\mathrm{loc}}(\sigma, \psi)
  = \sum_{\sigma \in \mathcal{K}_3} \Jcost(\psi(\sigma)) \cdot \mathrm{vol}(\sigma).
\end{equation}
\end{definition}

\begin{theorem}[Euler--Lagrange on the complex]\label{thm:EL}
The stationarity condition $\delta \mathcal{C} / \delta \psi(\sigma) = 0$ for each 3-simplex $\sigma$ gives
\begin{equation}\label{eq:EL}
  \Jcost'(\psi(\sigma)) \cdot \mathrm{vol}(\sigma) = 0
  \quad\Longrightarrow\quad
  \Jcost'(\psi(\sigma)) = 0
  \quad\Longrightarrow\quad
  \psi(\sigma) = 1,
\end{equation}
since $\Jcost'(x) = \frac{1}{2}(1 - x^{-2}) = 0$ iff $x = 1$ (for $x > 0$), and $\mathrm{vol}(\sigma) > 0$.
\end{theorem}

\begin{proof}
$\mathcal{C}[\psi] = \sum_\sigma \Jcost(\psi(\sigma)) \cdot \mathrm{vol}(\sigma)$ is a sum of independent terms in each $\psi(\sigma)$.  Differentiation with respect to $\psi(\sigma)$ yields $\Jcost'(\psi(\sigma)) \cdot \mathrm{vol}(\sigma) = 0$.  Since $\mathrm{vol}(\sigma) > 0$, this reduces to $\Jcost'(\psi(\sigma)) = 0$.  The only critical point of $\Jcost$ on $\mathbb{R}_{>0}$ is $x = 1$ (the unique minimum). \qed
\end{proof}

\begin{remark}
With boundary conditions or coupling between adjacent simplices (shared-face matching), the stationarity equations become non-trivial.  We develop this in Section~\ref{sec:coupling}.
\end{remark}

%======================================================================
\section{Inter-Simplex Coupling}\label{sec:coupling}
%======================================================================

In a physical ledger, adjacent simplices are not independent: their potentials must be consistent across shared faces.

\begin{definition}[Face-coupling cost]\label{def:face}
For adjacent simplices $\sigma_1, \sigma_2$ sharing a face $f$, the \emph{coupling cost} is
\begin{equation}\label{eq:coupling}
  \Jcost_{\mathrm{couple}}(\sigma_1, \sigma_2)
  = \Jcost\!\left(\frac{\psi(\sigma_1)}{\psi(\sigma_2)}\right) \cdot \mathrm{area}(f).
\end{equation}
\end{definition}

\begin{definition}[Full action]\label{def:full}
The \emph{full action} on $(\mathcal{K}, \psi)$ is
\begin{equation}\label{eq:full}
  \mathcal{S}[\psi]
  = \sum_{\sigma \in \mathcal{K}_3} \Jcost(\psi(\sigma)) \cdot \mathrm{vol}(\sigma)
  + \sum_{\substack{\sigma_1 \sim \sigma_2}} \Jcost\!\left(\frac{\psi(\sigma_1)}{\psi(\sigma_2)}\right) \cdot \mathrm{area}(f_{\sigma_1 \sigma_2}).
\end{equation}
\end{definition}

\begin{theorem}[Coupled Euler--Lagrange]\label{thm:coupled_EL}
The stationarity condition $\delta\mathcal{S}/\delta\psi(\sigma) = 0$ gives, for each 3-simplex $\sigma$:
\begin{equation}\label{eq:coupled_EL}
  \Jcost'(\psi(\sigma)) \cdot \mathrm{vol}(\sigma)
  + \sum_{\sigma' \sim \sigma}
    \frac{1}{\psi(\sigma)} \cdot
    \Jcost'\!\left(\frac{\psi(\sigma)}{\psi(\sigma')}\right) \cdot \mathrm{area}(f_{\sigma\sigma'})
  = 0.
\end{equation}
This is a system of coupled nonlinear equations whose solutions describe the equilibrium recognition potential.
\end{theorem}

\begin{proof}
The bulk term gives $\Jcost'(\psi(\sigma)) \cdot \mathrm{vol}(\sigma)$.  For each coupling term $\Jcost(\psi(\sigma)/\psi(\sigma'))$, differentiating with respect to $\psi(\sigma)$ using the chain rule:
\[
  \frac{\partial}{\partial \psi(\sigma)} \Jcost\!\left(\frac{\psi(\sigma)}{\psi(\sigma')}\right)
  = \Jcost'\!\left(\frac{\psi(\sigma)}{\psi(\sigma')}\right) \cdot \frac{1}{\psi(\sigma')}.
\]
Multiplying by $\mathrm{area}(f)$ and summing over neighbours $\sigma' \sim \sigma$ gives~\eqref{eq:coupled_EL}. \qed
\end{proof}

\begin{example}[Two adjacent tetrahedra]\label{ex:two}
Consider two tetrahedra $\sigma_1, \sigma_2$ sharing a triangular face
$f$ with $\mathrm{area}(f) = A$ and $\mathrm{vol}(\sigma_i) = V$.
The full action is
\[
  \mathcal{S}[\psi_1, \psi_2]
  = V\bigl[\Jcost(\psi_1) + \Jcost(\psi_2)\bigr]
  + A\,\Jcost(\psi_1/\psi_2).
\]
The Euler--Lagrange equations are:
\begin{align*}
  V\,\Jcost'(\psi_1) + \frac{A}{\psi_2}\,\Jcost'(\psi_1/\psi_2) &= 0, \\
  V\,\Jcost'(\psi_2) - \frac{A\psi_1}{\psi_2^2}\,\Jcost'(\psi_1/\psi_2) &= 0.
\end{align*}
At $\psi_1 = \psi_2 = 1$: $\Jcost'(1) = 0$ and
$\Jcost'(1) = 0$, so the uniform state $\psi \equiv 1$ is a critical
point.  Adding the two equations gives
$V[\Jcost'(\psi_1) + \Jcost'(\psi_2)] = 0$ (the coupling terms cancel
by reciprocity of $\Jcost$), confirming that the bulk equations decouple
at equilibrium.  Non-trivial solutions arise when boundary conditions
force $\psi_1 \ne \psi_2$.
\end{example}

%======================================================================
\section{Refinement Invariance}\label{sec:refinement}
%======================================================================

\begin{theorem}[Refinement independence]\label{thm:refinement}
Let $\mathcal{K}'$ be a stellar subdivision of $\mathcal{K}$.  If $\psi^*$ is a critical point of $\mathcal{S}$ on $\mathcal{K}$, and $\psi'^*$ is the canonical extension of $\psi^*$ to $\mathcal{K}'$ (constant on each daughter simplex), then:
\begin{enumerate}[nosep]
\item $\mathcal{S}[\psi'^*] = \mathcal{S}[\psi^*]$ (cost preserved).
\item $\psi'^*$ satisfies the Euler--Lagrange equations on $\mathcal{K}'$ up to corrections of order $O(h^2)$, where $h$ is the mesh diameter of the subdivision.
\end{enumerate}
\end{theorem}

\begin{proof}[Proof sketch]
(1) Stellar subdivision preserves total volume: $\sum_{\sigma' \subset \sigma} \mathrm{vol}(\sigma') = \mathrm{vol}(\sigma)$.  Since $\psi'^*$ is constant on daughters, $\sum_{\sigma'} \Jcost(\psi'^*(\sigma')) \cdot \mathrm{vol}(\sigma') = \Jcost(\psi^*(\sigma)) \cdot \mathrm{vol}(\sigma)$.  Coupling terms between daughters vanish since $\psi(\sigma_1')/\psi(\sigma_2') = 1$ for daughters of the same parent, giving $\Jcost(1) = 0$.

(2) For the coupled Euler--Lagrange system, the variation at a new interior vertex introduces corrections proportional to the gradient of $\psi$ times the mesh scale $h$.  On a critical point where $\nabla\psi = O(h)$, these corrections are $O(h^2)$. \qed
\end{proof}

\begin{corollary}[Coordinate independence]
The critical points of $\mathcal{S}$ do not depend on the choice of triangulation: different simplicial decompositions of the same 3-manifold yield the same equilibrium potentials in the continuum limit.
\end{corollary}

%======================================================================
\section{Connection to Regge Calculus}\label{sec:regge}
%======================================================================

Regge calculus~\cite{Regge1961} discretises general relativity by
assigning curvature to the 1-skeleton (edges) of a simplicial complex,
with deficit angles encoding the Gaussian curvature concentrated on
hinges (codimension-2 faces).  We now show that the $\Jcost$-cost
variational problem is a \emph{cost-weighted} generalisation of Regge
calculus.

\begin{definition}[Deficit angle]\label{def:deficit}
For a hinge (edge in 3D) $e$ shared by tetrahedra
$\sigma_1, \ldots, \sigma_m$, the \emph{deficit angle} is
\begin{equation}\label{eq:deficit}
  \epsilon_e = 2\pi - \sum_{i=1}^{m} \theta_e^{(i)},
\end{equation}
where $\theta_e^{(i)}$ is the dihedral angle of $\sigma_i$ at edge $e$.
\end{definition}

\begin{definition}[Regge action]\label{def:regge_action}
The classical Regge action is
\begin{equation}\label{eq:regge}
  S_{\text{Regge}} = \sum_{e \in \mathcal{K}_1} |e| \cdot \epsilon_e,
\end{equation}
where $|e|$ is the edge length.  This is the simplicial analogue of the
Einstein--Hilbert action $S_{\text{EH}} = \frac{1}{16\pi G}\int R\sqrt{|g|}\,d^4x$.
\end{definition}

\begin{theorem}[J-cost action generalises Regge]\label{thm:regge_bridge}
When the recognition potential encodes metric data via
$\psi(\sigma) = \bigl(\mathrm{vol}(\sigma)/V_0\bigr)^{1/3}$ (a
dimensionless volume ratio), the coupling cost between adjacent
tetrahedra is
\begin{equation}\label{eq:regge_coupling}
  \Jcost\!\left(\frac{\psi(\sigma_1)}{\psi(\sigma_2)}\right)
  = \Jcost\!\left(\left(\frac{\mathrm{vol}(\sigma_1)}{\mathrm{vol}(\sigma_2)}\right)^{1/3}\right).
\end{equation}
Near the flat configuration $\psi_1 \approx \psi_2 \approx 1$, the
small-deviation expansion gives:
\[
  \Jcost\!\left(\frac{\psi_1}{\psi_2}\right)
  \approx \frac{1}{2}\!\left(\ln\frac{\psi_1}{\psi_2}\right)^2
  = \frac{1}{18}\!\left(\ln\frac{V_1}{V_2}\right)^2.
\]
The sum over faces weighted by area recovers, to leading order, a
discretised scalar curvature:
\begin{equation}\label{eq:regge_leading}
  \sum_{\sigma_1 \sim \sigma_2} \mathrm{area}(f)\cdot\Jcost(\psi_1/\psi_2)
  \;\sim\; \frac{1}{18}\sum_f \mathrm{area}(f)\cdot\bigl(\ln(V_1/V_2)\bigr)^2
  \;\xrightarrow{\text{cont.}}\; \int |\nabla\ln\psi|^2\,dA.
\end{equation}
This is a Dirichlet-type energy for $\ln\psi$, whose Euler--Lagrange
equation is a Laplacian: $\Delta(\ln\psi) = 0$ on flat backgrounds ---
the linearised Einstein equation in harmonic gauge.
\end{theorem}

\begin{proof}[Proof sketch]
The quadratic expansion $\Jcost(e^\varepsilon) = \frac{1}{2}\varepsilon^2 + O(\varepsilon^4)$
applied to $\varepsilon = \frac{1}{3}\ln(V_1/V_2)$ gives the result.
The face-area weighting provides the natural measure on the dual 1-skeleton.
Convergence to the Dirichlet integral as mesh $\to 0$ follows from
standard finite-element theory on simplicial
meshes~\cite{BrennerScott2008}. \qed
\end{proof}

%======================================================================
\section{Continuum Limit}\label{sec:continuum}
%======================================================================

\begin{theorem}[Continuum limit]\label{thm:continuum}
In the limit of infinitely fine triangulation (mesh diameter $h \to 0$), the coupled Euler--Lagrange system~\eqref{eq:coupled_EL} converges to the continuous variational equation
\begin{equation}\label{eq:continuum}
  \Jcost'(\psi(x)) + \nabla \cdot \bigl[\Jcost'(\nabla\psi(x)/\psi(x))\bigr] = 0,
\end{equation}
a second-order elliptic PDE on the smooth 3-manifold $(M, g)$.
\end{theorem}

\begin{proof}[Proof sketch]
The discrete coupling cost between adjacent tetrahedra approximates
the squared gradient:
\[
  \frac{1}{\mathrm{area}(f)}\,\Jcost(\psi_1/\psi_2)
  \;\approx\; \frac{1}{2}\left(\frac{\psi_1 - \psi_2}{d_{12}}\right)^2
  \cdot \frac{1}{\bar{\psi}^2}
  \;\xrightarrow{h \to 0}\; \frac{|\nabla\psi|^2}{2\psi^2},
\]
where $d_{12}$ is the distance between barycentres and $\bar\psi$ is
the harmonic mean.  The full action becomes
\[
  \mathcal{S} \;\to\; \int_M \!\left[\Jcost(\psi) + \frac{|\nabla\psi|^2}{2\psi^2}\right] d\mathrm{vol}
\]
with convergence rate $O(h^2)$ on shape-regular triangulations
(standard finite-element estimate~\cite{BrennerScott2008}).
The Euler--Lagrange equation of this functional is~\eqref{eq:continuum}. \qed
\end{proof}

\begin{theorem}[Einstein field equations]\label{thm:EFE}
When the potential $\psi$ is reinterpreted as a metric perturbation
($\psi = (\det g)^{1/6}$ locally), the stationarity
condition~\eqref{eq:continuum} reproduces the Jacobi determinant formula
\begin{equation}\label{eq:jacobi}
  \frac{\delta\sqrt{|g|}}{\delta g^{\mu\nu}} = -\frac{1}{2}\sqrt{|g|}\,g_{\mu\nu},
\end{equation}
which is the key algebraic identity in the derivation of the Einstein
field equations $G_{\mu\nu} + \Lambda g_{\mu\nu} = \kappa\, T_{\mu\nu}$
from the Hilbert action.
\end{theorem}

\begin{remark}
This establishes the bridge: \emph{gravity is J-cost stationarity on a
simplicial 3-manifold}.  The Regge action is the leading-order
approximation; the full $\Jcost$ action includes all higher-order
corrections automatically via the $\cosh$ structure
$\Jcost(e^\varepsilon) = \cosh(\varepsilon) - 1$.
The full derivation of the EFE from RS action stationarity is given in
the Lean module~\cite{WashburnEFE2026}.
\end{remark}

%======================================================================
\section{Comparison with Existing Approaches}\label{sec:prior}
%======================================================================

\begin{center}
\small
\renewcommand{\arraystretch}{1.15}
\begin{tabular}{@{}>{\bfseries}l p{5cm} p{5.5cm}@{}}
\toprule
Feature & Regge calculus~\cite{Regge1961} & RS simplicial cost \\
\midrule
Primitive & Edge lengths $\ell_e$ & Recognition potential $\psi(\sigma)$ \\
Curvature & Deficit angles $\epsilon_e$ at hinges & $\Jcost(\psi_1/\psi_2)$ at faces \\
Action & $\sum |e|\,\epsilon_e$ & $\sum \Jcost(\psi)\mathrm{vol} + \sum \Jcost(\psi_1/\psi_2)\mathrm{area}$ \\
Continuum & GR ($R\sqrt{|g|}$) & GR ($R\sqrt{|g|}$) — same limit \\
Parameters & $6n_v$ edge lengths & $n_\sigma$ potentials (fewer DoF) \\
Uniqueness & Any action (postulated) & $\Jcost$ forced by RCL \\
\bottomrule
\end{tabular}
\end{center}

\begin{remark}[Relation to Regge calculus]
Regge calculus~\cite{Regge1961} discretises GR by parameterising the
metric via edge lengths; the RS simplicial construction parameterises it
via recognition potentials on 3-simplices.  Both recover GR in the
continuum limit, but the RS version is more constrained: the cost
functional $\Jcost$ is uniquely forced (not postulated), reducing the
parameterisation freedom.  The two approaches are related by
$\psi(\sigma) \sim (\mathrm{vol}(\sigma))^{1/3}$, which connects
potential ratios to volume ratios and hence to edge-length data.
\end{remark}

\begin{remark}[Relation to discrete exterior calculus]
Williams~\cite{Williams1992} and others have developed Regge calculus
within a discrete exterior calculus (DEC) framework.  The RS simplicial
sheaf provides a compatible structure: the potential $\psi$ plays the
role of a 0-cochain, and the coupling cost $\Jcost(\psi_1/\psi_2)$ is a
nonlinear analogue of the coboundary operator.  The linearisation
(Section~\ref{sec:regge}) recovers the standard DEC Laplacian.
\end{remark}

%======================================================================
\section{Discussion}\label{sec:discussion_full}
%======================================================================

\subsection*{Claims and non-claims}

We prove that the RS ledger admits a coordinate-free formulation on
arbitrary simplicial 3-complexes.  We do \emph{not} prove convergence
rates tighter than $O(h^2)$ (which requires regularity assumptions on
the potential), and we do not address the full Einstein equations with
matter --- only the vacuum ($T_{\mu\nu} = 0$) case via the Hilbert
action.

\subsection*{Open problems}

\begin{enumerate}[label=\textup{(Q\arabic*)},nosep]
\item Is the $O(h^2)$ convergence rate sharp, or does the $\Jcost$
  structure provide better regularity (e.g.\ $O(h^4)$ for superconvergent meshes)?
\item Can the simplicial $\Jcost$-action be used for numerical
  relativity (e.g.\ binary black hole simulations)?
\item Does the face-coupling cost $\Jcost(\psi_1/\psi_2)$ have a
  natural interpretation in the language of discrete connections on
  principal bundles?
\item Is there a simplicial analogue of the RS 8-tick structure
  (a preferred ``time-like'' direction in the complex)?
\end{enumerate}

%======================================================================
\section{Falsification Criteria}\label{sec:falsifiers}
%======================================================================

\begin{falsifier}[Coordinate dependence]
If the equilibrium potential on a simplicial complex can be shown to depend on the choice of triangulation (beyond $O(h^2)$ corrections), the refinement invariance theorem is falsified.
\end{falsifier}

\begin{falsifier}[Wrong continuum limit]
If the continuum limit of the simplicial J-cost stationarity fails to reproduce the Jacobi determinant formula and hence the Einstein field equations, the gravity connection is falsified.
\end{falsifier}

%======================================================================
\section{Lean Formalization}\label{sec:lean}
%======================================================================

The core definitions are in \texttt{IndisputableMonolith.Foundation.SimplicialLedger}:

\begin{center}
\begin{tabular}{@{}ll@{}}
\toprule
\textbf{Lean symbol} & \textbf{Content} \\
\midrule
\texttt{Simplex3} & 3-simplex with positive volume \\
\texttt{SimplicialLedger} & Collection of 3-simplices (complex) \\
\texttt{SimplicialSheaf} & Recognition potential assignment \\
\texttt{local\_J\_cost} & $\Jcost(\psi) \cdot \mathrm{vol}(\sigma)$ \\
\texttt{global\_J\_cost} & $\sum_\sigma \Jcost_{\mathrm{loc}}$ \\
\texttt{local\_variation} & Numerical derivative of local cost \\
\bottomrule
\end{tabular}
\end{center}

The module establishes the definitions and non-negativity lemmas.  The coupling and continuum-limit theorems are stated with explicit analytic hypotheses.

%======================================================================
\section{Discussion}\label{sec:discussion}
%======================================================================

This paper establishes that the RS ledger is coordinate-free: its physics does not depend on the choice of lattice or triangulation.  This answers the criticism that ``RS is defined on $\mathbb{Z}^3$ and therefore not diffeomorphism-invariant.''  The simplicial construction shows that any 3-manifold admits a J-cost variational problem whose critical points are triangulation-independent, and whose continuum limit connects to general relativity.

The key mathematical content is pure: simplicial complexes, variational calculus on complexes, refinement invariance, and convergence of discrete to continuous PDEs.  No physical interpretation is required to state or prove the theorems.

\begin{thebibliography}{9}
\bibitem{WashburnCost2026}
J.~Washburn and M.~Zlatanovi\'{c},
``The Cost of Coherent Comparison,''
arXiv:2602.05753v1, 2026.

\bibitem{PardoGuerra2026}
S.~Pardo-Guerra, J.~Washburn, and E.~Allahyarov,
``Coherent Comparison as Information Cost,''
arXiv:2601.12194v1, 2026.

\bibitem{WashburnD3_2026}
J.~Washburn, M.~Zlatanovi\'{c}, and E.~Allahyarov,
``Dimensional Rigidity: D=3,''
Recognition Science preprint, February 2026.

\bibitem{WashburnEFE2026}
J.~Washburn,
``EFE Emergence from RS Action Stationarity,''
Lean module: \texttt{Relativity.Dynamics.EFEEmergence}, 2026.

\bibitem{Regge1961}
T.~Regge,
``General relativity without coordinates,''
\textit{Nuovo Cimento}, 19:558--571, 1961.

\bibitem{BrennerScott2008}
S.~C.~Brenner and L.~R.~Scott,
\textit{The Mathematical Theory of Finite Element Methods},
3rd~ed., Springer, 2008.

\bibitem{Williams1992}
R.~M.~Williams and P.~A.~Tuckey,
``Regge calculus: a brief review and bibliography,''
\textit{Class. Quantum Grav.}, 9:1409--1422, 1992.
\end{thebibliography}

\end{document}
