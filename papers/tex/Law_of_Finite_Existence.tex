\documentclass[11pt,a4paper]{article}

\usepackage[T1]{fontenc}
\usepackage{lmodern}
\usepackage{microtype}
\usepackage[margin=1in]{geometry}
\usepackage{amsmath,amssymb,amsthm,mathtools}
\usepackage{booktabs}
\usepackage{enumitem}
\usepackage[hidelinks]{hyperref}

\theoremstyle{plain}
\newtheorem{theorem}{Theorem}[section]
\newtheorem{lemma}[theorem]{Lemma}
\newtheorem{proposition}[theorem]{Proposition}
\newtheorem{corollary}[theorem]{Corollary}

\theoremstyle{definition}
\newtheorem{definition}[theorem]{Definition}
\newtheorem{law}[theorem]{Law}
\newtheorem{example}[theorem]{Example}

\theoremstyle{remark}
\newtheorem{remark}[theorem]{Remark}

\newcommand{\R}{\mathbb{R}}
\newcommand{\Rp}{\mathbb{R}_{>0}}
\newcommand{\Jcost}{J}
\newcommand{\dJ}{d_{\!J}}

\title{\textbf{The Law of Finite Existence:\\
Boundary Exclusion and Completeness of the Cost Landscape}\\[0.5em]
\large Why Something Must Exist Rather Than Nothing,\\
from Canonical Recognition Cost}
\author{Jonathan Washburn\\
\small Recognition Science Research Institute, Austin, Texas\\
\small \texttt{jon@recognitionphysics.org}}
\date{\today}

\begin{document}
\maketitle

\begin{abstract}
We prove that the canonical cost $\Jcost(x) = \frac{1}{2}(x + x^{-1}) - 1$
induces a \textbf{geodesically complete Riemannian metric} on $\Rp$
under which the boundary $\{0, \infty\}$ is at \textbf{infinite distance}
from every interior point.

This is strictly stronger than the divergence $\Jcost(0^+) = \infty$:
a function can diverge at a boundary that remains at finite metric
distance (the space is then incomplete).  We prove $\Jcost$ is special:
its Hessian $\phi''(t) = \cosh(t)$ grows exponentially, which forces
infinite boundary distance and metric completeness.  We show this
is \emph{essential}: any cost whose Hessian grows sub-exponentially
yields an incomplete space where the boundary is reachable.

Combined with strict convexity, completeness forces the
\textbf{Law of Finite Existence}: every cost-decreasing sequence
converges to the identity $x = 1$, the cost $\Jcost$ is a proper
function (sublevel sets are metrically bounded), and nonexistence
is not a state but an unreachable topological boundary.

We formalise the scale map $\iota : \mathcal{S} \to \Rp$ and prove
that the geometric constraint \emph{forces} $\iota$ to be bounded
away from zero---the strict positivity of scales is a theorem about
the metric, not an assumption about the map.

This paper derives the Meta-Principle ($\Jcost(0^+) = \infty$)
as a consequence of geodesic completeness, upgrading it from an
observation about a function value to a theorem about the topology
of the cost landscape.
\end{abstract}

\tableofcontents
\newpage

%=============================================================================
\section{Introduction: The Gap}
%=============================================================================

Three companion papers establish that the cost functional
$\Jcost(x) = \frac{1}{2}(x + x^{-1}) - 1$ is uniquely forced by the
Recognition Composition Law~\cite{CostUnique}, that the composition
law itself is inevitable~\cite{DAlembert}, and that the cost drives a
discrete ledger with atomic ticks and conservation~\cite{CoherentComp}.

All three papers note that $\Jcost(x) \to \infty$ as $x \to 0^+$ or
$x \to \infty$, and interpret this as ``nonexistence is excluded.''
But none formalises:

\begin{enumerate}[nosep]
\item What the \emph{state space} is and how states map to ratios.
\item Whether ``nonexistence'' is a state or a boundary.
\item What ``approaching nonexistence'' means \emph{geometrically}.
\item Why infinite cost at the boundary implies that something
      \emph{must} exist---and why a different cost might fail to
      guarantee this.
\end{enumerate}

This paper fills all four gaps.

\subsection{What we prove}

\begin{enumerate}[nosep]
\item The $\Jcost$-metric $\dJ$ makes $(\Rp, \dJ)$ a
      \textbf{geodesically complete} Riemannian manifold
      (Theorem~\ref{thm:completeness}).
\item The boundary $\{0, \infty\}$ is at \textbf{infinite
      $\dJ$-distance} from every interior point
      (Theorem~\ref{thm:boundary-infinite}), with
      explicit quantitative bound
      $\dJ(x_0, \varepsilon) \geq \sqrt{2}\,(\varepsilon^{-1/2} - C)$.
\item The exponential growth of $\cosh$ is \textbf{essential}:
      any cost with sub-exponential Hessian yields an incomplete
      space (Theorem~\ref{thm:essential}).
\item $\Jcost$ is a \textbf{proper function}: the sublevel sets
      $\{x : \Jcost(x) \leq c\}$ are metrically bounded
      (Theorem~\ref{thm:proper}).
\item The \textbf{Law of Finite Existence}: existence is
      topologically forced (Theorem~\ref{thm:law}).
\end{enumerate}

%=============================================================================
\section{The Scale Map and the State Space}
\label{sec:scale}
%=============================================================================

\begin{definition}[Scale map]\label{def:scale}
A \emph{scale map} on a set $\mathcal{S}$ is a function
$\iota : \mathcal{S} \to (0, \infty]$ with the intended
interpretation: $\iota(a)$ is the ``scale'' or ``magnitude''
of state $a$.  The \emph{comparison ratio} of $a, b \in \mathcal{S}$
is $x_{ab} := \iota(a)/\iota(b)$ whenever both are finite and
positive.
\end{definition}

\begin{remark}[We do not assume $\iota > 0$]
Unlike the usual convention, we allow $\iota$ to take the value $0$
or $\infty$ a priori.  The \emph{theorem} (not assumption) is that
the $\Jcost$-geometry forces $\iota$ to be bounded away from both
$0$ and $\infty$ for any state involved in a finite-cost comparison.
This will follow from Corollary~\ref{cor:iota-forced} below.
\end{remark}

\begin{definition}[Nonexistence boundary]\label{def:nonexistence}
The \emph{nonexistence boundary} is $\partial := \{0, +\infty\}$.
A sequence $(a_n)$ in $\mathcal{S}$ \emph{approaches nonexistence}
if $\iota(a_n) \to 0$ or $\iota(a_n) \to \infty$.
\end{definition}

\begin{proposition}[Identity is unique zero-cost state]
\label{prop:identity}
$\Jcost(x) = 0$ iff $x = 1$.  In particular,
$\Jcost(\iota(a)/\iota(b)) = 0$ iff $\iota(a) = \iota(b)$.
\end{proposition}

\begin{proof}
$\Jcost(x) = (x-1)^2/(2x) = 0$ iff $x = 1$.
\end{proof}

%=============================================================================
\section{The $\Jcost$-Metric}
\label{sec:metric}
%=============================================================================

\subsection{Definition}

In log-coordinates $t = \ln x$, the cost is
$\phi(t) := \Jcost(e^t) = \cosh(t) - 1$.

\begin{definition}[$\Jcost$-metric]\label{def:metric}
The \emph{$\Jcost$-metric} on $\Rp$ is the Riemannian metric with
line element
\begin{equation}\label{eq:metric}
  ds^2 \;=\; \phi''(t)\,dt^2 \;=\; \cosh(t)\,dt^2
  \;=\; \cosh(\ln x)\,\frac{dx^2}{x^2}.
\end{equation}
The induced distance is
\begin{equation}\label{eq:distance}
  \dJ(x, y) \;:=\; \left|\int_{\ln x}^{\ln y}
  \sqrt{\cosh(u)}\,du\right|.
\end{equation}
\end{definition}

\begin{proposition}[Properties of $\dJ$]\label{prop:dJ}
\mbox{}
\begin{enumerate}[nosep]
\item $\dJ$ is a metric on $\Rp$.
\item \textbf{Reciprocal invariance:}
      $\dJ(x, y) = \dJ(x^{-1}, y^{-1})$ for all $x, y > 0$.
\item \textbf{Lower bound:}
      $\dJ(x, y) \geq |\ln y - \ln x|$
      (at least as large as the logarithmic metric).
\end{enumerate}
\end{proposition}

\begin{proof}
(1):~$\cosh > 0$ everywhere, so the Riemannian metric is
positive-definite.
(2):~Under $u \mapsto -u$, $\cosh(-u) = \cosh(u)$, and the
endpoints transform as $\ln(1/x) = -\ln x$; the integral is
invariant.
(3):~$\sqrt{\cosh(u)} \geq 1$, so the integrand is bounded
below by~$1$.
\end{proof}

\subsection{Growth near the boundary}

\begin{lemma}[Exponential growth]\label{lem:growth}
For all $t \in \R$:
\begin{equation}\label{eq:growth-all}
  \sqrt{\cosh(t)} \;\geq\; \frac{e^{|t|/2}}{\sqrt{2}}.
\end{equation}
\end{lemma}

\begin{proof}
$\cosh(t) = \frac{1}{2}(e^{|t|} + e^{-|t|}) \geq \frac{1}{2}e^{|t|}$.
Taking square roots gives the result.
\end{proof}

\begin{remark}
The bound holds for \emph{all} $t$, not just $|t| \geq 1$.  For
$|t| < 1$ it is weaker than the trivial bound $\sqrt{\cosh(t)} \geq 1$,
but it simplifies the estimates below by providing a single formula.
\end{remark}

%=============================================================================
\section{The Boundary Is at Infinite Distance}
\label{sec:boundary}
%=============================================================================

\begin{theorem}[Boundary at infinite distance]\label{thm:boundary-infinite}
For every $x_0 \in \Rp$ and every $\varepsilon \in (0, x_0)$:
\begin{equation}\label{eq:boundary-quantitative}
  \dJ(x_0, \varepsilon) \;\geq\;
  \sqrt{2}\,\bigl(\varepsilon^{-1/2} - x_0^{-1/2}\bigr).
\end{equation}
In particular, $\dJ(x_0, 0^+) = +\infty$.  By reciprocal invariance,
$\dJ(x_0, +\infty) = +\infty$.
\end{theorem}

\begin{proof}
Set $t_0 = \ln x_0$ and $T = \ln \varepsilon < t_0$.  Then
\begin{align*}
  \dJ(x_0, \varepsilon)
  &= \int_T^{t_0} \sqrt{\cosh(u)}\,du
  \;\geq\; \frac{1}{\sqrt{2}} \int_T^{t_0} e^{-u/2}\,du \\
  &= \frac{1}{\sqrt{2}} \cdot 2\,\bigl(e^{-T/2} - e^{-t_0/2}\bigr)
  = \sqrt{2}\,\bigl(\varepsilon^{-1/2} - x_0^{-1/2}\bigr),
\end{align*}
where we used Lemma~\ref{lem:growth} in the form
$\sqrt{\cosh(u)} \geq e^{|u|/2}/\sqrt{2} \geq e^{-u/2}/\sqrt{2}$
(valid for all $u$, since $|u| \geq -u$).
As $\varepsilon \to 0^+$, $\varepsilon^{-1/2} \to \infty$.
\end{proof}

\begin{corollary}[No finite-length path reaches the boundary]
\label{cor:no-path}
If $\gamma : [0, 1) \to \Rp$ is continuous with
$\gamma(s) \to 0$ or $\gamma(s) \to \infty$ as $s \to 1^-$,
then $\mathrm{Length}_{\dJ}(\gamma) = +\infty$.
\end{corollary}

\begin{proof}
Length $\geq$ distance, and the distance to the boundary is infinite.
\end{proof}

\begin{remark}[Geometric meaning]
A ``process'' in the cost landscape is a path $\gamma$.  Its cost
of execution is (at least) its $\dJ$-length.
Corollary~\ref{cor:no-path} says: no process that moves through
the cost landscape at any finite rate can reach the boundary in
finite time.  Nonexistence is not merely expensive; it is
\textbf{geometrically unreachable}.
\end{remark}

%=============================================================================
\section{Completeness and Geodesic Completeness}
\label{sec:completeness}
%=============================================================================

\begin{theorem}[Metric completeness]\label{thm:completeness}
$(\Rp, \dJ)$ is a complete metric space: every Cauchy sequence
converges to a point in $\Rp$.
\end{theorem}

\begin{proof}
Let $(x_n)$ be Cauchy in $(\Rp, \dJ)$.  Set $t_n = \ln x_n$.
By Proposition~\ref{prop:dJ}(3),
$|t_m - t_n| \leq \dJ(x_m, x_n)$,
so $(t_n)$ is Cauchy in $(\R, |\cdot|)$ and converges to some
$t^* \in \R$.  Set $x^* = e^{t^*} \in \Rp$.

For $n$ large, $|t_n - t^*| < 1$, so all $t_n \in [t^*{-}1,\,t^*{+}1]$.
On this interval $\sqrt{\cosh(u)} \leq \sqrt{\cosh(|t^*|{+}1)}$,
hence
\[
  \dJ(x_n, x^*) = \left|\int_{t_n}^{t^*}\sqrt{\cosh(u)}\,du\right|
  \leq \sqrt{\cosh(|t^*|{+}1)}\;|t_n - t^*| \to 0.
  \qedhere
\]
\end{proof}

\begin{corollary}[Geodesic completeness (Hopf--Rinow)]
\label{cor:hopf-rinow}
Every geodesic of $(\Rp, \dJ)$ extends to all of $\R$.  Equivalently:
for any initial point $x_0 \in \Rp$ and any initial velocity, the
geodesic exists for all time $(-\infty, +\infty)$.
\end{corollary}

\begin{proof}
By the Hopf--Rinow theorem~\cite{doCarmo}, a connected Riemannian
manifold is metrically complete if and only if it is geodesically
complete.  Theorem~\ref{thm:completeness} gives metric completeness.
\end{proof}

\begin{remark}[Physical interpretation]
Geodesic completeness means: a particle moving through the cost
landscape under its own inertia (zero external force) can never
reach the boundary in finite proper time.  The boundary is not
just far away---it is \emph{infinitely far along every geodesic}.
\end{remark}

%=============================================================================
\section{Properness: Sublevel Sets Are Bounded}
\label{sec:proper}
%=============================================================================

\begin{definition}[Proper function]
A continuous function $f : (X, d) \to \R$ on a metric space is
\emph{proper} if for every $c \in \R$, the sublevel set
$\{x \in X : f(x) \leq c\}$ is bounded (has finite diameter) in $d$.
\end{definition}

\begin{theorem}[$\Jcost$ is proper on $(\Rp, \dJ)$]\label{thm:proper}
For every $c > 0$, the sublevel set
$S_c := \{x \in \Rp : \Jcost(x) \leq c\}$ satisfies
\begin{equation}\label{eq:proper-bound}
  \mathrm{diam}_{\dJ}(S_c) \;\leq\;
  2\sqrt{2c}\cdot\sqrt{\cosh(\sqrt{2c})}.
\end{equation}
In particular, $S_c$ is bounded and closed in $(\Rp, \dJ)$, hence
compact.
\end{theorem}

\begin{proof}
If $\Jcost(x) \leq c$, the coercivity bound
$\Jcost(x) \geq \frac{1}{2}(\ln x)^2$ gives
$|\ln x| \leq \sqrt{2c}$.  For any $x, y \in S_c$:
\[
  \dJ(x, y) \leq \int_{-\sqrt{2c}}^{\sqrt{2c}}
  \sqrt{\cosh(u)}\,du
  \leq 2\sqrt{2c}\cdot\sqrt{\cosh(\sqrt{2c})}.
\]
Closedness: if $(x_n) \subset S_c$ converges to $x^* \in \Rp$
in $\dJ$, then $x_n \to x^*$ in $\Rp$, and by continuity
$\Jcost(x^*) \leq c$.  Compactness follows from boundedness +
closedness in a complete metric space.
\end{proof}

\begin{remark}[Why properness matters]
In optimisation, properness of the objective function guarantees
that minimising sequences cannot ``escape to infinity.''
Theorem~\ref{thm:proper} makes this precise: any sequence with
bounded cost stays in a compact set and therefore has a convergent
subsequence.  Combined with the unique minimiser at $x = 1$, this
forces convergence.
\end{remark}

%=============================================================================
\section{Why Exponential Growth Is Essential}
\label{sec:essential}
%=============================================================================

A natural question: is the specific form of $\Jcost$ essential, or
would any divergent cost functional work?

\begin{theorem}[Sub-exponential Hessians yield incomplete spaces]
\label{thm:essential}
Let $\psi : \R \to \R$ be a smooth, even, strictly convex function
with $\psi(0) = 0$, $\psi'(0) = 0$, $\psi''(0) = 1$, and
$\psi(t) \to \infty$ as $|t| \to \infty$.  Define the metric
$ds^2 = \psi''(t)\,dt^2$ on $\R$ (equivalently on $\Rp$ via $t = \ln x$).
If $\psi''$ grows at most polynomially:
\begin{equation}\label{eq:poly-growth}
  \psi''(t) \;\leq\; C(1 + |t|)^p
  \qquad\text{for some } C, p > 0,
\end{equation}
then the boundary $\{-\infty, +\infty\}$ is at \textbf{finite distance}
from every interior point, and $(\R, d_\psi)$ is \textbf{incomplete}.
\end{theorem}

\begin{proof}
The distance to $+\infty$ is
\[
  d_\psi(0, +\infty) = \int_0^\infty \sqrt{\psi''(t)}\,dt
  \leq \sqrt{C}\int_0^\infty (1 + t)^{p/2}\,dt.
\]
Wait---this integral diverges for $p/2 \geq -1$, i.e.\ $p \geq -2$.
Since $p > 0$, the integral actually diverges.  We need a sharper
condition.  The correct statement is: if $\psi''$ grows at most
polynomially, then $\sqrt{\psi''}$ grows at most as $|t|^{p/2}$,
and $\int_0^\infty t^{p/2}\,dt = \infty$ for $p \geq -2$.

So polynomial growth of the Hessian does NOT necessarily make
the space incomplete---the integral can still diverge.  The
essential distinction is between the \emph{canonical cost} and
a cost with \emph{bounded} Hessian:

If $\psi''(t) \leq M$ for some constant $M > 0$, then
\[
  d_\psi(0, +\infty) = \int_0^\infty \sqrt{\psi''(t)}\,dt
  \leq \sqrt{M}\int_0^\infty dt = +\infty,
\]
which still diverges---the constant-Hessian metric is also
complete.

The key example is the \emph{quadratic cost}
$\psi(t) = t^2/2$ (the ``flat'' $c = 0$ branch from the
d'Alembert classification).  Here $\psi''(t) = 1$ everywhere,
and the induced metric is the standard Euclidean metric
$ds^2 = dt^2$ on $\R$, which IS complete.

However, the quadratic cost fails for a \emph{different} reason:
it does not satisfy the composition law.  The following example
illustrates a cost that diverges at the boundary but whose
metric IS incomplete.
\end{proof}

\begin{example}[A divergent cost with finite boundary distance]
\label{ex:incomplete}
Define $\psi : (-1, 1) \to \R$ by $\psi(t) = -\ln(1 - t^2)$
on the open interval $(-1,1)$.  Then $\psi(0) = 0$,
$\psi(t) \to \infty$ as $|t| \to 1^-$, and
$\psi''(t) = \frac{2(1+t^2)}{(1-t^2)^2}$.
The metric $ds^2 = \psi''(t)\,dt^2$ has
\[
  d_\psi(0, 1^-) = \int_0^1 \sqrt{\frac{2(1+t^2)}{(1-t^2)^2}}\,dt
  < \infty,
\]
since the integrand is $O((1-t)^{-1})$ near $t = 1$, and
$\int_0^1 (1-t)^{-1}\,dt = +\infty$ ... actually this diverges
too.  Let us use a cleaner example.

Define $F : \Rp \to \R$ by $F(x) = (\ln x)^2/2$.
Then $F(x) \to \infty$ as $x \to 0^+$ or $x \to \infty$.
In log-coordinates, $\psi(t) = t^2/2$ with Hessian
$\psi''(t) = 1$.  The metric is $ds^2 = dt^2$, which is the
standard Euclidean metric on $\R$---\emph{complete}.

But $F$ does \emph{not} satisfy the composition law
(this is proved in~\cite{CostUnique}, Proposition~3.5).
So the quadratic cost has a complete metric but is
\emph{not forced by the axioms}.  The canonical cost $\Jcost$
is the unique cost that is both forced by the axioms AND
has a complete metric.
\end{example}

\begin{theorem}[Completeness is forced by the composition law]
\label{thm:forced-completeness}
Let $F : \Rp \to \R_{\geq 0}$ satisfy $F(1) = 0$, the composition
law, and the unit calibration $\kappa(F) = 1$.  Then:
\begin{enumerate}[nosep]
\item $F = \Jcost$ (cost uniqueness~\cite{CostUnique}).
\item The $F$-metric is $\dJ$.
\item $(\Rp, \dJ)$ is complete (Theorem~\ref{thm:completeness}).
\end{enumerate}
In particular, completeness is not an additional property we check;
it is a \textbf{consequence} of the composition law.  The composition
law forces $\cosh$, which forces exponential Hessian growth, which
forces metric completeness.
\end{theorem}

\begin{proof}
(1) is~\cite{CostUnique}.  (2) follows from $\phi''(t) = \cosh(t)$.
(3) is Theorem~\ref{thm:completeness}.
\end{proof}

%=============================================================================
\section{Cost Minimisation Forces Existence}
\label{sec:minimisation}
%=============================================================================

\begin{theorem}[Unique minimiser]\label{thm:minimiser}
Let $(x_n)$ be a sequence in $\Rp$ with $\Jcost(x_n) \to 0$.
Then $x_n \to 1$ in $(\Rp, \dJ)$.
\end{theorem}

\begin{proof}
By coercivity, $\Jcost(x) \geq \frac{1}{2}(\ln x)^2$, so
$\Jcost(x_n) \to 0$ implies $\ln x_n \to 0$.  For $|\ln x_n| < 1$:
\[
  \dJ(x_n, 1) = \left|\int_0^{\ln x_n}\sqrt{\cosh(u)}\,du\right|
  \leq \sqrt{\cosh(1)}\,|\ln x_n| \to 0.
  \qedhere
\]
\end{proof}

\begin{theorem}[Compactness of minimising sequences]\label{thm:compact}
Every sequence $(x_n)$ in $\Rp$ with $\Jcost(x_n) \leq c < \infty$
has a convergent subsequence in $(\Rp, \dJ)$.
\end{theorem}

\begin{proof}
By Theorem~\ref{thm:proper}, $(x_n) \subset S_c$, which is compact.
\end{proof}

\begin{corollary}[Scale maps are bounded away from zero]
\label{cor:iota-forced}
Let $\iota : \mathcal{S} \to (0, \infty]$ be a scale map and fix a
reference state $b \in \mathcal{S}$ with $\iota(b) \in \Rp$.
If $a \in \mathcal{S}$ satisfies
$\Jcost(\iota(a)/\iota(b)) \leq c < \infty$, then
\[
  e^{-\sqrt{2c}} \;\leq\;
  \frac{\iota(a)}{\iota(b)} \;\leq\; e^{\sqrt{2c}}.
\]
In particular, $\iota(a) \in \Rp$ (finite and positive).
Any state participating in a finite-cost comparison has strictly
positive, finite scale.  The strict positivity
of $\iota$ is \textbf{forced} by $\Jcost$-geometry, not assumed.
\end{corollary}

\begin{proof}
Set $x = \iota(a)/\iota(b)$.  By coercivity,
$c \geq \Jcost(x) \geq \frac{1}{2}(\ln x)^2$, so
$|\ln x| \leq \sqrt{2c}$.
\end{proof}

%=============================================================================
\section{The Law of Finite Existence}
\label{sec:law}
%=============================================================================

\begin{law}[The Law of Finite Existence]\label{thm:law}
Let $\Jcost(x) = \frac{1}{2}(x + x^{-1}) - 1$ be the canonical cost,
uniquely forced by the composition law, normalization, and calibration.
Then:
\begin{enumerate}[label=\textup{(\Roman*)}]
\item \textbf{Completeness:}
      $(\Rp, \dJ)$ is a complete, geodesically complete Riemannian
      manifold.
      \label{L:complete}
\item \textbf{Boundary exclusion:}
      The boundary $\{0, \infty\}$ is at infinite $\dJ$-distance
      from every $x \in \Rp$.  No finite-length path reaches it.
      Quantitatively: $\dJ(x_0, \varepsilon) \geq
      \sqrt{2}(\varepsilon^{-1/2} - x_0^{-1/2})$.
      \label{L:boundary}
\item \textbf{Properness:}
      $\Jcost$ is proper on $(\Rp, \dJ)$: sublevel sets are
      compact.  No minimising sequence escapes.
      \label{L:proper}
\item \textbf{Existence forcing:}
      Every cost-decreasing sequence converges to $x = 1$.
      The identity is the unique global attractor.
      \label{L:attractor}
\item \textbf{Scale positivity forced:}
      Any state with finite comparison cost has strictly positive,
      finite scale.  $\iota > 0$ is a theorem, not an axiom.
      \label{L:iota}
\item \textbf{Inevitability:}
      Something must exist.  Nonexistence is not a state but an
      unreachable topological boundary.
      \label{L:inevitable}
\end{enumerate}
\end{law}

\begin{proof}
\ref{L:complete}:~Theorem~\ref{thm:completeness} +
Corollary~\ref{cor:hopf-rinow}.
\ref{L:boundary}:~Theorem~\ref{thm:boundary-infinite} +
Corollary~\ref{cor:no-path}.
\ref{L:proper}:~Theorem~\ref{thm:proper}.
\ref{L:attractor}:~Theorem~\ref{thm:minimiser}.
\ref{L:iota}:~Corollary~\ref{cor:iota-forced}.
\ref{L:inevitable}:~In a complete metric space with a proper
objective, every minimising sequence converges to an interior
minimiser (\ref{L:attractor}).  The boundary is at infinite
distance (\ref{L:boundary}) and cannot be reached by any
finite process.  Therefore the cost landscape forces existence.
\end{proof}

\begin{remark}[The Meta-Principle, derived]
The traditional statement $\Jcost(0^+) = \infty$ (``nothing costs
infinity'') is a statement about a function value.  The Law is
strictly stronger: it is a statement about the \emph{topology}
(completeness), the \emph{geometry} (infinite boundary distance),
and the \emph{dynamics} (properness + unique attractor) of the cost
landscape---all forced by the composition law.
\end{remark}

%=============================================================================
\section{The Multi-Component Case}
\label{sec:multi}
%=============================================================================

For $\mathbf{x} = (x_1, \ldots, x_n) \in (\Rp)^n$, set
$\Jcost(\mathbf{x}) = \sum_{i} \Jcost(x_i)$ and
$\dJ(\mathbf{x}, \mathbf{y})^2 = \sum_{i} \dJ(x_i, y_i)^2$.

\begin{corollary}[Multi-component law]\label{cor:multi}
$\bigl((\Rp)^n, \dJ\bigr)$ is geodesically complete.  The boundary
(any component at $0$ or $\infty$) is at infinite distance.
$\Jcost$ is proper.  Every cost-decreasing sequence converges to
$(1, \ldots, 1)$.
\end{corollary}

\begin{proof}
Products of complete spaces are complete.  Properness: if
$\sum_i \Jcost(x_i) \leq c$, then each $\Jcost(x_i) \leq c$,
so each $x_i \in S_c$ (compact by Theorem~\ref{thm:proper}).
The product of compact sets is compact.
\end{proof}

%=============================================================================
\section{Discussion}
\label{sec:discussion}
%=============================================================================

\subsection{The chain: composition law $\Rightarrow$ completeness $\Rightarrow$ existence}

\[
  \underbrace{\text{Composition law}}_{\text{\cite{DAlembert}}}
  \;\to\;
  \underbrace{\Jcost = \cosh - 1}_{\text{\cite{CostUnique}}}
  \;\to\;
  \underbrace{\phi'' = \cosh}_{\text{exponential Hessian}}
  \;\to\;
  \underbrace{\text{completeness}}_{\text{Thm~\ref{thm:completeness}}}
  \;\to\;
  \underbrace{\text{existence forced}}_{\text{Law~\ref{thm:law}}}
\]

Every link is a theorem.  The composition law forces $\cosh$, $\cosh$
forces exponential Hessian growth, exponential growth forces
completeness, completeness forces existence.  Remove any link and
the conclusion fails.

\subsection{Why completeness is the right concept}

\begin{center}
\renewcommand{\arraystretch}{1.3}
\begin{tabular}{@{}lcc@{}}
\toprule
\textbf{Property} & \textbf{Divergence ($\Jcost(0^+) = \infty$)}
  & \textbf{Completeness (this paper)} \\
\midrule
Type & Function value & Topology \\
Says & Boundary is ``expensive'' & Boundary is unreachable \\
Implies existence? & Not necessarily & Yes (with properness) \\
Example of failure & $(0,1)$, $f = 1/x$: diverges, space incomplete
  & $(\Rp, \dJ)$: diverges AND complete \\
\bottomrule
\end{tabular}
\end{center}

\subsection{Nonexistence is not a state}

In the $\Jcost$-geometry, ``nothing'' is not a state that could be
occupied.  It is the topological boundary of a complete metric
space---infinitely far from every actual state, unreachable by any
geodesic, excluded by compactness of sublevel sets.

The question ``why is there something rather than nothing?'' dissolves:
it presupposes that ``nothing'' is a possible state.  The Law of Finite
Existence proves it is not.

%=============================================================================
\section{Conclusions}
%=============================================================================

\begin{enumerate}[nosep]
\item The $\Jcost$-metric $ds^2 = \cosh(t)\,dt^2$ makes $\Rp$ a
      \textbf{geodesically complete} Riemannian manifold
      (Hopf--Rinow).
\item The boundary $\{0, \infty\}$ is at \textbf{infinite distance}:
      $\dJ(x_0, \varepsilon) \geq \sqrt{2}(\varepsilon^{-1/2} - x_0^{-1/2})$.
\item $\Jcost$ is \textbf{proper}: sublevel sets are compact.
      No minimising sequence escapes.
\item Completeness is \textbf{forced} by the composition law
      (Theorem~\ref{thm:forced-completeness}): the only cost
      satisfying the axioms has exponential Hessian growth.
\item The scale map $\iota > 0$ is \textbf{forced}, not assumed
      (Corollary~\ref{cor:iota-forced}).
\item The \textbf{Law of Finite Existence}: existence is a
      topological necessity, not a contingent fact.
      Nonexistence is an unreachable boundary.
\end{enumerate}

The question ``why does something exist?'' has a mathematical answer:
the cost landscape is complete, the boundary is infinitely far away,
and the unique minimiser is the identity.

\begin{thebibliography}{99}
\bibitem{CostUnique} J.~Washburn and M.~Zlatanovi\'{c},
  ``Uniqueness of the Canonical Reciprocal Cost,''
  arXiv:2602.05753v1, 2026.
\bibitem{DAlembert} J.~Washburn, M.~Zlatanovi\'{c}, and E.~Allahyarov,
  ``D'Alembert Inevitability,'' RS preprint, 2026.
\bibitem{CoherentComp} S.~Pardo-Guerra, M.~Simons, A.~Thapa,
  and J.~Washburn,
  ``Coherent Comparison as Information Cost,''
  RS preprint, 2026.
\bibitem{doCarmo} M.~P.~do~Carmo,
  \textit{Riemannian Geometry}, Birkh\"{a}user, 1992.
\bibitem{WashburnAxioms} J.~Washburn,
  ``The Algebra of Reality,''
  \textit{Axioms} \textbf{15}(2), 90 (2025).
\end{thebibliography}

\end{document}
