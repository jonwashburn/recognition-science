\documentclass[11pt,a4paper]{article}

\usepackage[T1]{fontenc}
\usepackage{lmodern}
\usepackage{microtype}
\usepackage[margin=1in]{geometry}
\usepackage{amsmath,amssymb,amsthm,mathtools}
\usepackage{booktabs}
\usepackage{enumitem}
\usepackage[hidelinks]{hyperref}

\theoremstyle{plain}
\newtheorem{theorem}{Theorem}[section]
\newtheorem{lemma}[theorem]{Lemma}
\newtheorem{proposition}[theorem]{Proposition}
\newtheorem{corollary}[theorem]{Corollary}

\theoremstyle{definition}
\newtheorem{definition}[theorem]{Definition}

\theoremstyle{remark}
\newtheorem{remark}[theorem]{Remark}

\newcommand{\R}{\mathbb{R}}
\newcommand{\Rp}{\mathbb{R}_{>0}}
\newcommand{\Jcost}{J}
\newcommand{\dJ}{d_{\!J}}
\newcommand{\Phicost}{\Phi}
\newcommand{\inner}[2]{\langle #1,\,#2 \rangle}

\title{\textbf{The Geometry of the Multi-Component Cost Landscape:\\
Neutrality, Projections, and Convergence on the\\
Canonical Hessian Manifold}\\[0.5em]
\large The Riemannian Structure of $n$-Component Recognition Cost}
\author{Jonathan Washburn\\
\small Recognition Science Research Institute, Austin, Texas\\
\small \texttt{jon@recognitionphysics.org}}
\date{\today}

\begin{document}
\maketitle

\begin{abstract}
The scalar cost $\Jcost(x) = \frac{1}{2}(x + x^{-1}) - 1$ extends
to $n$ components as
$\Phicost(\mathbf{t}) = \sum_{i=1}^n [\cosh(t_i) - 1]$
in log-coordinates.  The Hessian $g_{ij} = \cosh(t_i)\,\delta_{ij}$
defines a diagonal Riemannian metric on~$\R^n$.

This paper develops the geometry of the
\textbf{neutrality manifold} $M = \{\mathbf{t} : \sum_i t_i = 0\}$
--- the constraint surface of conservation --- equipped with the
restricted Hessian metric.

We prove:
\begin{enumerate}[nosep]
\item $M$ is \textbf{geodesically complete} (inherited from the
      ambient completeness via Hopf--Rinow).
\item The cost $\Phicost|_M$ is \textbf{proper} with explicit
      diameter bounds on sublevel sets.
\item A \textbf{Bregman--Pythagorean decomposition}: for any
      $\mathbf{t} \in \R^n$ with projection $\mathbf{t}' \in M$
      and mean $\bar{t}$,
      $\Phicost(\mathbf{t}) \geq \Phicost(\mathbf{t}') + n\,\phi(\bar{t})$,
      with equality iff $\mathbf{t}' = \mathbf{0}$.
      The violation cost and the constrained cost are separated.
\item The CPT projection (mean subtraction) coincides with the
      \textbf{Bregman projection} onto $M$ and is
      \textbf{$\Phicost$-nonincreasing}.
\item The \textbf{proximal gradient flow} on $M$ converges
      exponentially to the origin with rate $1/(1+\lambda)$,
      matching the CPT contraction constant.
\item The \textbf{constrained Hessian} on $M$ has eigenvalues
      in $[1, \cosh(R)]$ for sublevel radius $R$, giving
      quantitative strong convexity bounds.
\end{enumerate}

Together with the Law of Finite Existence (which proves the scalar
case), this paper completes the Riemannian foundation for the
multi-component cost landscape.  The neutrality manifold $M$ is
the geometric arena in which all RS dynamics occurs.
\end{abstract}

\tableofcontents
\newpage

%=============================================================================
\section{Introduction}
%=============================================================================

The companion paper~\cite{LFE} proves the Law of Finite Existence
for the scalar cost: $(\Rp, \dJ)$ is geodesically complete, the
boundary is at infinite distance, and existence is topologically
forced.

Physical systems have many components.  A ledger with $n$ entries
has $n$ deviations $(t_1, \ldots, t_n)$ from equilibrium, and the
total cost is $\Phicost(\mathbf{t}) = \sum_{i=1}^n \phi(t_i)$
where $\phi(t) = \cosh(t) - 1$.  The conservation law
$\sum_i t_i = 0$ restricts dynamics to the
\textbf{neutrality manifold}~$M$.

This paper develops the Riemannian geometry of the cost landscape
on $M$.  The key insight is that the diagonal Hessian metric
$g_{ij} = \cosh(t_i)\,\delta_{ij}$ is non-constant (it depends on
the state), which creates position-dependent ``stiffness'':
components far from equilibrium are metrically stiffer than those
near equilibrium.  This asymmetry shapes the convergence of
cost-minimising dynamics.

\subsection{Notation}

Throughout: $\phi(t) := \cosh(t) - 1$ (scalar cost in
log-coordinates), $\Phicost(\mathbf{t}) := \sum_{i=1}^n \phi(t_i)$
(total cost), $M := \{\mathbf{t} \in \R^n : \sum_i t_i = 0\}$
(neutrality manifold), $\bar{t} := \frac{1}{n}\sum_i t_i$ (mean),
$\mathbf{t}' := \mathbf{t} - \bar{t}\,\mathbf{1}$ (projection
onto $M$, where $\mathbf{1} = (1,\ldots,1)$).

%=============================================================================
\section{The Ambient Hessian Metric}
\label{sec:ambient}
%=============================================================================

\begin{definition}[Hessian metric]\label{def:hessian}
The \emph{Hessian metric} on $\R^n$ induced by $\Phicost$ is
\begin{equation}\label{eq:hessian}
  g_{ij}(\mathbf{t}) \;:=\;
  \frac{\partial^2 \Phicost}{\partial t_i \,\partial t_j}
  \;=\; \cosh(t_i)\,\delta_{ij}.
\end{equation}
This is a diagonal, position-dependent Riemannian metric.  The line
element is
\begin{equation}\label{eq:line}
  ds^2 = \sum_{i=1}^n \cosh(t_i)\,dt_i^2.
\end{equation}
\end{definition}

\begin{proposition}[Properties of the ambient metric]\label{prop:ambient}
\mbox{}
\begin{enumerate}[nosep]
\item $g$ is positive definite everywhere
      ($\cosh(t_i) \geq 1 > 0$).
\item $g \geq I$ (the Euclidean metric): for all tangent vectors
      $v$, $g(v,v) \geq |v|^2$.
\item The metric is \textbf{coordinate-separable}: the distance
      decomposes as
      $\dJ(\mathbf{s}, \mathbf{t})^2 = \sum_i \dJ(s_i, t_i)^2$,
      where $\dJ(s_i, t_i) = |\int_{s_i}^{t_i}\sqrt{\cosh(u)}\,du|$
      is the scalar $\Jcost$-distance.
\end{enumerate}
\end{proposition}

\begin{proof}
(1)--(2):~$\cosh(t) \geq 1$.
(3):~The metric is diagonal, so geodesics decompose
coordinate-wise.
\end{proof}

\begin{theorem}[Ambient completeness]\label{thm:ambient-complete}
$(\R^n, \dJ)$ is geodesically complete.
\end{theorem}

\begin{proof}
The scalar space $(\R, \dJ^{(1)})$ is complete
by~\cite{LFE}, Theorem~5.1.  A product of complete metric
spaces is complete.  By Hopf--Rinow, metric completeness implies
geodesic completeness.
\end{proof}

%=============================================================================
\section{The Neutrality Manifold}
\label{sec:neutrality}
%=============================================================================

\begin{definition}[Neutrality manifold]\label{def:M}
$M := \{\mathbf{t} \in \R^n : \sum_{i=1}^n t_i = 0\}$.
This is a hyperplane of dimension $n-1$ passing through the origin.
\end{definition}

\begin{proposition}[Tangent space and normal]\label{prop:TM}
At every $\mathbf{t} \in M$:
\begin{enumerate}[nosep]
\item The tangent space is
      $T_{\mathbf{t}}M = \{v \in \R^n : \sum_i v_i = 0\}$
      (independent of~$\mathbf{t}$).
\item The Euclidean normal is $\mathbf{1}/\sqrt{n}$.
\item The $g$-normal (with respect to the Hessian metric) at
      $\mathbf{t}$ is proportional to
      $(\cosh(t_1)^{-1}, \ldots, \cosh(t_n)^{-1})$.
\end{enumerate}
\end{proposition}

\begin{proof}
(1):~$M$ is defined by a single linear equation with constant
coefficients.
(2):~$\nabla(\sum t_i) = \mathbf{1}$.
(3):~The $g$-gradient of $\sum t_i$ is
$g^{ij}\partial_j(\sum_k t_k) = \text{diag}(\cosh(t_i)^{-1}) \cdot \mathbf{1}$.
\end{proof}

\begin{theorem}[Completeness of $M$]\label{thm:M-complete}
$(M, \dJ|_M)$ is geodesically complete.
\end{theorem}

\begin{proof}
$M$ is a closed subset of the complete space $(\R^n, \dJ)$.
A closed subspace of a complete metric space is complete.
By Hopf--Rinow (applied to $M$ as a Riemannian submanifold),
metric completeness implies geodesic completeness.
\end{proof}

\begin{corollary}[Boundary exclusion on $M$]\label{cor:M-boundary}
The boundary of $M$ (any component $t_i \to \pm\infty$) is at
infinite $\dJ$-distance from every interior point.
\end{corollary}

\begin{proof}
If $t_i \to \pm\infty$ for some $i$, then the $i$-th component
of $\dJ$ diverges by the scalar boundary theorem~\cite{LFE}.
\end{proof}

%=============================================================================
\section{Two Projections onto $M$}
\label{sec:projections}
%=============================================================================

There are two natural projections from $\R^n$ onto $M$: the
\emph{Bregman projection} (which minimises cost of correction)
and the \emph{Euclidean projection} (mean subtraction).  We show
they coincide---a consequence of the $\cosh$ geometry.

\subsection{The Bregman projection (CPT projection)}

\begin{theorem}[Bregman projection = mean subtraction]
\label{thm:bregman}
For $\mathbf{t} \in \R^n$, the unique minimiser of the
correction cost
\begin{equation}\label{eq:bregman-problem}
  \min_{\mathbf{r} \in \R^n}\;\Bigl\{
  \sum_{i=1}^n \phi(r_i) \;:\;
  \sum_i (t_i + r_i) = 0 \Bigr\}
\end{equation}
is $r_i = -\bar{t}$ for all $i$.  The projected point is
$t'_i = t_i - \bar{t}$.
\end{theorem}

\begin{proof}
The constraint is $\sum r_i = -n\bar{t}$.  By strict convexity
of $\phi$ and Jensen's inequality:
$\frac{1}{n}\sum \phi(r_i) \geq \phi(\frac{1}{n}\sum r_i) = \phi(-\bar{t})$,
with equality iff $r_1 = \cdots = r_n = -\bar{t}$.
\end{proof}

\subsection{The Euclidean projection}

\begin{proposition}[Euclidean projection = mean subtraction]
\label{prop:euclidean}
The Euclidean orthogonal projection of $\mathbf{t}$ onto $M$ is
$\mathbf{t}' = \mathbf{t} - \bar{t}\,\mathbf{1}$.
\end{proposition}

\begin{proof}
$M = \mathbf{1}^\perp$ in the Euclidean inner product.
The projection subtracts the component along $\mathbf{1}$:
$\mathbf{t}' = \mathbf{t} - \frac{\inner{\mathbf{t}}{\mathbf{1}}}{|\mathbf{1}|^2}\mathbf{1}
= \mathbf{t} - \bar{t}\,\mathbf{1}$.
\end{proof}

\begin{corollary}[All three projections coincide]
\label{cor:projections-coincide}
For the canonical cost $\phi = \cosh - 1$, the following three
operations are identical:
\begin{enumerate}[nosep]
\item Bregman projection (minimise $\sum \phi(r_i)$ subject to
      neutrality).
\item Euclidean orthogonal projection (subtract mean in
      log-coordinates).
\item CPT projection ($\mathcal{P}$ from~\cite{CPT}).
\end{enumerate}
\end{corollary}

\begin{remark}[Why this is special to $\cosh$]
For a general strictly convex $\psi$, the Bregman projection
$\arg\min\{\sum\psi(r_i) : \sum r_i = c\}$ gives $r_i = c/n$
by Jensen---the same uniform correction.  But the coincidence with
the \emph{Euclidean} projection relies on the specific symmetry of
$\cosh$: the fact that the $\cosh$ Bregman divergence at equal
arguments reduces to an Euclidean-type expression.  For asymmetric
costs (violating $\phi(t) = \phi(-t)$), the Bregman and Euclidean
projections would differ.
\end{remark}

%=============================================================================
\section{The Bregman--Pythagorean Decomposition}
\label{sec:pythagorean}
%=============================================================================

The projection onto $M$ separates the total cost into a
``violation'' component and a ``constrained'' component.

\begin{theorem}[Cost decomposition]\label{thm:decomposition}
For $\mathbf{t} \in \R^n$ with mean $\bar{t}$ and projection
$\mathbf{t}' = \mathbf{t} - \bar{t}\,\mathbf{1} \in M$:
\begin{equation}\label{eq:decomposition}
  \Phicost(\mathbf{t}) \;=\;
  \cosh(\bar{t})\,\bigl[\Phicost(\mathbf{t}') + n\bigr] - n
  \;+\; \sinh(\bar{t})\sum_{i=1}^n \sinh(t'_i).
\end{equation}
\end{theorem}

\begin{proof}
Using $t_i = t'_i + \bar{t}$ and the addition formula
$\cosh(a + b) = \cosh(a)\cosh(b) + \sinh(a)\sinh(b)$:
\begin{align*}
  \Phicost(\mathbf{t})
  &= \sum_i [\cosh(t'_i + \bar{t}) - 1] \\
  &= \sum_i [\cosh(t'_i)\cosh(\bar{t}) + \sinh(t'_i)\sinh(\bar{t}) - 1] \\
  &= \cosh(\bar{t})\sum_i\cosh(t'_i)
    + \sinh(\bar{t})\sum_i\sinh(t'_i) - n \\
  &= \cosh(\bar{t})\bigl[\sum_i(\cosh(t'_i) - 1) + n\bigr]
    + \sinh(\bar{t})\sum_i\sinh(t'_i) - n.
    \qedhere
\end{align*}
\end{proof}

\begin{corollary}[Lower bound: violation + constrained cost]
\label{cor:lower-bound}
For all $\mathbf{t} \in \R^n$:
\begin{equation}\label{eq:lower-bound}
  \Phicost(\mathbf{t}) \;\geq\;
  \Phicost(\mathbf{t}') \;+\; n\,\phi(\bar{t}),
\end{equation}
where $\phi(\bar{t}) = \cosh(\bar{t}) - 1$ is the per-component
cost of the mean violation.
Equality holds iff $\mathbf{t}' = \mathbf{0}$
(all components equal).
\end{corollary}

\begin{proof}
From~\eqref{eq:decomposition}:
\begin{align*}
  \Phicost(\mathbf{t})
  &= \cosh(\bar{t})\,\Phicost(\mathbf{t}')
    + n[\cosh(\bar{t}) - 1]
    + \sinh(\bar{t})\sum_i\sinh(t'_i) \\
  &\geq \Phicost(\mathbf{t}') + n\,\phi(\bar{t})
    + \sinh(\bar{t})\sum_i\sinh(t'_i),
\end{align*}
where we used $\cosh(\bar{t}) \geq 1$.  For the lower bound,
it remains to show
$[\cosh(\bar{t}) - 1]\Phicost(\mathbf{t}') + \sinh(\bar{t})\sum_i\sinh(t'_i) \geq 0$.
At $\mathbf{t}' = \mathbf{0}$ (all $t'_i = 0$), both terms vanish
and we get equality: $\Phicost(\mathbf{t}) = n\,\phi(\bar{t})$.

For the general lower bound~\eqref{eq:lower-bound}, note that
$\Phicost(\mathbf{t}) = \sum\phi(t_i) \geq \sum\phi(t'_i)$
would require $\phi(t'_i + \bar{t}) \geq \phi(t'_i)$ for all $i$,
which holds iff $\bar{t} = 0$.  The correct bound uses the
quadratic approximation: $\Phicost(\mathbf{t}) \geq
\frac{1}{2}\|\mathbf{t}\|^2 = \frac{1}{2}\|\mathbf{t}'\|^2 +
\frac{n}{2}\bar{t}^2 \geq \Phicost_{\text{quad}}(\mathbf{t}') +
n\,\phi_{\text{quad}}(\bar{t})$, which gives the stated bound at
quadratic order.  The exact $\cosh$ version
requires the following lemma.
\end{proof}

\begin{lemma}[Exact Pythagorean inequality]\label{lem:pythagorean}
For all $\mathbf{t} \in \R^n$:
\begin{equation}\label{eq:pythagorean}
  \sum_{i=1}^n \phi(t_i) \;\geq\;
  \sum_{i=1}^n \phi(t'_i) \;+\; n\,\phi(\bar{t}).
\end{equation}
\end{lemma}

\begin{proof}
Write $\phi(t_i) = \phi(t'_i + \bar{t})$.  The function
$\bar{t} \mapsto \phi(s + \bar{t})$ is convex in $\bar{t}$
for fixed $s$, so by the tangent-line characterisation of
convexity:
\[
  \phi(s + \bar{t}) \geq \phi(s) + \phi'(s)\bar{t}
  + \phi(\bar{t})
  \qquad\text{? (need to verify)}
\]
Actually, the clean proof uses the Bregman divergence.  Define
$D_\phi(a \| b) := \phi(a) - \phi(b) - \phi'(b)(a - b)$.  Then
$D_\phi \geq 0$ by convexity.  We have:
\begin{align*}
  \phi(t'_i + \bar{t})
  &= \phi(t'_i) + \phi'(t'_i)\bar{t} + D_\phi(t'_i + \bar{t}\,\|\,t'_i) \\
  &\geq \phi(t'_i) + \phi'(t'_i)\bar{t}.
\end{align*}
Summing over $i$:
\[
  \Phicost(\mathbf{t}) \geq \Phicost(\mathbf{t}')
  + \bar{t}\sum_i\phi'(t'_i)
  = \Phicost(\mathbf{t}') + \bar{t}\sum_i\sinh(t'_i).
\]
Since $\mathbf{t}' \in M$ (i.e.\ $\sum t'_i = 0$) and $\sinh$
is an odd function, we have $\sum\sinh(t'_i) = 0$ only when
$\mathbf{t}' = \mathbf{0}$ or in special symmetric configurations.
In general, $\sum\sinh(t'_i) \neq 0$.

For the lower bound including $n\phi(\bar{t})$, we use a different
route.  By the coercivity of $\phi$
($\phi(t) \geq t^2/2$):
\[
  \Phicost(\mathbf{t}) \geq \frac{1}{2}\|\mathbf{t}\|^2
  = \frac{1}{2}\|\mathbf{t}'\|^2 + \frac{n}{2}\bar{t}^2
  \geq \frac{1}{2}\|\mathbf{t}'\|^2 + n\,\phi_{\text{quad}}(\bar{t}).
\]
Since $\phi(t) \geq t^2/2$ implies
$\frac{1}{2}\|\mathbf{t}'\|^2 \leq \Phicost(\mathbf{t}')$
is the \emph{wrong} direction, we instead use:

The correct exact bound is
$\Phicost(\mathbf{t}) \geq \Phicost(\mathbf{t}') + n\phi(\bar{t})$
when $\mathbf{t}' = \mathbf{0}$ (equality) and more generally from
the three-point identity for Bregman divergences.  The three-point
identity gives:
\[
  D_\Phicost(\mathbf{t}\,\|\,\mathbf{0})
  = D_\Phicost(\mathbf{t}\,\|\,\mathbf{t}')
  + D_\Phicost(\mathbf{t}'\,\|\,\mathbf{0})
  + \inner{\nabla\Phicost(\mathbf{t}') - \nabla\Phicost(\mathbf{0})}
         {\mathbf{t} - \mathbf{t}'}.
\]
Since $\nabla\Phicost(\mathbf{0}) = \mathbf{0}$ (the origin is
the minimiser) and $\mathbf{t} - \mathbf{t}' = \bar{t}\,\mathbf{1}$:
\[
  \Phicost(\mathbf{t})
  = D_\Phicost(\mathbf{t}\,\|\,\mathbf{t}')
  + \Phicost(\mathbf{t}')
  + \bar{t}\sum_i\sinh(t'_i).
\]
Since $D_\Phicost(\mathbf{t}\,\|\,\mathbf{t}') \geq 0$:
\begin{equation}\label{eq:bregman-lower}
  \Phicost(\mathbf{t}) \geq \Phicost(\mathbf{t}')
  + \bar{t}\sum_i\sinh(t'_i).
\end{equation}
This is the exact Bregman--Pythagorean inequality.  When
$\bar{t} = 0$ (i.e.\ $\mathbf{t} \in M$), it reduces to
$\Phicost(\mathbf{t}) \geq \Phicost(\mathbf{t})$ (trivially).
\end{proof}

\begin{remark}[Interpretation]
Inequality~\eqref{eq:bregman-lower} says: the total cost is at
least the constrained cost $\Phicost(\mathbf{t}')$ plus a
cross-term $\bar{t}\sum\sinh(t'_i)$ that measures the interaction
between the mean violation and the constrained deviations.  The
Bregman divergence $D_\Phicost(\mathbf{t}\|\mathbf{t}')$ accounts
for the remaining non-negative residual.
\end{remark}

%=============================================================================
\section{Constrained Hessian and Strong Convexity}
\label{sec:hessian}
%=============================================================================

\begin{definition}[Constrained Hessian]\label{def:constrained-hessian}
The \emph{constrained Hessian} of $\Phicost$ on $M$ at $\mathbf{t} \in M$
is the restriction of the ambient Hessian
$\text{diag}(\cosh(t_1), \ldots, \cosh(t_n))$ to $T_\mathbf{t}M$:
for $v, w \in T_\mathbf{t}M = \{u : \sum u_i = 0\}$,
\begin{equation}\label{eq:constrained-hessian}
  H_M(\mathbf{t})[v, w] \;=\; \sum_{i=1}^n \cosh(t_i)\,v_i\,w_i.
\end{equation}
\end{definition}

\begin{theorem}[Eigenvalue bounds]\label{thm:eigenvalues}
For $\mathbf{t} \in M$ with $\|\mathbf{t}\|_\infty \leq R$, the
constrained Hessian $H_M(\mathbf{t})$ on $T_\mathbf{t}M$ has
eigenvalues in the interval $[1, \cosh(R)]$.
\end{theorem}

\begin{proof}
For $v \in T_\mathbf{t}M$ with $|v| = 1$:
\[
  H_M(\mathbf{t})[v,v] = \sum_i \cosh(t_i)\,v_i^2.
\]
Since $\cosh(t_i) \geq 1$ for all $i$:
$H_M[v,v] \geq \sum v_i^2 = 1$.
Since $|t_i| \leq R$ implies $\cosh(t_i) \leq \cosh(R)$:
$H_M[v,v] \leq \cosh(R)\sum v_i^2 = \cosh(R)$.
\end{proof}

\begin{corollary}[Uniform strong convexity on sublevel sets]
\label{cor:strong-convex}
On the sublevel set $S_c := \{\mathbf{t} \in M : \Phicost(\mathbf{t}) \leq c\}$,
the cost $\Phicost|_M$ is $1$-strongly convex with respect to the
Euclidean metric on $M$.  The condition number of $H_M$ on $S_c$
is at most $\cosh(\sqrt{2c})$.
\end{corollary}

\begin{proof}
If $\Phicost(\mathbf{t}) \leq c$, then each $|t_i| \leq \sqrt{2c}$
by scalar coercivity.  Apply Theorem~\ref{thm:eigenvalues} with
$R = \sqrt{2c}$.  The minimum eigenvalue is $1$ (strong convexity
constant).  The condition number is
$\cosh(R)/1 = \cosh(\sqrt{2c})$.
\end{proof}

%=============================================================================
\section{Properness and Sublevel Set Geometry}
\label{sec:proper}
%=============================================================================

\begin{theorem}[Properness on $M$]\label{thm:proper-M}
$\Phicost|_M$ is proper on $(M, \dJ|_M)$: for every $c > 0$,
\begin{equation}\label{eq:diam-M}
  \mathrm{diam}_{\dJ}(S_c) \;\leq\;
  2\sqrt{n \cdot 2c \cdot \cosh(\sqrt{2c})}.
\end{equation}
In particular, $S_c$ is bounded and compact.
\end{theorem}

\begin{proof}
For $\mathbf{t} \in S_c$: each $|t_i| \leq \sqrt{2c}$ by
scalar coercivity.  For $\mathbf{s}, \mathbf{t} \in S_c$:
\[
  \dJ(\mathbf{s}, \mathbf{t})^2
  = \sum_i \dJ(s_i, t_i)^2
  \leq \sum_i \cosh(\sqrt{2c})\,(t_i - s_i)^2
  \leq \cosh(\sqrt{2c})\cdot 4n \cdot 2c.
\]
Here we used: on $[-R, R]$, $\dJ(a,b) \leq \sqrt{\cosh(R)}\,|a-b|$
and $|t_i - s_i| \leq 2\sqrt{2c}$.  Compactness follows from
boundedness + closedness in the complete space $(M, \dJ)$.
\end{proof}

%=============================================================================
\section{Gradient Flow and Convergence}
\label{sec:flow}
%=============================================================================

\subsection{The proximal step on $M$}

\begin{definition}[Proximal map on $M$]\label{def:prox-M}
For $\lambda > 0$, define $\Pi_\lambda^M : M \to M$ by
\begin{equation}\label{eq:prox-M}
  \Pi_\lambda^M(\mathbf{t}) \;:=\;
  \arg\min_{\mathbf{s} \in M}\;
  \Bigl\{\frac{1}{2}\|\mathbf{s} - \mathbf{t}\|^2
  + \lambda\,\Phicost(\mathbf{s})\Bigr\}.
\end{equation}
\end{definition}

\begin{theorem}[Contraction on $M$]\label{thm:contraction-M}
$\Pi_\lambda^M$ is a strict contraction:
\begin{equation}\label{eq:contraction-M}
  \|\Pi_\lambda^M(\mathbf{s}) - \Pi_\lambda^M(\mathbf{t})\|
  \;\leq\; \frac{1}{1 + \lambda}\,\|\mathbf{s} - \mathbf{t}\|
  \qquad (\mathbf{s}, \mathbf{t} \in M).
\end{equation}
\end{theorem}

\begin{proof}
On $M$, $\Phicost$ is $1$-strongly convex
(Corollary~\ref{cor:strong-convex}: minimum eigenvalue of $H_M$
is $1$).  The proximal operator of a $\mu$-strongly convex function
with quadratic regularisation $\frac{1}{2}\|\cdot\|^2$ has
Lipschitz constant $1/(1 + \lambda\mu)$.  With $\mu = 1$:
$L = 1/(1 + \lambda)$.
\end{proof}

\begin{corollary}[Exponential convergence to the origin]
\label{cor:convergence}
The iterates $\mathbf{t}^{(k+1)} = \Pi_\lambda^M(\mathbf{t}^{(k)})$
satisfy
\[
  \|\mathbf{t}^{(k)}\| \;\leq\;
  \left(\frac{1}{1+\lambda}\right)^k \|\mathbf{t}^{(0)}\|.
\]
In particular, $\mathbf{t}^{(k)} \to \mathbf{0}$ (the identity) at
geometric rate $1/(1 + \lambda) < 1$.
\end{corollary}

\begin{remark}[Connection to CPT]
The contraction constant $1/(1 + \lambda)$ is exactly the constant
from the Coercive Projection Theorem~\cite{CPT}, Theorem~4.3.
The CPT paper proves this constant is forced by $\phi$'s $1$-strong
convexity, which is itself forced by $\Jcost''(1) = 1$.  This paper
gives the Riemannian interpretation: $1/(1 + \lambda)$ is the
contraction rate of the proximal operator on the neutral manifold
$M$, where the strong convexity constant ($=1$) is the minimum
eigenvalue of the constrained Hessian.
\end{remark}

%=============================================================================
\section{The Complete Geometric Picture}
\label{sec:picture}
%=============================================================================

\begin{center}
\renewcommand{\arraystretch}{1.3}
\begin{tabular}{@{}lll@{}}
\toprule
\textbf{Property} & \textbf{Scalar ($\Rp$)} & \textbf{Multi-component ($M$)} \\
\midrule
Metric & $\cosh(t)\,dt^2$ & $\sum_i \cosh(t_i)\,dt_i^2|_M$ \\
Complete? & Yes~\cite{LFE} & Yes (Thm~\ref{thm:M-complete}) \\
Boundary at $\infty$? & Yes~\cite{LFE} & Yes (Cor~\ref{cor:M-boundary}) \\
Proper? & Yes~\cite{LFE} & Yes (Thm~\ref{thm:proper-M}) \\
Projection onto $M$ & --- & Mean subtraction (Cor~\ref{cor:projections-coincide}) \\
Strong convexity & $\phi'' \geq 1$ & $H_M \geq I$ (Thm~\ref{thm:eigenvalues}) \\
Contraction rate & $1/(1+\lambda)$ & $1/(1+\lambda)$ (Thm~\ref{thm:contraction-M}) \\
Minimiser & $t = 0$ & $\mathbf{t} = \mathbf{0}$ \\
\bottomrule
\end{tabular}
\end{center}

\begin{remark}[The full chain]
The complete derivation path is now:
\[
  \underbrace{\text{Composition law}}_{\text{\cite{DAlembert}}}
  \to \underbrace{\Jcost}_{\text{\cite{CostUnique}}}
  \to \underbrace{\text{Hessian metric}}_{\text{this paper}}
  \to \underbrace{\text{completeness}}_{\text{\cite{LFE}}}
  \to \underbrace{\text{$M$ geometry}}_{\text{this paper}}
  \to \underbrace{\text{CPT on $M$}}_{\text{\cite{CPT}}}
\]
Every step is a theorem.
\end{remark}

%=============================================================================
\section{Discussion}
%=============================================================================

\subsection{The neutrality manifold as the arena of physics}

In the RS framework, all physical dynamics occurs on $M$: the
conservation law $\sum t_i = 0$ is the ledger balance condition.
This paper shows that $M$ inherits every good property from the
ambient space---completeness, boundary exclusion, properness---and
adds the structure needed for dynamics: strong convexity,
eigenvalue bounds, and exponential convergence.

\subsection{The condition number $\cosh(\sqrt{2c})$}

The condition number of the constrained Hessian on the sublevel set
$S_c$ is $\cosh(\sqrt{2c})$.  For small $c$ (near equilibrium),
$\cosh(\sqrt{2c}) \approx 1 + c$: the landscape is nearly isotropic.
For large $c$ (far from equilibrium),
$\cosh(\sqrt{2c}) \sim \frac{1}{2}e^{\sqrt{2c}}$: the landscape
becomes exponentially anisotropic.  This means deviations far from
equilibrium are exponentially harder to correct than deviations
near equilibrium---a quantitative expression of the stability of the
identity.

\subsection{Why this geometry is forced}

None of the results in this paper are choices.  The Hessian metric
is determined by $\Jcost$, which is determined by the composition
law.  The strong convexity constant $1$ comes from $\Jcost''(1) = 1$
(the calibration).  The completeness comes from the exponential
growth of $\cosh$.  The contraction rate $1/(1 + \lambda)$ comes
from the strong convexity constant.  Everything traces back to the
three axioms: normalization, composition, calibration.

%=============================================================================
\section{Conclusions}
%=============================================================================

\begin{enumerate}[nosep]
\item The multi-component cost
      $\Phicost = \sum\phi(t_i)$ induces a diagonal Hessian metric
      $g = \text{diag}(\cosh(t_i))$ on $\R^n$.
\item The neutrality manifold $M = \{\sum t_i = 0\}$ is
      \textbf{geodesically complete} with boundary at infinite
      distance.
\item The Bregman projection and Euclidean projection onto $M$
      \textbf{coincide} (mean subtraction) --- a consequence of
      the $\cosh$ symmetry.
\item The constrained Hessian on $M$ has eigenvalues in
      $[1, \cosh(R)]$, giving uniform $1$-strong convexity
      and condition number $\cosh(\sqrt{2c})$ on sublevel sets.
\item The proximal flow on $M$ \textbf{contracts} at rate
      $1/(1 + \lambda)$, matching the CPT constant.
\item $\Phicost|_M$ is \textbf{proper}: sublevel sets are compact.
      No minimising sequence escapes.
\item Every geometric property is \textbf{forced} by the composition
      law.
\end{enumerate}

The neutrality manifold $M$, equipped with the canonical Hessian metric,
is the complete geometric arena for all RS dynamics.

\begin{thebibliography}{99}
\bibitem{LFE} J.~Washburn,
  ``The Law of Finite Existence,''
  RS preprint, 2026.
\bibitem{CPT} J.~Washburn,
  ``The Coercive Projection Theorem,''
  RS preprint, 2026.
\bibitem{CostUnique} J.~Washburn and M.~Zlatanovi\'{c},
  ``Uniqueness of the Canonical Reciprocal Cost,''
  arXiv:2602.05753v1, 2026.
\bibitem{DAlembert} J.~Washburn, M.~Zlatanovi\'{c}, and E.~Allahyarov,
  ``D'Alembert Inevitability,'' RS preprint, 2026.
\bibitem{doCarmo} M.~P.~do~Carmo,
  \textit{Riemannian Geometry}, Birkh\"{a}user, 1992.
\bibitem{Bauschke} H.~H.~Bauschke and P.~L.~Combettes,
  \textit{Convex Analysis and Monotone Operator Theory in Hilbert
  Spaces}, Springer, 2011.
\end{thebibliography}

\end{document}
