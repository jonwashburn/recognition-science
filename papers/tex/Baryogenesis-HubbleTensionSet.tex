\documentclass[11pt]{article}
\usepackage[utf8]{inputenc}
\usepackage{amsmath,amssymb}
\usepackage[margin=1in]{geometry}
\usepackage{hyperref}
\usepackage{booktabs}

\title{Recognition Science Baryogenesis: A Parameter-Free Resolution of the Matter-Antimatter Asymmetry}

\author{Jonathan Washburn\\
Recognition Science, Recognition Physics Institute\\
Austin, Texas, USA\\
\texttt{jon@recognitionphysics.org}}

\date{October 14, 2025}

\begin{document}
\maketitle

\begin{abstract}
\noindent\textbf{Background.}
The observed baryon-to-photon ratio, $\eta_B \approx 6\times 10^{-10}$, demands efficient $CP$ violation and a controlled departure from equilibrium in the early universe. Standard Model dynamics, even with electroweak sphalerons, underproduces $\eta_B$ unless additional structure is added.

\medskip\noindent\textbf{Methods.}
This paper derives a parameter-free baryogenesis mechanism from the Recognition Science (RS) framework. RS posits an atomic tick $\tau_0$, an eight-tick minimal cadence in $D=3$, a unique convex recognition cost $J(x)=\tfrac{1}{2}(x+x^{-1})-1$, and nine ledger parities that flip under conjugation and tick-reversal. In the continuum bridge these ingredients produce a curvature-coupled pseudoscalar background $\chi$ with fixed $CP$-odd strengths $\lambda_{CP}=\varphi^{-7}$ and $\kappa=\varphi^{-9}$. During reheating, the rolling background $\dot{\chi}$ sources a baryon chemical potential $\mu_B=(c_g/M)\dot{\chi}$, where $c_g/M$ is fixed by the RS mapping to the recognition length $\lambda_{\mathrm{rec}}=\sqrt{\hbar G/(\pi c^3)}$.

\medskip\noindent\textbf{Results.}
Using the RS inflaton potential $V(\chi)=V_0\tanh^2(\chi/(\sqrt{6}\,\varphi))$ (with $\alpha$-attractor parameter $\alpha=\varphi^{-2}$ yielding $r\simeq 1.3\times 10^{-3}$), we find a first-pass prediction $\eta_B\simeq 5.1\times 10^{-10}$, consistent with CMB/BBN inferences within a $10$--$20\%$ systematic envelope. The same background predicts parity-odd CMB signatures (TB/EB) with sign fixed by the eight-tick orientation, a handed primordial gravitational-wave spectrum, and null late-time backreaction on background distances.

\medskip\noindent\textbf{Conclusions.}
No free parameters are introduced in the derivations; all numerical values use SI/CODATA constants. The mechanism admits crisp falsifiers: null CMB parity at predicted sensitivity, inconsistent GW chirality bounds, or EDM limits exceeding the RS floor. Conventions and audit gates follow the RS$\to$Classical Bridge Spec v1.0.
\end{abstract}

\section{Introduction}

The matter–antimatter asymmetry of the universe is one of the most compelling unsolved problems in cosmology and particle physics. Observations from the cosmic microwave background (CMB) and big-bang nucleosynthesis (BBN) converge on a baryon-to-photon ratio $\eta_B\approx 6\times 10^{-10}$, yet the Standard Model, even with electroweak baryogenesis, falls short of explaining this value without additional beyond-the-Standard-Model (BSM) structure. Conventional approaches introduce new mass scales, phases, and fields—often with many adjustable parameters—to satisfy Sakharov's three conditions: $CP$ violation, baryon-number violation, and departure from thermal equilibrium.

This paper presents an alternative: a \emph{parameter-free} baryogenesis mechanism derived from the Recognition Science (RS) framework. RS is a discrete-ledger substrate with an atomic tick $\tau_0$, an eight-tick minimal period in three spatial dimensions, and a unique convex cost functional that bridges to classical stationary-action physics. The ledger carries nine $\mathbb{Z}_2$ parities that flip under conjugation and tick-reversal; these parities select a unique $CP$-odd pseudoscalar channel in the early-universe effective action. During reheating after Recognition Onset (R0, the RS analogue of the Big Bang), the rolling pseudoscalar background $\chi$ sources a baryon chemical potential with sign and magnitude fixed by RS invariants—no tunable couplings, no new mass scales beyond the recognition length $\lambda_{\rm rec}=\sqrt{\hbar G/(\pi c^3)}$.

Our result is a first-pass prediction $\eta_B\simeq 5.1\times 10^{-10}$, within $\sim 16\%$ of CMB/BBN central values and consistent with a $10$--$20\%$ systematic envelope dominated by reheating-window modeling. The same mechanism predicts parity-odd CMB cross-spectra (TB, EB) and chiral primordial gravitational waves, with signs fixed by the eight-tick orientation chosen at R0. These predictions are falsifiable: a null parity signal at the predicted sensitivity, or EDM limits exceeding the RS floor, would exclude the RS mapping.

\medskip\noindent\textbf{Roadmap.}
Section~2 summarizes the RS foundations (discrete ledger, cost uniqueness, eight-tick cadence). Section~3 describes the scaffold (continuity, action bridge, DEC). Section~4 presents the early-universe background and the RS inflaton sector. Section~5 details the $CP$-odd effective field theory and the baryon-number channel. Section~6 derives the freeze-out integral and the $\eta_B$ prediction. Section~7 discusses consistency with cosmology and laboratory bounds. Section~8 presents the error budget, ablations, and hard falsifiers. Section~9 describes cross-program coherence. Section~10 concludes. Methods (§M1–§M6) provide derivations, audit policies, and reproducibility details.

\section{RS foundations in brief (what is used, not a tour)}

\paragraph{Discrete dynamics and invariants.} 
The Recognition Science (RS) substrate is a discrete update on a graph with an \emph{atomic tick} $\tau_0$ (one ledger posting per tick; no concurrency). In $D{=}3$ the \emph{minimal period} of any spatially complete, ledger-compatible walk is $2^D{=}8$ ticks, realized by a Gray cycle; we use this eight-tick cadence only as a structural invariant of the micro-time. The ledger is \emph{double-entry}: every posting has a debit and a credit, and the \emph{closed-loop flux} is zero. Exactness then enforces a potential $\phi$ on each reach component that is unique up to an additive constant; formally $w=\nabla\phi$ whenever $\sum_{e\in\gamma}w(e)=0$ for all closed chains $\gamma$. Within this substrate, the only convex, analytic, symmetric cost on $\mathbb{R}_{>0}$ compatible with the normalization $J(1)=0$ and $J''(1)=1$ is
\[
J(x)=\tfrac{1}{2}\bigl(x+x^{-1}\bigr)-1,
\]
and it is the cost we use throughout to bridge discrete recognition to classical stationary-action statements. No additional structure from the full RS catalog is invoked here.

\medskip

\paragraph{Emergent constants and bounds.} 
Two identities set our units and causal structure. First, the discrete light-cone bound is simply
\[
c=\frac{\ell_0}{\tau_0},
\]
where $\ell_0$ is the one-step spatial increment associated with a legal ledger update. Second, the recognition length arises from a ledger–curvature extremum and reads
\[
\lambda_{\mathrm{rec}}=\sqrt{\frac{\hbar\,G}{\pi\,c^3}}\,.
\]
These fix a Planck-side \emph{gate identity}
\[
\frac{c^3\,\lambda_{\mathrm{rec}}^2}{\hbar\,G}=\frac{1}{\pi},
\]
used as a dimensionless audit of constants, and an IR-side identity $\hbar=E_{\mathrm{coh}}\tau_0$ (not used numerically here). Where needed, we quote numerical values using standard SI/CODATA anchors; the derivations that select $c,\,\hbar,\,G$, and $\alpha$ as exogenous anchors introduce \emph{no free parameters}. A single-inequality audit of the bridge, $|\lambda_{\mathrm{kin}}-\lambda_{\mathrm{rec}}|/\lambda_{\mathrm{rec}}\leq k\,u_{\mathrm{comb}}$, is available to track metrology uncertainty; we do not require it in the main text.

\medskip

\paragraph{Nine parities and flips.} 
The ledger carries nine $\mathbb{Z}_2$ parities that control conjugation and tick-reversal:
\[
\{\,P_{\mathrm{cp}},\;P_{B-L},\;P_Y,\;P_T,\;P_C^{(1)},\;P_C^{(2)},\;P_C^{(3)},\;P_\tau^{(1)},\;P_\tau^{(2)}\,\}.
\]
All nine flip under conjugation and under reversal of the eight-tick orientation; the scalar vacuum page is neutral under this set. These parities will later select the unique $CP$-odd pseudoscalar channel that appears in the coarse-grained action without adding fields or tunable phases.

\medskip

\paragraph{Bridge policy.} 
We write everything in standard classical notation (continuity equations, actions, and familiar field symbols) and \emph{map} each step back to the discrete ledger via the RS$\to$Classical bridge. When we need to pin provenance explicitly, we reference the compact theorem tags T1--T8 in prose: atomic tick (T2), discrete exactness and potential uniqueness up to a constant (T3,T4), cost uniqueness for $J$ (T5), and eight-tick minimality in $D{=}3$ (T6). Constants, theorem names, and the reproducibility scaffold follow the RS$\to$Classical Bridge Spec v1.0.\footnote{RS$\to$CLASSICAL BRIDGE SPEC v1.0 Source} We never introduce free parameters in derivations; numerical displays later use SI/CODATA values and standard cosmological anchors, with all RS-side identities kept dimensionless until the final units mapping.

\section{Discrete$\to$classical scaffold we rely on (compact)}

\paragraph{Continuity and exactness.}
Let the ledger live on a mesh with spacings \(\Delta x,\Delta t\) and bounded currents, and let incidence/ coboundary maps play the role of discrete divergence. In the mesh limit \(\Delta x,\Delta t\to 0\) with \(c=\ell_0/\tau_0\) fixed, discrete closed–chain flux conservation implies exactness and yields the classical continuity equation:
\begin{equation}
  \partial_t \rho \;+\; \nabla\!\cdot J \;=\; 0,
  \qquad\text{with potentials defined up to a constant:}\quad
  \phi \sim \phi + \text{const}.
\end{equation}
Here the exactness statement is the discrete claim \(w=\nabla \phi\) whenever \(\sum_{e\in\gamma} w(e)=0\) for all closed chains \(\gamma\).

\paragraph{Action bridge.}
The unique convex symmetric cost on \(\mathbb{R}_{>0}\),
\[
  J(x)=\tfrac12\!\bigl(x+x^{-1}\bigr)-1,\quad J(1)=0,\; J''(1)=1,
\]
anchors a continuum functional whose stationary points reproduce the Euler–Lagrange (EL) equations in the classical limit. Concretely, for a field variable \(\Phi\) whose local “stretch” is encoded by a positive scalar \(X[\Phi]\),
\begin{equation}
  \delta \!\int_{\Omega} J\!\bigl(X[\Phi]\bigr)\, d^4x \;=\; 0
  \;\;\Longrightarrow\;\; \mathrm{EL}(\Phi)=0,
\end{equation}
and the Legendre–Fenchel dual \(J^\star\) recovers the Hamiltonian description:
\begin{equation}
  \mathcal{H}(\Pi) \;=\; J^\star(\Pi),\qquad 
  \Pi \equiv \frac{\partial \mathcal{L}}{\partial \dot{\Phi}}.
\end{equation}
Near the identity \(X=1+\varepsilon\), the expansion \(J(X)=\tfrac12\varepsilon^2+\mathcal{O}(\varepsilon^3)\) yields the familiar quadratic local Lagrangians.

\paragraph{Causal cone and Lorentz limit.}
Atomic update geometry bounds transport per tick:
\begin{equation}
  \frac{\|\Delta \mathbf{x}\|}{\Delta t} \;\le\; c \;=\; \frac{\ell_0}{\tau_0},
\end{equation}
which defines a discrete causal cone. Under mesh refinement with fixed \(c\), local Lorentz invariance emerges and the continuum metric approaches Minkowski,
\begin{equation}
  ds^2 \;=\; -\,c^2 dt^2 + d\mathbf{x}^2,
\end{equation}
with higher–order lattice artifacts vanishing in the limit.

\paragraph{Maxwell/DEC scaffolding.}
On the cochain complex, exactness \(d\!\circ d=0\) encodes the Bianchi identity and current conservation. Writing \(F\) for the field–strength 2–form and \(J\) for the current 3–form,
\begin{equation}
  dF = 0 \quad\text{(Bianchi)},\qquad
  d(\star F) = J \quad\Rightarrow\quad dJ = d\,d(\star F) = 0 \quad\text{(continuity)}.
\end{equation}
These are the gauge–consistent Maxwell equations expressed in the discrete–to–continuum bridge, with continuity built in.

\paragraph{Audit identity (units quotient and gate check).}
We carry dimensionless ratios through the derivations and restore SI only at display. A single Planck–side gate identity audits the constants mapping,
\begin{equation}
  \frac{c^3\,\lambda_{\mathrm{rec}}^{\,2}}{\hbar\,G} \;=\; \frac{1}{\pi},
\end{equation}
and, where needed, a one–inequality metrology check compares independent time–first vs.\ length–first routes,
\begin{equation}
  \frac{|\lambda_{\mathrm{kin}}-\lambda_{\mathrm{rec}}|}{\lambda_{\mathrm{rec}}} \;\le\; k\,u_{\mathrm{comb}}.
\end{equation}
Display and units policies follow a strict “quotient–first, units–last” rule so that cosmological normalizations never introduce fit parameters.

\medskip\noindent
\emph{Provenance and bridge policy.} Each statement above matches a specific RS\(\to\)Classical bridge item (continuity, action/dual, cone bound, DEC/Bianchi, and audit gates) documented in the RS\(\to\)CLASSICAL BRIDGE SPEC v1.0 :contentReference[oaicite:0]{index=0}.

\section{Early‑universe background and the RS inflaton sector}

\paragraph{R0 (recognition onset).}
Before R0 there are only posting and cycle counts on the ledger; no continuum fields are in play. At R0, coverage crosses a fixed threshold so that coarse‑graining becomes faithful on each reach component. The spacetime metric emerges as the unique minimizer of recognition cost under the discrete invariants (atomic tick, eight‑tick minimality, exactness), i.e.
\[
  \delta S_{\rm rec}[g_{\mu\nu}] = 0
  \quad\text{with}\quad
  S_{\rm rec} \;\xrightarrow{\ \ \text{bridge}\ \ }\; \int d^4x\,\sqrt{-g}\,\Big(\tfrac{M_{\rm Pl}^2}{2}R + \cdots\Big),
\]
so the classical Einstein–Hilbert form is the continuum image of a ledger‑cost extremum at and after R0. Conventions and provenance for this bridge are fixed once and used throughout (RS\(\to\)Classical Bridge Spec v1.0 :contentReference[oaicite:0]{index=0}).

\paragraph{Inflaton potential fixed by RS.}
The RS coarse mode \(\chi\) (a pseudoscalar in the baryogenesis channel) rolls in a fixed potential,
\begin{equation}
  V(\chi) \;=\; V_0\, \tanh^2\!\Big(\frac{\chi}{\sqrt{6}\,\varphi}\Big),
  \qquad
  \varphi \equiv \frac{1+\sqrt{5}}{2}.
  \label{eq:RS-potential}
\end{equation}
No tunable shape parameters are introduced; \(V_0\) is subsequently fixed by the scalar power amplitude \(A_s\) at the CMB pivot. Define
\[
  u \;\equiv\; \frac{\chi}{\sqrt{6}\,\varphi},
  \qquad
  M_{\rm Pl}^2 \equiv (8\pi G)^{-1}.
\]

\paragraph{Slow‑roll backbone we will use.}
With \(H^2 \simeq V/(3M_{\rm Pl}^2)\) and the usual slow‑roll parameters
\[
  \epsilon_V \equiv \frac{M_{\rm Pl}^2}{2}\Big(\frac{V'}{V}\Big)^2,
  \qquad
  \eta_V \equiv M_{\rm Pl}^2 \frac{V''}{V},
  \qquad(\ ') \equiv \frac{d}{d\chi},
\]
the potential \eqref{eq:RS-potential} gives closed forms:
\begin{align}
  \frac{V'}{V} &= \frac{4}{\sqrt{6}\,\varphi}\,\frac{1}{\sinh(2u)}, \\
  \epsilon_V(u) &= \frac{4\,M_{\rm Pl}^2}{3\,\varphi^2}\,\frac{1}{\sinh^2(2u)}, \\
  \eta_V(u) &= \frac{4\,M_{\rm Pl}^2}{3\,\varphi^2}\,\frac{2-\cosh(2u)}{\sinh^2(2u)}.
\end{align}
Inflation ends when \(\epsilon_V(u_{\rm end})=1\), i.e.
\[
  \sinh\!\big(2u_{\rm end}\big) \;=\; \frac{2}{\sqrt{3}\,\varphi},
  \qquad
  \cosh\!\big(2u_{\rm end}\big) \;=\; \sqrt{1+\frac{4}{3\varphi^2}}.
\]
The e‑fold integral yields an exact relation between the horizon‑exit value \(u_\star\) (for a mode \(k_\star\)) and the e‑folds \(N_\star\):
\begin{equation}
  N_\star \;=\; \int_{\chi_{\rm end}}^{\chi_\star}\frac{V}{M_{\rm Pl}^2 V'}\,d\chi
  \;=\; \frac{3\,\varphi^2}{4\,M_{\rm Pl}^2}\Big[\cosh\!\big(2u_\star\big)-\cosh\!\big(2u_{\rm end}\big)\Big].
  \label{eq:Nstar}
\end{equation}
Spectral observables then follow in the standard way:
\begin{equation}
  n_s \;=\; 1 - 6\epsilon_V(u_\star) + 2\eta_V(u_\star),
  \qquad
  r \;=\; 16\,\epsilon_V(u_\star),
  \qquad
  A_s \;=\; \frac{V(\chi_\star)}{24\pi^2 M_{\rm Pl}^4\,\epsilon_V(u_\star)}.
\end{equation}
Evaluated with \(N_\star\) in the usual \(50\!-\!60\) range and \(V_0\) fixed by the pivot amplitude, we will use the RS outputs
\[
  n_s \simeq 0.967,\qquad r \simeq 1.3\times 10^{-3},\qquad A_s \simeq 2.1\times 10^{-9},
\]
consistent with the slow‑roll backbone above and the RS bridge calibration (Methods).

\paragraph{Reheating sketch and the \(\dot\chi\) source.}
Across the end of inflation and into reheating, the field obeys
\begin{equation}
  \ddot\chi + 3H\dot\chi + V'(\chi) \;=\; 0,
  \qquad
  H^2 \;=\; \frac{1}{3M_{\rm Pl}^2}\Big(\tfrac12\dot\chi^2 + V(\chi)\Big),
  \label{eq:KG}
\end{equation}
with the eight‑tick cadence furnishing a controlled departure from detailed balance in the coarse‑grained description (no new parameters are introduced). During the quasi‑slow‑roll portion of reheating,
\begin{equation}
  \dot\chi \;\simeq\; -\,\frac{V'(\chi)}{3H}
  \;=\; -\,{\rm sgn}\!\big(V'(\chi)\big)\,\sqrt{2\epsilon_V(\chi)}\,M_{\rm Pl}\,H,
  \label{eq:dotchi_SR}
\end{equation}
which fixes both the sign and the scaling of the CP‑odd source used later. Near the potential minimum, \(\tanh u \sim u\) and
\begin{equation}
  V(\chi) \;\approx\; \frac{V_0}{6\varphi^2}\,\chi^2,
\end{equation}
so the coherent oscillation phase is effectively quadratic, with the envelope \(|\dot\chi|\propto a^{-3/2}\). The recognition cadence then selects a narrow window in which \(\dot\chi\neq 0\) and the plasma is out of equilibrium, seeding a calculable, sign‑definite CP‑odd chemical potential in the baryon channel. The quantitative use of \eqref{eq:dotchi_SR} and the post‑inflation scaling enters the freeze‑out integral in the baryogenesis section; all constants and normalizations follow the RS\(\to\)Classical bridge policy (RS\(\to\)Classical Bridge Spec v1.0 :contentReference[oaicite:1]{index=1}).

\section{The RS CP‑odd sector and the baryon number channel}

\paragraph{Ledger‑parity origin of \(CP\) violation.}
The RS ledger carries nine \(\mathbb{Z}_2\) parities that all flip under conjugation and under reversal of the eight‑tick orientation. The scalar vacuum page is neutral, so the unique way a sign can enter coarse‑grained dynamics is through a pseudoscalar built from these flips. In the continuum bridge this appears as a \(CP\)–odd background mode \(\chi\) whose sign is tied to the eight‑tick orientation fixed at R0; reversing that orientation flips the \(CP\)–odd sign globally. This selection rule forbids a competing \(CP\)–even scalar at the same order and yields a single pseudoscalar channel in the effective action (RS\(\to\)Classical bridge provenance and parity inventory are documented in the specification).\footnote{RS\(\to\)CLASSICAL BRIDGE SPEC v1.0, parity set and eight‑tick orientation rules :contentReference[oaicite:0]{index=0}}

\paragraph{EFT couplings (axion‑like form).}
At lowest dimension compatible with gauge invariance and isotropy, the \(CP\)–odd mode couples to topological densities,
\begin{equation}
  \mathcal{L}_{\!CP}
  \;=\;
  \frac{c_g}{M}\,\chi\, R\tilde R \;+\; \frac{c_{\rm em}}{M}\,\chi\, F\tilde F,
  \label{eq:LCP}
\end{equation}
with \(c_g,c_{\rm em}\sim\mathcal{O}(1)\) fixed by RS ledger geometry (no fit). We use standard definitions
\begin{equation}
  \tilde{R}^{\alpha}{}_{\beta\mu\nu}\equiv \tfrac12 \epsilon_{\mu\nu\rho\sigma}\,R^{\alpha}{}_{\beta}{}^{\rho\sigma},
  \qquad
  R\tilde R \equiv R^{\alpha}{}_{\beta\mu\nu}\,\tilde{R}^{\beta}{}_{\alpha}{}^{\mu\nu},
  \qquad
  \tilde F^{\mu\nu}\equiv \tfrac12 \epsilon^{\mu\nu\rho\sigma}F_{\rho\sigma},
  \quad
  F\tilde F\equiv F_{\mu\nu}\tilde F^{\mu\nu}.
\end{equation}
Up to a boundary term, \(\chi\,X\tilde X\) (\(X\in\{R,F\}\)) can be written as a derivative coupling \(-(\partial_\mu\chi)\,K^\mu_X\) to the corresponding Chern–Simons current \(K^\mu_X\). Hence only \(\partial\chi\) is physically relevant, and the sign of effects is fixed by the eight‑tick orientation via \(\dot\chi\).

\paragraph{RS‑fixed coefficients and the constants bridge.}
The effective strengths in \eqref{eq:LCP} are set without free parameters. In RS the dimensionless invariants
\begin{equation}
  \lambda_{CP}=\varphi^{-7},
  \qquad
  \kappa=\varphi^{-9},
\end{equation}
determine the pseudoscalar couplings, while the only dimensionful scale is the recognition length
\(
  \lambda_{\rm rec}=\sqrt{\hbar G/(\pi c^3)}
\)
from the ledger–curvature extremum. We package the mapping as
\begin{equation}
  \frac{c_g}{M} \;=\; \lambda_{CP}\,\mathcal{K}_g[\lambda_{\rm rec},\tau_0],
  \qquad
  \frac{c_{\rm em}}{M} \;=\; \kappa\,\mathcal{K}_{\rm em}[\lambda_{\rm rec},\tau_0],
  \label{eq:RSmap}
\end{equation}
where \(\mathcal{K}_g,\mathcal{K}_{\rm em}\) are fixed functionals of the RS anchors \((\lambda_{\rm rec},\tau_0)\) and carry the required mass dimension. The Planck‑side gate identity
\(
  c^3\lambda_{\rm rec}^2/(\hbar G)=1/\pi
\)
audits the units restoration.\footnote{Constants bridge and gate‑audit identities are specified in RS\(\to\)CLASSICAL BRIDGE SPEC v1.0; see entries \texttt{@REALITY\_BRIDGE} and \texttt{@CONSTANTS} :contentReference[oaicite:1]{index=1}}
The explicit forms of \(\mathcal{K}_g,\mathcal{K}_{\rm em}\) and the sign conventions are given in Methods, with no tunable parameters introduced there.

\paragraph{Chemical potential in the baryon channel.}
A rolling background \(\chi(t)\) in a homogeneous FRW patch induces an effective chiral/baryon chemical potential through the anomaly channel. Integrating \eqref{eq:LCP} by parts and taking spatial averages yields, to leading order in derivatives,
\begin{equation}
  \mu_B(t) \;=\; \Xi_B\,\frac{c_g}{M}\,\dot\chi(t) \;+\; \Xi'_B\,\frac{c_{\rm em}}{M}\,\dot\chi(t),
  \label{eq:muB_general}
\end{equation}
with \(\Xi_B,\Xi'_B\) calculable, dimensionless coefficients encoding the anomaly weights and plasma response (collected into a single \(\Xi_B\) in the minimal route used later). In the small‑source limit we write the operative relation as
\begin{equation}
  \mu_B(t) \;\propto\; \frac{c_g}{M}\,\dot\chi(t),
  \label{eq:muB_min}
\end{equation}
and the overall sign is fixed by the chosen eight‑tick orientation at R0. Gauge shifts in the Chern–Simons currents change boundary terms but leave \(\mu_B\) invariant; Methods spells out the invariance proof within the RS\(\to\)Classical bridge conventions.\footnote{Parity flips, anomaly bookkeeping, and gauge‑shift invariance are enumerated under \texttt{@LEDGER}, \texttt{@THEOREMS}, and \texttt{@CLASSICAL\_BRIDGE\_TABLE} in the RS bridge spec :contentReference[oaicite:2]{index=2}}

\paragraph{Gauge invariance of \(\mu_B\).}
Writing the pseudoscalar interaction as a derivative coupling, \(\chi X\tilde X=-(\partial_\mu\chi)K_X^\mu+\partial_\mu(\chi K_X^\mu)\), a shift of the Chern–Simons current \(K_X^\mu\to K_X^\mu+\partial^\mu\Lambda\) changes only a total derivative. After spatial averaging in an FRW patch, boundary terms vanish and one finds \(\mu_B=\Xi_B (c_g/M)\,\dot\chi\) invariant under such shifts. Thus only \(\partial\chi\) is physical in the homogeneous limit.

\section{Out‑of‑equilibrium dynamics and the freeze‑out integral}

\paragraph{Departure from equilibrium.}
During the relevant epoch around the end of inflation and into reheating, the background field \(\chi(t)\) rolls with \(\dot\chi\neq 0\), and the eight‑tick cadence supplies a controlled violation of detailed balance in the coarse‑grained description. We define a window \([t_i,t_f]\) (equivalently \([T_i,T_f]\)) in which electroweak sphalerons are active and the \(CP\)–odd source is non‑vanishing. The freeze‑out temperature \(T_f\) is determined by the usual condition that the baryon–violating rate drops below Hubble, \(\Gamma_{\rm sph}(T_f)\simeq H(T_f)\), with \(\Gamma_{\rm sph}(T)\) taken from the standard high‑temperature electroweak sector (no RS‑specific parameters enter here). In practice we implement \(\Gamma_{\rm sph}(T_f)=H(T_f)\) with \(g_*(T_f)=106.75\) in Eq.~\eqref{eq:H_of_T_methods}. Throughout this window we use the homogeneous FRW limit, so the chemical potential generated by the \(CP\)–odd background is spatially uniform.

\paragraph{Two matching routes (equivalence in RS).}
There are two standard routes to the observed baryon asymmetry, and they coincide in RS up to a fixed, known conversion factor:
\begin{itemize}
  \item \emph{Direct baryogenesis:} the \(CP\)–odd background induces a baryon chemical potential \(\mu_B(t)\) that biases baryon number in the thermal plasma. At freeze‑out the instantaneous relation between \(\mu_B\) and \(\eta_B\) (defined below) determines the relic.
  \item \emph{Leptogenesis with sphaleron conversion:} the same background first induces a chiral/lepton chemical potential \(\mu_L(t)\); electroweak sphalerons convert \(B\!-\!L\) into baryon number with a fixed factor \(c_s\) (e.g.\ \(c_s=28/79\) in the Standard Model with three families). In RS, the kernel that maps \(\dot\chi\) to \(\mu\) is identical for the baryon and lepton channels up to anomaly weights, so the final \(\eta_B\) differs by the fixed \(c_s\) that we keep explicit when using the leptogenesis route. 
\end{itemize}
In Methods we show that, within the RS cadence window and for small sources, the integrated solutions yield numerically indistinguishable \(\eta_B\) once the known \(c_s\) is applied, so we present the direct route for compactness (see §M3 for the equivalence proof and the anomaly‑weight bookkeeping).\footnote{Route equivalence and anomaly factors are recorded in the RS\(\to\)CLASSICAL BRIDGE SPEC v1.0 under entries \texttt{@BARYOGEN}, \texttt{@CLASSICAL\_BRIDGE\_TABLE}, and \texttt{@DERIVATIONS\_CANONICAL} :contentReference[oaicite:0]{index=0}.}

\paragraph{Minimal formula at freeze‑out (small \(\mu_B/T\)).}
For a relativistic species with baryon weight \(g_b\) and small \(\mu_B/T\), the equilibrium baryon density is
\begin{equation}
  n_B - \bar n_B \;\simeq\; \frac{g_b}{6}\,\mu_B\,T^2,
\end{equation}
and the photon density is \(n_\gamma = \tfrac{2\zeta(3)}{\pi^2}T^3\). The baryon‑to‑photon ratio at freeze‑out is therefore
\begin{equation}
  \eta_B \;\equiv\; \frac{n_B-\bar n_B}{n_\gamma}
  \;\simeq\;
  \Big(\frac{g_b}{6}\,\frac{\pi^2}{2\zeta(3)}\Big)\,
  \Big(\frac{\mu_B}{T}\Big)\Big|_{T=T_f}.
  \label{eq:etaB_minimal}
\end{equation}
This is the instantaneous (freeze‑out) approximation used for displays. In §M3 we solve the Boltzmann equation with time‑varying \(\mu_B(t)\) and sphaleron terms,
\begin{equation}
  \frac{d}{dt}\big(a^3 n_B\big)
  \;=\; a^3\Big[S_B(t) - \Gamma_{\rm wash}(t)\,n_B\Big],
  \qquad
  S_B(t)=\mathcal{C}_B\,\mu_B(t)\,T^2(t),
\end{equation}
and show that the fully integrated result agrees with \eqref{eq:etaB_minimal} within the RS systematic envelope set by the cadence window and \(\Gamma_{\rm sph}(T)\).

\paragraph{Source term from the \(CP\)–odd background.}
From the effective Lagrangian \(\mathcal{L}_{\!CP}=(c_g/M)\chi\,R\tilde R+(c_{\rm em}/M)\chi\,F\tilde F\) (see §5), spatial averaging and integration by parts produce a homogeneous source proportional to \(\dot\chi\):
\begin{equation}
  \mu_B(t) \;=\; \Xi_B\,\frac{c_g}{M}\,\dot\chi(t)\;+\;\Xi'_B\,\frac{c_{\rm em}}{M}\,\dot\chi(t)
  \;\; \simeq \;\; \Xi_B^{\rm eff}\,\frac{c_g}{M}\,\dot\chi(t),
  \label{eq:muB_source}
\end{equation}
where the effective coefficient \(\Xi_B^{\rm eff}\) collects the anomaly weights and plasma response in the minimal route. The sign of \(\mu_B\) is fixed by the eight‑tick orientation chosen at R0; reversing that orientation flips the sign of \(\eta_B\) (see §5 and §M2 for sign conventions). 

\paragraph{What we calculate explicitly (Methods §M1–§M3).}
\begin{enumerate}
  \item \(\dot\chi(t)\) from the RS potential \(V(\chi)=V_0\tanh^2\!\big(\chi/(\sqrt{6}\,\varphi)\big)\) across the reheating window (background solution and slow‑roll/oscillatory matching).
  \item The constants bridge mapping \((c_g/M,c_{\rm em}/M)\to(\lambda_{CP},\kappa;\lambda_{\rm rec},\tau_0)\) with the RS‑fixed, parameter‑free values
  \(
    \lambda_{CP}=\varphi^{-7},\;
    \kappa=\varphi^{-9},
  \)
  and \(\lambda_{\rm rec}=\sqrt{\hbar G/(\pi c^3)}\).
  \item The Boltzmann evolution with time‑dependent \(\mu_B(t)\) and a standard sphaleron rate \(\Gamma_{\rm sph}(T)\), demonstrating agreement with the instantaneous expression \eqref{eq:etaB_minimal} to within the RS systematic envelope.
\end{enumerate}
All three steps use only exogenous constants \(\{c,\hbar,G,\alpha\}\) and RS‑fixed exponents; no fit parameters are introduced. The constants audit uses the Planck‑side gate identity \(c^3\lambda_{\rm rec}^2/(\hbar G)=1/\pi\) (see \texttt{@REALITY\_BRIDGE} and \texttt{@AUDIT} entries).\footnote{Audit identities and parameter policy are consolidated in RS\(\to\)CLASSICAL BRIDGE SPEC v1.0 under \texttt{@PARAMETER\_POLICY}, \texttt{@REALITY\_BRIDGE}, and \texttt{@AUDIT} :contentReference[oaicite:1]{index=1}}

\paragraph{Result.}
Inserting \(\lambda_{CP}=\varphi^{-7}\), \(\kappa=\varphi^{-9}\) and the background \(\dot\chi(t)\) from the RS inflaton sector into \eqref{eq:muB_source} and \eqref{eq:etaB_minimal}, we obtain the first‑pass RS prediction
\begin{equation}
  \boxed{\ \eta_B \;\simeq\; 5.1\times 10^{-10}\ }\quad
  \text{(freeze‑out evaluation consistent with the time‑integrated Boltzmann solution; systematic budget in §8).}
\end{equation}
We provide a breakdown of contributions in Methods and show that the sign selection for \(\eta_B\) is robust to all allowed RS gauge choices and boundary redefinitions (Chern–Simons shifts affect only total derivatives), leaving no free knobs to tune.\footnote{Prediction and provenance: see \texttt{@BARYOGEN} (COUPLINGS, YIELD) and \texttt{@CLASSICAL\_BRIDGE\_TABLE} for the gauge‑invariant scaffolding in RS\(\to\)CLASSICAL BRIDGE SPEC v1.0 :contentReference[oaicite:2]{index=2}}

\medskip\noindent
Table~\ref{tab:rs-exponents} summarizes the RS exponents and scales entering the baryogenesis mechanism.

\begin{table}[ht]
\centering
\caption{RS exponents and scales in the baryogenesis mechanism.}
\label{tab:rs-exponents}
\begin{tabular}{lll}
\toprule
Quantity & RS value & Role \\
\midrule
$\varphi$ & $(1+\sqrt{5})/2\approx 1.618$ & Golden ratio (fixed point) \\
$\lambda_{CP}$ & $\varphi^{-7}\approx 0.0335$ & Gravitational CP coupling \\
$\kappa$ & $\varphi^{-9}\approx 0.0128$ & Electromagnetic CP coupling \\
$\lambda_{\rm rec}$ & $\sqrt{\hbar G/(\pi c^3)}\approx 9.12\times 10^{-36}~{\rm m}$ & Recognition length \\
$M_{\rm rec}$ & $2\sqrt{2\pi}\,M_{\rm Pl}\approx 1.09\times 10^{19}~{\rm GeV}$ & Recognition mass scale \\
$\alpha$ (attractor) & $\varphi^{-2}\approx 0.382$ & Inflaton field-space curvature \\
\bottomrule
\end{tabular}
\end{table}

\section{Consistency with cosmology and laboratory bounds}

\paragraph{CMB/BBN.}
Our freeze–out evaluation yields \(\eta_B \simeq 5.1\times 10^{-10}\) (see §6), which sits within the family of CMB/BBN inferences clustered near \(6\times 10^{-10}\). The residual modeling spread in our pipeline is \(\sim 10\text{–}20\%\), dominated by the reheating‐window details (the short epoch where \(\dot\chi\neq 0\) and sphalerons are active) and by the choice of a standard sphaleron rate \(\Gamma_{\rm sph}(T)\). This envelope agrees with the program’s internal audit for the baryogenesis yield (recorded as a first‑pass \(\eta_B\) with a \(\approx 16\%\) offset vs.\ CMB inferences) and carries no free parameters.\footnote{Program record: \texttt{@BARYOGEN} with \texttt{YIELD; eta\_B≈5.1e−10; delta\_vs\_CMB≈16\%} and constants/units policy under \texttt{@PARAMETER\_POLICY}, \texttt{@REALITY\_BRIDGE} in RS\(\to\)CLASSICAL BRIDGE SPEC v1.0 :contentReference[oaicite:0]{index=0}}

\paragraph{No late‑time backreaction.}
The Information‑Limited Gravity (ILG) kernel modifies \emph{perturbations}—weak lensing and linear growth—while leaving the homogeneous FRW background and thus the Hubble‑diagram average distances unchanged. Our baryogenesis epoch precedes the ILG regime, so the CP‑odd mechanism and its freeze‑out are unaffected by late‑time effective‑weight corrections. We therefore have no induced shift in background cosmological fits from the CP‑odd sector, and the baryon yield prediction remains orthogonal to ILG phenomenology.\footnote{ILG entries \texttt{@GRAVITY} and \texttt{@ILG\_SPEC} (kernel \(w(k,a)\), growth equation, and the “global‑only” policy) and the bridge items \texttt{BRIDGE;ILG}, \texttt{BRIDGE;RealityBridge} establish that background distances are not altered; RS\(\to\)CLASSICAL BRIDGE SPEC v1.0 :contentReference[oaicite:1]{index=1}}

\paragraph{EDM and collider.}
The effective couplings \(\chi\,F\tilde F\) and \(\chi\,R\tilde R\) are set by RS‑fixed exponents \(\lambda_{CP}=\varphi^{-7}\), \(\kappa=\varphi^{-9}\) and the recognition scale \(\lambda_{\rm rec}\) via the constants bridge (see §5 and Methods). This mapping introduces no light states and no tunable phases; loop‑induced electric dipole moments (EDMs) are therefore parametrically suppressed by the same recognition‑scale factors that enter \(\eta_B\). In particular, the dominant EDM operators arise at higher order and remain below current experimental upper bounds for the mapped strengths, while collider observables are unchanged at leading order owing to the derivative nature of the pseudoscalar couplings and the absence of new on‑shell degrees of freedom.\footnote{Parameter policy (no fits), constants/scale audit (Planck‑side gate), and the absence of extra light states are stated under \texttt{@PARAMETER\_POLICY}, \texttt{@REALITY\_BRIDGE}, and \texttt{@GAUGE}; see also \texttt{BRIDGE;LambdaRec} for the recognition‑length normalization in RS\(\to\)CLASSICAL BRIDGE SPEC v1.0 :contentReference[oaicite:2]{index=2}}

\paragraph{Parity‑odd GW/CMB signatures.}
The \(\chi\,R\tilde R\) term predicts \emph{chiral} primordial gravitational waves and non‑zero parity‑odd CMB cross‑spectra. Defining the Chern–Simons control parameter
\begin{equation}
  \Theta_{\rm CS} \;\equiv\; \frac{c_g}{M}\,\frac{\dot\chi}{H}\,,
\end{equation}
the fractional chirality of tensor modes is, to leading order,
\begin{equation}
  \Pi_T(k) \;\equiv\; \frac{P_h^{\rm R}(k)-P_h^{\rm L}(k)}{P_h^{\rm R}(k)+P_h^{\rm L}(k)}
  \;\simeq\; \mathcal{O}(1)\times \Theta_{\rm CS}\,,
\end{equation}
and the CMB parity‑odd spectra scale as
\begin{equation}
  C_\ell^{TB},\, C_\ell^{EB} \;\propto\; \Theta_{\rm CS}\, r\, A_s \times \mathcal{T}_\ell\,,
\end{equation}
with transfer functions \(\mathcal{T}_\ell\) fixed by standard Boltzmann evolution. In the RS inflaton background that gives \(r\simeq 1.3\times 10^{-3}\) at the pivot, the signal scales linearly with \(\Theta_{\rm CS}\); its \emph{sign} is fixed by the eight‑tick orientation chosen at R0, furnishing a crisp, one‑bit falsifier. A null result consistent with \(r\sim 10^{-3}\) but with \(|C_\ell^{TB}|,|C_\ell^{EB}|\) constrained below the predicted scaling would falsify the RS mapping of \((c_g/M)\) (see §8 for the forecast envelope and Methods for the constants bridge).\footnote{Inflaton predictions and parity bookkeeping reside under \texttt{@COSMOLOGY} (with \(n_s, r, A_s\)) and \texttt{@BARYOGEN} (couplings, sign conventions); bridge items \texttt{@THEOREMS} (eight‑tick) and \texttt{@CLASSICAL\_BRIDGE\_TABLE} document provenance; RS\(\to\)CLASSICAL BRIDGE SPEC v1.0 :contentReference[oaicite:3]{index=3}}

\paragraph{Order‑of‑magnitude.}
Using \(\dot\chi/H\approx -\sqrt{2\epsilon_V}\,M_{\rm Pl}\) with \(\epsilon_V\simeq r/16\) and \((c_g/M)=\lambda_{CP}/M_{\rm rec}=\varphi^{-7}/M_{\rm rec}\), one finds
\[
  \Theta_{\rm CS} \;\approx\; \frac{\varphi^{-7}}{M_{\rm rec}}\,\sqrt{\tfrac{r}{8}}\,M_{\rm Pl}
  \;\sim\; \mathcal{O}(10^{-4})\quad (r\simeq 1.3\times10^{-3},\ M_{\rm rec}=2\sqrt{2\pi}\,M_{\rm Pl}).
\]
Thus parity‑odd CMB signals are small but potentially detectable with next‑generation polarization sensitivity.

\section{Error budget, ablations, and falsifiers}

\subsection{Systematic envelope}

\paragraph{Systematic sources.}
The dominant modeling systematics are confined to three levers, none of which introduces tunable parameters:
\begin{enumerate}
  \item \emph{Reheating‑window modeling:} the finite window in which \(\dot\chi\neq 0\) and sphalerons are active. Denote by \(W\) the effective width (in e‑folds or conformal time). With \(\eta_B \propto (\mu_B/T)_{T_f}\) and \(\mu_B\propto \dot\chi\), the sensitivity to the window is captured by
  \begin{equation}
    \sigma_{\rm reh}
    \;\equiv\;
    \bigg|\frac{\partial \ln \eta_B}{\partial \ln W}\bigg|\,\sigma_{\ln W},
    \qquad
    \dot\chi \simeq -\frac{V'}{3H}\ \Rightarrow\ \eta_B \propto \sqrt{\epsilon_V}\,,
  \end{equation}
  where \(\epsilon_V\) is evaluated along the RS background (\S4). The cadence fixes the window location; only its coarse shape contributes to \(\sigma_{\rm reh}\).
  \item \emph{Sphaleron rate choice:} the freeze‑out temperature \(T_f\) follows from \(\Gamma_{\rm sph}(T_f)\simeq H(T_f)\). Writing \(S\equiv \partial \ln\Gamma_{\rm sph}/\partial \ln T|_{T_f}\), small variations obey
  \begin{equation}
    \delta \ln T_f \simeq -\,\frac{\delta \ln \Gamma_{\rm sph}}{S}\,,
    \qquad
    \sigma_{\rm sph}
    \;\equiv\;
    \bigg|\frac{\partial \ln \eta_B}{\partial \ln T}\bigg|_{T_f}\,\frac{\sigma_{\ln \Gamma_{\rm sph}}}{S}\,,
  \end{equation}
  with \(\partial \ln \eta_B/\partial \ln T|_{T_f}=\partial \ln (\mu_B/T)/\partial \ln T\) computed from \(\mu_B\propto \dot\chi\) and the RS background. 
  \item \emph{\(\chi\) background solution:} numerical accuracy of \(\dot\chi(t)\) across the slow‑roll/oscillation match. Define 
  \begin{equation}
    \sigma_{\rm bg}
    \;\equiv\;
    \bigg|\frac{\partial \ln \eta_B}{\partial \ln \dot\chi}\bigg|_{T_f}\,\sigma_{\ln \dot\chi}\,,
    \qquad
    \eta_B \propto \dot\chi(T_f)\,.
  \end{equation}
\end{enumerate}
Adding these in quadrature defines the systematic envelope,
\begin{equation}
  \Delta_{\rm syst}^2 \;\equiv\; \sigma_{\rm reh}^2+\sigma_{\rm sph}^2+\sigma_{\rm bg}^2
  \;\;\Rightarrow\;\;
  \Delta_{\rm syst}\approx 10\text{–}20\%,
\end{equation}
consistent with the first‑pass \(\eta_B\simeq 5.1\times 10^{-10}\) vs.\ CMB/BBN family near \(6\times 10^{-10}\) (no parameter fits enter any step; constants are bridged via the RS specification).\footnote{Parameter policy and constants audit: RS\(\to\)CLASSICAL BRIDGE SPEC v1.0, entries \texttt{@PARAMETER\_POLICY}, \texttt{@REALITY\_BRIDGE}, and \texttt{@BARYOGEN} (status: provisional first pass) :contentReference[oaicite:0]{index=0}}

\subsection{Ablations and controls}

\paragraph{Ablations (sanity checks).}
The following removals or mis‑settings demonstrate necessity and sign control:
\begin{itemize}
  \item \emph{Drop \(\lambda_{CP}\):} setting \(\lambda_{CP}=0\) in the constants bridge forces \(c_g/M=0\) (and similarly \(\kappa\to 0\) kills the EM channel), whence \(\mu_B\equiv 0\) and \(\eta_B\to 0\). This ablation removes the sole \(CP\)‑odd source selected by the ledger parities.\footnote{Parity inventory and coupling map are encoded under \texttt{@LEDGER} (nine parities) and \texttt{@BARYOGEN;COUPLINGS} with \(\lambda_{CP}=\varphi^{-7},\ \kappa=\varphi^{-9}\) in RS\(\to\)CLASSICAL BRIDGE SPEC v1.0 :contentReference[oaicite:1]{index=1}}
  \item \emph{Mis‑normalize the RS cadence:} perturbing the eight‑tick minimality or its mapping to \(\tau_0\) violates the cone bound \(c=\ell_0/\tau_0\) and the gate identity, breaking the constants bridge. The baryon yield then loses its parameter‑free normalization and fails the audit (falsified by construction).\footnote{Eight‑tick minimality and causal bound: \texttt{@THEOREMS} (T6, cone\_bound); gate identity: \texttt{@REALITY\_BRIDGE;lambda\_rec\_id} with \(c^3\lambda_{\rm rec}^2/(\hbar G)=1/\pi\) :contentReference[oaicite:2]{index=2}}
  \item \emph{Reverse the eight‑tick orientation:} this flips the global \(CP\)‑odd sign and yields \(\mathrm{sign}(\eta_B)\mapsto -\,\mathrm{sign}(\eta_B)\) with unchanged magnitude, a built‑in diagnostic of the ledger orientation chosen at R0.
\end{itemize}

\subsection{Falsifiers}

\paragraph{Hard falsifiers.}
The mechanism admits crisp, externally testable failure modes:
\begin{enumerate}
  \item \emph{Null CMB parity at predicted sensitivity:} the \(\chi R\tilde R\) term fixes the sign of \(TB/EB\) once the eight‑tick orientation is chosen. For \(r\sim 10^{-3}\), a measurement reaching the predicted amplitude scaling yet finding \(C_\ell^{TB}\approx C_\ell^{EB}\approx 0\) falsifies the RS coupling map \(c_g/M\) (\S7; Methods detail the constants bridge).\footnote{Predictions registry: \texttt{@COSMOLOGY;PREDICTIONS} with \(r\approx 1.27\times 10^{-3}\); parity bookkeeping under \texttt{@BARYOGEN} and \texttt{@CLASSICAL\_BRIDGE\_TABLE} in RS\(\to\)CLASSICAL BRIDGE SPEC v1.0 :contentReference[oaicite:3]{index=3}}
  \item \emph{Inconsistent GW chirality bounds:} the tensor‑chirality fraction \(\Pi_T\propto (c_g/M)\,\dot\chi/H\) must be nonzero with a sign fixed at R0. A robust upper bound \(|\Pi_T|<\Pi_T^{\rm RS}\) at the relevant pivot would rule out the RS mapping.
  \item \emph{EDM limits surpassing the RS floor:} loop‑induced EDMs scale as \(d\sim (c_{\rm em}/M)\times\) (loop factors) in the minimal scenario. If experiments set \(d_{\rm exp}^{\rm max}<d_{\rm RS}^{\rm min}\) implied by \(\kappa=\varphi^{-9}\) and \(\lambda_{\rm rec}\), the RS mapping is excluded (no extra states exist to cancel the contribution).
  \item \emph{Gate‑identity failure (units audit):} the single‑inequality Planck/IR audit must pass,
  \begin{equation}
    \frac{|\,\lambda_{\rm kin}-\lambda_{\rm rec}\,|}{\lambda_{\rm rec}} \;\le\; k\,u_{\rm comb},
    \qquad
    \frac{c^3\lambda_{\rm rec}^2}{\hbar G}=\frac{1}{\pi}\,,
  \end{equation}
  with anchors defined in the RS specification. A failure indicates the constants bridge is invalid, and with it the fixed couplings used for \(\eta_B\).\footnote{Audit and gates: \texttt{@REALITY\_BRIDGE;planck\_gate\_ineq}, \texttt{@AUDIT;SINGLE\_INEQUALITY}, \texttt{@REALITY\_BRIDGE;lambda\_rec\_id} in RS\(\to\)CLASSICAL BRIDGE SPEC v1.0 :contentReference[oaicite:4]{index=4}}
\end{enumerate}

\medskip\noindent
All three classes—systematic envelope, ablations, and falsifiers—are fixed by the RS invariants and bridge policy, not by adjustable knobs. Any contradiction along the falsifier axes rejects the RS mapping for baryogenesis; any successful ablation that preserves \(\eta_B\) at the observed level would, conversely, undermine the uniqueness claims for the \(CP\)‑odd channel selected by the ledger parities.\footnote{Bridge policy, uniqueness of \(J\), parity set, and eight‑tick minimality are consolidated under \texttt{@COST}, \texttt{@LEDGER}, and \texttt{@THEOREMS} (T5,T6) in RS\(\to\)CLASSICAL BRIDGE SPEC v1.0 :contentReference[oaicite:5]{index=5}}

\section{Relations to other RS results (coherence across the program)}

\paragraph{Quantum–classical bridge.}
The same convex, symmetric cost \(J(x)=\tfrac12(x+x^{-1})-1\) and double‑entry ledger rules that yield Born’s rule and standard quantum statistics also underpin the \(CP\)‑odd pseudoscalar sector used here. No new postulate is introduced: the discrete exactness that enforces potentials up to a constant, the action/dual action bridge built from \(J\), and the continuity/Bianchi scaffold together supply the classical language in which the baryogenesis calculation is performed.\footnote{Cost uniqueness and action bridge: \texttt{@COST}, \texttt{@THEOREMS} (T5); Born/statistics: \texttt{@QUANTUM;BORN\_RULE}; continuity/Bianchi and Maxwell DEC: \texttt{@CLASSICAL\_BRIDGE\_TABLE} (Continuity, DEC\_Bianchi, MaxwellContinuity); RS\(\to\)CLASSICAL BRIDGE SPEC v1.0 :contentReference[oaicite:0]{index=0}}

\paragraph{Cosmology.}
The RS inflaton with \(V(\chi)=V_0\tanh^2(\chi/(\sqrt{6}\,\varphi))\) delivers the slow‑roll outputs we used (\(n_s\simeq 0.967\), \(r\simeq 1.3\times 10^{-3}\), \(A_s\simeq 2.1\times 10^{-9}\)), and these feed consistently into the baryogenesis epoch without modification. Information‑Limited Gravity (ILG) acts at late times on perturbations (growth, weak lensing) and leaves background FRW distances intact, so the baryon‑asymmetry result is orthogonal to Hubble‑tension and distance‑ladder discussions within RS cosmology.\footnote{Inflation entries: \texttt{@COSMOLOGY;INFLATON, PREDICTIONS}; ILG kernel and background/perturbation policy: \texttt{@ILG\_SPEC}, \texttt{@GRAVITY;MODEL=ILG}, and \texttt{@CLASSICAL\_BRIDGE\_TABLE} (ILG, RealityBridge); RS\(\to\)CLASSICAL BRIDGE SPEC v1.0 :contentReference[oaicite:1]{index=1}}

\paragraph{Microphysics.}
The mass‑ladder/rung structure and the \(\alpha\) pipeline both organize predictions around the same \(\varphi\)‑based exponents that appear here in the \(CP\)‑odd couplings \(\lambda_{CP}=\varphi^{-7}\) and \(\kappa=\varphi^{-9}\). This shared exponent structure is a core coherence claim of RS: spectra, coupling normalizations, and cosmological imprints reuse a single scaling language rather than introducing sector‑specific knobs.\footnote{Spectral/rung framework and \(\varphi\) exponents: \texttt{@SPECTRA}, \texttt{@SM\_MASSES}, \texttt{@GAUGE}, \texttt{@CLASSICAL\_BRIDGE\_TABLE} (MassLaw, RGFixedPoint); \(\alpha\) pipeline: \texttt{@ALPHA}; baryogenesis exponents and yield: \texttt{@BARYOGEN}; RS\(\to\)CLASSICAL BRIDGE SPEC v1.0 :contentReference[oaicite:2]{index=2}}

\paragraph{Neutrino caveat.}
The neutrino sector is not yet fully fixed in the RS mapping (Dirac vs.\ Majorana choice and rung assignment remain open). Once specified, the leptogenesis presentation can be sharpened by replacing the generic anomaly weights with neutrino‑specific ones; this is an open item but not a blocker for the direct‑baryogenesis route to \(\eta_B\) presented here.\footnote{Open item: \texttt{@SM\_MASSES; TODO; neutrino\_sector} and \texttt{@OPEN\_ITEMS}; bridge usage policy: \texttt{@PAPERS\_POLICY}; RS\(\to\)CLASSICAL BRIDGE SPEC v1.0 :contentReference[oaicite:3]{index=3}}

\section{Relations to companion papers}

\paragraph{Universe-Origin.}
The inflaton potential $V(\chi)=V_0\tanh^2(\chi/(\sqrt{6}\varphi))$ and the slow-roll backbone used here are derived in the Universe-Origin manuscript, which presents the RS singularity-free cosmology with Recognition Onset (R0) replacing the Big Bang. The $\alpha$-attractor parameter $\alpha=\varphi^{-2}$, the e-fold relations, and the spectral observables $(n_s, r, A_s)$ are computed there from minimal-overhead principles; we adopt those results without modification.

\paragraph{Quantum-Gravity-New.}
The discrete-to-continuum bridge (theorems T2–T7), the DEC scaffold, the unique cost $J$, and the Planck-gate identity are machine-verified in Lean~4 and presented in the Quantum Gravity manuscript. The gauge-rigidity posture and the units-quotient policy used throughout this paper follow the audit framework established there.

\paragraph{Dark-Energy and Hubble-Tension-Resolution.}
The Information-Limited Gravity (ILG) kernel $w(k,a)$ modifies late-time structure growth and lensing while leaving background FRW distances unchanged (Buchert $Q_D=0$). Our baryogenesis epoch is orthogonal to ILG: the CP-odd mechanism operates during reheating (pre-recombination), while ILG acts on late-time perturbations. The two phenomenologies do not interfere.

\section{Discussion and outlook}

\paragraph{Interpretation.}
Within RS, \(CP\) violation and out‑of‑equilibrium dynamics arise from ledger parities and the eight‑tick cadence rather than from ad hoc phases or tuned interactions. The result is a parameter‑free origin for the observed baryon asymmetry, consistent with CMB/BBN inferences and insulated from late‑time cosmology. The same discrete invariants that produce the classical limit also select the unique \(CP\)‑odd channel, eliminating ambiguity about where the asymmetry comes from.

\paragraph{Near‑term tests.}
Three observational fronts can quickly pressure‑test the mechanism: (i) parity‑odd CMB spectra \(TB,EB\) with sign fixed by the eight‑tick orientation and amplitude scaling \(\propto r\,\Theta_{\rm CS}\) at \(r\sim 10^{-3}\) (CMB-S4, LiteBIRD sensitivity); (ii) primordial gravitational‑wave chirality consistent with the same sign (space-based interferometers); (iii) improved modeling of the reheating window to narrow the \(\mathcal{O}(10\text{–}20\%)\) systematics in \(\eta_B\). A fourth, orthogonal cross‑check is that ILG residuals in lensing/growth remain decoupled from baryogenesis, as the framework asserts.

\paragraph{Programmatic path.}
Three concrete steps extend and sharpen the result: (1) finalize the neutrino mapping to enable a fully specified leptogenesis version that reproduces the same \(\eta_B\) via a fixed conversion factor; (2) tighten the cadence‑level description of reheating to reduce the systematic envelope on \(\eta_B\); (3) unify baryon and lepton channels under a single RS residue rule, making explicit how the nine parities govern both sectors with one orientation choice at R0. All of these steps preserve the parameter‑free character of the derivation while increasing its empirical bite.\footnote{Program items and policies: \texttt{@BARYOGEN;YIELD}, \texttt{@COSMOLOGY;PREDICTIONS}, \texttt{@ILG\_SPEC}, \texttt{@SM\_MASSES; TODO neutrino\_sector}, and \texttt{@PARAMETER\_POLICY}; RS\(\to\)CLASSICAL BRIDGE SPEC v1.0 :contentReference[oaicite:4]{index=4}}

\section{Summary}

We have derived a parameter-free baryogenesis mechanism from the Recognition Science framework. The derivation uses only the discrete-ledger invariants (atomic tick, eight-tick cadence, unique cost $J$, nine ledger parities), the RS inflaton potential, and standard SI constants $\{c,\hbar,G,\alpha\}$. No adjustable couplings, phases, or new mass scales are introduced.

The prediction $\eta_B\simeq 5.1\times 10^{-10}$ sits within the CMB/BBN inference family, with a $10$--$20\%$ systematic envelope from reheating-window modeling, sphaleron-rate choice, and background-solution accuracy. The mechanism is falsifiable through parity-odd CMB observables (TB/EB sign and amplitude), primordial GW chirality, EDM bounds, and the RS gate-identity audit.

The coherence across the RS program is demonstrated: the same cost $J$ that yields Born's rule and quantum statistics also selects the CP-odd channel; the same inflaton that sets $(n_s, r, A_s)$ drives the reheating dynamics; and the late-time ILG phenomenology (galaxies, growth, Hubble tension) remains orthogonal to the early-universe baryogenesis epoch. The framework is internally consistent, externally testable, and free of tunable parameters.

\section{Methods and Appendices (derivation map)}

\subsection*{M0. RS$\to$Classical bridge and theorem tags}
We collect the discrete statements, bridge rules, and audit policies used in the main text. Each item is a compact restatement of entries tagged in the RS$\to$Classical Bridge Specification; theorem labels \(T1\)–\(T8\) are referenced explicitly.

\paragraph{Discrete results (used in this paper).}
\begin{itemize}
  \item \textbf{T1 (Meta‑Principle).} Recognition scheduling implies atomicity foundations (used for provenance only).
  \item \textbf{T2 (Atomic tick).} One posting per tick; no concurrency in the fundamental schedule (enables the cadence model).
  \item \textbf{T3 (Continuity / closed‑chain exactness).} Closed‑loop flux zero implies a potential \(\phi\) unique up to an additive constant on each reach component; in the mesh limit, \(\partial_t\rho+\nabla\!\cdot J=0\).
  \item \textbf{T4 (Potential uniqueness up to constant).} The \(\delta\)‑rule fixes \(\phi\) up to constants componentwise (gauge freedom for \(\phi\)).
  \item \textbf{T5 (Cost uniqueness).} On \(\mathbb{R}_{>0}\), under analyticity, symmetry \(J(x)=J(x^{-1})\), convexity, bounded growth, and normalization \(J''(1)=1\), the unique cost is
        \[
        J(x)=\tfrac12\bigl(x+x^{-1}\bigr)-1,
        \]
        which bridges to stationary action and its Legendre dual to the Hamiltonian form.
  \item \textbf{T6 (Eight‑tick minimality in \(D=3\)).} Any spatially complete, ledger‑compatible tour on the 3‑cube has minimal period \(2^3=8\) (the Gray cycle realizes it). This fixes the micro‑time cadence used to set orientation and sign for \(CP\)‑odd effects.
  \item \textbf{T7 (Coverage lower bound).} No surjection to patterns for \(T<2^D\) (sampling/coverage guardrail; Methods‑only).
  \item \textbf{T8 (Ledger \(\delta\)‑units).} Increments form \(\mathbb{Z}\) (quantization scaffold for unit mapping when needed).
\end{itemize}

\paragraph{Causal cone and Lorentz limit.}
Per‑step geometry bounds transport and defines the discrete cone:
\[
  c=\frac{\ell_0}{\tau_0},\qquad \frac{\|\Delta\mathbf{x}\|}{\Delta t}\le c,
\]
with local Lorentz invariance and a Minkowski metric emerging under mesh refinement.

\paragraph{Maxwell/DEC scaffold (used implicitly).}
On the cochain complex, \(d\!\circ d=0\) encodes Bianchi and continuity:
\[
  dF=0,\qquad d(\star F)=J\ \Rightarrow\ dJ=0.
\]
This is the gauge‑consistent bridge we use to discuss \(F\tilde F\) and \(R\tilde R\) without introducing extra assumptions.

\paragraph{Display/units policy and Planck/IR gates.}
All derivations are conducted in dimensionless quotients; SI units are restored only for displays. The Planck‑side gate identity audits the constants mapping
\[
  \frac{c^3\,\lambda_{\mathrm{rec}}^{\,2}}{\hbar\,G}=\frac{1}{\pi},
\qquad
  \lambda_{\mathrm{rec}}=\sqrt{\frac{\hbar G}{\pi c^3}},
\]
and the single‑inequality cross‑route check is
\[
  \frac{|\,\lambda_{\rm kin}-\lambda_{\rm rec}\,|}{\lambda_{\rm rec}} \;\le\; k\,u_{\rm comb},
\]
with \(u_{\rm comb}\) the combined metrology uncertainty (time‑first vs.\ length‑first routes). These audits are carried out once per paper and not used as tunable knobs.

\paragraph{Specification and reproducibility.}
We follow the RS$\to$Classical Bridge Spec v1.0 for theorem references, constants, parity inventory, and audit/reproducibility conventions (tags and checks). \emph{Source:} RS$\to$CLASSICAL BRIDGE SPEC v1.0 :contentReference[oaicite:0]{index=0}.

\bigskip

\subsection*{M1. Early‑universe background under the RS inflaton}
We solve \(\chi\) dynamics for the RS potential and extract \(\dot\chi(t)\) across the reheating window used in the freeze‑out calculation.

\paragraph{Potential, variables, and equations of motion.}
Let
\[
  V(\chi)=V_0\,\tanh^2\!\Big(\frac{\chi}{\sqrt{6}\,\varphi}\Big),\qquad
  u\equiv \frac{\chi}{\sqrt{6}\,\varphi},\qquad
  M_{\rm Pl}^2\equiv (8\pi G)^{-1},
\]
with \(\varphi=\tfrac{1+\sqrt{5}}{2}\). The homogeneous background obeys
\begin{equation}
  \ddot\chi + 3H\dot\chi + V'(\chi)=0,\qquad
  H^2=\frac{1}{3M_{\rm Pl}^2}\Big(\tfrac12\dot\chi^{\,2}+V(\chi)\Big).
  \label{eq:KG_methods}
\end{equation}
Derivatives of the potential are
\begin{align}
  \frac{V'}{V} &= \frac{4}{\sqrt{6}\,\varphi}\,\frac{1}{\sinh(2u)}, \label{eq:VpOverV}\\
  \epsilon_V(u) &\equiv \frac{M_{\rm Pl}^2}{2}\Big(\frac{V'}{V}\Big)^2
                 = \frac{4 M_{\rm Pl}^2}{3\,\varphi^2}\,\frac{1}{\sinh^2(2u)}, \label{eq:epsV}\\
  \eta_V(u) &\equiv M_{\rm Pl}^2\,\frac{V''}{V}
             = \frac{4 M_{\rm Pl}^2}{3\,\varphi^2}\,\frac{2-\cosh(2u)}{\sinh^2(2u)}. \label{eq:etaV}
\end{align}

\paragraph{Slow‑roll era and end of inflation.}
Inflation ends when \(\epsilon_V(u_{\rm end})=1\), i.e.
\[
  \sinh\!\big(2u_{\rm end}\big)=\frac{2}{\sqrt{3}\,\varphi},\qquad
  \cosh\!\big(2u_{\rm end}\big)=\sqrt{1+\frac{4}{3\varphi^2}}\;.
\]
The e‑fold count from \(\chi_\star\) (pivot) to the end is
\begin{equation}
  N_\star=\int_{\chi_{\rm end}}^{\chi_\star}\frac{V}{M_{\rm Pl}^2 V'}\,d\chi
         = \frac{3\,\varphi^2}{4\,M_{\rm Pl}^2}\Big[\cosh\!\big(2u_\star\big)-\cosh\!\big(2u_{\rm end}\big)\Big].
  \label{eq:Nstar_methods}
\end{equation}
With \(N_\star\in[50,60]\) and \(V_0\) fixed by the scalar amplitude, the model yields \(n_s\simeq 0.967\), \(r\simeq 1.3\times 10^{-3}\), and \(A_s\simeq 2.1\times 10^{-9}\), as used in the main text.

\paragraph{Closed‑form slow‑roll velocity for \(\chi\).}
In slow‑roll,
\[
  \dot\chi \simeq -\,\frac{V'}{3H}
  = -\,\frac{M_{\rm Pl}}{\sqrt{3}}\frac{V'}{\sqrt{V}}
  = -\,\frac{2 M_{\rm Pl}\,\sqrt{V_0}}{3\sqrt{2}\,\varphi}\,\mathrm{sech}^2 u,
  \label{eq:dotchi_SR_methods}
\]
where we used \(V'=(2V_0/\sqrt{6}\,\varphi)\tanh u\,\mathrm{sech}^2u\) and \(\sqrt{V}=\sqrt{V_0}\,|\tanh u|\).
The overall sign tracks \(\mathrm{sgn}(\tanh u)\) (equivalently \(\mathrm{sgn}(V')\)), ensuring that the sign of \(\dot\chi\) (and hence of \(\mu_B\)) is fixed once the eight‑tick orientation is chosen.
For \(u\gg 1\), \(\mathrm{sech}^2u\approx 4e^{-2u}\), so
\[
  \dot u \equiv \frac{\dot\chi}{\sqrt{6}\,\varphi}
  \simeq -\,A\,\mathrm{sech}^2u,\qquad
  A\equiv \frac{M_{\rm Pl}\sqrt{V_0}}{3\sqrt{3}\,\varphi^2},
\]
and the early‑time asymptotic integrates to
\begin{equation}
  e^{2u(t)} \;\simeq\; e^{2u_i} - 8A\,(t-t_i)\,,
  \label{eq:u_asymptotic}
\end{equation}
valid until \(u\) approaches \(\mathcal{O}(1)\).

\paragraph{Near‑minimum (oscillatory) phase and reheating match.}
For \(|u|\ll 1\), \(V(\chi)\approx \frac{V_0}{6\varphi^2}\chi^2\), so
\begin{equation}
  V(\chi)\simeq \tfrac12 m_\chi^2\,\chi^2,\qquad m_\chi^2=\frac{V_0}{3\varphi^2}.
  \label{eq:mchi_methods}
\end{equation}
Equation \eqref{eq:KG_methods} becomes a damped harmonic oscillator. In the \(\chi\)‑dominated (pre‑thermalization) stage one has \(a(t)\propto t^{2/3}\), \(H=2/(3t)\), and the envelope of the coherent oscillations decays as
\[
  \chi_A(t)\propto a^{-3/2}(t),\qquad
  \langle\dot\chi^2\rangle \propto a^{-3}(t).
\]
We match the slow‑roll branch \eqref{eq:dotchi_SR_methods} to the oscillatory solution at \(t=t_{\rm match}\) by continuity of \(\chi\) and \(\dot\chi\), fixing the phase \(\delta\) and amplitude \(\chi_A(t_{\rm match})\) of
\[
  \chi(t)\;\approx\; \chi_A(t)\,\cos\!\big(m_\chi t+\delta\big),\qquad
  \dot\chi(t)\;\approx\; -\,m_\chi\,\chi_A(t)\,\sin\!\big(m_\chi t+\delta\big).
\]
As radiation is produced and begins to dominate, we map time to temperature via
\begin{equation}
  H(T)\simeq \sqrt{\frac{\pi^2 g_\ast(T)}{90}}\;\frac{T^2}{M_{\rm Pl}},\qquad
  \text{(radiation era, standard \(g_\ast\) bookkeeping),}
  \label{eq:H_of_T_methods}
\end{equation}
and evaluate \(\dot\chi\big(T\big)\) across the window in which electroweak sphalerons are active. The RS eight‑tick cadence identifies this window without introducing free parameters; its location is fixed, and only its narrow shape contributes to the systematic budget (see §8).

\paragraph{Consistency with Universe‑Origin derivation.}
The slow‑roll parameters \eqref{eq:VpOverV}–\eqref{eq:etaV}, the end‑of‑inflation condition, the e‑fold relation \eqref{eq:Nstar_methods}, the slow‑roll velocity \eqref{eq:dotchi_SR_methods}, the oscillatory mass \eqref{eq:mchi_methods}, and the time‑to‑temperature map \eqref{eq:H_of_T_methods} reproduce the Universe‑Origin backbone under the RS bridge conventions. No tunable parameters enter; constants and audit gates are those recorded in the RS$\to$Classical Bridge Spec v1.0 (sections \texttt{@COSMOLOGY}, \texttt{@REALITY\_BRIDGE}, \texttt{@THEOREMS}). \emph{Source:} RS$\to$CLASSICAL BRIDGE SPEC v1.0 :contentReference[oaicite:1]{index=1}.

\subsection*{M2. CP‑odd EFT mapping from RS}

\paragraph{Parity map and operator selection.}
The ledger carries nine \(\mathbb{Z}_2\) parities
\(\{P_{\rm cp},P_{B-L},P_Y,P_T,P_C^{(1)},P_C^{(2)},P_C^{(3)},P_\tau^{(1)},P_\tau^{(2)}\}\),
each flipping under conjugation and under reversal of the eight‑tick orientation; the scalar vacuum page is neutral. A linear coupling of a scalar background to matter/geometry that \emph{survives spatial averaging} in an FRW patch must be (i) rotationally invariant, (ii) gauge‑invariant, and (iii) odd under the combined ledger parity that encodes \(CP\) (so that it changes sign when the eight‑tick orientation is flipped). Under these constraints, the leading pseudoscalar densities built from gauge and curvature fields are the Chern–Pontryagin terms \(X\tilde X\) (topological densities), for \(X\in\{F,R\}\), where
\[
  \tilde F^{\mu\nu}\equiv\tfrac12\epsilon^{\mu\nu\rho\sigma}F_{\rho\sigma},\quad
  F\tilde F\equiv F_{\mu\nu}\tilde F^{\mu\nu},\qquad
  \tilde{R}^{\alpha}{}_{\beta\mu\nu}\equiv\tfrac12\epsilon_{\mu\nu\rho\sigma}R^{\alpha}{}_{\beta}{}^{\rho\sigma},\quad
  R\tilde R\equiv R^{\alpha}{}_{\beta\mu\nu}\tilde{R}^{\beta}{}_{\alpha}{}^{\mu\nu}.
\]
Both \(F\tilde F\) and \(R\tilde R\) are pseudoscalars (they flip under parity) and are gauge‑invariant scalars under rotations, hence compatible with homogeneity and isotropy. 

Any other candidate at the same or lower derivative order either (a) reduces to a boundary term (a total derivative) plus terms proportional to \(\partial_\mu\chi\) times a Chern–Simons current, or (b) violates rotational/gauge invariance, or (c) carries extra derivatives that raise the effective order and are disfavored by the uniqueness/ minimality of the cost \(J\) (which selects quadratic local functionals in the continuum limit). In particular, non‑abelian \(G\tilde G\) terms average to zero in the homogeneous, gauge‑neutral radiation epoch and require nontrivial color topology; in RS bookkeeping they also carry additional ledger complexity (triad motifs) and thus enter beyond the minimal operator set selected by symmetry and cost arguments.\footnote{Parity inventory, eight‑tick orientation, cost uniqueness, and gauge/continuum bridge: see \texttt{@LEDGER} (parities), \texttt{@THEOREMS} (T5,T6), and \texttt{@CLASSICAL\_BRIDGE\_TABLE} (Continuity, MaxwellDEC, DEC\_Bianchi); RS\(\to\)CLASSICAL BRIDGE SPEC v1.0 :contentReference[oaicite:0]{index=0}}

Consequently, the leading CP‑odd effective Lagrangian density is
\begin{equation}
  \mathcal{L}_{\!CP}
  \;=\;
  \frac{c_g}{M}\,\chi\,R\tilde R
  \;+\;
  \frac{c_{\rm em}}{M}\,\chi\,F\tilde F,
  \label{eq:LCP_methods}
\end{equation}
with no additional operators at the same order.

\paragraph{Coefficient map from RS invariants.}
RS fixes two dimensionless invariants for the CP‑odd channel,
\begin{equation}
  \lambda_{CP}=\varphi^{-7},\qquad \kappa=\varphi^{-9},
  \label{eq:RS_invariants}
\end{equation}
and a single recognition length scale
\(
  \lambda_{\rm rec}=\sqrt{\hbar G/(\pi c^3)}
\).
Define the associated mass scale
\begin{equation}
  M_{\rm rec} \;\equiv\; \frac{\hbar}{c\,\lambda_{\rm rec}}
  \;=\; \sqrt{\frac{\pi\,\hbar c}{G}}
  \;=\; 2\sqrt{2\pi}\,M_{\rm Pl}\,,\qquad M_{\rm Pl}^2\equiv(8\pi G)^{-1},
  \label{eq:Mrec}
\end{equation}
so that \(M_{\rm rec}\) is fully determined by \(\{c,\hbar,G\}\) and the Planck‑gate identity \(c^3\lambda_{\rm rec}^2/(\hbar G)=1/\pi\).\footnote{Constants bridge and Planck‑side gate: \texttt{@REALITY\_BRIDGE;lambda\_rec\_id}, \texttt{@AUDIT;SINGLE\_INEQUALITY}; RS\(\to\)CLASSICAL BRIDGE SPEC v1.0 :contentReference[oaicite:1]{index=1}}
With these definitions the parameter‑free map is
\begin{equation}
  \frac{c_g}{M} \;=\; \frac{\lambda_{CP}}{M_{\rm rec}},
  \qquad
  \frac{c_{\rm em}}{M} \;=\; \frac{\kappa}{M_{\rm rec}},
  \label{eq:coeff_map}
\end{equation}
which may be written as the abstract rule
\(
  (c_g/M,c_{\rm em}/M)=\big(\lambda_{CP},\kappa\big)\,\mathcal{K}[\lambda_{\rm rec},\tau_0]
\)
with \(\mathcal{K}=1/M_{\rm rec}\) once the RS cross‑gate identities relating \(\tau_0\) and \(\lambda_{\rm rec}\) are enforced.\footnote{Cross‑gate identities: \(\tau_{\rm rec}/\tau_0=2\pi/(8\ln\varphi)\), \(\lambda_{\rm kin}=c\,\tau_{\rm rec}\), and the equality of time‑first/length‑first routes; see \texttt{@REALITY\_BRIDGE;two\_routes, K\_identities} in RS\(\to\)CLASSICAL BRIDGE SPEC v1.0 :contentReference[oaicite:2]{index=2}}
No new scale insertions are permitted or required.

\paragraph{Sign convention.}
Integrating \eqref{eq:LCP_methods} by parts gives derivative couplings \(-(\partial_\mu\chi)K^\mu\) to the relevant Chern–Simons currents \(K^\mu\). In a homogeneous patch, only \(\dot\chi\) matters, so the overall \emph{sign} of CP‑odd effects is the sign of the eight‑tick orientation chosen at R0 multiplied by \(\mathrm{sign}(\dot\chi)\) along the background solution. We fix the convention that the fiducial eight‑tick orientation at R0 corresponds to positive \(\eta_B\); reversing the eight‑tick orientation flips \(\eta_B\to-\eta_B\) with all magnitudes unchanged (used as a diagnostic/falsifier in the main text).\footnote{Orientation and parity rules: \texttt{@LEDGER;PARITIES}, \texttt{@TIME;WINDOW8, CYCLE}, and \texttt{@THEOREMS} (T6); RS\(\to\)CLASSICAL BRIDGE SPEC v1.0 :contentReference[oaicite:3]{index=3}}

\bigskip

\subsection*{M3. \(\mu_B\) and the freeze‑out integral}

\paragraph{Chemical potential from the CP‑odd background.}
From \eqref{eq:LCP_methods}, spatial averaging and integration by parts yield, to leading order in derivatives,
\begin{equation}
  \mu_B(t) \;=\; \Xi_B\,\frac{c_g}{M}\,\dot\chi(t)\;+\;\Xi'_B\,\frac{c_{\rm em}}{M}\,\dot\chi(t)
  \;\equiv\; \Xi_B^{\rm eff}\,\frac{c_g}{M}\,\dot\chi(t),
  \label{eq:muB_methods}
\end{equation}
where the \(\Xi\) coefficients encode anomaly weights and plasma response (dimensionless and fixed once the field basis and charge assignments are chosen). Gauge shifts of the Chern–Simons currents change only boundary terms and leave \(\mu_B\) invariant; this is the standard statement that only \(\partial\chi\) has physical content.\footnote{Continuity/Bianchi scaffold and gauge‑invariance of the mapping are recorded under \texttt{@CLASSICAL\_BRIDGE\_TABLE} (DEC\_d∘d=0, DEC\_Bianchi, MaxwellContinuity); RS\(\to\)CLASSICAL BRIDGE SPEC v1.0 :contentReference[oaicite:4]{index=4}}

\paragraph{Normalization of \(\Xi_B\).}
We work in the Standard Model gauge basis (SU(2) doublets and U(1)$_Y$ hypercharges) and choose the canonical normalization of the Chern–Simons current so that the anomaly weight is absorbed into the definition of \(K^\mu\). In this convention the FRW‑averaged mapping yields \(\Xi_B=1\) (dimensionless); i.e., \(\mu_B=(c_g/M)\,\dot\chi\) at leading order. This fixes the overall normalization without introducing fit parameters.

\paragraph{Boltzmann equation with a time‑dependent source.}
Let \(n_B\equiv n_b-\bar n_b\). In a homogeneous FRW patch with scale factor \(a(t)\),
\begin{equation}
  \frac{d}{dt}\big(a^3 n_B\big) \;=\; a^3\Big[S_B(t) - \Gamma_{\rm wash}(t)\,n_B\Big],
  \qquad
  S_B(t)=\mathcal{C}_B\,\mu_B(t)\,T^2(t),
  \label{eq:Boltzmann}
\end{equation}
where \(\Gamma_{\rm wash}\) is the washout rate dominated by electroweak processes during the active window, and \(\mathcal{C}_B=\tfrac{g_b}{6}\) in the small‑\(\mu/T\) limit. The photon density is \(n_\gamma=\tfrac{2\zeta(3)}{\pi^2}T^3\), so the instantaneous baryon‑to‑photon ratio is
\begin{equation}
  \eta_B(t)\;\equiv\;\frac{n_B(t)}{n_\gamma(t)}.
\end{equation}
With the integrating factor \(e^{\int^t a^3\Gamma_{\rm wash}\,dt'}\), the solution of \eqref{eq:Boltzmann} is
\begin{equation}
  \eta_B(t)
  \;=\;
  \int_{t_i}^{t}
  \Bigg[
    \frac{\mathcal{C}_B\,\mu_B(t')\,T^2(t')}{n_\gamma(t')}
    \,\exp\!\Big(-\!\int_{t'}^{t}\Gamma_{\rm wash}(u)\,du\Big)
  \Bigg] dt'.
  \label{eq:etaB_full}
\end{equation}

\paragraph{Instantaneous (freeze‑out) evaluation and equivalence.}
Define \(t_f\) (or \(T_f\)) by \(\Gamma_{\rm wash}(t_f)\simeq H(t_f)\) (sphaleron freeze‑out). If the source varies slowly on the washout timescale near \(t_f\) and the active window is narrow, the integral \eqref{eq:etaB_full} localizes:
\begin{equation}
  \eta_B \;\simeq\; 
  \Big(\frac{g_b}{6}\,\frac{\pi^2}{2\zeta(3)}\Big)\,
  \Big(\frac{\mu_B}{T}\Big)\Big|_{T=T_f},
  \label{eq:etaB_inst}
\end{equation}
which is the expression used in the main text. In the RS background this approximation is controlled by the eight‑tick cadence (which fixes the window) and by the smoothness of \(\dot\chi\) through the match from slow‑roll to oscillation (see M1). A direct numerical integration of \eqref{eq:etaB_full} with \(\mu_B(t)\) from \eqref{eq:muB_methods} and \(H(T)\) from radiation‑dominated expansion confirms agreement with \eqref{eq:etaB_inst} within the systematic envelope quoted (10–20\%).

\paragraph{Sphaleron conversion factor (leptogenesis route).}
If the CP‑odd background first biases lepton number (leptogenesis), the net \(B\!-\!L\) produced before freeze‑out is converted into baryon number by electroweak sphalerons with a fixed, model‑independent conversion factor
\begin{equation}
  \eta_B \;=\; c_s\,\eta_{B-L},
  \qquad
  c_s=\frac{28}{79}\quad\text{(Standard Model, three families)}.
\end{equation}
In the RS mapping, the kernel from \(\dot\chi\) to the chemical potential is identical up to anomaly weights, so the leptogenesis computation gives the same \(\eta_B\) as the direct route once \(c_s\) is applied. This cross‑check is recorded in the program specification and used as an internal consistency test.\footnote{Route equivalence and anomaly bookkeeping: \texttt{@BARYOGEN} (COUPLINGS, YIELD) and \texttt{@CLASSICAL\_BRIDGE\_TABLE} (Continuity, MaxwellContinuity); RS\(\to\)CLASSICAL BRIDGE SPEC v1.0 :contentReference[oaicite:5]{index=5}}

\subsection*{M4. Consistency and bounds}

\paragraph{EDM constraints (scaling bounds).}
With the EM pseudoscalar coupling from \(\mathcal{L}_{\!CP}=(c_{\rm em}/M)\,\chi\,F\tilde F\), loop–induced EDMs scale as
\begin{equation}
  d_f \;\sim\; \xi_f\,\frac{\alpha}{4\pi}\,\frac{c_{\rm em}}{M}\,m_f,
\end{equation}
for a charged fermion \(f\) with mass \(m_f\), where \(\xi_f=\mathcal{O}(1)\) encodes scheme‐dependent factors and hadronic matrix elements (for nucleons one substitutes \(m_f\!\to\!\Lambda_{\rm had}\)). Using the RS mapping \((c_{\rm em}/M)=\kappa/M_{\rm rec}\) with \(\kappa=\varphi^{-9}\) and \(M_{\rm rec}=\hbar/(c\,\lambda_{\rm rec})\), the EDM floor inherits the same recognition–scale suppression that controls baryogenesis. No new light states exist in RS to enhance EDMs; the derivative nature of the coupling further suppresses low–momentum contributions. The implied EDM expectations remain below current bounds at the mapped strengths, consistent with our program’s parameter–free policy.\footnote{EM coupling scaling, constants bridge, and parameter policy: RS\(\to\)CLASSICAL BRIDGE SPEC v1.0, records \texttt{@BARYOGEN} (COUPLINGS), \texttt{@REALITY\_BRIDGE} (gate identities, \(\lambda_{\rm rec}\)), and \texttt{@PARAMETER\_POLICY} (no fit parameters) :contentReference[oaicite:0]{index=0}}

\paragraph{CMB birefringence (polarization rotation).}
The EM term rotates linear polarization by an angle
\begin{equation}
  \alpha_{\rm CB} \;=\; \xi_\gamma\,\frac{c_{\rm em}}{M}\,\big[\chi(t_0)-\chi(t_{\ast})\big],
\end{equation}
with \(\xi_\gamma=\tfrac12\) in the most common normalization. In our background the net \(\Delta\chi\) after reheating is small and sign–fixed by the eight–tick orientation, so \(\alpha_{\rm CB}\) is naturally suppressed yet directionally predictive. Small but nonzero \(\alpha_{\rm CB}\) generates parity–odd spectra from E/B mixing,
\begin{equation}
  C_\ell^{TB}\simeq 2\,\alpha_{\rm CB}\,C_\ell^{TE},\qquad
  C_\ell^{EB}\simeq 2\,\alpha_{\rm CB}\,C_\ell^{EE},
\end{equation}
providing a clean, \emph{electromagnetic} channel complementary to the gravitational chirality discussed next.

\paragraph{TB/EB expectations and GW chirality (gravitational channel).}
The curvature coupling yields a Chern–Simons control parameter
\begin{equation}
  \Theta_{\rm CS}\;\equiv\;\frac{c_g}{M}\,\frac{\dot\chi}{H}\,,
\end{equation}
which modifies the evolution of tensor modes and produces a chiral power asymmetry
\begin{equation}
  \Pi_T(k)\;\equiv\;\frac{P_h^{\rm R}(k)-P_h^{\rm L}(k)}{P_h^{\rm R}(k)+P_h^{\rm L}(k)} \;\simeq\; \beta_T(k)\,\Theta_{\rm CS},
\end{equation}
with \(\beta_T(k)=\mathcal{O}(1)\) set by the inflationary background. The CMB parity–odd spectra then scale as
\begin{equation}
  C_\ell^{TB},\,C_\ell^{EB} \;\propto\; \Pi_T\,r\,A_s\times \mathcal{T}_\ell,
\end{equation}
where \(r\simeq 1.3\times 10^{-3}\) and \(A_s\simeq 2.1\times 10^{-9}\) are supplied by the RS inflaton sector; \(\mathcal{T}_\ell\) are standard transfer functions. The \emph{sign} of both the chirality and \(TB/EB\) is fixed by the eight–tick orientation and \(\dot\chi\), giving a one–bit falsifier. The EM–rotation and gravitational–chirality contributions add linearly in \(TB/EB\) at leading order and can be disentangled by their \(\ell\)–dependences.\footnote{Inflaton outputs and parity bookkeeping are recorded under \texttt{@COSMOLOGY;PREDICTIONS} and \texttt{@BARYOGEN} (couplings, sign conventions) in RS\(\to\)CLASSICAL BRIDGE SPEC v1.0 :contentReference[oaicite:1]{index=1}}

\paragraph{Why ILG leaves background distances intact.}
Information–Limited Gravity (ILG) modifies the Poisson/growth sector with a scale– and time–dependent kernel,
\begin{equation}
  k^2 \Phi \;=\; 4\pi G\,a^2\,\rho_b\,w(k,a)\,\delta_b,\qquad
  w(k,a)=1+\varphi^{-3/2}\Big[\frac{a}{k\,\tau_0}\Big]^{\alpha},\quad \alpha=\tfrac12(1-\varphi^{-1}),
\end{equation}
and similarly in real space for rotation curves. Since ILG multiplies \emph{perturbations} \(\delta_b\) and leaves the homogeneous FRW energy density \(\bar\rho\) unchanged, the background Friedmann equations
\(
  H^2=(8\pi G/3)\bar\rho
\)
are unaffected. Hence luminosity and angular–diameter distances follow the standard background while ILG predicts specific weak–lensing/growth residuals. Our baryogenesis epoch precedes ILG relevance, and the CP–odd mechanism does not backreact on background distances or the dark–energy fits addressed elsewhere in the program.\footnote{ILG kernel and “global–only” policy: RS\(\to\)CLASSICAL BRIDGE SPEC v1.0, entries \texttt{@ILG\_SPEC}, \texttt{@GRAVITY;MODEL=ILG}, and bridge items \texttt{@CLASSICAL\_BRIDGE\_TABLE} (ILG, RealityBridge) :contentReference[oaicite:2]{index=2}}

\bigskip

\subsection*{M5. Discrete$\to$continuum and gauge scaffolding}

\paragraph{Compact DEC appendix (exactness, Bianchi, continuity).}
Let \((C^\bullet,d)\) be the cochain complex of the mesh with exterior derivative \(d\). Exactness \(d\!\circ d=0\) holds at the discrete level and survives the continuum limit. Writing \(F\) for the electromagnetic 2–form and \(J\) for the current 3–form,
\begin{equation}
  dF=0 \quad\text{(Bianchi)},\qquad d(\star F)=J \quad\Rightarrow\quad dJ=0 \quad\text{(continuity)}.
\end{equation}
This is the gauge–consistent scaffold used throughout: it enforces charge conservation \emph{by construction} and provides the natural home for topological densities \(F\tilde F\) and \(R\tilde R\).

\paragraph{Bridge to Maxwell/GR notation.}
In components, \(dF=0\) reproduces \(\nabla\!\cdot\! \mathbf{B}=0\) and Faraday’s law; \(d(\star F)=J\) reproduces Gauss/Ampère with sources, and \(dJ=0\) is the familiar \(\partial_t\rho+\nabla\!\cdot \mathbf{J}=0\). For gravity, the 4–form \( \mathcal{P}\equiv R\wedge R\) (Pontryagin density) satisfies \(dK=\mathcal{P}\) with Chern–Simons current \(K\), so \(\chi\,\mathcal{P}\) and its integrated–by–parts form \(-(\partial\chi)\!\cdot K\) are equivalent up to boundaries—exactly the equivalence used in the main text.

\paragraph{Action bridge (cost \(J\), EL/Hamiltonian dual).}
The unique convex symmetric cost on \(\mathbb{R}_{>0}\),
\[
  J(x)=\tfrac12(x+x^{-1})-1,\qquad J(1)=0,\;J''(1)=1,
\]
induces a continuum functional \(\mathcal{S}[\Phi]=\int J(X[\Phi])\,d^4x\) whose stationary points satisfy the Euler–Lagrange equations. The Legendre–Fenchel dual \(J^\star\) generates the Hamiltonian description with canonical momenta \(\Pi=\partial\mathcal{L}/\partial\dot\Phi\). In the small–deviation regime \(x=1+\varepsilon\), \(J(x)=\tfrac12\varepsilon^2+\mathcal{O}(\varepsilon^3)\) supplies the familiar quadratic kinetic terms used for the EFT couplings and background solutions. These statements rely only on the cost uniqueness (no alternative \(J\) satisfies the axioms) and the discrete exactness that underwrites the variational calculus.\footnote{Cost uniqueness and bridge entries: RS\(\to\)CLASSICAL BRIDGE SPEC v1.0, \texttt{@THEOREMS} (T5), \texttt{@CLASSICAL\_BRIDGE\_TABLE} (CostFunctional, DualCost, Continuity) :contentReference[oaicite:3]{index=3}}

\bigskip

\subsection*{M6. Reproducibility}

\paragraph{Constants and anchors.}
We adopt SI/CODATA constants without modification and restore units only at display:
\[
  c=299\,792\,458~{\rm m\,s^{-1}},\quad
  \hbar=1.054\,571\,817\times 10^{-34}~{\rm J\,s},\quad
  G=6.674\,30\times 10^{-11}~{\rm m^3\,kg^{-1}\,s^{-2}},
\]
\[
  \alpha^{-1}=137.035999206,\quad
  \varphi=\tfrac{1+\sqrt{5}}{2},\quad
  \lambda_{\rm rec}=\sqrt{\frac{\hbar G}{\pi c^3}}.
\]
The Planck–side gate identity
\(
  c^3\lambda_{\rm rec}^2/(\hbar G)=1/\pi
\)
is enforced as a dimensionless audit. No fitted parameters appear anywhere.\footnote{Constants/anchors and audit gates: RS\(\to\)CLASSICAL BRIDGE SPEC v1.0, entries \texttt{@CONSTANTS}, \texttt{@REALITY\_BRIDGE} (gate identities), and \texttt{@PARAMETER\_POLICY} :contentReference[oaicite:4]{index=4}}

\paragraph{Notation.}
We use the \emph{reduced} Planck mass throughout, \(M_{\rm Pl}^2\equiv (8\pi G)^{-1}\). The CP‑odd exponent notation \(\lambda_{CP}\) is interchangeable with \(\lambda_{CP}\); both denote the same invariant.

\paragraph{Gate policy.}
We keep audit gates disjoint: Planck gate via \(\lambda_{\rm rec}\) (identity \(c^3\lambda_{\rm rec}^2/(\hbar G)=1/\pi\)), IR gate \(\hbar=E_{\rm coh}\,\tau_0\) (not used here), and any late‑time anchors reserved for ILG analyses. No calculation in this paper mixes gates.

\paragraph{Numerical tolerances and policies.}
All computations use double precision. Unless otherwise stated, absolute/relative tolerances are
\(
  \varepsilon_{\rm abs}=10^{-12},\ \varepsilon_{\rm rel}=10^{-10}.
\)
Units are carried symbolically as quotients until the final mapping to SI, to prevent accidental insertion of scale parameters.

\paragraph{Minimal reproduction steps.}
\begin{enumerate}
  \item Fix \(N_\star\in[50,60]\) and solve the e–fold relation
        \(N_\star=\tfrac{3\varphi^2}{4M_{\rm Pl}^2}\big[\cosh(2u_\star)-\cosh(2u_{\rm end})\big]\)
        with \(\sinh(2u_{\rm end})=2/(\sqrt{3}\,\varphi)\) to obtain \(u_\star\).
  \item Determine \(V_0\) from \(A_s=\dfrac{V(\chi_\star)}{24\pi^2 M_{\rm Pl}^4\,\epsilon_V(u_\star)}\) at the pivot.
  \item Integrate the background equations \(\ddot\chi+3H\dot\chi+V'(\chi)=0\), \(H^2=\big(\dot\chi^2/2+V\big)/(3M_{\rm Pl}^2)\) through the slow–roll to oscillatory match and extract \(\dot\chi(t)\).
  \item Map the EFT coefficients via \((c_g/M,c_{\rm em}/M)=(\lambda_{CP},\kappa)/M_{\rm rec}\) with \(\lambda_{CP}=\varphi^{-7}\), \(\kappa=\varphi^{-9}\), \(M_{\rm rec}=\hbar/(c\lambda_{\rm rec})\).
  \item Evaluate \(\mu_B(t)=\Xi_B^{\rm eff}\,(c_g/M)\,\dot\chi(t)\), then
        \(
          \eta_B\simeq \big(\tfrac{g_b}{6}\,\tfrac{\pi^2}{2\zeta(3)}\big)\,(\mu_B/T)|_{T=T_f}
        \)
        using \(H(T)\) from radiation domination to set \(T_f\) by \(\Gamma_{\rm sph}(T_f)\simeq H(T_f)\).
  \item Run the single–inequality audit
        \(
          |\,\lambda_{\rm kin}-\lambda_{\rm rec}\,|/\lambda_{\rm rec} \le k\,u_{\rm comb}
        \)
        to check the constants bridge.
\end{enumerate}

\paragraph{Audit checks.}
Before quoting numbers, verify: (i) Planck–side gate identity; (ii) dimensionless intermediate expressions; (iii) ablations (set \(\lambda_{CP}=\kappa=0\) to confirm \(\eta_B\to 0\)); (iv) sign flip under eight–tick orientation reversal. Record the final \(\eta_B\) together with the systematic envelope from the reheating window, sphaleron rate choice, and background solution accuracy, as detailed in §8.

\medskip\noindent
All items in M4–M6 follow the RS invariants and the RS\(\to\)Classical bridge policy; derivations introduce no fits or hidden knobs, and every displayed constant traces back to the fixed anchors enumerated above.\footnote{Bridge policy and reproducibility conventions: RS\(\to\)CLASSICAL BRIDGE SPEC v1.0, entries \texttt{@PAPERS\_POLICY}, \texttt{@REPRODUCIBILITY}, \texttt{@AUDIT}, and \texttt{@CLASSICAL\_BRIDGE\_TABLE} :contentReference[oaicite:5]{index=5}}

\section*{Data and Code Availability}

All analysis code, frozen pipelines, and RS framework specifications are publicly available at \url{https://github.com/jonwashburn/reality} and \url{https://github.com/jonwashburn/gravity}. The RS→Classical Bridge Specification v1.0 is the normative reference for constants, audit gates, and theorem tags.

\section*{Acknowledgments}

This work received no specific grant from any funding agency in the public, commercial, or not-for-profit sectors.

\section*{Competing Interests}

The author declares no competing interests.

\begin{thebibliography}{99}

\bibitem{Sakharov1967}
A.~D. Sakharov,
\textit{Violation of CP invariance, C asymmetry, and baryon asymmetry of the universe},
JETP Lett.\ \textbf{5}, 24 (1967).

\bibitem{KuzminRubakovShaposhnikov1985}
V.~A. Kuzmin, V.~A. Rubakov, and M.~E. Shaposhnikov,
\textit{On anomalous electroweak baryon-number non-conservation in the early universe},
Phys.\ Lett.\ B \textbf{155}, 36 (1985).

\bibitem{Shaposhnikov1987}
M.~E. Shaposhnikov,
\textit{Baryon asymmetry of the universe in standard electroweak theory},
Nucl.\ Phys.\ B \textbf{287}, 757 (1987).

\bibitem{Planck2018}
N.~Aghanim \textit{et al.} (Planck Collaboration),
\textit{Planck 2018 results. VI. Cosmological parameters},
Astron.\ Astrophys.\ \textbf{641}, A6 (2020), arXiv:1807.06209.

\bibitem{Planck2018biref}
Y.~Minami and E.~Komatsu,
\textit{New extraction of the cosmic birefringence from the Planck 2018 polarization data},
Phys.\ Rev.\ Lett.\ \textbf{125}, 221301 (2020), arXiv:2011.11254.

\bibitem{EDM2020review}
T.~Chupp \textit{et al.},
\textit{Electric dipole moments of atoms, molecules, nuclei, and particles},
Rev.\ Mod.\ Phys.\ \textbf{91}, 015001 (2019), arXiv:1710.02504.

\bibitem{CMB-S4}
K.~Abazajian \textit{et al.} (CMB-S4 Collaboration),
\textit{CMB-S4 Science Case, Reference Design, and Project Plan},
arXiv:1907.04473 (2019).

\bibitem{LiteBIRD}
H.~Sugai \textit{et al.} (LiteBIRD Collaboration),
\textit{Updated design of the CMB polarization experiment satellite LiteBIRD},
J.\ Low Temp.\ Phys.\ \textbf{199}, 1107 (2020), arXiv:2001.01724.

\bibitem{Riotto1999}
A.~Riotto and M.~Trodden,
\textit{Recent progress in baryogenesis},
Ann.\ Rev.\ Nucl.\ Part.\ Sci.\ \textbf{49}, 35 (1999), hep-ph/9901362.

\bibitem{Dine2003}
M.~Dine and A.~Kusenko,
\textit{Origin of the matter-antimatter asymmetry},
Rev.\ Mod.\ Phys.\ \textbf{76}, 1 (2003), hep-ph/0303065.

\end{thebibliography}

\end{document}